\documentclass[a4paper,11pt]{report}
\usepackage{../packages/canevas_modele}
\settasks{
  % the next two should be set to the same value so labels are aligned to the
  % left
  label-width = 2em ,
  item-indent = 3em ,
  %before-skip = -\parskip , % undo paragraph skip
  %after-skip = -\parskip , % undo paragraph skip
  label-offset=1em,
  after-item-skip = 1em % undo paragraph skip
}
\usepackage{enumitem}
\setlist[enumerate]{align=left,leftmargin=1cm,itemsep=10pt,parsep=0pt,topsep=0pt,rightmargin=0.5cm}
\setlist[itemize]{align=left,labelsep=1em,leftmargin=*,itemsep=0pt,parsep=0pt,topsep=0pt,rightmargin=0cm}
\renewcommand{\titreChapitre}{Limites et continuité} 
\renewcommand{\regroupement}{Mathématiques 3$^{\text{e}}$ niveau normal}
\begin{document}
	\begin{center}
	{\Large \bfseries  \titreChapitre}
\end{center}
{\bfseries Tableau récapitulatif} 
\vspace{-0.8cm}
\begin{center}
\begin{longtblr}[
  caption = {},
]{
  colspec={|X[1,m,l]|c|c|},
  rowhead = 1,
  row{1} = {font=\bfseries},
  cell{2}{1} = {r=1}{}, % multirow
  cell{2}{2} = {r=1}{}, % multirow
  cell{2}{3} = {r=1}{}, % multirow
  cell{3}{1} = {r=1}{},
  cell{3}{2} = {r=1}{},
  cell{3}{3} = {r=1}{},
  cell{4}{1} = {r=1}{},
  cell{4}{2} = {r=1}{},
  cell{4}{3} = {r=1}{},
  cell{5}{1} = {r=1}{},
  cell{5}{2} = {r=1}{},
  cell{5}{3} = {r=1}{},
  cell{6}{1} = {r=1}{},
  cell{6}{2} = {r=1}{},
  cell{6}{3} = {r=1}{},
  cell{7}{1} = {r=1}{},
  cell{7}{2} = {r=1}{},
  cell{7}{3} = {r=1}{},
  cell{8}{1} = {r=1}{},
  cell{8}{2} = {r=1}{},
  cell{8}{3} = {r=1}{},
  cell{9}{1} = {r=1}{},
  cell{9}{2} = {r=1}{},
  cell{9}{3} = {r=1}{},
  cell{10}{1} = {r=1}{},
  cell{10}{2} = {r=1}{},
  cell{10}{3} = {r=1}{},
}
\hline[0.8pt]
&Activités &\makecell{Exercices\\ à faire}\\
\hline[0.8pt]
\makecell{{\bf Continuité}\\Définition intuitive\\Théorème valeur intermédiaire} 
&\makecell{1, 2}&\makecell{1 à 4}\\ 
\hline
\makecell{{\bf Limites en un nombre}\\Définitions et exemples\\Lien avec continuité} 
&\makecell{4, 5}&\makecell{5 à 7}\\
\hline
\makecell{Limites de fonctions\\élémentaires}
&\makecell{}&\makecell{8, 9}\\
\hline
\makecell{Propriétés des limites}
&\makecell{}&\makecell{10 à 19}\\
\hline
\makecell{Indéterminations\\types $\frac{0}{0}$ et $\frac{1}{0}$}
&\makecell{6, 7, 8}&\makecell{20 à 34}\\
\hline
\makecell{Asymptotes verticales}
&\makecell{9}&\makecell{35, 36}\\
\hline
\makecell{{\bf Limites en l'infini}\\Définition et algèbre\\de l'infini\\
Indéterminations avec\\de l'infini}
&\makecell{10, 11}&\makecell{37 à 45}\\
\hline
\makecell{Asymptotes horizontales}
&\makecell{12, 13, 14, 15}&\makecell{46 à 52}\\
\hline
\makecell{{\bf Limites trigonométriques}\\$\lim_{x \to 0} \frac{\sin x}{x} = 1$\\Indéterminations trigonométriques}
&\makecell{16, 17}&\makecell{53 à 55}\\
\hline[1pt]
\end{longtblr}
Terminé le~:~\ligne{6} 

Revu aux dates~:
\ligne{6}
\ligne{6}

Difficultés rencontrées~:
\ligne{18}
\ligne{18}
\ligne{18}
\ligne{18}
\ligne{18}
\ligne{18}
\ligne{18}
\ligne{18}
Remarques~:
\ligne{18}
\ligne{18}
\ligne{18}
\ligne{18}
\ligne{18}
\ligne{18}
\ligne{18}
\ligne{18}
\end{center}
\end{document}
