\documentclass[a4paper,12pt]{article}
\usepackage{../packages/coursCollege}
\newcommand{\Chapitre}{Limites et continuité}
\renewcommand{\path}{../}
\externaldocument{output/3M-limites/3M-limites}

\usepackage[    %backend=biber, 
    natbib=true,
    style=numeric,
    sorting=none]{biblatex}  % Load biblatex for bibliography handling
\addbibresource{biblio-der.bib}  
\renewcommand\refname{Sources}
\renewcommand{\cours}{3MA1~--~EG~--~ns~--~2025-2026}
\begin{document}
\tocloftpagestyle{fancy}
% Reduce space between section entries
\setlength{\cftbeforesecskip}{2pt}

% Reduce indentation for section entries
\setlength{\cftsecindent}{1em}

\begin{center}
	{\bfseries \Huge Chapitre 1~: \\Fonctions continues et calcul de limites}
	

\end{center}
\vspace{-1cm}
\tableofcontents

\newpage

\section{Exercices}
Dans les corrigés vous trouverez seulement des éléments de réponses. {\bfseries Une réponse détaillée et justifiée} à l'aide des résultats du cours est attendue. L'enseignant est là pour valider votre réponse et vérifier votre rédaction. {\bfseries En évaluation, une réponse qui n'est pas justifiée selon les attentes ne recueillera pas de point.} 

\subsection{Continuité}
\insertexo{1n93r}{false}{both}
\insertexo{59swy}{false}{both}
\insertexo{qp2mw}{false}{both}
\insertexo{6csvs}{false}{both}
\subsection{Limites en un nombre}
\insertexo{4dptb}{false}{both}
\insertexo[exo:p1f]{g4gbb}{false}{both}
\insertexo{n9sxp}{false}{both}
\insertexo[exo:p2d]{tecfn}{false}{both}
\insertexo{wnd21}{false}{both}
\insertexo{3vr28}{false}{both}
\insertexo{c3uc9}{false}{both}
\insertexo{1bqh9}{false}{both}
\insertexo{myjcq}{false}{both}
\insertexo{dpjwk}{false}{both}
\insertexo{97ryj}{false}{both}
\insertexo{n1axs}{false}{both}

\insertexo{gb3ag}{false}{both}
\insertexo{5911p}{false}{both}

\insertexo{hsytn}{false}{both}
\subsubsection{Limites infinies « $\dfrac{1}{0}$ » et indétermination du type « $\dfrac{0}{0}$ »}

\insertexo[exo:p1d]{vyj67}{false}{both}
\insertexo{ac9v2}{false}{both}
\insertexo{98752}{false}{both}
\insertexo{gg7hk}{false}{both}
\insertexo{f7dft}{false}{both}
\insertexo{u44h9}{false}{both}
\insertexo{pp62m}{false}{both}
\insertexo{b3p6g}{false}{both}
\insertexo{n41yn}{false}{both}
\insertexo{8j18h}{false}{both}
\insertexo{yhzh8}{false}{both}
\insertexo{9yqnb}{false}{both}
%--> corrigés vérifiés après ici.
\insertexo{k3bry}{false}{both}
\insertexo{1w6s3}{false}{both}
\insertexo{nxc2s}{false}{both}
\subsubsection{Asymptotes verticales}
\insertexo{89av7}{false}{both}
\insertexo{1w3pd}{false}{both}
\subsection{Limites en l'infini}
\insertexo{x5h3r}{false}{exo}
\insertexo{n9qmn}{false}{exo}
\insertexo{t4gwt}{false}{exo}
\insertexo{bqknk}{false}{exo}
\insertexo{du7be}{false}{exo}
\insertexo{6qym1}{false}{exo}
\insertexo{aatq8}{false}{exo}
\insertexo{gm9uk}{false}{exo}
\insertexo{taje8}{false}{exo}
\insertexo{y4dv7}{false}{exo}
\insertexo{y4385}{false}{exo}
\insertexo{2at6d}{false}{exo}
%MA2\insertexo{us8px}{false}{exo}
\insertexo{5en1r}{false}{exo}
\insertexo{zu8c5}{false}{exo}
\insertexo{1gayt}{false}{exo}
%MA2\insertexo{pe1ew}{false}{exo}
%MA2\insertexo{mj5qq}{false}{exo}
%MA2\insertexo{hgcu8}{false}{exo}
%MA2\insertexo{ek51g}{false}{exo}
%MA2\insertexo{955md}{false}{exo}
%MA2\insertexo{mkn1h}{false}{exo}
%MA2\insertexo{pzy7s}{false}{exo}
\insertexo{3yjzh}{false}{exo}

\subsection{Limites de fonctions trigonométriques}
\insertexo{byf5j}{false}{exo}
\insertexo{h9w1s}{false}{exo}
\insertexo{jqhm4}{false}{exo}


%Pas inclus 
%\insertexo[exo:p1d]{vyj67}{false}{exo}
%\insertexo{4dptb}{false}{exo}
%\insertexo[exo:p1f]{g4gbb}{false}{exo}
%%\insertexo{extez}{false}{exo}
%%\insertexo{znghn}{false}{exo}
%\insertexo[exo:p2d]{tecfn}{false}{exo}
%%\insertexo{fazjq}{false}{exo}
%\insertexo{f7dft}{false}{exo}
%\insertexo{ac9v2}{false}{exo}
%\insertexo{nxc2s}{false}{exo}
%\insertexo{n9qmn}{false}{exo}
%\insertexo{bqknk}{false}{exo}
%\insertexo{gg7hk}{false}{exo}
%\insertexo{n41yn}{false}{exo}
%\insertexo{8j18h}{false}{exo}
%\insertexo{1w3pd}{false}{exo}
%\insertexo{x5h3r}{false}{exo}
%\insertexo{1bqh9}{false}{exo}
%\insertexo{du7be}{false}{exo}
%\insertexo{k3bry}{false}{exo}
%\insertexo{gm9uk}{false}{exo}
%\insertexo{taje8}{false}{exo}
%\insertexo{3vr28}{false}{exo}
%\insertexo{gb3ag}{false}{exo}
%\insertexo{c3uc9}{false}{exo}
%\insertexo{myjcq}{false}{exo}
%\insertexo{dpjwk}{false}{exo}
%\insertexo{97ryj}{false}{exo}
%\insertexo{n1axs}{false}{exo}
%\insertexo{wnd21}{false}{exo}
%\insertexo{98752}{false}{exo}
%\insertexo{t4gwt}{false}{exo}
%\insertexo{1w6s3}{false}{exo}
%\insertexo{yhzh8}{false}{exo}
%\insertexo{9yqnb}{false}{exo}
%\insertexo{byf5j}{false}{exo}
%\insertexo{h9w1s}{false}{exo}
%\insertexo{jqhm4}{false}{exo}
%\insertexo{1n93r}{false}{exo}
%\insertexo{qp2mw}{false}{exo}
%\insertexo{6qym1}{false}{exo}
%\insertexo{aatq8}{false}{exo}
\newpage
\section{Pistes pour réviser}
\begin{itemize}
	\item Quelles sont les trois conditions qu'une fonction doit satisfaire {\bfseries simultanément} pour être continue en un point~? Donner des exemples de fonctions qui satisfont seulement deux des trois conditions.
	\item Quelle est la première étape à effectuer dans tout calcul de limite~?
	\item Une limite peut être déterminée ou indéterminée, en un nombre ou en l'infini. Donner des exemples de chacun de ces cas. 
	\item Qu'est-ce qu'une indétermination~? Quels sont les différents cas, leurs particularités et les différentes manières de les lever~? Donner des exemples.
	\item Quelle technique a-t-on à diposition pour calculer une limite qui contient une expression littérale avec une racine~?
%MA2	\item Quelles sont les différences entre les asymptotes verticales, horizontales et obliques~? Expliquer et donner des exemples.
	\item Donner des conditions pour l'existence d'asymptotes verticales. Exemples. 
	\item Donner des conditions pour l'existence d'asymptotes horizontales. Exemples.
%MA2	\item Donner des conditions pour l'existence d'asymptotes obliques. Exemples.
	\item Quelles sont les deux façons qui ont étudiées pour calculer une limite trigonométrique~? Exemples. 
\end{itemize}
 \nocite{*}
 \vspace{-10pt}
\defbibnote{myprenote}{Les sources suivantes ont majoritairement été utilisées pour construire ce cours. Les exercices et activités proviennent également principalement de ces ouvrages. D'autres exercices ont été adaptés ou sont inspirés de ressources partagées par des collègues ou trouvées sur internet. Leur contribution mineure les exclut de cette liste.}
 \printbibliography[prenote=myprenote,title={Sources du cours}] 
\end{document}

