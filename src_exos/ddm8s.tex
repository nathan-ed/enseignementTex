\titre{}
\theme{dérivées}
\auteur{Nathan Scheinmann}
\niveau{3M}
\source{fundamentum}
\type{serie}
\piments{2}
\pts{}
\annee{2425}

\contenu{
\tcblower
La portée $P = \text{OA}$ d'un projectile lancé (dans le vide) avec une vitesse initiale $v_0$ et un angle d'élévation $\varphi$ est donnée par $P = \dfrac{v_0^2 \cdot \sin(2\varphi)}{g}$, $g$ étant l'accélération de la pesanteur. Pour une vitesse initiale donnée, déterminer la valeur de l'angle $\varphi$ pour laquelle la portée est maximale.
\begin{center}
\begin{tikzpicture}[scale=0.7]
    \tkzDefPoint(0,0){O}
    \tkzDefPoint(9,0){A}
    \tkzDefPoint(47:3){v_dir}
    
    \draw[->, thick] (O) -- (v_dir) node[above left] {$\vec{v_0}$};
    \draw[thick] (O) .. controls (3,3.5) and (6,3.5) .. (A);
    
    \tkzDrawSegment(O,A)
    \tkzLabelPoints[below](O,A)
    
    \tkzMarkAngle[size=1.2, fill=gray!20](A,O,v_dir)
    \tkzLabelAngle[pos=1.5](A,O,v_dir){$\varphi$}
\end{tikzpicture}
\end{center}
}
\correction{
\tcblower
}

