\titre{1}
\theme{trigo}
\auteur{Nathan Scheinmann}
\niveau{1M}
\source{sesamath-1M-trigo}
\type{serie}
\piments{2}
\pts{}
\annee{2425}

\contenu{
	\tcblower
	Construire un triangle $\triangle ABC$ tel que $\overline{AB}=4{,}5~\text{cm}, \widehat{BAC}=27^\circ$ et $\widehat{CBA}=63^\circ$.
	\begin{tasks}
		\task Ce triangle est-il rectangle~? Pourquoi~?
		\task Calculer la les longueurs $\overline{AC}$ et $\overline{BC}$ arrondies au millième.
	\end{tasks}
}
\correction{
\tcblower

\textit{Generated by AI}

\begin{tasks}
\task \textbf{Le triangle est-il rectangle ?}

Pour qu'un triangle soit rectangle, la somme des trois angles doit être égale à $180^\circ$.

Calculons le troisième angle :
\[\widehat{ACB} = 180^\circ - \widehat{BAC} - \widehat{CBA} = 180^\circ - 27^\circ - 63^\circ = 90^\circ\]

Donc oui, le triangle $\triangle ABC$ est rectangle en $C$ car $\widehat{ACB} = 90^\circ$.

\task \textbf{Calcul des longueurs $\overline{AC}$ et $\overline{BC}$}

Puisque le triangle est rectangle en $C$, nous pouvons utiliser les rapports trigonométriques dans le triangle rectangle.

\textbf{Calcul de $\overline{AC}$ :}

Dans le triangle rectangle $\triangle ABC$, rectangle en $C$ :
\[\cos(\widehat{BAC}) = \dfrac{\overline{AC}}{\overline{AB}}\]

Donc :
\[\overline{AC} = \overline{AB} \times \cos(\widehat{BAC}) = 4{,}5 \times \cos(27^\circ) \approx 4{,}5 \times 0{,}891 \approx 4{,}010~\text{cm}\]

\textbf{Calcul de $\overline{BC}$ :}

De même :
\[\sin(\widehat{BAC}) = \dfrac{\overline{BC}}{\overline{AB}}\]

Donc :
\[\overline{BC} = \overline{AB} \times \sin(\widehat{BAC}) = 4{,}5 \times \sin(27^\circ) \approx 4{,}5 \times 0{,}454 \approx 2{,}043~\text{cm}\]

\textbf{Réponses :}
\begin{itemize}
\item $\overline{AC} \approx 4{,}010~\text{cm}$
\item $\overline{BC} \approx 2{,}043~\text{cm}$
\end{itemize}
\end{tasks}

}

