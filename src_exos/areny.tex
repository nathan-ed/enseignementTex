\titre{}
\theme{derivation2}
\auteur{Nathan Scheinmann}
\niveau{3M}
\source{sesamath3e}
\type{serie}
\piments{2}
\pts{}
\annee{2425}

\contenu{
\tcblower
Déterminer les dérivées suivantes et donner la réponse sans exposant négatif ou fractionnaire :
\begin{tasks}(2)
\task $\sin(x^4 - \dfrac{1}{x})$
\task $\cos(5\sqrt{x})$
\task $\tan(\cos(x))$
\task $\sin^3(x)$
\task $\sqrt{\cos(x)}$
\task $\sin^{-1}(2x)$
\end{tasks}
}
\correction{
\tcblower
{\scriptsize \textit{Correction générée par IA}}

\begin{tasks}
\task Pour $f(x)=\sin(x^4 - \dfrac{1}{x})$, on utilise la formule de dérivation des fonctions composées.

Posons $u(x)=x^4-\dfrac{1}{x}=x^4-x^{-1}$, alors $u'(x)=4x^3+x^{-2}=4x^3+\dfrac{1}{x^2}$.

Donc :
\[
f'(x)=\cos(u(x)) \cdot u'(x)=\cos\left(x^4-\dfrac{1}{x}\right)\cdot\left(4x^3+\dfrac{1}{x^2}\right)
\]

\task Pour $f(x)=\cos(5\sqrt{x})=\cos(5x^{\frac{1}{2}})$.

Posons $u(x)=5x^{\frac{1}{2}}$, alors $u'(x)=5\cdot\dfrac{1}{2}x^{-\frac{1}{2}}=\dfrac{5}{2\sqrt{x}}$.

Donc :
\[
f'(x)=-\sin(u(x)) \cdot u'(x)=-\sin(5\sqrt{x})\cdot\dfrac{5}{2\sqrt{x}}=-\dfrac{5\sin(5\sqrt{x})}{2\sqrt{x}}
\]

\task Pour $f(x)=\tan(\cos(x))$.

Posons $u(x)=\cos(x)$, alors $u'(x)=-\sin(x)$.

La dérivée de $\tan(u)$ est $\dfrac{1}{\cos^2(u)}$, donc :
\[
f'(x)=\dfrac{1}{\cos^2(u(x))} \cdot u'(x)=\dfrac{-\sin(x)}{\cos^2(\cos(x))}
\]

\task Pour $f(x)=\sin^3(x)=(\sin(x))^3$.

Posons $u(x)=\sin(x)$, alors $u'(x)=\cos(x)$.

Donc :
\[
f'(x)=3(\sin(x))^2 \cdot \cos(x)=3\sin^2(x)\cos(x)
\]

\task Pour $f(x)=\sqrt{\cos(x)}=(\cos(x))^{\frac{1}{2}}$.

Posons $u(x)=\cos(x)$, alors $u'(x)=-\sin(x)$.

Donc :
\[
f'(x)=\dfrac{1}{2}(\cos(x))^{-\frac{1}{2}} \cdot (-\sin(x))=-\dfrac{\sin(x)}{2\sqrt{\cos(x)}}
\]

\task Pour $f(x)=\sin^{-1}(2x)=\arcsin(2x)$.

Posons $u(x)=2x$, alors $u'(x)=2$.

La dérivée de $\arcsin(u)$ est $\dfrac{1}{\sqrt{1-u^2}}$, donc :
\[
f'(x)=\dfrac{1}{\sqrt{1-(2x)^2}} \cdot 2=\dfrac{2}{\sqrt{1-4x^2}}
\]
\end{tasks}
}

