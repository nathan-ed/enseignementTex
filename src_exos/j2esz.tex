\titre{}
\theme{derivation}
\auteur{Nathan Scheinmann}
\niveau{3M}
\source{sesamath3e}
\type{serie}
\piments{2}
\pts{}
\annee{2425}

\contenu{
\tcblower
   Calculer le taux de variation de la fonction (la pente donnée de la droite entre deux points du graphe de la fonction) \(f\) donnée entre les deux points donnés et interpréter graphiquement :
    \begin{tasks}
      \task \(f(x)=x^2,\ A=(1;f(1)),\ P=(2;f(2))\).
      \task \(f(x)=x^2,\ A=(1;f(1)),\ P=(1{,}5;f(1{,}5))\).
      \task \(f(x)=x^2,\ A=(-2;f(-2)),\ P=(2;f(2))\).
      \task \(f(x)=x^3,\ A=(1;f(1)),\ P=(2;f(2))\).
    \end{tasks}
}
\correction{
	\tcblower
	On calcule le rapport $\dfrac{A_y-P_y}{A_x-P_x}$.
	\begin{tasks}
\task $f(x) = x^2$, $A = (1; 1)$, $P = (2; 4)$
$\frac{f(2) - f(1)}{2 - 1} = \frac{4 - 1}{1} = 3$
\task $f(x) = x^2$, $A = (1; 1)$, $P = (1{,}5; 2{,}25)$
$\frac{f(1{,}5) - f(1)}{1{,}5 - 1} = \frac{2{,}25 - 1}{0{,}5} = 2{,}5$

\task $f(x) = x^2$, $A = (-2; 4)$, $P = (2; 4)$
$\frac{f(2) - f(-2)}{2 - (-2)} = \frac{4 - 4}{4} = 0$

\task $f(x) = x^3$, $A = (1; 1)$, $P = (2; 8)$
$\frac{f(2) - f(1)}{2 - 1} = \frac{8 - 1}{1} = 7$
\end{tasks}
}

