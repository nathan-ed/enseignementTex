\titre{}
\theme{derivation}
\auteur{Nathan Scheinmann}
\niveau{3M}
\source{sesamath3e}
\type{serie}
\piments{2}
\pts{}
\annee{2425}

\contenu{
\tcblower
   Calculer le taux de variation de la fonction \(f\) donnée entre les deux points donnés et interpréter graphiquement :
    \begin{tasks}
      \task \(f(x)=x^2,\ A=(1;f(1)),\ P=(2;f(2))\).
      \task \(f(x)=x^2,\ A=(1;f(1)),\ P=(1{.}5;f(1{.}5))\).
      \task \(f(x)=x^2,\ A=(-2;f(-2)),\ P=(2;f(2))\).
      \task \(f(x)=x^3,\ A=(1;f(1)),\ P=(2;f(2))\).
    \end{tasks}
}
\correction{

}

