\titre{}
\theme{trigo}
\auteur{Nathan Scheinmann}
\niveau{1M}
\source{sophie1M}
\type{serie}
\piments{3}
\pts{}
\annee{2425}

\contenu{
\tcblower
Du sommet d'un phare, haut de $175$ m , le gardien observe les positions successives d'un bateau
qui avance en direction du phare. L'angle de dépression de la première position vaut $12^\circ$, celui de
la deuxième position vaut $20^\circ$.
Calculer la distance dont a avancé le bateau.

NB. L'angle de dépression est l'angle entre l'horizontale et la direction de visée du bateau.
}
\correction{
$342,5$ m
}

