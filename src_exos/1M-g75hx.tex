\titre{}
\theme{derivation}
\auteur{Nathan Scheinmann}
\niveau{3M}
\source{sesamath3e}
\type{serie}
\piments{2}
\pts{}
\annee{2425}

\contenu{
\tcblower
Calculer, à partir de la définition de la fonction dérivée, la dérivée des fonctions réelles ci-dessous ; si possible, représenter \(f\) et \(f'\) dans un même repère et interpréter graphiquement le résultat :
\begin{tasks}(2)
  \task \(f(x)=7\).
  \task \(f(x)=x\).
  \task \(f(x)=3x\).
  \task \(f(x)=ax+b\), avec \(a,b\in\mathbb{R}\).
  \task \(f(x)=x^2\).
  \task \(f(x)=ax^2+bx+c\), avec \(a,b,c\in\mathbb{R}\).
  \task \(f(x)=x^3\).
  \task \(f(x)=\sqrt{x}\).
  \task \(f(x)=\dfrac{1}{x}\).
  \task \(f(x)=\dfrac{1}{x^2+1}\).
\end{tasks}
}
\correction{

}

