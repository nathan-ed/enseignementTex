\titre{}
\theme{dérivées}
\auteur{Nathan Scheinmann}
\niveau{3M}
\source{analysis}
\type{serie}
\piments{2}
\pts{}
\annee{2425}

\contenu{
\tcblower
Trouver les points critiques et les extrema locaux.

\begin{tasks}(3)
\task* $x^3 + 3x - 2$
\task $2x^4 - 4x^2 + 6$
\task* $x + \dfrac{1}{x}$
\task $x^{-1}(1-x)$
\task* $x(x+1)(x+2)$
\task $(1-x)^2(1+x)$
\task* $\dfrac{1}{x-2}$
\task $\dfrac{1+x}{1-x}$
\task* $\dfrac{2-3x}{2+x}$
\task $\dfrac{2}{x(x+1)}$
\task* $|x^2 - 16|$
\task $x^3(1-x)^2$
\task* $\left(\dfrac{x-2}{x+2}\right)^3$
\task $(1-2x)(x-1)^3$
\task* $(1-x)(1+x)^3$
\task $\dfrac{x^2}{1+x}$
\task* $\dfrac{|x|}{1+|x|}$
\task $(3x-5)^3$
\task* $|x-1||x+2|$
\task $x\sqrt[3]{1-x}$
\task* $-\dfrac{x^3}{x+1}$
\task $\dfrac{1}{x+1} - \dfrac{1}{x-2}$
\task* $\dfrac{1}{x+1} - \dfrac{1}{x+2}$
\task $|x-3| + |2x+1|$
\end{tasks}
}
\correction{
\tcblower
%% GENERATED BY AI %%
Pour chaque fonction, nous cherchons les points critiques où $f'(x) = 0$ ou $f'(x)$ n'existe pas, puis nous déterminons les extrema locaux.

\begin{tasks}(3)
\task* $f(x) = x^3 + 3x - 2$

$f'(x) = 3x^2 + 3 = 3(x^2 + 1) > 0$ pour tout $x \in \mathbb{R}$.

Pas de point critique, fonction strictement croissante, pas d'extremum local.

\task $f(x) = 2x^4 - 4x^2 + 6$

$f'(x) = 8x^3 - 8x = 8x(x^2 - 1) = 0 \Rightarrow x = 0, \pm 1$

$f''(x) = 24x^2 - 8$ donc $f''(0) = -8 < 0$ (maximum local), $f''(\pm 1) = 16 > 0$ (minima locaux).

\task* $f(x) = x + \dfrac{1}{x}$, $D_f = \mathbb{R}^*$

$f'(x) = 1 - \dfrac{1}{x^2} = \dfrac{x^2 - 1}{x^2} = 0 \Rightarrow x = \pm 1$

$f''(x) = \dfrac{2}{x^3}$ donc $f''(1) = 2 > 0$ (minimum local), $f''(-1) = -2 < 0$ (maximum local).

\task $f(x) = x^{-1}(1-x) = \dfrac{1-x}{x}$, $D_f = \mathbb{R}^*$

$f'(x) = \dfrac{-x - (1-x)}{x^2} = \dfrac{-1}{x^2} < 0$ pour tout $x \neq 0$.

Pas de point critique, fonction strictement décroissante sur chaque composante de $D_f$.

\task* $f(x) = x(x+1)(x+2) = x^3 + 3x^2 + 2x$

$f'(x) = 3x^2 + 6x + 2 = 0 \Rightarrow x = \dfrac{-6 \pm \sqrt{36-24}}{6} = \dfrac{-6 \pm 2\sqrt{3}}{6} = \dfrac{-3 \pm \sqrt{3}}{3}$

$f''(x) = 6x + 6$. Test des dérivées secondes pour déterminer la nature des extrema.

\task $f(x) = (1-x)^2(1+x) = (1-2x+x^2)(1+x) = 1 - x - x^2 + x^3$

$f'(x) = -1 - 2x + 3x^2 = 0 \Rightarrow 3x^2 - 2x - 1 = 0 \Rightarrow (3x+1)(x-1) = 0$

Points critiques : $x = -\dfrac{1}{3}$ et $x = 1$.

\task* $f(x) = \dfrac{1}{x-2}$, $D_f = \mathbb{R} \setminus \{2\}$

$f'(x) = -\dfrac{1}{(x-2)^2} < 0$ pour tout $x \neq 2$.

Pas de point critique, fonction strictement décroissante sur chaque composante.

\task $f(x) = \dfrac{1+x}{1-x}$, $D_f = \mathbb{R} \setminus \{1\}$

$f'(x) = \dfrac{(1-x) - (1+x)(-1)}{(1-x)^2} = \dfrac{1-x+1+x}{(1-x)^2} = \dfrac{2}{(1-x)^2} > 0$ pour tout $x \neq 1$.

Pas de point critique, fonction strictement croissante sur chaque composante.

\task* $f(x) = \dfrac{2-3x}{2+x}$, $D_f = \mathbb{R} \setminus \{-2\}$

$f'(x) = \dfrac{-3(2+x) - (2-3x)}{(2+x)^2} = \dfrac{-6-3x-2+3x}{(2+x)^2} = \dfrac{-8}{(2+x)^2} < 0$

Pas de point critique, fonction strictement décroissante.

\task $f(x) = \dfrac{2}{x(x+1)}$, $D_f = \mathbb{R} \setminus \{0, -1\}$

$f'(x) = -\dfrac{2(2x+1)}{x^2(x+1)^2} = 0 \Rightarrow x = -\dfrac{1}{2}$

Point critique : $x = -\dfrac{1}{2}$ (extremum local).

\task* $f(x) = |x^2 - 16|$

Non dérivable en $x = \pm 4$ (où $x^2 - 16 = 0$).

Pour $|x| < 4$ : $f(x) = 16 - x^2$, $f'(x) = -2x = 0 \Rightarrow x = 0$ (maximum local).

Pour $|x| > 4$ : $f(x) = x^2 - 16$, $f'(x) = 2x$ (pas de zéro dans ces régions).

Points critiques : $x = 0, \pm 4$. Minimum local en $x = \pm 4$, maximum local en $x = 0$.

\task $f(x) = x^3(1-x)^2 = x^3(1-2x+x^2) = x^3 - 2x^4 + x^5$

$f'(x) = 3x^2 - 8x^3 + 5x^4 = x^2(3 - 8x + 5x^2) = 0$

$x = 0$ ou $5x^2 - 8x + 3 = 0 \Rightarrow x = \dfrac{8 \pm 2}{10}$ donc $x = 1$ ou $x = \dfrac{3}{5}$.

\task* $f(x) = \left(\dfrac{x-2}{x+2}\right)^3$, $D_f = \mathbb{R} \setminus \{-2\}$

$f'(x) = 3\left(\dfrac{x-2}{x+2}\right)^2 \cdot \dfrac{(x+2) - (x-2)}{(x+2)^2} = 3\left(\dfrac{x-2}{x+2}\right)^2 \cdot \dfrac{4}{(x+2)^2} > 0$ pour tout $x \neq \pm 2$.

Fonction strictement croissante, pas d'extremum local.

\task $f(x) = (1-2x)(x-1)^3$

$f'(x) = -2(x-1)^3 + 3(1-2x)(x-1)^2 = (x-1)^2[-2(x-1) + 3(1-2x)] = (x-1)^2(-2x+2+3-6x) = (x-1)^2(5-8x) = 0$

Points critiques : $x = 1$ et $x = \dfrac{5}{8}$.

\task* $f(x) = (1-x)(1+x)^3$

$f'(x) = -(1+x)^3 + 3(1-x)(1+x)^2 = (1+x)^2[-(1+x) + 3(1-x)] = (1+x)^2(2-4x) = 0$

Points critiques : $x = -1$ et $x = \dfrac{1}{2}$.

\task $f(x) = \dfrac{x^2}{1+x}$, $D_f = \mathbb{R} \setminus \{-1\}$

$f'(x) = \dfrac{2x(1+x) - x^2}{(1+x)^2} = \dfrac{x(2+x)}{(1+x)^2} = 0 \Rightarrow x = 0$ ou $x = -2$.

Points critiques : $x = 0$ et $x = -2$.

\task* $f(x) = \dfrac{|x|}{1+|x|}$

Non dérivable en $x = 0$.

Pour $x > 0$ : $f(x) = \dfrac{x}{1+x}$, $f'(x) = \dfrac{1}{(1+x)^2} > 0$.

Pour $x < 0$ : $f(x) = \dfrac{-x}{1-x}$, $f'(x) = \dfrac{-1}{(1-x)^2} < 0$.

Point critique : $x = 0$ (minimum absolu).

\task $f(x) = (3x-5)^3$

$f'(x) = 9(3x-5)^2 \geq 0$ pour tout $x$, égal à zéro uniquement en $x = \dfrac{5}{3}$.

Pas d'extremum local (point d'inflexion horizontal).

\task* $f(x) = |x-1||x+2|$

Non dérivable en $x = 1$ et $x = -2$.

Pour $x < -2$ : $f(x) = (1-x)(-x-2) = x^2 + x - 2$, $f'(x) = 2x + 1 = 0 \Rightarrow x = -\dfrac{1}{2} \notin ]-\infty; -2[$.

Pour $-2 < x < 1$ : $f(x) = (1-x)(x+2) = -x^2 - x + 2$, $f'(x) = -2x - 1 = 0 \Rightarrow x = -\dfrac{1}{2}$ (maximum local).

Pour $x > 1$ : $f(x) = (x-1)(x+2) = x^2 + x - 2$, $f'(x) = 2x + 1 > 0$.

Points critiques : $x = -2, -\dfrac{1}{2}, 1$. Maximum local en $x = -\dfrac{1}{2}$, minima locaux en $x = -2$ et $x = 1$.

\task $f(x) = x\sqrt[3]{1-x}$

$f'(x) = \sqrt[3]{1-x} + x \cdot \dfrac{-1}{3(1-x)^{2/3}} = \sqrt[3]{1-x} - \dfrac{x}{3(1-x)^{2/3}} = \dfrac{3(1-x) - x}{3(1-x)^{2/3}} = \dfrac{3-4x}{3(1-x)^{2/3}} = 0$

Point critique : $x = \dfrac{3}{4}$. La dérivée n'existe pas en $x = 1$ (également point critique).

\task* $f(x) = -\dfrac{x^3}{x+1}$, $D_f = \mathbb{R} \setminus \{-1\}$

$f'(x) = -\dfrac{3x^2(x+1) - x^3}{(x+1)^2} = -\dfrac{3x^3 + 3x^2 - x^3}{(x+1)^2} = -\dfrac{2x^3 + 3x^2}{(x+1)^2} = -\dfrac{x^2(2x+3)}{(x+1)^2} = 0$

Points critiques : $x = 0$ et $x = -\dfrac{3}{2}$.

\task $f(x) = \dfrac{1}{x+1} - \dfrac{1}{x-2}$, $D_f = \mathbb{R} \setminus \{-1, 2\}$

$f'(x) = -\dfrac{1}{(x+1)^2} + \dfrac{1}{(x-2)^2} = 0 \Rightarrow (x+1)^2 = (x-2)^2 \Rightarrow x = \dfrac{1}{2}$

Point critique : $x = \dfrac{1}{2}$.

\task* $f(x) = \dfrac{1}{x+1} - \dfrac{1}{x+2}$, $D_f = \mathbb{R} \setminus \{-1, -2\}$

$f'(x) = -\dfrac{1}{(x+1)^2} + \dfrac{1}{(x+2)^2} = 0 \Rightarrow (x+2)^2 = (x+1)^2$

Impossible car les distances sont différentes. Pas de point critique.

$f'(x) = \dfrac{(x+1)^2 - (x+2)^2}{(x+1)^2(x+2)^2} = \dfrac{-2x-3}{(x+1)^2(x+2)^2} = 0 \Rightarrow x = -\dfrac{3}{2}$

Point critique : $x = -\dfrac{3}{2}$.

\task $f(x) = |x-3| + |2x+1|$

Non dérivable en $x = 3$ et $x = -\dfrac{1}{2}$.

Pour $x < -\dfrac{1}{2}$ : $f(x) = (3-x) + (-2x-1) = 2 - 3x$, $f'(x) = -3 < 0$.

Pour $-\dfrac{1}{2} < x < 3$ : $f(x) = (3-x) + (2x+1) = 4 + x$, $f'(x) = 1 > 0$.

Pour $x > 3$ : $f(x) = (x-3) + (2x+1) = 3x - 2$, $f'(x) = 3 > 0$.

Minimum local en $x = -\dfrac{1}{2}$.
\end{tasks}
}

