\titre{}
\theme{numerique}
\auteur{Nathan Scheinmann}
\niveau{1M}
\source{musy}
\type{serie}
\piments{1}
\pts{}
\annee{2425}

\contenu{
\tcblower
	Donner l'ensemble des diviseurs pour chacun des entiers allant de 1 à 10, sous la forme habituelle~: 

\[\operatorname{Div}_1=\{1\} ;\quad \operatorname{Div}_2=\{1 ; 2\} ;\quad \operatorname{Div}_3=\{1 ; 3\} ;\quad \ldots ;\quad \operatorname{Div}_{10}=\ldots.\]
\begin{enumerate}
\item Relever la liste des entiers de 1 à 10 qui ont un nombre impair de diviseurs~:
\begin{enumerate}
\item Pouvez-vous trouver un point commun à ces entiers, ou leur nom~?
\item Donner la liste des quinze premiers nombres entiers qui ont cette caractéristique.
\end{enumerate}
\item Relever la liste des entiers de 1 à 10 qui ont exactement deux diviseurs~:
\begin{enumerate}
\item Pouvez-vous trouver un point commun à ces entiers, ou leur nom~?
\item Donner la liste des nombres entiers inférieurs à 50 qui ont cette caractéristique.
\end{enumerate}
\end{enumerate}
}
\correction{
\tcblower
\begin{tasks}
	\task $1; 4; 9$, on les appelle des carrés parfaits. 
	\task Ce sont des nombres premiers. $\{2;3;5;7;11;13;17;\ldots\}$. 
\end{tasks}
}

