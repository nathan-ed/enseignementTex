\titre{}
\theme{limites}
\auteur{Nathan Scheinmann}
\niveau{3M}
\source{musy}
\type{serie}
\piments{2}
\pts{}
\annee{2425}

\contenu{
\tcblower
Soit la fonction $f$ définie sur $\mathbb{R}$ par 
\[f(x)=\begin{cases}\dfrac{x^3-8}{x-2}& \text{ si } x\neq 2,\\
c &\text{ si } x=2\end{cases}\]
pour $c\in \mathbb{R}$
\begin{tasks}
	\task Expliquez pourquoi cette fonction est continue sur $\mathbb{R}\setminus \{2\}$.
	\task Pour quelles valeurs de $c$ la fonction est-elle continue sur $\mathbb{R}$~?
\end{tasks}
}
\correction{
	\tcblower
\begin{tasks}
	\task Si $x\neq 2$, alors on peut simplifier l'expression de la fonction (division polynomiale) $f(x)=x^2+2x+4$ qui est une expression polynomiale, donc continue.
	\task $f$ est continue sur $\R\setminus \{2\}$. Il faut à présent faire en sorte qu'elle soit continue en $x=2$. Pour cela, on doit avoir $\lim_{x\to 2}f(x)=c$. On calcule la limite $\lim_{x\to 2} f(x)=12$, donc pour $c=12$.
\end{tasks}
}

