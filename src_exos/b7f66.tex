\titre{}
\theme{geometrie}
\auteur{Nathan Scheinmann}
\niveau{1M}
\source{}
\type{serie}
\piments{2}
\pts{}
\annee{2425}

\contenu{
	\tcblower
	\begin{minipage}[t]{0.4\textwidth}{
		\vspace{0pt}
		On a $BH=18 cm$ et $BC= 30 cm$.  
Calculer la longueur de $CH, AH$ et $AC$.	
		}
		\end{minipage}
\begin{minipage}[t]{0.5\textwidth}{
		\vspace{0pt}
\begin{tikzpicture}[scale=.75]
\tkzDefPoint(5,3.5){C}
\tkzDefPoint(0,0){B}
\tkzDefPoint(7,0){A}
\tkzDefPointBy[projection=onto A--B](C) \tkzGetPoint{H}
\tkzMarkRightAngle(A,C,B)
\tkzMarkRightAngle(C,H,A)
\tkzDrawPoints(A,B,C,H)
\tkzDrawSegment(C,H)
\tkzDrawPolygon(A,B,C)
\tkzLabelPoints[below](B,A,H)
\tkzLabelPoints[above](C)
\end{tikzpicture}}
\end{minipage}}
\correction{
	\tcblower
{\scriptsize \textit{Correction générée par IA}}

Dans le triangle rectangle $ABC$, la hauteur $CH$ partage le triangle en deux triangles rectangles plus petits. Nous pouvons utiliser les relations métriques dans le triangle rectangle.

\textbf{Calcul de $CH$ :}

Dans le triangle rectangle $BCH$, nous avons :
\[
BC^2 = BH^2 + CH^2
\]

D'où :
\[
30^2 = 18^2 + CH^2
\]

\[
900 = 324 + CH^2
\]

\[
CH^2 = 576
\]

\[
CH = 24 \text{ cm}
\]

\textbf{Calcul de $AH$ :}

Dans le triangle rectangle $ACH$, nous utilisons la relation de la hauteur dans un triangle rectangle.

On sait que dans un triangle rectangle, si $H$ est le pied de la hauteur issue de l'angle droit, alors :
\[
CH^2 = AH \cdot BH
\]

En fait, cette formule n'est pas directe. Utilisons plutôt le triangle rectangle $ACH$ :

Par le théorème de Pythagore dans le triangle $ACH$ :
\[
AC^2 = AH^2 + CH^2
\]

Mais nous devons d'abord trouver $AC$. Utilisons la relation :
\[
BC \cdot AC = AB \cdot CH
\]

Ou plus directement, remarquons que $AB = AH + BH = AH + 18$.

Dans le triangle rectangle $ABC$, nous avons aussi :
\[
CH^2 = AH \cdot BH
\]

Donc :
\[
24^2 = AH \cdot 18
\]

\[
576 = 18 \cdot AH
\]

\[
AH = 32 \text{ cm}
\]

\textbf{Calcul de $AC$ :}

Dans le triangle rectangle $ACH$ :
\[
AC^2 = AH^2 + CH^2 = 32^2 + 24^2 = 1024 + 576 = 1600
\]

\[
AC = 40 \text{ cm}
\]

\textbf{Réponses :} $CH = 24$ cm, $AH = 32$ cm, $AC = 40$ cm.
}

