\titre{}
\theme{geometrie}
\auteur{Nathan Scheinmann}
\niveau{1M}
\source{}
\type{serie}
\piments{2}
\pts{}
\annee{2425}

\contenu{
	\tcblower
	\begin{minipage}[t]{0.4\textwidth}{
		\vspace{0pt}
		On a $BH=18 cm$ et $BC= 30 cm$.  
Calculer la longueur de $CH, AH$ et $AC$.	
		}
		\end{minipage}
\begin{minipage}[t]{0.5\textwidth}{
		\vspace{0pt}
\begin{tikzpicture}[scale=.75]
\tkzDefPoint(5,3.5){C}
\tkzDefPoint(0,0){B}
\tkzDefPoint(7,0){A}
\tkzDefPointBy[projection=onto A--B](C) \tkzGetPoint{H}
\tkzMarkRightAngle(A,C,B)
\tkzMarkRightAngle(C,H,A)
\tkzDrawPoints(A,B,C,H)
\tkzDrawSegment(C,H)
\tkzDrawPolygon(A,B,C)
\tkzLabelPoints[below](B,A,H)
\tkzLabelPoints[above](C)
\end{tikzpicture}}
\end{minipage}}
\correction{
\tcblower
{\scriptsize \textit{Correction générée par IA}}

Par Pythagore dans le triangle $BCH$~:
\[
CH^2 = BC^2 - BH^2 = 900 - 324 = 576 \implies CH = 24 \text{ cm}
\]

Relation de la hauteur~: $CH^2 = AH \cdot BH$
\[
AH = \dfrac{576}{18} = 32 \text{ cm}
\]

Par Pythagore dans le triangle $ACH$~:
\[
AC^2 = AH^2 + CH^2 = 1024 + 576 = 1600 \implies AC = 40 \text{ cm}
\]
}

