\titre{}
\theme{derivation}
\auteur{Nathan Scheinmann}
\niveau{3M}
\source{musy}
\type{serie}
\piments{1}
\pts{}
\annee{2526}

\contenu{
\tcblower
Dériver les fonctions en $x$ suivantes, à l'aide des propriétés de la dérivée ($a, b, c, d$ et $\pi$ sont des nombres réels) :
\begin{tasks}(3)
\task $f(x)=2 x-3$
\task $f(x)=\pi x^2$
\task $f(x)=4 x^2-5 x+6$
\task $f(x)=a x^2+b x+c$
\task $f(x)=(2 x-3)^2$
\task $f(x)=2 x-\frac{1}{x^2-1}$
\task $f(x)=\frac{x+5}{x-1}$
\task $f(x)=\frac{a}{x^2}$
\task $f(x)=\frac{1}{x^4}$
\task*(2) $f(x)=(2 x-1)(3-4 x)$
\task $f(x)=\frac{10}{x^3-4 x^2-2}$
\task $f(x)=\frac{a x+b}{c x+d}$
\end{tasks}
}
\correction{
\tcblower
}

