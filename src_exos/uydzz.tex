\titre{17}
\theme{trigo}
\auteur{Nathan Scheinmann}
\niveau{1M}
\source{sesamath-1M-trigo}
\type{serie}
\piments{2}
\pts{}
\annee{2425}

\contenu{
\tcblower
Pour déterminer la hauteur du Mont Ticule, Sophie a mesuré les angles $\alpha, \beta$ et la distance $d$. Calculer $h$ en sachant que $\alpha=30^\circ, \beta=45^\circ$ et $d=200~\text{m}$. 
\begin{center}
	\includegraphics[scale=1]{../medias/1M/trigo/1M-exo-17}
\end{center}
}
\correction{
\tcblower
\textit{Generated by AI}

Notons $x$ la distance horizontale entre le point de mesure de l'angle $\beta$ et la base du Mont Ticule.

À partir du point de mesure de $\beta$ (45°), nous avons :
$$\tan(\beta) = \dfrac{h}{x}$$
$$\tan(45°) = \dfrac{h}{x}$$
$$1 = \dfrac{h}{x}$$
$$x = h$$

À partir du point de mesure de $\alpha$ (30°), qui est à une distance $d = 200$ m plus loin, nous avons :
$$\tan(\alpha) = \dfrac{h}{x + d}$$
$$\tan(30°) = \dfrac{h}{h + 200}$$
$$\dfrac{1}{\sqrt{3}} = \dfrac{h}{h + 200}$$

En multipliant en croix :
$$h + 200 = h\sqrt{3}$$
$$200 = h\sqrt{3} - h$$
$$200 = h(\sqrt{3} - 1)$$
$$h = \dfrac{200}{\sqrt{3} - 1}$$

Rationalisons le dénominateur :
$$h = \dfrac{200}{\sqrt{3} - 1} \times \dfrac{\sqrt{3} + 1}{\sqrt{3} + 1} = \dfrac{200(\sqrt{3} + 1)}{3 - 1} = \dfrac{200(\sqrt{3} + 1)}{2} = 100(\sqrt{3} + 1)$$

Calculons la valeur numérique :
$$h = 100(1{,}732 + 1) = 100 \times 2{,}732 = 273{,}2 \text{ m}$$

La hauteur du Mont Ticule est donc $\boxed{h = 100(\sqrt{3} + 1) \approx 273{,}2 \text{ m}}$.
}

