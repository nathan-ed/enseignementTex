\titre{}
\theme{derivation}
\auteur{Nathan Scheinmann}
\niveau{3M}
\source{sesamath3e}
\type{serie}
\piments{2}
\pts{}
\annee{2425}

\contenu{
\tcblower
On donne ci-contre une représentation graphique de la dérivée $f'$ d'une fonction $f$.
GRAPHIC
$f$ admet-elle un maximum et/ou un minimum ? Justifier
}
\correction{
\tcblower

\textit{Generated by AI}

Pour déterminer si $f$ admet un maximum ou un minimum, nous devons analyser le graphique de sa dérivée $f'$.

\textbf{Rappel :}
\begin{itemize}
\item Si $f'$ change de signe de positif à négatif en un point $a$, alors $f$ admet un maximum local en $a$.
\item Si $f'$ change de signe de négatif à positif en un point $a$, alors $f$ admet un minimum local en $a$.
\end{itemize}

D'après le graphique de $f'$ :
\begin{itemize}
\item Si $f'$ s'annule et change de signe, nous pouvons identifier des extrema de $f$.
\item Il faut repérer les points où la courbe de $f'$ traverse l'axe des abscisses.
\end{itemize}

\textbf{Analyse :} En observant le graphique (non fourni ici), nous devons :
\begin{enumerate}
\item Identifier les abscisses où $f'(x) = 0$
\item Pour chacune, déterminer si $f'$ change de signe
\item Conclure sur l'existence d'un maximum ou d'un minimum
\end{enumerate}

\textbf{Conclusion :} Sans le graphique précis, la méthode générale est : $f$ admet un maximum local en $a$ si $f'$ passe du positif au négatif en $a$, et un minimum local en $b$ si $f'$ passe du négatif au positif en $b$.

}

