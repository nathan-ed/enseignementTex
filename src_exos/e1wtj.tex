\titre{}
\theme{derivation}
\auteur{Nathan Scheinmann}
\niveau{3M}
\source{sesamath3e}
\type{serie}
\piments{2}
\pts{}
\annee{2425}

\contenu{
\tcblower
Parmi tous les rectangles inscrits dans un cercle de rayon $r$, trouver celui dont l'aire est maximale.
}
\correction{
\tcblower
\textit{Generated by AI}

Un rectangle inscrit dans un cercle de rayon $r$ a sa diagonale égale au diamètre du cercle, soit $2r$.

Notons $x$ et $y$ les côtés du rectangle. Par le théorème de Pythagore :
\[x^2 + y^2 = (2r)^2 = 4r^2\]

D'où : $y = \sqrt{4r^2 - x^2}$

L'aire du rectangle est :
\[A(x) = x \cdot y = x\sqrt{4r^2 - x^2}\]

Pour maximiser l'aire, dérivons par rapport à $x$ :
\[A'(x) = \sqrt{4r^2 - x^2} + x \cdot \frac{-2x}{2\sqrt{4r^2 - x^2}} = \sqrt{4r^2 - x^2} - \frac{x^2}{\sqrt{4r^2 - x^2}}\]

\[A'(x) = \frac{4r^2 - x^2 - x^2}{\sqrt{4r^2 - x^2}} = \frac{4r^2 - 2x^2}{\sqrt{4r^2 - x^2}}\]

$A'(x) = 0$ lorsque :
\[4r^2 - 2x^2 = 0 \implies x^2 = 2r^2 \implies x = r\sqrt{2}\]

Alors :
\[y = \sqrt{4r^2 - 2r^2} = \sqrt{2r^2} = r\sqrt{2}\]

On obtient $x = y = r\sqrt{2}$, donc le rectangle est un carré.

\textbf{Réponse :} Le rectangle d'aire maximale inscrit dans un cercle de rayon $r$ est un \textbf{carré de côté $r\sqrt{2}$}, d'aire $A_{\text{max}} = 2r^2$.
}

