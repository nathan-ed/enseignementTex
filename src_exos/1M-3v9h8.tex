\titre{musy-169}
\theme{fonctions}
\auteur{Nathan Scheinmann}
\niveau{1M}
\source{}
\type{serie}
\piments{2}
\pts{}
\annee{2425}

\contenu{
	\tcblower
Voici des équations de droites :
\begin{tasks}(3)
\task $3 x+2 y-9=0$;
\task $3 x=4 y$;
\task $y=-x+3$;
\task $x=\dfrac{1}{2} y+3$;
\task $\dfrac{x}{2}+\dfrac{y}{5}=1$
\end{tasks}
\begin{tasks}
	\task[1)] Déterminer la pente et l'ordonnée à l'origine de chaque droite.
	\task[2)] Donner la représentation graphique de ces droites (un seul repère).
	\task[3)] Calculer les coordonnées des différents points d'intersections de ces droites.
\end{tasks}
}
\correction{
	\tcblower
Les points d'intersection sont
 $a \cap b=\left(2 ; \dfrac{3}{2}\right) ; a \cap c=(3 ; 0) ; a \cap d=(3 ; 0) ; a \cap e=\left(\dfrac{1}{2} ; \dfrac{15}{4}\right) ; b \cap c=\left(\dfrac{12}{7} ; \dfrac{9}{7}\right) ;$ $b \cap d=\left(\dfrac{24}{5} ; \dfrac{18}{5}\right);$ $ b \cap e=\left(\dfrac{20}{13} ; \dfrac{15}{13}\right) ; c \cap d=(3 ; 0) ; c \cap e=\left(\dfrac{4}{3} ;\dfrac{5}{3}\right);$ $d \cap e=\left(\dfrac{22}{9} ;-\dfrac{10}{9}\right)$
}

