\titre{}
\theme{fonctions}
\auteur{Nathan Scheinmann}
\niveau{1M}
\source{}
\type{serie}
\piments{2}
\pts{}
\annee{2425}

\contenu{
	\tcblower
Pour quelle valeur de $x$ l'expression $x^2 +(x+1)^2 +(x+2)^2$ est-elle minimale, et que vaut alors cette expression ? Interpréter graphiquement.
}
\correction{
	\tcblower
	$f(x) = x^2 +(x+1)^2 +(x+2)^2 = 3x^2 + 6x +5 $. Son mimimum est atteint en $x=-\dfrac{b}{2a}=-\dfrac{6}{2\cdot 3}=-1$, donc vaut $f(-1)=3-6+5=2$.
}

