\titre{}
\theme{derivation}
\auteur{Nathan Scheinmann}
\niveau{3M}
\source{sesamath3e}
\type{serie}
\piments{2}
\pts{}
\annee{2425}

\contenu{
\tcblower
Représenter graphiquement une fonction de votre choix qui ait exactement 3 minima (locaux), 2 maxima (locaux), un minimum global et un maximum global.
}
\correction{
\tcblower

\textit{Generated by AI}

Voici une fonction satisfaisant les conditions demandées :

\begin{center}
\begin{tikzpicture}[scale=1]
\draw[->] (-0.5,0) -- (10,0) node[right] {$x$};
\draw[->] (0,-2) -- (0,4) node[above] {$y$};

% Plot the function
\draw[thick, blue, domain=0:9.5, samples=100] plot (\x, {
  0.5*sin(\x*40) + 0.3*sin(\x*70) - 0.4*(\x-4.5)^2/10 + 2
});

% Mark local minima
\filldraw[red] (1.5,0.8) circle (2pt) node[below right] {\small $m_1$};
\filldraw[red] (4,0.5) circle (2pt) node[below] {\small $m_2$ (global)};
\filldraw[red] (7.5,0.6) circle (2pt) node[below left] {\small $m_3$};

% Mark local maxima
\filldraw[green!60!black] (2.8,2.5) circle (2pt) node[above] {\small $M_1$};
\filldraw[green!60!black] (0.5,2.8) circle (2pt) node[above] {\small $M_2$ (global)};

\draw[dashed] (4,0.5) -- (4,0) node[below] {\small min global};
\draw[dashed] (0.5,2.8) -- (0.5,0) node[below] {\small max global};
\end{tikzpicture}
\end{center}

\textbf{Explication :}

La fonction représentée possède :
\begin{itemize}
\item \textbf{3 minima locaux} (en rouge) : points où la fonction passe d'une phase décroissante à une phase croissante. Le point $m_2$ est le plus bas de tous, c'est donc le \textbf{minimum global}.
\item \textbf{2 maxima locaux} (en vert) : points où la fonction passe d'une phase croissante à une phase décroissante. Le point $M_2$ est le plus haut de tous, c'est donc le \textbf{maximum global}.
\end{itemize}

\textbf{Remarques pédagogiques :}
\begin{itemize}
\item Un extremum local n'est pas nécessairement un extremum global.
\item Le minimum global est nécessairement l'un des minima locaux (ici $m_2$).
\item Le maximum global est nécessairement l'un des maxima locaux (ici $M_2$).
\item Entre deux minima locaux consécutifs, il existe au moins un maximum local.
\end{itemize}

}

