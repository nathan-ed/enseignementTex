\titre{}
\theme{remediation}
\auteur{Nathan Scheinmann}
\niveau{1M}
\source{co}
\type{serie}
\piments{1}
\pts{}
\annee{2425}

\contenu{
	\qrcode{https://coopmaths.fr/alea?uuid=feb39&id=11FA10-7&n=6&d=10&cd=1&cols=2&es=2211001&lang=fr-CH&title=&v=eleve}
\tcblower
% @see : 
Pour chaque équation, calculer le discriminant et déterminer le nombre de solutions de cette équation dans $\mathbb{R}$.
\begin{tasks}(2)
	\task  $-5x^2+20x-22=0$ 
	\task  $-5x^2+20x-17=0$ 
	\task  $2x^2+16x+32=0$ 
	\task $2x^2+20x+50=0$
	\task  $-5x^2+30x-47=0$ 
	\task  $x^2+4x+3=0$
\end{tasks}
}
\correction{
\tcblower
\begin{tasks}(2)
\task $\Delta = 20^2-4\cdot(-5)\cdot(-22)=-40$\\$\Delta<0$ donc l'équation n'admet pas de solution.\\$\mathcal{S}=\emptyset$.
\task $\Delta = 20^2-4\cdot(-5)\cdot(-17)=60$\\$\Delta>0$ donc l'équation admet deux solutions.
\task $\Delta = 16^2-4\cdot2\cdot32=0$\\$\Delta=0$ donc l'équation admet une unique solution.
\task $\Delta = 20^2-4\cdot2\cdot50=0$\\$\Delta=0$ donc l'équation admet une unique solution.
\task $\Delta = 30^2-4\cdot(-5)\cdot(-47)=-40$\\$\Delta<0$ donc l'équation n'admet pas de solution.\\$\mathcal{S}=\emptyset$.
\task $\Delta = 4^2-4\cdot1\cdot3=4$\\$\Delta>0$ donc l'équation admet deux solutions.
\end{tasks}


}

