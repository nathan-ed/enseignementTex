\titre{}
\theme{derivation}
\auteur{Nathan Scheinmann}
\niveau{3M}
\source{sesamath3e}
\type{serie}
\piments{2}
\pts{}
\annee{2425}

\contenu{
\tcblower
Déterminer les dimensions de la boîte de conserve cylindrique d'une contenance de 1 litre construite avec le moins de matériau possible (on négligera l'épaisseur des parois et les déchets de construction).
}
\correction{
\tcblower
\textit{Generated by AI}

Pour minimiser la quantité de matériau, nous devons minimiser l'aire totale du cylindre.

Notons $r$ le rayon de la base et $h$ la hauteur.

\textbf{Contrainte :} Volume = $1~\text{L} = 1000~\text{cm}^3$
\[\pi r^2 h = 1000 \implies h = \frac{1000}{\pi r^2}\]

\textbf{Aire totale à minimiser :}
\[A = 2\pi r^2 + 2\pi rh\]
(deux bases circulaires + surface latérale)

En substituant $h$ :
\[A(r) = 2\pi r^2 + 2\pi r \cdot \frac{1000}{\pi r^2} = 2\pi r^2 + \frac{2000}{r}\]

Pour minimiser, dérivons :
\[A'(r) = 4\pi r - \frac{2000}{r^2}\]

\[A'(r) = 0 \implies 4\pi r = \frac{2000}{r^2} \implies 4\pi r^3 = 2000 \implies r^3 = \frac{500}{\pi}\]

\[r = \sqrt[3]{\frac{500}{\pi}} \approx 5{,}42~\text{cm}\]

La hauteur correspondante :
\[h = \frac{1000}{\pi r^2} = \frac{1000}{\pi \cdot (500/\pi)^{2/3}} = 2\sqrt[3]{\frac{500}{\pi}} \approx 10{,}84~\text{cm}\]

On remarque que $h = 2r$ (la hauteur optimale est égale au diamètre).

\textbf{Réponse :} Rayon $\approx \boxed{5{,}42~\text{cm}}$ et hauteur $\approx \boxed{10{,}84~\text{cm}}$
}

