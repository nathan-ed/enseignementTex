\titre{}
\theme{geometrie}
\auteur{Nathan Scheinmann}
\niveau{1M}
\source{alloprof}
\type{serie}
\piments{2}
\pts{}
\annee{2425}

\contenu{
	\tcblower
 \begin{minipage}[t]{0.45\textwidth}{
		\vspace{0pt}
		Calculer la longueur de $\overline{DE}$ si $\overline{BC}=100\text{ cm}$ et $\overline{AC}=116{,}62 \text{ cm}$.	
		}
		\end{minipage}
		\hfill
\begin{minipage}[t]{0.4\textwidth}{
		\vspace{0pt}

\begin{tikzpicture}[scale=.75]

% Old: \tkzDefPoint(5,3.5){C}  ==> C→A
\tkzDefPoint(5,3.5){A}

% Old: \tkzDefPoint(0,0){A}  ==> A→B
\tkzDefPoint(0,0){B}

% Old: \tkzDefPoint(7,0){B}  ==> B→E
\tkzDefPoint(7,0){E}

% Old: \tkzDefPointBy[projection=onto A--B](C)  ==> A→B, B→E, C→A
\tkzDefPointBy[projection=onto B--E](A) \tkzGetPoint{D}

% Old: \tkzDefLine[orthogonal=through A](A,C)  ==> A→B, C→A
\tkzDefLine[orthogonal=through B](B,A) \tkzGetPoint{h}

% Old: \tkzInterLL(A,h)(C,D) \tkzGetPoint{E}  ==> A→B, C→A, E→C
\tkzInterLL(B,h)(A,D) \tkzGetPoint{C}

% Old: \tkzMarkRightAngle(B,C,A)  ==> B→E, C→A, A→B
\tkzMarkRightAngle(E,A,B)

% Old: \tkzMarkRightAngle(A,D,C)  ==> A→B, C→A
\tkzMarkRightAngle(B,D,A)

% Old: \tkzMarkRightAngle(C,A,E)  ==> C→A, A→B, E→C
\tkzMarkRightAngle(A,B,C)

% Old: \tkzDrawSegment(C,D)  ==> C→A
\tkzDrawSegment(A,D)

% Old: \tkzDrawPolygon(A,B,C)  ==> A→B, B→E, C→A
\tkzDrawPolygon(B,E,A)

% Old: \tkzDrawPolygon(A,C,E)  ==> A→B, C→A, E→C
\tkzDrawPolygon(B,A,C)

% Old: \tkzLabelPoints[above left](C)  ==> C→A
\tkzLabelPoints[above left](A)

% Old: \tkzLabelPoints[below](E,A,B)  ==> E→C, A→B, B→E
\tkzLabelPoints[below](C,B,E)

% Old: \tkzLabelPoints[below right](D)
\tkzLabelPoints[below right](D)

\end{tikzpicture}
}
\end{minipage}

}
\correction{
	\tcblower
	%% GENERATED BY AI %%
	{\scriptsize \textit{Correction générée par IA}}

	Nous avons un triangle $ABE$ rectangle en $B$, et un triangle $ABD$ rectangle en $D$.

	Les triangles $ABC$ et $BDE$ sont rectangles en $B$ et $D$ respectivement, et ils partagent l'angle en $A$. De plus, puisque $BD \perp AE$ et $AB \perp BC$, les triangles $ABE$ et $ABD$ sont semblables.

	Par le théorème de Pythagore dans le triangle $ABC$ rectangle en $B$ :
	\[AC^2 = AB^2 + BC^2\]
	\[116{,}62^2 = AB^2 + 100^2\]
	\[13600{,}2244 = AB^2 + 10000\]
	\[AB^2 = 3600{,}2244\]
	\[AB = 60{,}00187 \approx 60 \text{ cm}\]

	Les triangles $ABE$ et $ABD$ sont semblables avec le triangle $ABD$ inscrit dans $ABE$.

	Dans le triangle $ABE$ rectangle en $B$, nous avons $AB = 60$ cm et $BE = BC = 100$ cm.

	Les triangles $ABD$ et $ABE$ sont semblables (ils partagent l'angle $\widehat{BAD}$ et ont chacun un angle droit).

	Par similitude :
	\[\frac{BD}{AB} = \frac{AB}{AE}\]

	Calculons d'abord $AE$ avec Pythagore dans le triangle $ABE$ :
	\[AE^2 = AB^2 + BE^2 = 60^2 + 100^2 = 3600 + 10000 = 13600\]
	\[AE = \sqrt{13600} = 20\sqrt{34} \approx 116{,}62 \text{ cm}\]

	Donc :
	\[\frac{BD}{60} = \frac{60}{116{,}62}\]
	\[BD = \frac{60 \times 60}{116{,}62} = \frac{3600}{116{,}62} \approx 30{,}86 \text{ cm}\]

	De même, par similitude :
	\[\frac{AD}{AB} = \frac{AB}{AE}\]
	\[AD = \frac{AB^2}{AE} = \frac{60^2}{116{,}62} = \frac{3600}{116{,}62} \approx 30{,}86 \text{ cm}\]

	Et finalement :
	\[\frac{DE}{BE} = \frac{AB}{AE}\]
	\[DE = \frac{AB \times BE}{AE} = \frac{60 \times 100}{116{,}62} = \frac{6000}{116{,}62} \approx 51{,}44 \text{ cm}\]

	La longueur de $\overline{DE}$ est donc $\boxed{51{,}44 \text{ cm}}$ (ou plus précisément $\frac{3000\sqrt{34}}{34} = \frac{1500\sqrt{34}}{17}$ cm).
}

