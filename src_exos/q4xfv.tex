\titre{}
\theme{derivation}
\auteur{Nathan Scheinmann}
\niveau{3M}
\source{crm}
\type{serie}
\piments{1}
\pts{}
\annee{2526}

\contenu{
\tcblower
Déterminer les points du graphe de $f$ en lesquels la tangente passe par le point $P$ et indiquer l'équation de cette tangente.
\begin{tasks}(2)
	\task $f(x)=x^2$ et $P(1;0)$.
	\task $f(x)=x^3+x^2$ et $P(0;0)$.
	\task $f(x)=\dfrac{1}{x}$ et $P(-3;1)$.
	\task $f(x)=x^3-2x^2+4x$ et $P(0;0)$.
\end{tasks}
}
\correction{
\tcblower
\begin{tasks}(1)
	\task tangentes $y=0$ en $(0;0)$ et $y=4x-4$ en $(2;4)$
	\task tangentes $y=0$ en $(0;0)$ et $y=-\dfrac{1}{4}x$ en $\left -\dfrac{1}{2};\dfrac{1}{8}\right)$
	\task tangentes $y=-x-2$ en $(-1;-1)$ et $y=-\dfrac{1}{9}+\dfrac{2}{3}$ en $\left(3;\dfrac{1}{3}\right)$
	\task tangentes $y=4x$ en $(0;0)$ et $y=3x$ en $(1;3)$

\end{tasks}
}

