\titre{}
\theme{dérivées}
\auteur{Nathan Scheinmann}
\niveau{3M}
\source{analysis}
\type{serie}
\piments{2}
\pts{}
\annee{2425}

\contenu{
\tcblower
Trouver $a$ et $b$ sachant que le graphe de $f(x) = ax^3 + bx^2$ passe par $(-1, 1)$ et a un point d'inflexion quand $x = \frac{1}{3}$.
}
\correction{
\tcblower
%% GENERATED BY AI %%
Nous avons deux conditions :
\begin{enumerate}
\item Le graphe passe par $(-1, 1)$ : $f(-1) = 1$
\item Il y a un point d'inflexion en $x = \dfrac{1}{3}$ : $f''\left(\dfrac{1}{3}\right) = 0$
\end{enumerate}

Pour $f(x) = ax^3 + bx^2$, calculons les dérivées :
\begin{align*}
f'(x) &= 3ax^2 + 2bx \\
f''(x) &= 6ax + 2b
\end{align*}

\textbf{Condition 2 :} $f''\left(\dfrac{1}{3}\right) = 0$
\begin{align*}
6a \cdot \dfrac{1}{3} + 2b &= 0 \\
2a + 2b &= 0 \\
b &= -a
\end{align*}

\textbf{Condition 1 :} $f(-1) = 1$
\begin{align*}
a(-1)^3 + b(-1)^2 &= 1 \\
-a + b &= 1
\end{align*}

En substituant $b = -a$ dans cette équation :
\begin{align*}
-a + (-a) &= 1 \\
-2a &= 1 \\
a &= -\dfrac{1}{2}
\end{align*}

Donc $b = -a = \dfrac{1}{2}$.

\textbf{Vérification :}
\begin{itemize}
\item $f(x) = -\dfrac{1}{2}x^3 + \dfrac{1}{2}x^2 = \dfrac{1}{2}x^2(1-x)$
\item $f(-1) = -\dfrac{1}{2}(-1) + \dfrac{1}{2}(1) = \dfrac{1}{2} + \dfrac{1}{2} = 1$ \checkmark
\item $f''(x) = -3x + 1$, $f''\left(\dfrac{1}{3}\right) = -1 + 1 = 0$ \checkmark
\end{itemize}

$$\boxed{a = -\dfrac{1}{2} \quad \text{et} \quad b = \dfrac{1}{2}}$$
}

