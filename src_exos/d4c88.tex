\titre{}
\theme{derivation2}
\auteur{Nathan Scheinmann}
\niveau{3M}
\source{sesamath3e}
\type{serie}
\piments{2}
\pts{}
\annee{2425}

\contenu{
\tcblower
Soit la fonction polynomiale $P$ définie par
$P(x) = 3x^4 - 11x^3 + 12x^2 - 4x + 2$. Montrer
que $P'$ s'annule au moins une fois sur $]0,1[$.
}
\correction{
\tcblower
\textit{Generated by AI}

Soit $P(x) = 3x^4 - 11x^3 + 12x^2 - 4x + 2$.

Calculons $P'(x)$ :
\[P'(x) = 12x^3 - 33x^2 + 24x - 4\]

Pour montrer que $P'$ s'annule au moins une fois sur $]0, 1[$, nous allons utiliser le théorème de la valeur intermédiaire.

Calculons $P'$ aux bornes de l'intervalle :

\textbf{En $x = 0$ :}
\[P'(0) = 12(0)^3 - 33(0)^2 + 24(0) - 4 = -4 < 0\]

\textbf{En $x = 1$ :}
\[P'(1) = 12(1) - 33(1) + 24(1) - 4 = 12 - 33 + 24 - 4 = -1 < 0\]

Hmm, les deux valeurs sont négatives. Essayons une valeur intermédiaire, par exemple $x = 0{,}5$ :

\[P'(0{,}5) = 12(0{,}125) - 33(0{,}25) + 24(0{,}5) - 4 = 1{,}5 - 8{,}25 + 12 - 4 = 1{,}25 > 0\]

Puisque $P'$ est continue (polynôme), que $P'(0) = -4 < 0$ et $P'(0{,}5) = 1{,}25 > 0$, le théorème de la valeur intermédiaire garantit l'existence d'au moins un point $c \in ]0, 0{,}5[ \subset ]0, 1[$ tel que $P'(c) = 0$.

\textbf{Conclusion :} $P'$ s'annule au moins une fois sur $]0, 1[$.
}

