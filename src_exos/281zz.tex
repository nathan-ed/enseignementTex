\titre{14}
\theme{trigo}
\auteur{Nathan Scheinmann}
\niveau{1M}
\source{sesamath-1M-trigo}
\type{serie}
\piments{2}
\pts{}
\annee{2425}

\contenu{
	\tcblower
Calculer en valeur exacte les côtés manquants des triangles $\triangle EFG$ rectangles en $G$ dans les cas suivants. 
\begin{tasks}(3)
	\task $\widehat{FEG}=30^\circ\;;\overline{EG}=3~\text{cm}$
	\task $\widehat{FEG}=45^\circ\;;\overline{FG}=5~\text{cm}$
	\task $\widehat{FEG}=60^\circ\;;\overline{EG}=9~\text{cm}$
\end{tasks}
}
\correction{
\tcblower

\textit{Generated by AI}

Dans chaque cas, nous avons un triangle $\triangle EFG$ rectangle en $G$, avec l'angle $\widehat{FEG}$ donné.

\begin{tasks}(1)
\task $\widehat{FEG}=30^\circ$ et $\overline{EG}=3~\text{cm}$ :

Le côté $EG$ est adjacent à l'angle de $30°$. Nous cherchons l'hypoténuse $EF$ et le côté opposé $FG$.

Pour l'hypoténuse :
\[\cos(30°) = \frac{EG}{EF} = \frac{3}{EF}\]
\[EF = \frac{3}{\cos(30°)} = \frac{3}{\frac{\sqrt{3}}{2}} = \frac{6}{\sqrt{3}} = \frac{6\sqrt{3}}{3} = \boxed{2\sqrt{3}~\text{cm}}\]

Pour le côté opposé :
\[\tan(30°) = \frac{FG}{EG} = \frac{FG}{3}\]
\[FG = 3 \times \tan(30°) = 3 \times \frac{1}{\sqrt{3}} = \frac{3}{\sqrt{3}} = \frac{3\sqrt{3}}{3} = \boxed{\sqrt{3}~\text{cm}}\]

\task $\widehat{FEG}=45^\circ$ et $\overline{FG}=5~\text{cm}$ :

Le côté $FG$ est opposé à l'angle de $45°$. Dans un triangle rectangle à $45°$, les deux cathètes sont égaux.

\[\tan(45°) = \frac{FG}{EG} = 1 \implies EG = FG = \boxed{5~\text{cm}}\]

Pour l'hypoténuse, utilisons le théorème de Pythagore :
\[EF = \sqrt{EG^2 + FG^2} = \sqrt{5^2 + 5^2} = \sqrt{50} = \boxed{5\sqrt{2}~\text{cm}}\]

\task $\widehat{FEG}=60^\circ$ et $\overline{EG}=9~\text{cm}$ :

Le côté $EG$ est adjacent à l'angle de $60°$.

Pour le côté opposé :
\[\tan(60°) = \frac{FG}{EG} = \frac{FG}{9}\]
\[FG = 9 \times \tan(60°) = 9 \times \sqrt{3} = \boxed{9\sqrt{3}~\text{cm}}\]

Pour l'hypoténuse :
\[\cos(60°) = \frac{EG}{EF} = \frac{9}{EF}\]
\[EF = \frac{9}{\cos(60°)} = \frac{9}{\frac{1}{2}} = \boxed{18~\text{cm}}\]
\end{tasks}

\textbf{Rappel des valeurs remarquables :}
\begin{center}
\begin{tabular}{|c|c|c|c|}
\hline
Angle & $\sin$ & $\cos$ & $\tan$ \\
\hline
$30°$ & $\frac{1}{2}$ & $\frac{\sqrt{3}}{2}$ & $\frac{1}{\sqrt{3}} = \frac{\sqrt{3}}{3}$ \\
\hline
$45°$ & $\frac{\sqrt{2}}{2}$ & $\frac{\sqrt{2}}{2}$ & $1$ \\
\hline
$60°$ & $\frac{\sqrt{3}}{2}$ & $\frac{1}{2}$ & $\sqrt{3}$ \\
\hline
\end{tabular}
\end{center}

}

