\titre{}
\theme{dérivées}
\auteur{Nathan Scheinmann}
\niveau{3M}
\source{analysis}
\type{serie}
\piments{2}
\pts{}
\annee{2425}

\contenu{
\tcblower
Tracer le graphe de
\[
f(x) =
\begin{cases}
2 + x^3, & x \leq 1 \\
3x, & x > 1
\end{cases}
\]
et calculer sa dérivée. Déterminer si \(f\) satisfait les conditions du théorème des accroissemtns finis sur l’intervalle \([1,2]\) et, si oui, trouver toutes les valeurs de $c$ prédites par le théorème.
}
\correction{
\tcblower
\textit{Generated by AI}

Pour vérifier si le théorème des accroissements finis (TAF) s'applique sur $[1,2]$, $f$ doit être continue sur $[1,2]$ et dérivable sur $]1,2[$.

\textbf{Étude de la continuité en $x=1$ :}

À gauche : $\lim_{x \to 1^-} f(x) = 2 + 1^3 = 3$

À droite : $\lim_{x \to 1^+} f(x) = 3 \cdot 1 = 3$

De plus, $f(1) = 2 + 1^3 = 3$

Donc $f$ est continue en $x=1$.

\textbf{Étude de la dérivabilité en $x=1$ :}

À gauche : $f'(1^-) = 3x^2|_{x=1} = 3$

À droite : $f'(1^+) = 3$

Les dérivées à gauche et à droite sont égales, donc $f$ est dérivable en $x=1$ avec $f'(1) = 3$.

\textbf{Dérivée de $f$ :}
\[
f'(x) =
\begin{cases}
3x^2, & x < 1 \\
3, & x \geq 1
\end{cases}
\]

Les conditions du TAF sont satisfaites sur $[1,2]$.

\textbf{Application du TAF :}

On cherche $c \in ]1,2[$ tel que $f'(c) = \frac{f(2) - f(1)}{2-1}$.

$f(1) = 3$ et $f(2) = 3 \cdot 2 = 6$

\[\frac{f(2) - f(1)}{2-1} = \frac{6-3}{1} = 3\]

Sur $]1,2[$, on a $f'(x) = 3$ pour tout $x$, donc $f'(c) = 3$ pour tout $c \in ]1,2[$.

Conclusion : \textbf{Toutes les valeurs $c \in ]1,2[$} vérifient le TAF.
}

