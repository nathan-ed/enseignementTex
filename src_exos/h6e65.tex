\titre{}
\theme{dérivées}
\auteur{Nathan Scheinmann}
\niveau{3M}
\source{fundamentum}
\type{serie}
\piments{2}
\pts{}
\annee{2425}

\contenu{
\tcblower
Un fil de longueur $L$ doit être coupé en deux parties. Avec l'une on forme un triangle équilatéral et avec l'autre un carré. Où faut-il couper ce fil pour que l'aire totale des deux figures construites soit maximale ? minimale ?
}
\correction{
\tcblower
%% GENERATED BY AI %%
{\scriptsize \textit{Correction générée par IA}}

Soit $x$ la longueur du fil utilisée pour le triangle équilatéral et $L - x$ celle utilisée pour le carré.

\textbf{Triangle équilatéral :}
\begin{itemize}
\item Périmètre : $3a = x$, donc côté $a = \frac{x}{3}$
\item Aire : $A_{\triangle} = \frac{\sqrt{3}}{4}a^2 = \frac{\sqrt{3}}{4} \cdot \frac{x^2}{9} = \frac{\sqrt{3}x^2}{36}$
\end{itemize}

\textbf{Carré :}
\begin{itemize}
\item Périmètre : $4c = L - x$, donc côté $c = \frac{L - x}{4}$
\item Aire : $A_{\square} = c^2 = \frac{(L - x)^2}{16}$
\end{itemize}

\textbf{Aire totale :}
\[A(x) = \frac{\sqrt{3}x^2}{36} + \frac{(L - x)^2}{16}\]

Dérivons :
\[A'(x) = \frac{2\sqrt{3}x}{36} + \frac{2(L - x) \cdot (-1)}{16} = \frac{\sqrt{3}x}{18} - \frac{L - x}{8}\]

\textbf{Pour le minimum :}

Annulation : $\frac{\sqrt{3}x}{18} = \frac{L - x}{8}$

\[8\sqrt{3}x = 18(L - x) \implies 8\sqrt{3}x = 18L - 18x \implies x(8\sqrt{3} + 18) = 18L\]

\[x = \frac{18L}{18 + 8\sqrt{3}} = \frac{18L}{18 + 8\sqrt{3}} \cdot \frac{18 - 8\sqrt{3}}{18 - 8\sqrt{3}} = \frac{18L(18 - 8\sqrt{3})}{324 - 192} = \frac{18L(18 - 8\sqrt{3})}{132}\]

Simplifions : $\boxed{x = \frac{9L}{9 + 4\sqrt{3}}}$ pour l'aire minimale.

\textbf{Pour le maximum :}

Étudions les extrémités :
\begin{itemize}
\item Si $x = 0$ : tout le fil forme un carré, $A = \frac{L^2}{16}$
\item Si $x = L$ : tout le fil forme un triangle, $A = \frac{\sqrt{3}L^2}{36}$
\end{itemize}

Comparons : $\frac{L^2}{16} = 0{,}0625L^2$ et $\frac{\sqrt{3}L^2}{36} \approx 0{,}048L^2$.

L'aire maximale est obtenue pour $\boxed{x = 0}$ (tout le fil en carré).
}

