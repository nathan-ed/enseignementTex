\titre{}
\theme{geometrie}
\auteur{Nathan Scheinmann}
\niveau{1M}
\source{alloprof}
\type{serie}
\piments{2}
\pts{}
\annee{2425}

\contenu{
	\tcblower
 \begin{minipage}[t]{0.45\textwidth}{
		\vspace{0pt}
		Calculer la longueur de $\overline{CD}$ si $\overline{AB}=13\text{ cm}$ et $\overline{AC}=12 \text{ cm}$.	
		}
		\end{minipage}
		\hfill
\begin{minipage}[t]{0.4\textwidth}{
		\vspace{0pt}
\begin{tikzpicture}[scale=.75, rotate=167]
\tkzDefPoint(5,3.5){C}
\tkzDefPoint(0,0){A}
\tkzDefPoint(7,0){B}
\tkzDefPointBy[projection=onto A--B](C) \tkzGetPoint{D}
\tkzMarkRightAngle(B,C,A)
\tkzMarkRightAngle(A,D,C)
\tkzDrawPoints(A,B,C,D)
\tkzDrawSegment(C,D)
\tkzDrawPolygon(A,B,C)
\tkzLabelPoints[above left](B,A,D)
\tkzLabelPoints[below](C)
\end{tikzpicture}}
\end{minipage}

}
\correction{
	\tcblower
{\scriptsize \textit{Correction générée par IA}}

Dans le triangle rectangle $ABC$, avec l'angle droit en $C$, le point $D$ est le pied de la hauteur issue de $C$.

Nous utilisons les relations métriques dans un triangle rectangle. On sait que :
\[
CD^2 = AD \cdot BD
\]

Mais nous ne connaissons pas encore $BD$. Utilisons plutôt le fait que dans un triangle rectangle :
\[
AC^2 = AD \cdot AB
\]

On nous donne $AB = 13$ cm et $AC = 12$ cm.

Donc :
\[
12^2 = AD \cdot 13
\]

\[
144 = 13 \cdot AD
\]

\[
AD = \dfrac{144}{13} \text{ cm}
\]

Puisque $AB = AD + DB$, on a :
\[
DB = AB - AD = 13 - \dfrac{144}{13} = \dfrac{169 - 144}{13} = \dfrac{25}{13} \text{ cm}
\]

Maintenant, calculons $CD$ en utilisant :
\[
CD^2 = AD \cdot DB = \dfrac{144}{13} \times \dfrac{25}{13} = \dfrac{3600}{169}
\]

\[
CD = \dfrac{60}{13} \text{ cm}
\]

La longueur de $\overline{CD}$ est de $\dfrac{\textbf{60}}{\textbf{13}}$ cm, soit environ $4{,}62$ cm.
}

