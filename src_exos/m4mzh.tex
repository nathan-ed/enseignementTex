\titre{}
\theme{derivation}
\auteur{Nathan Scheinmann}
\niveau{3M}
\source{sesamath3e}
\type{serie}
\piments{2}
\pts{}
\annee{2425}

\contenu{
\tcblower
Déterminer les dimensions d'un rectangle de plus grande aire ayant un périmètre de 30 cm.
}
\correction{
\tcblower
\textit{Generated by AI}

Notons $L$ la longueur et $\ell$ la largeur du rectangle.

Le périmètre est donné par :
\[
2L + 2\ell = 30 \quad \Rightarrow \quad L + \ell = 15 \quad \Rightarrow \quad \ell = 15 - L
\]

L'aire du rectangle est :
\[
\mathcal{A}(L) = L \times \ell = L(15 - L) = 15L - L^2
\]

avec $L \in ]0 ; 15[$.

Pour maximiser l'aire, calculons la dérivée :
\[
\mathcal{A}'(L) = 15 - 2L
\]

Cherchons le point critique :
\[
\mathcal{A}'(L) = 0 \quad \Leftrightarrow \quad 15 - 2L = 0 \quad \Leftrightarrow \quad L = 7{,}5~\text{cm}
\]

Vérifions qu'il s'agit bien d'un maximum :
\[
\mathcal{A}''(L) = -2 < 0
\]

Donc $L = 7{,}5~\text{cm}$ correspond à un maximum.

La largeur correspondante est :
\[
\ell = 15 - 7{,}5 = 7{,}5~\text{cm}
\]

L'aire maximale est :
\[
\mathcal{A}(7{,}5) = 7{,}5 \times 7{,}5 = 56{,}25~\text{cm}^2
\]

\textbf{Réponse :} Le rectangle de plus grande aire ayant un périmètre de 30 cm est un carré de côté $7{,}5~\text{cm}$.
}

