\titre{}
\theme{dérivées}
\auteur{Nathan Scheinmann}
\niveau{3M}
\source{fundamentum}
\type{serie}
\piments{2}
\pts{}
\annee{2425}

\contenu{
\tcblower
La résistance d'une poutre de section rectangulaire est proportionnelle au produit de la largeur par le carré de la hauteur de sa section transversale. Quelle est la forme de la poutre la plus résistante que l'on peut tailler dans un tronc d'arbre de section circulaire ?
}
\correction{
\tcblower
\textit{Generated by AI}
{\scriptsize \textit{Correction générée par IA}}

Considérons un tronc d'arbre de section circulaire de rayon $R$. Nous voulons tailler une poutre de section rectangulaire de largeur $x$ et de hauteur $y$.

Par le théorème de Pythagore dans la section circulaire, nous avons la contrainte :
\[\left(\frac{x}{2}\right)^2 + \left(\frac{y}{2}\right)^2 = R^2\]
\[\frac{x^2 + y^2}{4} = R^2\]
\[x^2 + y^2 = 4R^2\]

D'où $x^2 = 4R^2 - y^2$.

La résistance de la poutre est proportionnelle à $xy^2$. Notons $r = kxy^2$ où $k > 0$ est une constante.

Nous voulons maximiser :
\[r(y) = k \cdot \sqrt{4R^2 - y^2} \cdot y^2 = ky^2\sqrt{4R^2 - y^2}\]

Pour simplifier, maximisons $f(y) = y^2\sqrt{4R^2 - y^2}$ pour $y \in (0, 2R)$.

Posons $u = y^2$, alors $\sqrt{4R^2 - y^2} = \sqrt{4R^2 - u}$ et nous maximisons :
\[g(u) = u\sqrt{4R^2 - u}\]

Équivalemment, maximisons $h(u) = u^2(4R^2 - u)$ (en élevant au carré).

Développons : $h(u) = 4R^2u^2 - u^3$

Dérivons :
\[h'(u) = 8R^2u - 3u^2 = u(8R^2 - 3u)\]

L'annulation de la dérivée donne :
\[u(8R^2 - 3u) = 0\]

Puisque $u > 0$, nous avons $8R^2 - 3u = 0$, d'où :
\[u = \frac{8R^2}{3}\]

Donc $y^2 = \frac{8R^2}{3}$, ce qui donne $y = \frac{2\sqrt{2}R}{\sqrt{3}} = \frac{2\sqrt{6}R}{3}$.

La largeur correspondante est :
\[x^2 = 4R^2 - y^2 = 4R^2 - \frac{8R^2}{3} = \frac{12R^2 - 8R^2}{3} = \frac{4R^2}{3}\]
\[x = \frac{2R}{\sqrt{3}} = \frac{2\sqrt{3}R}{3}\]

Le rapport entre largeur et hauteur est :
\[\frac{x}{y} = \frac{\frac{2\sqrt{3}R}{3}}{\frac{2\sqrt{6}R}{3}} = \frac{2\sqrt{3}R}{2\sqrt{6}R} = \frac{\sqrt{3}}{\sqrt{6}} = \frac{1}{\sqrt{2}} = \frac{\sqrt{2}}{2}\]

La poutre la plus résistante a donc un rapport $\boxed{\frac{x}{y} = \frac{1}{\sqrt{2}}}$, c'est-à-dire que la hauteur doit être $\sqrt{2}$ fois la largeur : $y = \sqrt{2} \cdot x$.
}

