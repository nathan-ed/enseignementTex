\titre{}
\theme{dérivées}
\auteur{Nathan Scheinmann}
\niveau{3M}
\source{analysis}
\type{cote}
\piments{2}
\pts{}
\annee{2425}

\contenu{
\tcblower
Montrer que l’équation \(x^3 + ax + b = 0\) a exactement une racine réelle si \(a \geq 0\) et au plus une racine réelle entre \(-\dfrac{1}{3}\sqrt{3|a|}\) et \(\dfrac{1}{3}\sqrt{3|a|}\) si \(a < 0\).
}
\correction{
\tcblower
\textit{Generated by AI}

Considérons $f(x) = x^3 + ax + b$.

\textbf{Cas 1 : $a \geq 0$}

Calculons la dérivée :
\[
f'(x) = 3x^2 + a
\]

Si $a \geq 0$, alors $f'(x) = 3x^2 + a \geq 0$ pour tout $x \in \mathbb{R}$, et $f'(x) = 0$ uniquement si $a = 0$ et $x = 0$.

Donc $f$ est strictement croissante sur $\mathbb{R}$ (ou croissante si $a = 0$).

Une fonction strictement monotone sur $\mathbb{R}$ coupe l'axe des abscisses exactement une fois car :
\begin{itemize}
\item $\lim_{x \to -\infty} f(x) = -\infty$
\item $\lim_{x \to +\infty} f(x) = +\infty$
\end{itemize}

Par le théorème des valeurs intermédiaires, $f$ admet exactement une racine réelle.

\textbf{Cas 2 : $a < 0$}

Si $a < 0$, alors $f'(x) = 3x^2 + a = 0$ admet deux solutions :
\[
x = \pm \sqrt{-\frac{a}{3}} = \pm \sqrt{\frac{|a|}{3}} = \pm \frac{1}{\sqrt{3}}\sqrt{|a|}
\]

Tableau de variation :
\begin{itemize}
\item $f$ croissante sur $\left]-\infty ; -\frac{1}{\sqrt{3}}\sqrt{|a|}\right]$
\item $f$ décroissante sur $\left[-\frac{1}{\sqrt{3}}\sqrt{|a|} ; \frac{1}{\sqrt{3}}\sqrt{|a|}\right]$
\item $f$ croissante sur $\left[\frac{1}{\sqrt{3}}\sqrt{|a|} ; +\infty\right[$
\end{itemize}

La fonction $f$ admet :
\begin{itemize}
\item Un maximum local en $x_1 = -\frac{1}{\sqrt{3}}\sqrt{|a|}$
\item Un minimum local en $x_2 = \frac{1}{\sqrt{3}}\sqrt{|a|}$
\end{itemize}

Entre ces deux extrema, c'est-à-dire sur l'intervalle $\left[-\frac{1}{\sqrt{3}}\sqrt{|a|} ; \frac{1}{\sqrt{3}}\sqrt{|a|}\right]$, la fonction est décroissante.

Dans cet intervalle, $f$ peut avoir au plus une racine (car $f$ est monotone sur cet intervalle).

De plus :
\begin{itemize}
\item $f$ peut avoir une racine avant le maximum (à gauche)
\item $f$ peut avoir une racine dans l'intervalle de décroissance
\item $f$ peut avoir une racine après le minimum (à droite)
\end{itemize}

Donc au total, $f$ peut avoir 1, 2 ou 3 racines réelles selon les valeurs de $b$.

Mais dans l'intervalle $\left[-\frac{1}{3}\sqrt{3|a|} ; \frac{1}{3}\sqrt{3|a|}\right]$, qui contient l'intervalle de décroissance, $f$ est au plus monotone par morceaux et admet au plus une racine dans la partie décroissante.

\textbf{Note :} L'expression $\frac{1}{3}\sqrt{3|a|}$ devrait probablement être $\frac{1}{\sqrt{3}}\sqrt{|a|} = \sqrt{\frac{|a|}{3}}$.
}

