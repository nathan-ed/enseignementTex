\titre{}
\theme{limites}
\auteur{Nathan Scheinmann}
\niveau{3M}
\source{musy}
\type{serie}
\piments{2}
\pts{}
\annee{2425}

\contenu{
\tcblower
Montrer que l'équation 
\[2^x=x^2\]
admet une solution sur l'intervalle $[-1;0]$. 
}
\correction{
	\tcblower
\begin{itemize}
  \item On considère la fonction \( f(x) = 2^x - x^2 \), continue sur \( \mathbb{R} \) car somme de fonctions continues.
  \item On calcule :
    \[
    \begin{aligned}
    f(-1) &= 2^{-1} - (-1)^2 = \dfrac{1}{2} - 1 = -\dfrac{1}{2} \\
    f(0) &= 2^0 - 0^2 = 1 - 0 = 1
    \end{aligned}
    \]
  \item On a \( f(-1) < 0 \) et \( f(0) > 0 \), donc \( f(-1) \cdot f(0) < 0 \).
  \item D’après le corollaire du théorème de la valeur intermédiaire, il existe \( c \in [-1; 0] \) tel que \( f(c) = 0 \).
\end{itemize}
}

