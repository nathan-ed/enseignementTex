\titre{20}
\theme{trigo}
\auteur{Nathan Scheinmann}
\niveau{1M}
\source{sesamath-1M-trigo}
\type{serie}
\piments{2}
\pts{}
\annee{2425}

\contenu{
	\tcblower
	\begin{minipage}[t]{0.45\textwidth}{
	\vspace{0pt}
	Soit un cercle de centre $C$ et de rayon $10$. Une corde $[BD]$ est regardée sous un angle de $20^\circ$ depuis un point $A$ du cercle. Calculer la longueur de la corde $[BD]$ et de l'arc $BD$. 	
	}
	\end{minipage}
	\hfill
	\begin{minipage}[t]{0.5\textwidth}{
	\vspace{0pt}
	\includegraphics[scale=1]{../medias/1M/trigo/1M-exo-20}
	}
	\end{minipage}
}
\correction{
\tcblower
\textit{Generated by AI}

Un cercle de rayon $R = 10$ avec centre $C$. Un point $A$ sur le cercle observe la corde $[BD]$ sous un angle de $20°$.

\textbf{Propriété : Angle inscrit et angle au centre}

L'angle inscrit $\widehat{BAD} = 20°$ intercepte le même arc que l'angle au centre $\widehat{BCD}$.

Par la propriété des angles inscrits : l'angle au centre est le double de l'angle inscrit.

Donc :
\[
\widehat{BCD} = 2 \times \widehat{BAD} = 2 \times 20° = 40°
\]

\textbf{Calcul de la longueur de la corde $[BD]$ :}

Dans le triangle isocèle $\triangle BCD$ (car $CB = CD = 10$), traçons la hauteur issue de $C$ qui coupe $[BD]$ en son milieu $M$.

On a :
\[
\widehat{BCM} = \frac{40°}{2} = 20°
\]

Dans le triangle rectangle $\triangle BCM$ :
\[
\sin(20°) = \frac{BM}{BC} = \frac{BM}{10}
\]

Donc :
\[
BM = 10 \sin(20°) \approx 10 \times 0{,}342 = 3{,}42
\]

La longueur de la corde est :
\[
BD = 2 \times BM = 20 \sin(20°) \approx 6{,}84
\]

\textbf{Calcul de la longueur de l'arc $BD$ :}

L'arc correspond à un angle au centre de $40° = \frac{40\pi}{180} = \frac{2\pi}{9}$ radians.

La longueur de l'arc est :
\[
\text{Arc } BD = R \times \theta = 10 \times \frac{2\pi}{9} = \frac{20\pi}{9} \approx 6{,}98
\]

\textbf{Réponses :}
\begin{itemize}
\item Longueur de la corde : $BD = 20 \sin(20°) \approx 6{,}84$
\item Longueur de l'arc : $\text{Arc } BD = \dfrac{20\pi}{9} \approx 6{,}98$
\end{itemize}
}

