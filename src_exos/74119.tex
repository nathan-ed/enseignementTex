\titre{}
\theme{dérivées}
\auteur{Nathan Scheinmann}
\niveau{3M}
\source{fundamentum}
\type{serie}
\piments{2}
\pts{}
\annee{2425}

\contenu{
\tcblower
Une feuille rectangulaire doit contenir 392 cm$^2$ de texte imprimé. Les marges supérieure et inférieure doivent être de 2 cm chacune; les marges latérales de 1 cm chacune. Déterminer les dimensions de la feuille nécessitant le moins de papier.
}
\correction{
\tcblower
\textit{Generated by AI}
{\scriptsize \textit{Correction générée par IA}}

Soit $x$ la largeur du texte imprimé et $y$ sa hauteur. Nous avons $xy = 392$ cm$^2$.

Les dimensions de la feuille sont :
\begin{itemize}
\item Largeur totale : $x + 2 \times 1 = x + 2$
\item Hauteur totale : $y + 2 \times 2 = y + 4$
\end{itemize}

L'aire totale de la feuille est :
\[A = (x + 2)(y + 4)\]

De la contrainte $xy = 392$, nous tirons $y = \frac{392}{x}$. Substituons :
\[A(x) = (x + 2)\left(\frac{392}{x} + 4\right) = (x + 2) \cdot \frac{392 + 4x}{x} = \frac{(x + 2)(392 + 4x)}{x}\]

\[= \frac{392x + 4x^2 + 784 + 8x}{x} = \frac{4x^2 + 400x + 784}{x} = 4x + 400 + \frac{784}{x}\]

Pour minimiser $A(x)$, dérivons :
\[A'(x) = 4 - \frac{784}{x^2}\]

Annulation de la dérivée :
\[4 - \frac{784}{x^2} = 0 \implies 4x^2 = 784 \implies x^2 = 196 \implies x = 14 \text{ cm}\]

Donc $y = \frac{392}{14} = 28$ cm.

Les dimensions de la feuille sont :
\[\boxed{\text{Largeur} = 14 + 2 = 16 \text{ cm}, \quad \text{Hauteur} = 28 + 4 = 32 \text{ cm}}\]
}

