\titre{10}
\theme{trigo}
\auteur{Nathan Scheinmann}
\niveau{1M}
\source{sesamath-1M-trigo}
\type{serie}
\piments{2}
\pts{}
\annee{2425}

\contenu{
	\tcblower
	Soit $\triangle EFG$ un triangle isocèle en $F$. On a $\overline{EG}=42~\text{cm}$ et $\widehat{EFG}=62^\circ$. Calculer l'aire de $\triangle EFG$.
}
\correction{
\tcblower
\textit{Generated by AI}

Le triangle $EFG$ est isocèle en $F$, donc $\overline{FE} = \overline{FG}$. Notons cette longueur $\ell$.

L'angle au sommet est $\widehat{EFG} = 62°$ et la base est $\overline{EG} = 42~\text{cm}$.

Pour calculer l'aire, nous pouvons utiliser la formule :
\[\text{Aire} = \frac{1}{2} \cdot \overline{FE} \cdot \overline{FG} \cdot \sin(\widehat{EFG})\]

Il faut d'abord trouver $\ell = \overline{FE} = \overline{FG}$.

Utilisons la loi des cosinus dans le triangle :
\[\overline{EG}^2 = \overline{FE}^2 + \overline{FG}^2 - 2 \cdot \overline{FE} \cdot \overline{FG} \cdot \cos(\widehat{EFG})\]

\[42^2 = \ell^2 + \ell^2 - 2\ell^2 \cos(62°)\]
\[1764 = 2\ell^2(1 - \cos(62°))\]
\[\ell^2 = \frac{1764}{2(1 - \cos(62°))} = \frac{1764}{2(1 - 0{,}469)} = \frac{1764}{1{,}062} \approx 1661\]
\[\ell \approx 40{,}76~\text{cm}\]

L'aire est :
\[\text{Aire} = \frac{1}{2} \cdot \ell^2 \cdot \sin(62°) = \frac{1}{2} \cdot 1661 \cdot 0{,}883 \approx \boxed{733~\text{cm}^2}\]
}

