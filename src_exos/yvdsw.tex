\titre{18}
\theme{trigo}
\auteur{Nathan Scheinmann}
\niveau{1M}
\source{sesamath-1M-trigo}
\type{serie}
\piments{2}
\pts{}
\annee{2425}

\contenu{
	\tcblower
	$P$ et $Q$ sont deux points d'un cercle de centre $O$ et de diamètre $[AB]$ tels que $\widehat{AQP}=35^\circ$. On donne $\overline{AB}=5~\text{cm}$. 
	\begin{tasks}
		\task Faire un croquis de la situation. 
		\task Déterminer l'angle $\widehat{ABP}$ et justifier. 
		\task Quelle est la nature du triangle $\triangle ABP~?$
		\task Calculer la longueur $\overline{AP}$.
		\task Déterminer l'angle $\widehat{POB}$ et justifier. 
	\end{tasks}
}
\correction{
\tcblower
\textit{Generated by AI}

\begin{tasks}
\task \textit{Croquis :} Dessiner un cercle de centre $O$ avec le diamètre $[AB]$. Placer les points $P$ et $Q$ sur le cercle tels que l'angle $\widehat{AQP} = 35°$.

\task \textbf{Angle $\widehat{ABP}$ :}

Les points $P$ et $Q$ sont sur le cercle de diamètre $[AB]$. Par le théorème de l'angle inscrit dans un demi-cercle, tout angle inscrit qui intercepte le diamètre est un angle droit.

Donc $\widehat{APB} = 90°$ (angle inscrit qui intercepte le diamètre $[AB]$).

De même, $\widehat{AQB} = 90°$.

Dans le triangle $APQ$, nous avons l'angle inscrit $\widehat{AQP} = 35°$.

L'angle au centre correspondant est $\widehat{AOP}$, et par la propriété des angles inscrits, $\widehat{ABP}$ est lié à $\widehat{AQP}$ car ils interceptent le même arc.

Par le théorème de l'angle inscrit : $\widehat{ABP} = \widehat{AQP} = \boxed{35°}$

\task \textbf{Nature du triangle $\triangle ABP$ :}

Puisque $[AB]$ est le diamètre du cercle et que $P$ est sur le cercle, par le théorème de Thalès, l'angle $\widehat{APB} = 90°$.

Donc le triangle $\triangle ABP$ est \boxed{\text{rectangle en } P}.

\task \textbf{Longueur $\overline{AP}$ :}

Dans le triangle rectangle $ABP$ (rectangle en $P$) :
$$\sin(\widehat{ABP}) = \dfrac{AP}{AB}$$
$$\sin(35°) = \dfrac{AP}{5}$$
$$AP = 5 \times \sin(35°) \approx 5 \times 0{,}574 = 2{,}87 \text{ cm}$$

$$\boxed{AP \approx 2{,}87 \text{ cm}}$$

\task \textbf{Angle $\widehat{POB}$ :}

L'angle au centre $\widehat{POB}$ est le double de l'angle inscrit $\widehat{PAB}$ qui intercepte le même arc $\widehat{PB}$.

Dans le triangle rectangle $ABP$ : $\widehat{PAB} = 90° - 35° = 55°$

Donc $\widehat{POB} = 2 \times 55° = \boxed{110°}$
\end{tasks}
}

