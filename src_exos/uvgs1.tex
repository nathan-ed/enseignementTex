\titre{}
\theme{derivation}
\auteur{Nathan Scheinmann}
\niveau{3M}
\source{sesamath3e}
\type{serie}
\piments{2}
\pts{}
\annee{2425}

\contenu{
\tcblower
Déterminer les dimensions du cylindre de volume maximal inscrit dans une sphère de rayon $R$.
}
\correction{
\tcblower
\textit{Generated by AI}

Plaçons la sphère de centre $O$ et de rayon $R$ dans un repère, avec le cylindre d'axe vertical. Soit $r$ le rayon du cylindre et $h$ sa hauteur.

La section de la sphère par un plan horizontal passant par la base supérieure du cylindre donne un cercle de rayon $r$. La distance du centre de ce cercle au centre $O$ de la sphère est $\dfrac{h}{2}$ (la moitié de la hauteur du cylindre).

Par le théorème de Pythagore dans la sphère :
$$r^2 + \left(\dfrac{h}{2}\right)^2 = R^2$$

D'où :
$$r^2 = R^2 - \dfrac{h^2}{4}$$

Le volume du cylindre est :
$$V(h) = \pi r^2 h = \pi \left(R^2 - \dfrac{h^2}{4}\right)h = \pi R^2 h - \dfrac{\pi h^3}{4}$$

Pour maximiser le volume, nous calculons la dérivée :
$$V'(h) = \pi R^2 - \dfrac{3\pi h^2}{4}$$

Résolvons $V'(h) = 0$ :
$$\pi R^2 = \dfrac{3\pi h^2}{4}$$
$$R^2 = \dfrac{3h^2}{4}$$
$$h^2 = \dfrac{4R^2}{3}$$
$$h = \dfrac{2R}{\sqrt{3}} = \dfrac{2R\sqrt{3}}{3}$$

Vérifions qu'il s'agit d'un maximum : $V''(h) = -\dfrac{3\pi h}{2} < 0$ pour $h > 0$, donc c'est bien un maximum.

Le rayon correspondant est :
$$r^2 = R^2 - \dfrac{h^2}{4} = R^2 - \dfrac{R^2}{3} = \dfrac{2R^2}{3}$$
$$r = R\sqrt{\dfrac{2}{3}} = \dfrac{R\sqrt{6}}{3}$$

Les dimensions du cylindre de volume maximal sont :
$$\boxed{h = \dfrac{2R\sqrt{3}}{3} \text{ et } r = \dfrac{R\sqrt{6}}{3}}$$
}

