\titre{}
\theme{limites}
\auteur{Nathan Scheinmann}
\niveau{3M}
\source{sesamath3e}
\type{serie}
\piments{2}
\pts{}
\annee{2425}

\contenu{
\tcblower
Déterminer les asymptotes obliques des fonctions $f$ définies par :

\begin{tasks}(2)
\task $f(x) = \dfrac{x^3 - 2x - 3}{x^2 + 2x + 1}$
\task $f(x) = \dfrac{x^2 - x + 2}{x - 2}$
%\task $f(x) = x\sqrt{\dfrac{x}{x + 1}}$
%\task $f(x) = \sqrt{x^2 + 2px + k}$, où $p$ et $k$ sont tels que la fonction $f$ est définie sur $\mathbb{R}$
\end{tasks}
{\itshape Pourquoi parle-t-on d'asymptotes obliques dans ces cas~? }
}
\correction{
	\tcblower
\begin{tasks}(4)
	\task $y=x-2$
	\task $y=x+1$
\end{tasks}
}

