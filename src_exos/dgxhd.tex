\titre{}
\theme{geometrie}
\auteur{Nathan Scheinmann}
\niveau{1M}
\source{}
\type{serie}
\piments{2}
\pts{}
\annee{2425}

\contenu{
	\tcblower
\pgfmathsetmacro{\a}{4}%  
\pgfmathsetmacro{\b}{6}%  
\pgfmathsetmacro{\h}{3}%

	\begin{minipage}[t]{0.5\textwidth}{
	\vspace{0pt}
\begin{tasks}
	\task Calculer la hauteur de la pyramide droite à base carrée suivante.
	\task Calculer l’aire de la base et le volume de la Pyramide de Kheops si le côté de la base mesure $231$ m et la hauteur de la pyramide mesure $147$ m.
\end{tasks}	
	}
	\end{minipage}
	\begin{minipage}[t]{0.4\textwidth}{
	\vspace{0pt}
\begin{tikzpicture}[%scale=0.7,
font=\footnotesize, 
z ={(0,0,-cos(45))},
]

\coordinate[label=left:$A$] (A) at (0,0,0); 
\coordinate[label=right:$B$] (B) at (\a,0,0); 
\coordinate[label=right:$C$] (C) at (\a,0,\b); 
\coordinate[label=below:$D$] (D) at (0,0,\b); 
\coordinate[] (M) at (0.5*\a,0,0.5*\b); 
\path[] (M) --+ (0,\h,0) coordinate[label=$S$] (S);

% Grundfläche
\fill[lightgray] (A) -- (B) -- (C) -- (D) --cycle;
\draw[] (A) -- (B)  -- (C) node[pos=0.6, sloped, below]{$6$\,cm}; 
\draw[densely dashed] (A) -- (D) -- (C); 
% Höhe
\draw[] ($(M)!-5pt!(C)$) -- ($(M)!5pt!(C)$) ; 
\draw[] ($(M)!-5pt!(B)$) -- ($(M)!5pt!(B)$) ; 
\draw[] (M) -- (S); 
\draw[] (C) -- (S) node[midway,sloped, above]{$9$\,cm}; 
% Mantellinien
\draw[densely dashed] (D) -- (S); 
\foreach \P in {A,B,C}{  \draw[] (\P) -- (S);     }


\end{tikzpicture}
	}
	\end{minipage}
}
\correction{
	\tcblower
{\scriptsize \textit{Correction générée par IA}}

\begin{tasks}
\task Pour calculer la hauteur de la pyramide, nous utilisons le triangle rectangle formé par la hauteur $h$, la demi-diagonale de la base et l'arête latérale.

Le côté de la base carrée mesure $6$ cm, donc la demi-diagonale de la base vaut :
\[
\dfrac{6\sqrt{2}}{2} = 3\sqrt{2} \text{ cm}
\]

L'arête latérale (de $C$ à $S$) mesure $9$ cm.

Dans le triangle rectangle, nous avons :
\[
9^2 = h^2 + (3\sqrt{2})^2
\]

\[
81 = h^2 + 9 \cdot 2
\]

\[
81 = h^2 + 18
\]

\[
h^2 = 63
\]

\[
h = \sqrt{63} = 3\sqrt{7} \text{ cm} \approx 7{,}94 \text{ cm}
\]

\task Pour la Pyramide de Kheops :

\textbf{Aire de la base :}

Le côté de la base mesure $231$ m, donc l'aire de la base carrée est :
\[
A = 231^2 = 53\,361 \text{ m}^2
\]

\textbf{Volume de la pyramide :}

Le volume d'une pyramide est donné par :
\[
V = \dfrac{1}{3} \times A_{\text{base}} \times h
\]

Avec $h = 147$ m :
\[
V = \dfrac{1}{3} \times 53\,361 \times 147 = \dfrac{7\,844\,067}{3} = 2\,614\,689 \text{ m}^3
\]

Le volume de la Pyramide de Kheops est d'environ $2\,614\,689$ m$^3$.
\end{tasks}
}

