\titre{102-musy17}
\theme{equations}
\auteur{Nathan Scheinmann}
\niveau{1M}
\source{muzy-2017}
\type{serie}
\piments{3}
\pts{}
\annee{2425}

\contenu{
	\tcblower
Le rectangle représenté ci-dessous a été découpé en 5 carrés. Le périmètre du rectangle est de 1 m . Déterminer son aire.
\begin{center}
	\begin{tikzpicture}[scale=0.5]
    \tkzDefPoint[label=below:{}](0,0){A}
    \tkzDefPoint[label=right:{}](0,2.25){B}
    \tkzDefPoint[label=above:{}](0,4.5){C}
    \tkzDefPoint[label=above:{}](-2.25,4.5){D}
    \tkzDefPoint[label=above:{}](-5.25,4.5){E}
    \tkzDefPoint[label=above:{}](-5.25,1.5){F}
    \tkzDefPoint[label=above:{}](-5.25,0){G}
    \tkzDefPoint[label=above:{}](-3.75,0){H}
    \tkzDefPoint[label=above:{}](-2.25,0){I}
    \tkzDefPoint[label=above:{}](-2.25,2.25){J}
    \tkzDefPoint[label=above:{}](-2.25,1.5){K}
    \tkzDefPoint[label=above:{}](-3.75,1.5){L}
\tkzDrawPolygon(A,B,J,I)
\tkzDrawPolygon(B,C,D,J)
\tkzDrawPolygon(D,E,F,K)
\tkzDrawPolygon(F,G,H,L)
\tkzDrawPolygon(H,I,K,L)
 \end{tikzpicture}
\end{center}

}
\correction{
	\tcblower
$\dfrac{21}{338}$ m$^{2}$
}
\correctionDetaillee{
On note $a$ le côté du grand carré, $b$ le côté des moyens, $d$ le côté des plus petits étant donné que les moyens et les petits sont isométriques entre-eux, partageant un côté. On a que $a=2d$ et donc que 
\[\dfrac{3}{2}a=2b\implies a=\dfrac{4b}{3}.\]
Par ailleurs, $a+3b=\dfrac{1}{2}$ et donc 
\[\dfrac{4b}{3}+3b=\dfrac{1}{2}\implies 13b=\dfrac{3}{2} \implies b=\dfrac{3}{26}\]
Ainsi, $a=\dfrac{2}{13}$. On calcule l'aire 
\[\dfrac{3}{2}a(a+b)=\dfrac{21}{338}\text{m}^2\]
}

