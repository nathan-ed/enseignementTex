\titre{}
\theme{dérivées}
\auteur{Nathan Scheinmann}
\niveau{3M}
\source{fundamentum}
\type{serie}
\piments{2}
\pts{}
\annee{2425}

\contenu{
\tcblower
Un navire doit parcourir 40 km contre un courant de 10 km/h. Il consomme par heure une quantité de carburant proportionnelle au carré de sa vitesse. En supposant qu'il navigue à vitesse constante, quelle doit être sa vitesse pour minimiser la quantité de carburant consommée ?
}
\correction{
\tcblower
%% GENERATED BY AI %%
{\scriptsize \textit{Correction générée par IA}}

Notons $v$ la vitesse du navire par rapport à l'eau (en km/h). La vitesse effective contre le courant est $v - 10$ km/h.

Le temps nécessaire pour parcourir 40 km contre le courant est :
\[t = \frac{40}{v - 10} \text{ heures}\]

La consommation de carburant par heure est proportionnelle à $v^2$, soit $kv^2$ pour une constante $k > 0$.

La consommation totale de carburant est donc :
\[C(v) = kv^2 \cdot t = kv^2 \cdot \frac{40}{v - 10} = \frac{40kv^2}{v - 10}\]

Pour minimiser la consommation, nous dérivons par rapport à $v$ :
\[C'(v) = 40k \cdot \frac{2v(v - 10) - v^2 \cdot 1}{(v - 10)^2} = 40k \cdot \frac{2v^2 - 20v - v^2}{(v - 10)^2} = 40k \cdot \frac{v^2 - 20v}{(v - 10)^2}\]

L'annulation de la dérivée donne :
\[v^2 - 20v = 0\]
\[v(v - 20) = 0\]

Puisque $v > 10$ (le navire doit avancer contre le courant), nous avons :
\[v = 20 \text{ km/h}\]

Vérifions qu'il s'agit bien d'un minimum en étudiant le signe de $C'(v)$ :
\begin{itemize}
\item Pour $10 < v < 20$ : $v^2 - 20v < 0$, donc $C'(v) < 0$ (décroissant)
\item Pour $v > 20$ : $v^2 - 20v > 0$, donc $C'(v) > 0$ (croissant)
\end{itemize}

Donc $v = 20$ km/h correspond bien à un minimum.

La vitesse optimale du navire est $\boxed{20 \text{ km/h}}$ par rapport à l'eau.

Remarque : La vitesse effective contre le courant est alors $20 - 10 = 10$ km/h.
}

