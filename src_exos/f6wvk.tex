\titre{}
\theme{dérivées}
\auteur{Nathan Scheinmann}
\niveau{3M}
\source{fundamentum}
\type{serie}
\piments{2}
\pts{}
\annee{2425}

\contenu{
\tcblower
Un navire doit parcourir 40 km contre un courant de 10 km/h. Il consomme par heure une quantité de carburant proportionnelle au carré de sa vitesse. En supposant qu'il navigue à vitesse constante, quelle doit être sa vitesse pour minimiser la quantité de carburant consommée ?
}
\correction{
\tcblower
}

