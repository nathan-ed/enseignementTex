\titre{25}
\theme{trigo}
\auteur{Nathan Scheinmann}
\niveau{1M}
\source{sesamath-1M-trigo}
\type{serie}
\piments{2}
\pts{}
\annee{2425}

\contenu{
	\tcblower

\begin{minipage}[t]{0.53\textwidth}{
\vspace{0pt}
\begin{tasks}(1)
	\task Exprimer l'aire $A$ de la surface hachurée en fonction de $r$ et $\alpha$. 
	\task Calculer $A$ lorsque $r=10~\text{cm}$ et $\alpha=60^\circ$.
\end{tasks}
}
\end{minipage}
\hfill
\begin{minipage}[t]{0.40\textwidth}{
\vspace{0pt}
\includegraphics[scale=1]{../medias/1M/trigo/1M-exo-25}
}
\end{minipage}
}
\correction{
\tcblower
\textit{Generated by AI}

La surface hachurée est un secteur circulaire d'angle $\alpha$ et de rayon $r$.

\begin{tasks}(1)
\task \textbf{Expression de l'aire $A$ en fonction de $r$ et $\alpha$ :}

L'aire d'un disque complet de rayon $r$ est $\pi r^2$.

Un secteur circulaire d'angle $\alpha$ (en degrés) représente la fraction $\dfrac{\alpha}{360°}$ du disque complet.

Donc l'aire du secteur est :
\[A = \frac{\alpha}{360°} \times \pi r^2\]

Ou, si $\alpha$ est en radians, l'aire s'écrit :
\[A = \frac{1}{2}r^2 \alpha\]

\textbf{Réponse :} $\boxed{A = \dfrac{\alpha}{360} \times \pi r^2}$ (avec $\alpha$ en degrés) ou $\boxed{A = \dfrac{1}{2}r^2 \alpha}$ (avec $\alpha$ en radians)

\task \textbf{Calcul de $A$ lorsque $r=10~\text{cm}$ et $\alpha=60°$ :}

Nous utilisons la formule avec l'angle en degrés :
\[A = \frac{60}{360} \times \pi \times 10^2\]

Simplifions :
\[A = \frac{1}{6} \times \pi \times 100 = \frac{100\pi}{6} = \frac{50\pi}{3}~\text{cm}^2\]

Calculons la valeur numérique :
\[A = \frac{50\pi}{3} \approx \frac{50 \times 3{,}14159}{3} \approx \frac{157{,}08}{3} \approx 52{,}36~\text{cm}^2\]

\textbf{Réponse :} $\boxed{A = \dfrac{50\pi}{3}~\text{cm}^2 \approx 52{,}4~\text{cm}^2}$
\end{tasks}
}

