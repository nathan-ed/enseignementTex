\titre{}
\theme{dérivées}
\auteur{Nathan Scheinmann}
\niveau{3M}
\source{analysis}
\type{serie}
\piments{2}
\pts{}
\annee{2425}

\contenu{
\tcblower
Déterminer $A$ et $B$ pour que la courbe $y = Ax^{1/2} + Bx^{-1/2}$ ait un point d'inflexion en $(1, 4)$.
}
\correction{
\tcblower
\textit{Generated by AI}

Pour qu'un point soit un point d'inflexion, deux conditions doivent être satisfaites :
\begin{enumerate}
\item Le point appartient à la courbe
\item La dérivée seconde s'annule en ce point
\end{enumerate}

Posons $f(x) = Ax^{1/2} + Bx^{-1/2}$.

\textbf{Condition 1 : La courbe passe par $(1, 4)$}

\[
f(1) = A \cdot 1^{1/2} + B \cdot 1^{-1/2} = A + B = 4
\]

\textbf{Condition 2 : Point d'inflexion en $x = 1$}

Calculons les dérivées :
\[
f'(x) = \frac{A}{2}x^{-1/2} - \frac{B}{2}x^{-3/2} = \frac{A}{2\sqrt{x}} - \frac{B}{2x\sqrt{x}}
\]

\[
f''(x) = -\frac{A}{4}x^{-3/2} + \frac{3B}{4}x^{-5/2} = -\frac{A}{4x\sqrt{x}} + \frac{3B}{4x^2\sqrt{x}}
\]

Pour un point d'inflexion en $x = 1$ :
\[
f''(1) = 0 \quad \Rightarrow \quad -\frac{A}{4} + \frac{3B}{4} = 0 \quad \Rightarrow \quad 3B = A
\]

\textbf{Résolution du système :}

Nous avons :
\begin{align*}
A + B &= 4 \\
A - 3B &= 0
\end{align*}

De la seconde équation : $A = 3B$

Substituons dans la première :
\[
3B + B = 4 \quad \Rightarrow \quad 4B = 4 \quad \Rightarrow \quad B = 1
\]

Donc :
\[
A = 3B = 3
\]

\textbf{Réponse :} $A = 3$ et $B = 1$.
}

