\titre{}
\theme{limites}
\auteur{Nathan Scheinmann}
\niveau{3M}
\source{sesamath3e}
\type{serie}
\piments{2}
\pts{}
\annee{2425}

\contenu{
\tcblower
\noindent Déterminer pour chaque cas l’expression algébrique \(f(x)\) une fonction telle que :  
\begin{tasks}(3)
\task \(\displaystyle\lim_{x\to 2}f(x)\) n’existe pas
\task \(\displaystyle\lim_{x\to 2}f(x)=+\infty\)
\task \(\displaystyle\lim_{x\to 2}f(x)=-\infty\)
\end{tasks}
Justifier votre réponse. 
Y a-t-il plusieurs réponses possibles dans tous ces cas ?

}
\correction{
	\tcblower
	* Oui, il y a plusieurs cas possibles. 
	Sans justification.
\begin{tasks}(3)
	\task $f(x)=\dfrac{1}{x-2}$ 
	\task $f(x)=\dfrac{1}{(x-2)^2}$
	\task $f(x)=-\dfrac{1}{(x-2)^2}$
\end{tasks}
}

