\titre{}
\theme{geometrie}
\auteur{Nathan Scheinmann}
\niveau{1M}
\source{alloprof}
\type{serie}
\piments{2}
\pts{}
\annee{2425}

\contenu{
	\tcblower
 \begin{minipage}[t]{0.4\textwidth}{
		\vspace{0pt}
		Calculer la longueur de $\overline{BD}$ si $\overline{DC}=6\text{ cm}$ et $\overline{AD}=12 \text{ cm}$.	
		}
		\end{minipage}
		\hfill
\begin{minipage}[t]{0.4\textwidth}{
		\vspace{0pt}
\begin{tikzpicture}[scale=.75, rotate=35]
\tkzDefPoint(5,3.5){C}
\tkzDefPoint(0,0){A}
\tkzDefPoint(7,0){B}
\tkzDefPointBy[projection=onto A--B](C) \tkzGetPoint{D}
\tkzMarkRightAngle(B,C,A)
\tkzMarkRightAngle(A,D,C)
\tkzDrawPoints(A,B,C,D)
\tkzDrawSegment(C,D)
\tkzDrawPolygon(A,B,C)
\tkzLabelPoints[below right](B,A,D)
\tkzLabelPoints[above](C)
\end{tikzpicture}}
\end{minipage}

}
\correction{
	\tcblower
{\scriptsize \textit{Correction générée par IA}}

Dans le triangle rectangle $ABC$, avec l'angle droit en $C$, le point $D$ est le pied de la hauteur issue de $C$.

Nous utilisons les relations métriques dans un triangle rectangle. On sait que :
\[
BD^2 = DC \cdot DA
\]

On nous donne $DC = 6$ cm et $AD = 12$ cm.

Donc :
\[
BD^2 = 6 \times 12 = 72
\]

\[
BD = \sqrt{72} = \sqrt{36 \times 2} = 6\sqrt{2} \text{ cm}
\]

La longueur de $\overline{BD}$ est de $\textbf{6}\sqrt{\textbf{2}}$ cm, soit environ $8{,}49$ cm.
}

