\titre{}
\theme{derivation}
\auteur{Nathan Scheinmann}
\niveau{3M}
\source{sesamath3e}
\type{serie}
\piments{2}
\pts{}
\annee{2425}

\contenu{
\tcblower
Un fermier veut délimiter un champ rectangulaire avec une clôture longue de 400m. La pièce étant située le long d'une rivière, il suffit au fermier de poser la clôture sur 3 des 4 côtés. Calculer les dimensions du champ d'aire maximale.
}
\correction{
\tcblower
\textit{Generated by AI}

Notons $x$ la largeur du champ (parallèle à la rivière) et $y$ la profondeur (perpendiculaire à la rivière).

Puisque le côté le long de la rivière ne nécessite pas de clôture, le fermier utilise :
\[x + 2y = 400\]

D'où : $x = 400 - 2y$

L'aire du champ est :
\[A(y) = x \cdot y = (400 - 2y) \cdot y = 400y - 2y^2\]

Pour maximiser l'aire, calculons la dérivée et trouvons où elle s'annule :
\[A'(y) = 400 - 4y\]

\[A'(y) = 0 \implies 400 - 4y = 0 \implies y = 100\]

Vérifions qu'il s'agit bien d'un maximum :
\[A''(y) = -4 < 0\]

Donc $y = 100$ donne bien un maximum.

La largeur correspondante est :
\[x = 400 - 2(100) = 200\]

\textbf{Conclusion :} Les dimensions optimales sont \boxed{200~\text{m} \times 100~\text{m}} (largeur le long de la rivière : 200 m, profondeur : 100 m).

L'aire maximale est $A = 200 \times 100 = 20\,000~\text{m}^2$.
}

