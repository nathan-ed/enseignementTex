\titre{}
\theme{limites}
\auteur{Nathan Scheinmann}
\niveau{3M}
\source{sesamath3e}
\type{serie}
\piments{2}
\pts{}
\annee{2425}

\contenu{
\tcblower
\noindent Calculer les limites suivantes en justifiant chaque étape.  
\begin{tasks}(3)
\task \(\displaystyle\lim_{x\to -2}\frac{-2}{2x+4}\)
\task \(\displaystyle\lim_{x\to 1}\frac{x-2}{x^2-2x+1}\)
\task \(\displaystyle\lim_{x\to 2}\frac{3x-7}{x-2}\)
\task \(\displaystyle\lim_{x\to -2}\frac{-3x+2}{x+2}\)
\task \(\displaystyle\lim_{x\to 1}\frac{3x}{1-x^2}\)
\task \(\displaystyle\lim_{x\to -3}\frac{5x}{x+3}\)
\end{tasks}
}
\correction{
	\tcblower

\begin{tasks}(2)
	\task 
Type \(« \dfrac{1}{0} »\) : 

	\(\begin{aligned}
		\displaystyle \lim_{x \to -2} \dfrac{-2}{2x + 4}
  &\stackrel{\text{P4}}{=} \dfrac{-2}{\lim_{x \to -2} (2x + 4)}\\
&\stackrel{\text{cor. \ref{thm:poly}}}{=} \dfrac{-2}{2\cdot(-2) + 4}\\
&= \dfrac{-2}{-4 + 4} = «\dfrac{-2}{0}»
	\end{aligned}\)  

 $\displaystyle \lim_{x \to -2^-} \dfrac{-2}{2x + 4} = «\dfrac{-2}{0^-}» = +\infty,$

 $\displaystyle \lim_{x \to -2^+} \dfrac{-2}{2x+4}=«\dfrac{-2}{0^+}» = -\infty$
donc la limite n’existe pas.


  \task Même justification que ci-dessus : 

  \( \displaystyle \lim_{x \to 1} \dfrac{x - 2}{x^2 - 2x +1}
  = \dfrac{1 - 2}{1^2 - 2\cdot1 + 1}
  =« \dfrac{-1}{0} »\)

  Type \(« \dfrac{1}{0} »\) :   on factorise : \( x^2 + 2x - 1 = (x - 1)^2 \)  

  $\displaystyle \lim_{x \to 1^-} \dfrac{x - 2}{(x - 1)^2} = «\dfrac{-1}{(0^-)^2}»=«\dfrac{-1}{0^+}» = -\infty,$

  $\displaystyle \lim_{x \to 1^+} \dfrac{x - 2}{(x - 1)^2}= «\dfrac{-1}{(0^+)}» = -\infty$

  donc \( \lim_{x \to 1} f(x) = -\infty \)

  \task Type \(« \dfrac{1}{0} »\) :\( \displaystyle \lim_{x \to 2} \dfrac{3x - 7}{x - 2}
  = «\dfrac{6 - 7}{0}» = «\dfrac{-1}{0}» \)
  
  $
  \displaystyle \lim_{x \to 2^-} \dfrac{3x - 7}{x - 2}= «\dfrac{-1}{0^-}» = +\infty, $

  $\displaystyle \lim_{x \to 2^+} \dfrac{3x - 7}{x - 2}= «\dfrac{-1}{0^+}» = -\infty$

  donc la limite n’existe pas.

  \task   Type \(« \dfrac{1}{0} »\) : \( \displaystyle \lim_{x \to -2} \dfrac{-3x + 2}{x + 2}
  = \dfrac{6 + 2}{0} = «\dfrac{8}{0}» \)

  $
   \displaystyle \lim_{x \to -2^-} \dfrac{-3x + 2}{x + 2}
= «\dfrac{8}{0^-}» = -\infty,$

$\displaystyle \lim_{x \to -2^+} \dfrac{-3x + 2}{x + 2}
= «\dfrac{8}{0^+}» = +\infty
$

  donc la limite n’existe pas.

  \task Type \(« \dfrac{1}{0} »\) :\( \displaystyle \lim_{x \to 1} \dfrac{3x}{1 - x^2}
  = \dfrac{3}{1 - 1} = «\dfrac{3}{0}» \)

  \( 1 - x^2 = -(x - 1)(x + 1) \), donc :
 
  $
   \displaystyle \lim_{x \to 1^-} \dfrac{3x}{1 - x^2} = «\dfrac{3}{-(0^-)(2)}» = -\infty,$
  
   $\displaystyle \lim_{x \to 1^+}  \dfrac{3x}{1 - x^2}= «\dfrac{3}{-(0^+)(2)}» = +\infty$

  donc la limite n'existe pas. 

  \task Type \(« \dfrac{1}{0} »\) :\( \displaystyle \lim_{x \to -3} \dfrac{5x}{x + 3} = «\dfrac{-15}{0}» \)

$  \displaystyle \lim_{x \to -3^-}\dfrac{5x}{x + 3}= «\dfrac{-15}{0^-} »= +\infty,$

 $\displaystyle \lim_{x \to -3^+} \dfrac{5x}{x + 3}= «\dfrac{-15}{0^+}» = -\infty$

  donc la limite n’existe pas.
\end{tasks}
}

