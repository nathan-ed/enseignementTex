\titre{}
\theme{limites}
\auteur{Nathan Scheinmann}
\niveau{1M}
\source{co}
\type{serie}
\piments{1}
\pts{}
\annee{2425}

\contenu{
\tcblower
Soit $P$ un polynôme de degré $3$ à coefficients réels. Montrer que le poylnôme $P$ possède au moins une racine réelle. 

{\itshape On admet pour cet exercice que les fonctions définies par des polynômes sont continues.}
}
\correction{
\tcblower
On écrit $P(x)=ax^3+bx^2+cx+d$ avec $a,b,c,d\in \mathbb{R}$ et $a\neq 0$. On a que pour un nombre $n$ assez petit, $\text{signe}(P(n))=-\text{signe}(a)$ et que pour un nombre $m$ assez grand, $\text{signe}(P(m))=\text{signe}(a)$. La fonction $x\mapsto P(x)$ est continue sur tout $\mathbb{R}$, donc aussi sur $\interval{n}{m}$. Ainsi, par la corollaire du théorème de la valeur intermédiaire, il existe $r\in\interval{n}{m}$ tel que $P(r)=0$. 
}

