\titre{}
\theme{equations}
\auteur{Nathan Scheinmann}
\niveau{1M}
\source{MER11e}
\type{serie}
\piments{2}
\pts{}
\annee{2425}

\contenu{
	\tcblower
Un confiseur répartit des truffes dans des cornets de $200$ g. S'il avait réparti ses truffes dans des cornets de $150$ g, il y aurait eu $12$ cornets de plus. 

Quelle quantité de truffes a-t-il préparée?
}
\correction{
	\tcblower
On pose les inconnues
\begin{align*}
	&x=\text{Le nombre de sachet de 200g} &&y=\text{Le nombre de sachet de 150g}
\end{align*}
On obtient le système
$
\begin{cases}
y-x=12\\
200x=150y
\end{cases}
$
On résout le système (par exemple par substitution) et on obtient que $x=36$ et $y=48$. Le confiseur a donc préparé $36\cdot 200=7,2$kg de truffes.
}

