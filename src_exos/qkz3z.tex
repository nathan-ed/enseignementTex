\titre{}
\theme{calcLit}
\auteur{Nathan Scheinmann}
\niveau{1M}
\source{musy}
\type{serie}
\piments{2}
\pts{}
\annee{2425}

\contenu{
\tcblower
Connaître par coeur et savoir démontrer les identités suivantes :
\begin{tasks}(2)
\task $(a+b)^2=a^2+2 a b+b^2$
\task $(a-b)^2=a^2-2 a b+b^2$
\task $(a+b)(a-b)=a^2-b^2$
\task* $(x+a)(x+b)=x^2+(a+b) x+a b$
\end{tasks}
}
\correction{
\tcblower
{\scriptsize \textit{Correction générée par IA}}

\begin{tasks}
\task $(a+b)^2=a^2+2ab+b^2$

\textbf{Démonstration :}

\( \begin{aligned}
(a+b)^2 &= (a+b)(a+b) \\
&= a \cdot a + a \cdot b + b \cdot a + b \cdot b \\
&= a^2 + ab + ab + b^2 \\
&= a^2 + 2ab + b^2
\end{aligned} \)

\task $(a-b)^2=a^2-2ab+b^2$

\textbf{Démonstration :}

\( \begin{aligned}
(a-b)^2 &= (a-b)(a-b) \\
&= a \cdot a + a \cdot (-b) + (-b) \cdot a + (-b) \cdot (-b) \\
&= a^2 - ab - ab + b^2 \\
&= a^2 - 2ab + b^2
\end{aligned} \)

\task $(a+b)(a-b)=a^2-b^2$

\textbf{Démonstration :}

\( \begin{aligned}
(a+b)(a-b) &= a \cdot a + a \cdot (-b) + b \cdot a + b \cdot (-b) \\
&= a^2 - ab + ab - b^2 \\
&= a^2 - b^2
\end{aligned} \)

\task $(x+a)(x+b)=x^2+(a+b)x+ab$

\textbf{Démonstration :}

\( \begin{aligned}
(x+a)(x+b) &= x \cdot x + x \cdot b + a \cdot x + a \cdot b \\
&= x^2 + bx + ax + ab \\
&= x^2 + (a+b)x + ab
\end{aligned} \)
\end{tasks}
}

