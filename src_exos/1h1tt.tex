\titre{}
\theme{derivation}
\auteur{Nathan Scheinmann}
\niveau{3M}
\source{sesamath3e}
\type{serie}
\piments{2}
\pts{}
\annee{2425}

\contenu{
\tcblower
Représenter graphiquement trois fonctions de votre choix qui soit croissantes sur $\displaystyle\left]-\infty ; -2\right] \cup \left[1 ; 3\right] \cup \left[5 ; 8\right]$ et décroissantes sur $\displaystyle\left[-2 ; 1\right] \cup \left[3 ; 5\right] \cup \left[8 ; +\infty\right[$.
}
\correction{
\tcblower

\textit{Generated by AI}

Voici trois exemples de fonctions satisfaisant les conditions demandées :

\textbf{Fonction 1 : Fonction polynomiale par morceaux}

\begin{center}
\begin{tikzpicture}[scale=0.6]
\draw[->] (-3,0) -- (9,0) node[right] {$x$};
\draw[->] (0,-2) -- (0,6) node[above] {$y$};
\draw[thick, blue, domain=-3:-2] plot (\x, {0.5*(\x+3)+1});
\draw[thick, blue, domain=-2:1] plot (\x, {-0.3*(\x+2)^2+1.5});
\draw[thick, blue, domain=1:3] plot (\x, {0.4*(\x-1)+0.9});
\draw[thick, blue, domain=3:5] plot (\x, {-0.3*(\x-3)^2+1.7});
\draw[thick, blue, domain=5:8] plot (\x, {0.3*(\x-5)+1.1});
\draw[thick, blue, domain=8:9] plot (\x, {-0.4*(\x-8)+2});
\foreach \x in {-2,1,3,5,8}
  \draw (\x,0.1) -- (\x,-0.1) node[below] {\small $\x$};
\end{tikzpicture}
\end{center}

\textbf{Fonction 2 : Fonction sinusoïdale modifiée}

\begin{center}
\begin{tikzpicture}[scale=0.6]
\draw[->] (-3,0) -- (9,0) node[right] {$x$};
\draw[->] (0,-2) -- (0,4) node[above] {$y$};
\draw[thick, red, domain=-3:-2, samples=50] plot (\x, {1.2*sin((\x+3)*90)+1});
\draw[thick, red, domain=-2:1, samples=50] plot (\x, {-0.5*(\x+0.5)^2+2.5});
\draw[thick, red, domain=1:3, samples=50] plot (\x, {0.6*(\x-1)+0.8});
\draw[thick, red, domain=3:5, samples=50] plot (\x, {-0.4*(\x-3)^2+2});
\draw[thick, red, domain=5:8, samples=50] plot (\x, {0.3*(\x-5)+0.8});
\draw[thick, red, domain=8:9, samples=50] plot (\x, {-0.5*(\x-8)+1.7});
\foreach \x in {-2,1,3,5,8}
  \draw (\x,0.1) -- (\x,-0.1) node[below] {\small $\x$};
\end{tikzpicture}
\end{center}

\textbf{Fonction 3 : Fonction avec extrema locaux}

\begin{center}
\begin{tikzpicture}[scale=0.6]
\draw[->] (-3,0) -- (9,0) node[right] {$x$};
\draw[->] (0,-1) -- (0,5) node[above] {$y$};
\draw[thick, green!60!black, domain=-3:-2, samples=50] plot (\x, {0.4*(\x+2.5)^2+0.5});
\draw[thick, green!60!black, domain=-2:1, samples=50] plot (\x, {-0.25*(\x+0.5)^2+2.2});
\draw[thick, green!60!black, domain=1:3, samples=50] plot (\x, {0.5*(\x-1)+0.8});
\draw[thick, green!60!black, domain=3:5, samples=50] plot (\x, {-0.35*(\x-4)^2+2.3});
\draw[thick, green!60!black, domain=5:8, samples=50] plot (\x, {0.35*(\x-6)^2+1});
\draw[thick, green!60!black, domain=8:9, samples=50] plot (\x, {-0.6*(\x-8)+4.1});
\foreach \x in {-2,1,3,5,8}
  \draw (\x,0.1) -- (\x,-0.1) node[below] {\small $\x$};
\end{tikzpicture}
\end{center}

\textbf{Remarques :}
\begin{itemize}
\item Les fonctions sont croissantes sur $]-\infty ; -2]$, $[1 ; 3]$ et $[5 ; 8]$ (les courbes montent).
\item Les fonctions sont décroissantes sur $[-2 ; 1]$, $[3 ; 5]$ et $[8 ; +\infty[$ (les courbes descendent).
\item Les extrema locaux se situent aux points de changement de monotonie : $x = -2, 1, 3, 5, 8$.
\end{itemize}

}

