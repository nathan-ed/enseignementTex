\titre{}
\theme{dérivées}
\auteur{Nathan Scheinmann}
\niveau{3M}
\source{fundamentum}
\type{serie}
\piments{2}
\pts{}
\annee{2425}

\contenu{
\tcblower
Un fil de longueur $L$ doit être coupé en deux parties. Avec l'une on forme un triangle équilatéral et avec l'autre un carré. Où faut-il couper ce fil pour que l'aire totale des deux figures construites soit maximale ? minimale ?
}
\correction{
\tcblower
}

