\titre{}
\theme{derivation}
\auteur{Nathan Scheinmann}
\niveau{3M}
\source{sesamath3e}
\type{serie}
\piments{2}
\pts{}
\annee{2425}

\contenu{
\tcblower
Étudier les fonctions ci-dessous :

\begin{tasks}(2)
\task $f(x) = 3x^5 - 5x^3$
\task $f(x) = \dfrac{2x^3 + 2x - 12}{-x^2 + 3x - 2}$
\task $f(x) = \dfrac{x^2 - 4x + 6}{x^2 - 4x + 4}$
\task $f(x) = \dfrac{x^2 - 6x + x^3}{4 - x^2}$
\task $f(x) = \dfrac{x^2 + 2x + 2}{x - 1}$
\end{tasks}
}
\correction{
\tcblower
{\scriptsize \textit{Correction générée par IA}}

\begin{tasks}(1)
\task $f(x) = 3x^5 - 5x^3$

\textbf{Domaine~:} $D_f = \mathbb{R}$

\textbf{Dérivée~:} $f'(x) = 15x^4 - 15x^2 = 15x^2(x^2-1) = 15x^2(x-1)(x+1)$

\textbf{Zéros de $f'$~:} $x = -1$, $x = 0$, $x = 1$

\textbf{Tableau de signes de $f'$ et variations de $f$~:}
\begin{center}
\begin{tabular}{c|ccccccccc}
$x$ & $-\infty$ && $-1$ && $0$ && $1$ && $+\infty$ \\
\hline
$f'(x)$ && $-$ & $0$ & $+$ & $0$ & $+$ & $0$ & $-$ & \\
\hline
$f(x)$ && $\searrow$ & $-2$ & $\nearrow$ & $0$ & $\nearrow$ & $2$ & $\searrow$ &
\end{tabular}
\end{center}

\task $f(x) = \dfrac{2x^3 + 2x - 12}{-x^2 + 3x - 2}$

\textbf{Domaine~:} On factorise le dénominateur~: $-x^2 + 3x - 2 = -(x^2 - 3x + 2) = -(x-1)(x-2)$.

Donc $D_f = \mathbb{R} \setminus \{1, 2\}$

\textbf{Dérivée~:} En utilisant la formule $(u/v)' = (u'v - uv')/v^2$~:

$f'(x) = \dfrac{(6x^2+2)(-x^2+3x-2) - (2x^3+2x-12)(-2x+3)}{(-x^2+3x-2)^2}$

Après simplification, on obtient une expression rationnelle dont le signe doit être étudié sur chaque intervalle de $D_f$.

\task $f(x) = \dfrac{x^2 - 4x + 6}{x^2 - 4x + 4}$

\textbf{Domaine~:} Le dénominateur est $(x-2)^2$, donc $D_f = \mathbb{R} \setminus \{2\}$

\textbf{Dérivée~:}
\( \begin{aligned}
f'(x) &= \dfrac{(2x-4)(x^2-4x+4) - (x^2-4x+6)(2x-4)}{(x^2-4x+4)^2} \\
&= \dfrac{(2x-4)[(x^2-4x+4) - (x^2-4x+6)]}{(x-2)^4} \\
&= \dfrac{(2x-4)(-2)}{(x-2)^4} \\
&= \dfrac{-4(x-2)}{(x-2)^4} \\
&= \dfrac{-4}{(x-2)^3}
\end{aligned} \)

$f'(x) < 0$ pour $x > 2$ et $f'(x) > 0$ pour $x < 2$. La fonction est croissante sur $]-\infty, 2[$ et décroissante sur $]2, +\infty[$.

\task $f(x) = \dfrac{x^2 - 6x + x^3}{4 - x^2}$

\textbf{Domaine~:} $4 - x^2 = 0 \Leftrightarrow x = \pm 2$, donc $D_f = \mathbb{R} \setminus \{-2, 2\}$

\textbf{Dérivée~:} En appliquant la formule de dérivation du quotient et en simplifiant, on obtient~:

$f'(x) = \dfrac{(3x^2+2x-6)(4-x^2) - (x^3+x^2-6x)(-2x)}{(4-x^2)^2}$

Le signe de $f'$ doit être étudié sur chaque intervalle de $D_f$.

\task $f(x) = \dfrac{x^2 + 2x + 2}{x - 1}$

\textbf{Domaine~:} $D_f = \mathbb{R} \setminus \{1\}$

\textbf{Dérivée~:}
\( \begin{aligned}
f'(x) &= \dfrac{(2x+2)(x-1) - (x^2+2x+2)(1)}{(x-1)^2} \\
&= \dfrac{2x^2-2x+2x-2 - x^2-2x-2}{(x-1)^2} \\
&= \dfrac{x^2-2x-4}{(x-1)^2}
\end{aligned} \)

Le numérateur s'annule pour $x = \dfrac{2 \pm \sqrt{4+16}}{2} = 1 \pm \sqrt{5}$.

$f'(x) > 0$ pour $x < 1-\sqrt{5}$ ou $x > 1+\sqrt{5}$, et $f'(x) < 0$ pour $1-\sqrt{5} < x < 1$ ou $1 < x < 1+\sqrt{5}$.
\end{tasks}
}

