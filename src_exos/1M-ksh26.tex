\titre{}
\theme{racines}
\auteur{Nathan Scheinmann}
\niveau{1M}
\source{sophie}
\type{serie}
\piments{2}
\pts{}
\annee{2425}

\contenu{
\tcblower
Montrer que $a$ est égal à $b$ dans les cas suivants~:
\begin{tasks}(2)
\task $a = \dfrac{1}{\sqrt{2}}$ ; $b = \dfrac{\sqrt{2}}{2}$
\task $a = \dfrac{1}{\sqrt{27}}$ ; $b = \dfrac{\sqrt{3}}{9}$
\task $a = \dfrac{2 + \sqrt{8}}{2}$ ; $b = 1 + \sqrt{2}$
\task $a = \dfrac{1}{\sqrt{5} - \sqrt{2}}$ ; $b = \dfrac{\sqrt{5} + \sqrt{2}}{3}$
\end{tasks}
}
\correction{
\tcblower
On utilise la multiplication par l'expression conjuguée et les propriétés des racines. 
}

