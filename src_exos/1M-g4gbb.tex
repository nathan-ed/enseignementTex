\titre{}
\theme{limites}
\auteur{Nathan Scheinmann}
\niveau{3M}
\source{sesamath3e}
\type{serie}
\piments{2}
\pts{}
\annee{2425}

\contenu{
\tcblower
\noindent Esquisser une fonction \(f\) qui vérifie chaque fois les conditions données :  
\begin{tasks}
	\task \(\displaystyle\lim_{x\to 3} f(x)\to 2\) et \(f(3)=2\).
	\task \(\displaystyle\lim_{x\to 3} f(x)\to 2\) et \(f(3)=1\).
	\task \(\displaystyle\lim_{x\to 3^+} f(x)=2\), \(\displaystyle\lim_{x\to 3^-} f(x)=4\) et \(f(3)=2\).
\task \(\displaystyle\lim_{x\to 3^+} f(x)=2\), \(\displaystyle\lim_{x\to 3^-} f(x)= 4\) et \(f(3)=4\).
\task \(\displaystyle\lim_{x\to 3^+} f(x)=2\), \(\displaystyle\lim_{x\to 3^-} f(x)=4\) et \(f(3)\neq 4\).
\task \(\displaystyle\lim_{x\to 3} f(x)\) n'existe pas et \(f(3)=1\).
\task \(\displaystyle\lim_{x\to 2} f(x)=3\) et \(2\notin D_{f}\).
\task \(\displaystyle\lim_{x\to3^-} f(x)=-1\), \(\displaystyle\lim_{x\to 3^+} f(x)=1\) et \(f(3)=2\).
\end{tasks}
}
\correction{
	\tcblower
Vérifier la réponse avec l'enseignant.
}

