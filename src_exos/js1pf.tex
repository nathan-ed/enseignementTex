\titre{}
\theme{dérivées}
\auteur{Nathan Scheinmann}
\niveau{3M}
\source{fundamentum}
\type{serie}
\piments{2}
\pts{}
\annee{2425}

\contenu{
\tcblower
On construit un conteneur de forme cylindrique sans couvercle de volume $648\,\pi$ cm$^3$. Le matériau utilisé pour le fond coûte 15 centimes par cm$^2$ et celui utilisé pour la paroi latérale 5 centimes par cm$^2$. Si la fabrication ne donne lieu à aucun déchet, quelles sont les dimensions du conteneur le plus économique ?
}
\correction{

\tcblower
{\scriptsize \textit{Correction générée par IA}}

Soit $r$ le rayon de la base du cylindre et $h$ sa hauteur. Le volume est $V = \pi r^2 h = 648\pi$, 
\[h = \dfrac{648}{r^2}\]

Le coût total de fabrication est :
\[C = 15 \cdot \pi r^2 + 5 \cdot 2\pi r h = 15\pi r^2 + 10\pi r h\]

Substituons $h = \dfrac{648}{r^2}$ :
\[C(r) = 15\pi r^2 + 10\pi r \cdot \dfrac{648}{r^2} = 15\pi r^2 + \dfrac{6480\pi}{r}\]

Pour minimiser le coût, dérivons :
\[C'(r) = 30\pi r - \dfrac{6480\pi}{r^2}\]

Annulation :
\[30\pi r - \dfrac{6480\pi}{r^2} = 0 \implies 30r = \dfrac{6480}{r^2} \implies 30r^3 = 6480 \implies r^3 = 216 \implies r = 6 \text{ cm}\]

\[h = \dfrac{648}{6^2} = \dfrac{648}{36} = 18 \text{ cm}\]

Les dimensions du conteneur le plus économique sont :
\[\boxed{\text{Rayon} = 6 \text{ cm}, \quad \text{Hauteur} = 18 \text{ cm}}\]

}

