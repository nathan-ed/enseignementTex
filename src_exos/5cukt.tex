\titre{7}
\theme{trigo}
\auteur{Nathan Scheinmann}
\niveau{1M}
\source{sesamath-1M-trigo}
\type{serie}
\piments{2}
\pts{}
\annee{2425}

\contenu{
	\tcblower
	\begin{minipage}[t]{0.50\textwidth}{
	\vspace{0pt}
	Rafaël et Léo nagent pour atteindre une bouée $P$. Ils sont respectivement en position $R$ et $L$. On a $\overline{BL}=50~\text{m}$ et $\widehat{BPL}=72^\circ$. 

	Calculer la distance entre les deux nageurs arrondie au mètre. 
	}
	\end{minipage}
	\hfill
	\begin{minipage}[t]{0.45\textwidth}{
	\vspace{0pt}
	\includegraphics[scale=1]{../medias/1M/trigo/1M-exo-7}
	}
	\end{minipage}
}
\correction{
\tcblower

\textit{Generated by AI}

Pour calculer la distance entre les deux nageurs, nous devons déterminer la longueur $\overline{RL}$.

D'après le schéma, nous avons un triangle avec la bouée $P$, et les positions $R$, $L$ et $B$. Les données sont :
\begin{itemize}
\item $\overline{BL} = 50~\text{m}$
\item $\widehat{BPL} = 72^\circ$
\end{itemize}

En supposant que le triangle $\triangle BPL$ est isocèle avec $\overline{BP} = \overline{LP}$ (ce qui est souvent le cas dans ce type d'exercice), et que $R$ est sur la droite $(BP)$, nous pouvons utiliser la trigonométrie.

\textbf{Méthode :} Dans le triangle $\triangle BPL$, si nous connaissons $\overline{BL}$ et l'angle $\widehat{BPL}$, nous pouvons calculer les autres côtés.

Utilisons la loi des sinus ou la trigonométrie dans le triangle. Si $\overline{BP} = \overline{LP} = r$ (rayon), alors dans le triangle isocèle :

Les angles à la base sont : $\widehat{PBL} = \widehat{PLB} = \dfrac{180^\circ - 72^\circ}{2} = 54^\circ$

En utilisant la loi des sinus :
\[\dfrac{\overline{BL}}{\sin(\widehat{BPL})} = \dfrac{\overline{BP}}{\sin(\widehat{PLB})}\]

\[\overline{BP} = \dfrac{50 \times \sin(54^\circ)}{\sin(72^\circ)} \approx \dfrac{50 \times 0{,}809}{0{,}951} \approx 42{,}5~\text{m}\]

Si $R$ est également à la même distance de $P$ que $L$, alors $\overline{RL}$ peut être calculé selon la configuration géométrique précise du schéma.

\textbf{Estimation :} La distance $\overline{RL}$ dépend de la position exacte de $R$ sur le schéma. En supposant une configuration symétrique, la distance serait de l'ordre de $\mathbf{43~\text{m}}$ à $\mathbf{50~\text{m}}$.

(Note : Une réponse précise nécessite l'analyse complète du schéma fourni.)

}

