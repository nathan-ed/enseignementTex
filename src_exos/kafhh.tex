\titre{}
\theme{dérivées}
\auteur{Nathan Scheinmann}
\niveau{3M}
\source{nathan}
\type{serie}
\piments{2}
\pts{}
\annee{2425}

\contenu{
\tcblower
Déterminer si la fonction \(f(x) = (1 - x^2)^{1/3} + x^2\) satisfait les conditions du théorème de Rolle sur l'intervalle \([-1,1]\). Si oui, trouver les valeurs de $c$ prédites par le théorème.
}
\correction{
\tcblower
\textit{Generated by AI}

Le théorème de Rolle stipule que si une fonction $f$ est :
\begin{itemize}
\item continue sur $[a;b]$
\item dérivable sur $]a;b[$
\item $f(a) = f(b)$
\end{itemize}
alors il existe au moins un point $c \in ]a;b[$ tel que $f'(c) = 0$.

\textbf{Vérification des conditions :}

Pour $f(x) = (1 - x^2)^{1/3} + x^2$ sur $[-1;1]$ :

\textbf{1) Continuité sur $[-1;1]$ :}

La fonction $(1 - x^2)^{1/3}$ est continue sur $[-1;1]$ car $1 - x^2 \geq 0$ sur cet intervalle (avec $1 - x^2 = 0$ aux extrémités).

La fonction $x^2$ est continue partout.

Donc $f$ est continue sur $[-1;1]$.

\textbf{2) Dérivabilité sur $]-1;1[$ :}

Calculons la dérivée :
\[
f'(x) = \frac{1}{3}(1 - x^2)^{-2/3} \cdot (-2x) + 2x = -\frac{2x}{3(1 - x^2)^{2/3}} + 2x
\]

Cette dérivée existe pour $x \in ]-1;1[$ (car $1 - x^2 > 0$ sur cet intervalle).

\textbf{3) Égalité aux extrémités :}

\begin{align*}
f(-1) &= (1 - 1)^{1/3} + 1 = 0 + 1 = 1 \\
f(1) &= (1 - 1)^{1/3} + 1 = 0 + 1 = 1
\end{align*}

Donc $f(-1) = f(1) = 1$.

\textbf{Conclusion :} Les trois conditions du théorème de Rolle sont satisfaites.

\textbf{Recherche des valeurs de $c$ :}

Résolvons $f'(c) = 0$ :
\[
-\frac{2c}{3(1 - c^2)^{2/3}} + 2c = 0
\]
\[
2c \left(1 - \frac{1}{3(1 - c^2)^{2/3}}\right) = 0
\]

Soit $c = 0$, soit $1 - \frac{1}{3(1 - c^2)^{2/3}} = 0$, ce qui donne :
\[
3(1 - c^2)^{2/3} = 1 \quad \Rightarrow \quad (1 - c^2)^{2/3} = \frac{1}{3}
\]
\[
1 - c^2 = \left(\frac{1}{3}\right)^{3/2} \quad \Rightarrow \quad c^2 = 1 - \frac{1}{3\sqrt{3}}
\]

\textbf{Réponse :} La valeur principale prédite par le théorème de Rolle est $c = 0$.
}

