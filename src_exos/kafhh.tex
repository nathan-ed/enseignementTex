\titre{}
\theme{dérivées}
\auteur{Nathan Scheinmann}
\niveau{3M}
\source{nathan}
\type{serie}
\piments{2}
\pts{}
\annee{2425}

\contenu{
\tcblower
Déterminer si la fonction \(f(x) = (1 - x^2)^{1/3} + x^2\) satisfait les conditions du théorème de Rolle sur l'intervalle \([-1;1]\). Si oui, trouver les valeurs de $c$ prédites par le théorème.
}
\correction{
\tcblower
$f(x) = (1-x^2)^{1/3} + x^2$ sur $[-1; 1]$.
$f$ est continue sur $[-1, 1]$ et dérivable sur $]-1, 1[$.
$f(-1) = 1$ et $f(1) = 1$. Les conditions de Rolle sont satisfaites.
$f'(x) = \dfrac{-2x}{3(1-x^2)^{2/3}} + 2x = 0 \iff 2x \left( 1 - \dfrac{1}{3(1-x^2)^{2/3}} \right) = 0$.
Solutions : $c_1 = 0$, $c_2 = \sqrt{1 - (1/3\sqrt{3})} \approx 0,903$ et $c_3 \approx -0,903$.
}

