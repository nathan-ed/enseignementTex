\titre{}
\theme{dérivées}
\auteur{Nathan Scheinmann}
\niveau{3M}
\source{fundamentum}
\type{serie}
\piments{2}
\pts{}
\annee{2425}

\contenu{
\tcblower
Parmi tous les rectangles de périmètre donné $2p$, quel est celui dont l'aire est maximale ? Quelle est la valeur de cette aire ?
}
\correction{

\tcblower
{\scriptsize \textit{Correction générée par IA}}

Soit $a$ et $b$ les dimensions du rectangle avec $a + b = p$ (demi-périmètre).

L'aire du rectangle est :
\[A = ab = a(p - a) = ap - a^2\]

Pour maximiser l'aire, dérivons :
\[A'(a) = p - 2a\]

Annulation : $p - 2a = 0 \implies a = \dfrac{p}{2}$, donc $b = \dfrac{p}{2}$.

Le rectangle d'aire maximale est donc un carré de côté $\boxed{a = b = \dfrac{p}{2}}$.

L'aire maximale est :
\[\boxed{A_{\max} = \dfrac{p}{2} \cdot \dfrac{p}{2} = \dfrac{p^2}{4}}\]

}

