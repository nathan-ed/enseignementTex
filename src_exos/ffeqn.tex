\titre{}
\theme{fonctions}
\auteur{Nathan Scheinmann}
\niveau{1M}
\source{ns}
\type{serie}
\piments{2}
\pts{}
\annee{2425}

\contenu{
	\tcblower
\begin{enumerate}
            \item Parmi les fonctions suivantes, lesquelles sont des fonctions constantes, linéaires ou affines?
	\begin{alignat*}{3}
		 &f_1:x\mapsto 1& \quad\quad\quad\quad&f_2:x\mapsto 3x+2^2& \quad\quad\quad\quad&f_3:x\mapsto 1-x^2\\
&f_4:x\mapsto -x& &f_5:x\mapsto \dfrac{x-3}{4}& &f_6:x\mapsto x^2-(1-x)^2
	 \end{alignat*}
 \item Donner le coefficient de l'ordonnée à l'origine (o.o.) et le coefficient de la pente pour celles qui sont affines.
 \item Représenter les fonctions affines dans un repère orthonormé.
        \end{enumerate}

}
\correction{
	\tcblower
	\begin{tasks}(2)
		\task[] $f_1:$ constante, o.o. $1$ et pente $0$
		\task[] $f_2:$ affine, o.o. $4$ et pente $3$
		\task[] $f_3:$ pas affine.
		\task[] $f_4:$ linéaire, o.o. $0$ et pente $-1$
		\task[] $f_5:$ affine, o.o. $-\dfrac{3}{4}$ et pente $\dfrac{1}{4}$
		\task[] $f_6:$ affine, o.o. $2$ et pente $-1$
	\end{tasks}
	Pour la représentation, utiliser un grapheur.
}

