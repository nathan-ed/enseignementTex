\titre{}
\theme{derivation}
\auteur{Nathan Scheinmann}
\niveau{3M}
\source{sesamath3e}
\type{serie}
\piments{2}
\pts{}
\annee{2425}

\contenu{
\tcblower
Étudier la fonction suivante, y compris la deuxième dérivée :

\begin{tasks}(2)
\task $f(x) = \dfrac{x^2 - 4x + 6}{(x - 2)^2}$
\task $f(x) = \dfrac{x^2 + 2x + 2}{x - 1}$
\task $f(x) = \dfrac{x^2}{x} + \dfrac{1}{x}$
\end{tasks}
}
\correction{
\tcblower
\textit{Generated by AI}

\textbf{Question a) : $f(x) = \dfrac{x^2 - 4x + 6}{(x - 2)^2}$}

\textit{Domaine :} $\mathbb{R} \setminus \{2\}$

\textit{Dérivée première :}

En utilisant la règle du quotient :
\begin{align*}
f'(x) &= \frac{(2x - 4)(x - 2)^2 - (x^2 - 4x + 6) \cdot 2(x - 2)}{(x - 2)^4} \\
&= \frac{(2x - 4)(x - 2) - 2(x^2 - 4x + 6)}{(x - 2)^3} \\
&= \frac{2x^2 - 8x + 8 - 2x^2 + 8x - 12}{(x - 2)^3} \\
&= \frac{-4}{(x - 2)^3}
\end{align*}

$f'(x) < 0$ pour $x < 2$ et $f'(x) > 0$ pour $x > 2$, donc pas de point critique (asymptote verticale en $x = 2$).

\textit{Dérivée seconde :}
\[
f''(x) = \frac{12}{(x - 2)^4} > 0
\]

La fonction est concave vers le haut partout où elle est définie.

\textbf{Question b) : $f(x) = \dfrac{x^2 + 2x + 2}{x - 1}$}

\textit{Domaine :} $\mathbb{R} \setminus \{1\}$

\textit{Dérivée première :}
\begin{align*}
f'(x) &= \frac{(2x + 2)(x - 1) - (x^2 + 2x + 2)}{(x - 1)^2} \\
&= \frac{2x^2 - 2 - x^2 - 2x - 2}{(x - 1)^2} \\
&= \frac{x^2 - 2x - 4}{(x - 1)^2}
\end{align*}

Points critiques : $x = \frac{2 \pm \sqrt{4 + 16}}{2} = 1 \pm \sqrt{5}$

\textit{Dérivée seconde :}
\[
f''(x) = \frac{d}{dx}\left[\frac{x^2 - 2x - 4}{(x - 1)^2}\right]
\]

\textbf{Question c) : $f(x) = \dfrac{x^2}{x} + \dfrac{1}{x} = x + \dfrac{1}{x}$}

\textit{Domaine :} $\mathbb{R}^*$

\textit{Dérivée première :}
\[
f'(x) = 1 - \frac{1}{x^2} = \frac{x^2 - 1}{x^2}
\]

Points critiques : $x = \pm 1$

\textit{Dérivée seconde :}
\[
f''(x) = \frac{2}{x^3}
\]

Pour $x = 1$ : $f''(1) = 2 > 0$ donc minimum

Pour $x = -1$ : $f''(-1) = -2 < 0$ donc maximum
}

