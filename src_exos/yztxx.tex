\titre{24}
\theme{trigo}
\auteur{Nathan Scheinmann}
\niveau{1M}
\source{sesamath-1M-trigo}
\type{serie}
\piments{2}
\pts{}
\annee{2425}

\contenu{
\tcblower
Un géomètre doit déterminer la largeur d'une rivière. Voici le croquis qu'il a réalisé:

\begin{minipage}[t]{0.4\textwidth}{
\vspace{0pt}
$\overline{AB}=100~\text{m}$;

$\widehat{BAD}=60^\circ$;

$\widehat{BAC}=22^\circ$;

$\widehat{ABD}=90^\circ$;

Calculer la largeur de la rivière à un mètre près. 
}
\end{minipage}
\begin{minipage}[t]{0.55\textwidth}{
\vspace{0pt}
\includegraphics[scale=1]{../medias/1M/trigo/1M-exo-24}
}
\end{minipage}

}
\correction{
\tcblower
\textit{Generated by AI}

Pour déterminer la largeur de la rivière, nous devons identifier quelle distance représente cette largeur dans le schéma. La largeur de la rivière correspond à la hauteur du triangle rectangle par rapport au côté $\overline{AB}$.

\textbf{Données :}
\begin{itemize}
\item $\overline{AB} = 100$ m
\item $\widehat{BAD} = 60°$
\item $\widehat{BAC} = 22°$
\item $\widehat{ABD} = 90°$ (donc $BD \perp AB$)
\end{itemize}

D'après le schéma, la largeur de la rivière est la distance $\overline{BD}$, qui est perpendiculaire à $\overline{AB}$.

\textbf{Calcul de $\overline{BD}$ :}

Dans le triangle rectangle $\triangle ABD$ (rectangle en $B$), nous connaissons :
\begin{itemize}
\item L'angle $\widehat{BAD} = 60°$
\item Le côté adjacent $\overline{AB} = 100$ m
\end{itemize}

Nous utilisons la tangente :
\[\tan(\widehat{BAD}) = \frac{\overline{BD}}{\overline{AB}}\]

Donc :
\[\overline{BD} = \overline{AB} \times \tan(60°) = 100 \times \sqrt{3}\]

Calculons la valeur numérique :
\[\overline{BD} = 100 \times \sqrt{3} \approx 100 \times 1{,}732 = 173{,}2~\text{m}\]

\textbf{Réponse :} La largeur de la rivière est d'environ $\boxed{173~\text{m}}$ (arrondi au mètre près).

\textbf{Remarque :} L'angle $\widehat{BAC} = 22°$ permet de calculer d'autres distances dans la figure (comme $\overline{AC}$), mais n'est pas nécessaire pour déterminer la largeur de la rivière.
}

