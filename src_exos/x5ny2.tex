\titre{}
\theme{derivation2}
\auteur{Nathan Scheinmann}
\niveau{3M}
\source{sesamath3e}
\type{serie}
\piments{2}
\pts{}
\annee{2425}

\contenu{
\tcblower
La vitesse maximale autorisée sur une route nationale est de 110\,km/h. L’indicateur de vitesse d’une voiture marque 80\,km/h au moment où elle passe à hauteur d’une borne kilométrique le long d’une route. Quatre minutes plus tard, elle est 8\,km plus loin et son compteur marque 88\,km/h. Montrer qu’à un moment au moins, entre ces deux repères, la voiture a dépassé la vitesse de 118\,km/h.
}
\correction{
\tcblower
\textit{Generated by AI}

Nous allons utiliser le \textbf{théorème des accroissements finis}.

Notons $s(t)$ la position de la voiture en kilomètres en fonction du temps $t$ en heures. La vitesse est $v(t) = s'(t)$.

\textbf{Données :}
\begin{itemize}
\item À $t = 0$ : $v(0) = 80$ km/h et $s(0) = 0$ km
\item À $t = \dfrac{4}{60} = \dfrac{1}{15}$ h : $v\left(\dfrac{1}{15}\right) = 88$ km/h et $s\left(\dfrac{1}{15}\right) = 8$ km
\end{itemize}

\textbf{Vitesse moyenne :}
$$v_{\text{moy}} = \dfrac{s\left(\dfrac{1}{15}\right) - s(0)}{\dfrac{1}{15} - 0} = \dfrac{8}{\dfrac{1}{15}} = 8 \times 15 = 120 \text{ km/h}$$

\textbf{Application du théorème des accroissements finis :}

La fonction $s(t)$ est continue sur $\left[0, \dfrac{1}{15}\right]$ et dérivable sur $\left]0, \dfrac{1}{15}\right[$.

D'après le théorème des accroissements finis, il existe au moins un instant $c \in \left]0, \dfrac{1}{15}\right[$ tel que :
$$s'(c) = \dfrac{s\left(\dfrac{1}{15}\right) - s(0)}{\dfrac{1}{15} - 0} = 120 \text{ km/h}$$

Or, $s'(c) = v(c)$ est la vitesse instantanée à l'instant $c$.

\textbf{Conclusion :}

À un moment au moins entre les deux repères, la voiture a roulé à exactement 120 km/h. Puisque la vitesse passe de 80 km/h à 88 km/h et que la vitesse moyenne est de 120 km/h, par continuité de la fonction vitesse, la voiture a dû atteindre au moins 120 km/h, ce qui dépasse largement 118 km/h.

Donc la voiture a bien dépassé 118 km/h à un moment donné.
}

