\titre{}
\theme{equations}
\auteur{Nathan Scheinmann}
\niveau{1M}
\source{musy-2017}
\type{serie}
\piments{1}
\pts{}
\annee{2425}

\contenu{
	\tcblower
	\begin{itemize}
			\item Pour chaque système d'équations, donner l'opération à effectuer pour élimer la variable $x$.
				\begin{center}
					{\bfseries Exemple: }
\sysautonum*{\quad\hbox{(*)}}
				\systeme*{8 x+3 y=1,
					3 x+5 y=9}

					\vspace{5pt} 
				$E1\cdot 3+E2\cdot (-8)$ c'est-à-dire on multiplie $(1)$ par 3 et $(2)$ par $-8$ puis on addition $(1)$ et $(2)$.
				\end{center}

				\begin{tasks}(3)
					\task \sysautonum*{\hbox{(*)}}\systeme{2 x+5 y=7 , 3 x+4 y=9}
			\task \sysautonum*{\quad\hbox{(*)}}\systeme*{4 x-3 y=2 , -5 x+8 y=1}
			\task \sysautonum*{\quad\hbox{(*)}}\systeme*{2 x+10 y=9 , 8 x+5 y=7}
\end{tasks}
			\item Pour chaque système d'équations, donner l'opération à effectuer pour élimer la variable $y$.
\begin{tasks}(3)
\task \sysautonum*{\quad\hbox{(*)}}\systeme*[][:]{2 x+5 y=7:
3 x+4 y=9}
\task \sysautonum*{\quad\hbox{(*)}}\systeme*[][:]{4 x-3 y=2 : -5 x+8 y=1}
\task \sysautonum*{\quad\hbox{(*)}}\systeme*[][:]{2 x+10 y=9 : 8 x+5 y=7}
\end{tasks}
\end{itemize}
}
\correction{
	\tcblower
	Par exemple, pour éliminer $x$ :
\begin{tasks}(3)
	\task $E1\cdot 3 - E2\cdot 2$ 
	\task $E1\cdot 5 + E2 \cdot 4$
	\task $E1\cdot 4-E2$
\end{tasks}
Par exemple, pour éliminer $y$ :
\begin{tasks}(3)
	\task $E1\cdot 4-E2\cdot 5$
	\task $E1\cdot 8+E2\cdot 3$
	\task $E1-E2\cdot 2$
\end{tasks}
}

