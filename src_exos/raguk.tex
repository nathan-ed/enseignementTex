\titre{}
\theme{derivation2}
\auteur{Nathan Scheinmann}
\niveau{3M}
\source{sesamath3e}
\type{serie}
\piments{2}
\pts{}
\annee{2425}

\contenu{
\tcblower
On considère le théorème « Dérivée de la différence de deux fonctions »
\begin{tasks}
\task L'énoncer en identifiant clairement hypothèses et conclusions.
\task Illustrer son utilité par des exemples.
\task On donne ci-dessous une démonstration :
\end{tasks}

Démonstration : 

$\begin{aligned}(f - g)'(x) &=
\displaystyle\lim_{h \to 0} \dfrac{(f - g)(x + h) - (f - g)(x)}{h}\\
&= \displaystyle\lim_{h \to 0} \dfrac{(f(x + h) - g(x + h)) - (f(x) - g(x))}{h}\\
&= \displaystyle\lim_{h \to 0} \dfrac{f(x + h) - f(x) - g(x + h) + g(x)}{h}\\
&= \displaystyle\lim_{h \to 0} \dfrac{f(x + h) - f(x) - (g(x + h) - g(x))}{h}\\
&= \displaystyle\lim_{h \to 0} \dfrac{f(x + h) - f(x)}{h} - \dfrac{g(x + h) - g(x)}{h}\\
&= \displaystyle\lim_{h \to 0} \dfrac{f(x + h) - f(x)}{h} - \displaystyle\lim_{h \to 0} \dfrac{g(x + h) - g(x)}{h}\\
&= f'(x) - g'(x)\\
\end{aligned}$

Justifier chaque étape de la démonstration.
}
\correction{
	\tcblower
Faire vérifier par l'enseignant.
}

