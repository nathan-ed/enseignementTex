\titre{5}
\theme{trigo}
\auteur{Nathan Scheinmann}
\niveau{1M}
\source{sesamath-1M-trigo}
\type{serie}
\piments{2}
\pts{}
\annee{2425}

\contenu{
	\tcblower
Quand le sommet de la tour Eiffel est vu d'une distance de $60~\text{m}$ à partir de la base, l'angle d'élévation est de $79{,}2^\circ$. Estimer la hauteur de la tour Eiffer au mètre près.
}
\correction{
\tcblower
\textit{Generated by AI}

Nous avons un triangle rectangle où :
- La distance horizontale (base) = $60~\text{m}$
- L'angle d'élévation = $79{,}2°$
- La hauteur de la tour = $h$ (inconnue)

En utilisant la tangente de l'angle :
\[\tan(79{,}2°) = \frac{h}{60}\]

\[h = 60 \times \tan(79{,}2°) \approx 60 \times 5{,}145 \approx 308{,}7~\text{m}\]

\textbf{Réponse :} La hauteur de la tour Eiffel est d'environ \boxed{309~\text{m}}.
}

