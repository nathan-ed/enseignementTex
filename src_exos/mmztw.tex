\titre{}
\theme{derivation}
\auteur{Nathan Scheinmann}
\niveau{3M}
\source{sesamath3e}
\type{serie}
\piments{2}
\pts{}
\annee{2425}

\contenu{
\tcblower
Trouver la pente de la tangente à la parabole d’équation \(y=x^2\) au point \((-1;1)\).
}
\correction{
	\tcblower
	$\begin{aligned}
		f'(-1) &= \displaystyle\lim_{h \to 0} \frac{f(-1+h) - f(-1)}{h}
		       =\displaystyle\lim_{h \to 0} \frac{(-1+h)^2 - 1}{h}\\
		       &= \displaystyle\lim_{h \to 0} \frac{1 - 2h + h^2 - 1}{h}
		       = \displaystyle\lim_{h \to 0} \frac{-2h + h^2}{h} = \displaystyle\lim_{h \to 0} (-2 + h) = -2
	\end{aligned}$
}

