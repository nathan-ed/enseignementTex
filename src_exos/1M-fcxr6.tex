\titre{}
\theme{derivation}
\auteur{Nathan Scheinmann}
\niveau{3M}
\source{sesamath3e}
\type{serie}
\piments{2}
\pts{}
\annee{2425}

\contenu{
\tcblower
On considère la fonction définie par $f(x) = ax^3 + bx^2 + cx + d$.

\begin{tasks}(1)
\task Déterminer les constantes $a$, $b$, $c$ et $d$ afin que le graphe de la fonction possède un extremum en $(1; 16)$ et un autre en $(3; -16)$.
\task $f$ admet-elle un point d'inflexion ? Si oui, calculer ses coordonnées.
\task Esquisser la courbe représentative. de $f$.
\end{tasks}
}
\correction{

}

