\titre{}
\theme{limites}
\auteur{Nathan Scheinmann}
\niveau{3M}
\source{sesamath3e}
\type{serie}
\piments{4}
\pts{}
\annee{2425}

\contenu{
\tcblower
\noindent Soit la fonction \(f\) définie par  
\[
f(x)=\frac{x-1}{3-2x^2}.
\]
\begin{tasks}
\task Donner les limites à droite et à gauche de \(f(x)\) lorsque \(x\to0\) et lorsque \(x\to2\), et déterminer \(\displaystyle\lim_{x\to0}f(x)\) et \(\displaystyle\lim_{x\to2}f(x)\).
\task Trouver un polynôme \(g(x)\) tel que \(\displaystyle\lim_{x\to0}\bigl(f(x)\cdot g(x)\bigr)=3\).
\task Cette question admet-elle plusieurs réponses ? Si oui, en donner au moins une autre.
\task Trouver un polynôme \(h(x)\) tel que \(\displaystyle\lim_{x\to0}\bigl(f(x)\cdot h(x)\bigr)\) soit non nul, et calculer cette limite. Cette question admet-elle plusieurs réponses ? Si oui, en donner au moins une autre.
\task Trouver un polynôme \(k(x)\) tel que \(\displaystyle\lim_{x\to0}\bigl(f(x)\cdot k(x)\bigr)=0\).
\end{tasks}
}
\correction{
\tcblower
\begin{tasks}(2)
\task 
On a : \( \lim_{x \to 0^-} f(x) = \dfrac{-1}{3} \) et \( \lim_{x \to 0^+} f(x) = \dfrac{-1}{3} \), donc  
\( \lim_{x \to 0} f(x) = -\dfrac{1}{3} \) par le théorème 3.  
De le même manière, \( \lim_{x \to 2} f(x) = \dfrac{2 - 1}{3 - 2 \cdot 4} = \dfrac{1}{-5} = -\dfrac{1}{5} \).

\task 
On a \( \lim_{x \to 0} f(x) = -\dfrac{1}{3} \).  
On cherche \( g(x) \) tel que \( \lim_{x \to 0} f(x) \cdot g(x) = 3 \).  
Par la propriété P3 (produit), il suffit que \( \lim_{x \to 0} g(x) = -9 \).  
Exemple : \( g(x) = -9 \).

\task 
Oui, car toute fonction \( g \) telle que \( \lim_{x \to 0} g(x) = -9 \) convient.  
Par exemple : \( g(x) = -9 + x \).  
La limite du produit est assurée par P3.

\task 
Il suffit que \( \lim_{x \to 0} h(x) \ne 0 \) pour que \( \lim f(x) \cdot h(x) \ne 0 \), car \( \lim f(x) = -\dfrac{1}{3} \).  
Par exemple : \( h(x) = 1 \) convient.  
Alors \( \lim_{x \to 0} f(x) \cdot h(x) = -\dfrac{1}{3} \ne 0 \) par P3.

\task 
Il suffit que \( \lim_{x \to 0} k(x) = 0 \), car alors  
\( \lim f(x) \cdot k(x) = -\dfrac{1}{3} \cdot 0 = 0 \) (propriété P3).  
Exemple : \( k(x) = x \).
\end{tasks}
}

