\titre{}
\theme{limites}
\auteur{Nathan Scheinmann}
\niveau{3M}
\source{musy}
\type{serie}
\piments{2}
\pts{}
\annee{2425}

\contenu{
\tcblower
Montrer que l'équation 
\[x^5-3x-1=0\]
admet une solution sur l'intervalle $[1;2]$. 
}
\correction{
	\tcblower
\begin{itemize}
  \item La fonction \( f(x) = x^5 - 3x - 1 \) est un polynôme, donc continue sur \( \mathbb{R} \), en particulier sur \([1; 2]\).
  \item On a :
    \[
    \begin{aligned}
    f(1) &= 1^5 - 3 \cdot 1 - 1 = -3 \\
    f(2) &= 2^5 - 3 \cdot 2 - 1 = 32 - 6 - 1 = 25
    \end{aligned}
    \]
  \item Or \( f(1) < 0 \) et \( f(2) > 0 \), donc \( f(1) \cdot f(2) < 0 \).
  \item Par le corollaire du théorème de la valeur intermédiaire, il existe \( c \in [1; 2] \) tel que \( f(c) = 0 \).
\end{itemize}
}

