\titre{}
\theme{calcLit}
\auteur{Nathan Scheinmann}
\niveau{1M}
\source{musy}
\type{serie}
\piments{2}
\pts{}
\annee{2425}

\contenu{
\tcblower
On considère l'identité suivante, appelée égalité de Lagrange (mathématicien du XVI ${ }^e$ siècle):
\[
\left(a^2+b^2\right)\left(c^2+d^2\right)=(a c+b d)^2+(a d-b c)^2
\]
\begin{tasks}
\task Démontrer cette identité.
\task Appliquer cette identité à quatre entiers (par exemple $2,3,4,5$ ) en utilisant la calculatrice.
%\task Décrire cette identité par une phrase: \enquote{Le produit d...}
\end{tasks}
}
\correction{
\tcblower
\begin{tasks}
	\task On développe les deux membres. On constate qu'ils sont égaux à $a^2c^2+a^2d^2+b^2c^2+b^2d^2$. 
	\task à la calculatrice. 
\end{tasks}
}

