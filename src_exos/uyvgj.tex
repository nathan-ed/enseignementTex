\titre{}
\theme{derivation}
\auteur{Nathan Scheinmann}
\niveau{3M}
\source{sesamath3e}
\type{serie}
\piments{2}
\pts{}
\annee{2425}

\contenu{
\tcblower
Déterminer les asymptotes de la fonction $f$ définie par $f(x) = \dfrac{3x^2 - 6x + 18}{x - 2 + x^2}$.
}
\correction{
\tcblower
\textit{Generated by AI}

Réécrivons d'abord la fonction sous forme standard :
$$f(x) = \dfrac{3x^2 - 6x + 18}{x^2 + x - 2}$$

\textbf{Asymptotes verticales :}

Le dénominateur s'annule lorsque $x^2 + x - 2 = 0$. Factorisons :
$$x^2 + x - 2 = (x + 2)(x - 1) = 0$$

Donc $x = -2$ ou $x = 1$. Vérifions que le numérateur ne s'annule pas en ces points :
\begin{itemize}
\item Pour $x = -2$ : $3(-2)^2 - 6(-2) + 18 = 12 + 12 + 18 = 42 \neq 0$
\item Pour $x = 1$ : $3(1)^2 - 6(1) + 18 = 3 - 6 + 18 = 15 \neq 0$
\end{itemize}

Les asymptotes verticales sont : $\boxed{x = -2}$ et $\boxed{x = 1}$.

\textbf{Asymptote horizontale :}

Calculons la limite à l'infini. Comme le numérateur et le dénominateur sont des polynômes de même degré (2), la limite est le rapport des coefficients dominants :
$$\lim_{x \to \pm\infty} f(x) = \lim_{x \to \pm\infty} \dfrac{3x^2}{x^2} = 3$$

L'asymptote horizontale est : $\boxed{y = 3}$.
}

