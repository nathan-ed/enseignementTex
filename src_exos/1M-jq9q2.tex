\titre{}
\theme{fonctions}
\auteur{Nathan Scheinmann}
\niveau{1M}
\source{}
\type{serie}
\piments{2}
\pts{}
\annee{2425}

\contenu{
	\tcblower
Un terrain se trouve en bordure d'une rivière rectiligne.
        On désire délimiter une zone rectangulaire le long de la rivière à l'aide d'une barrière ayant une longueur totale de 120 mètres. Le côté de la zone le long de la rivière n'a pas besoin de barrière.
        Quelle est l'aire maximale possible de la zone délimitée par la barrière et la rivière ?
}
\correction{
	\tcblower
	$1800 \,\text{m}^2$.
}

