\titre{undefined}
\theme{equations}
\auteur{Nathan Scheinmann}
\niveau{1M}
\source{undefined}
\type{serie}
\piments{1}
\pts{}
\annee{2425}

\contenu{
	\tcblower
	Un terrain rectangulaire a un périmètre de $150$ m. Si l’on augmente sa largeur de $5$ m et
si l’on diminue sa longueur de $3$ m, alors son aire augmente de $120$ $\text{m}^2$.
Quelles sont les dimensions de ce rectangle?
}
\correction{
	\tcblower
On pose les inconnues
\begin{align*}
	&x=\text{La largeur du rectangle} &&y=\text{La longueur du rectangle}
\end{align*}
On obtient le système
$
\begin{cases}
2x+2y=150\\
(x+5)(y-3)=xy+120
\end{cases}
$
On résout le système après avoir simplifié au maximimum la deuxième équation
\[(x+5)(y-3)=xy+120 \iff xy-3x+5y-15=xy+120\iff -3x+5y=135.\]
On obtient que $x=30$cm et $y=45$cm. Le rectangle mesure $30$cm sur $45$cm.
}

