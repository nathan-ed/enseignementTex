\titre{}
\theme{calcLit}
\auteur{Nathan Scheinmann}
\niveau{1M}
\source{musy}
\type{serie}
\piments{2}
\pts{}
\annee{2425}

\contenu{
\tcblower
	Utiliser les identités remarquables pour calculer (sans calculatrice) les carrés suivants\,:
	\begin{tasks}
\task Avec $(a+b)^2\,: \quad  23^2\,;\quad  92^2\,;\quad 101^2\,;\quad 42^2$
\task Avec $(a-b)^2\,:\quad 39^2\,;\quad 68^2\,;\quad 99^2\,;\quad 298^2$
	\end{tasks}
}
\correction{
\tcblower
	Par exemple, $23^2=(20+3)^2=20^2+2\cdot 20\cdot 3+3^9=400+120+9=529$.
}

