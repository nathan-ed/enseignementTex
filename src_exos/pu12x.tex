\titre{30}
\theme{trigo}
\auteur{Nathan Scheinmann}
\niveau{1M}
\source{sesamath-1M-trigo}
\type{serie}
\piments{2}
\pts{}
\annee{2425}

\contenu{
	\tcblower
	Calculer la valeur exactes de $\cos(\alpha)$ et de $\tan(\alpha)$ sachant que $\alpha$ est un angle aigu tel que $\sin(\alpha)=\dfrac{\sqrt{6}-\sqrt{2}}{4}$.
}
\correction{
\tcblower
\textit{Generated by AI}

Sachant que $\sin(\alpha) = \dfrac{\sqrt{6}-\sqrt{2}}{4}$ et que $\alpha$ est un angle aigu (donc $0° < \alpha < 90°$).

\textbf{Calcul de $\cos(\alpha)$ :}

En utilisant l'identité fondamentale $\sin^2(\alpha) + \cos^2(\alpha) = 1$ :

\[
\cos^2(\alpha) = 1 - \sin^2(\alpha) = 1 - \left(\frac{\sqrt{6}-\sqrt{2}}{4}\right)^2
\]

Calculons $\sin^2(\alpha)$ :
\begin{align*}
\sin^2(\alpha) &= \frac{(\sqrt{6}-\sqrt{2})^2}{16} \\
&= \frac{6 - 2\sqrt{6}\sqrt{2} + 2}{16} \\
&= \frac{8 - 2\sqrt{12}}{16} \\
&= \frac{8 - 2 \cdot 2\sqrt{3}}{16} \\
&= \frac{8 - 4\sqrt{3}}{16} \\
&= \frac{2 - \sqrt{3}}{4}
\end{align*}

Donc :
\[
\cos^2(\alpha) = 1 - \frac{2 - \sqrt{3}}{4} = \frac{4 - 2 + \sqrt{3}}{4} = \frac{2 + \sqrt{3}}{4}
\]

Comme $\alpha$ est aigu, $\cos(\alpha) > 0$, donc :
\[
\cos(\alpha) = \sqrt{\frac{2 + \sqrt{3}}{4}} = \frac{\sqrt{2 + \sqrt{3}}}{2}
\]

On peut simplifier en remarquant que $2 + \sqrt{3} = \frac{(\sqrt{6}+\sqrt{2})^2}{4}$ :
\[
\cos(\alpha) = \frac{\sqrt{6}+\sqrt{2}}{4}
\]

\textbf{Calcul de $\tan(\alpha)$ :}

\[
\tan(\alpha) = \frac{\sin(\alpha)}{\cos(\alpha)} = \frac{\dfrac{\sqrt{6}-\sqrt{2}}{4}}{\dfrac{\sqrt{6}+\sqrt{2}}{4}} = \frac{\sqrt{6}-\sqrt{2}}{\sqrt{6}+\sqrt{2}}
\]

Rationalisons en multipliant par le conjugué :
\begin{align*}
\tan(\alpha) &= \frac{(\sqrt{6}-\sqrt{2})(\sqrt{6}-\sqrt{2})}{(\sqrt{6}+\sqrt{2})(\sqrt{6}-\sqrt{2})} \\
&= \frac{6 - 2\sqrt{12} + 2}{6 - 2} \\
&= \frac{8 - 4\sqrt{3}}{4} \\
&= 2 - \sqrt{3}
\end{align*}

\textbf{Réponses :}
\[
\cos(\alpha) = \frac{\sqrt{6}+\sqrt{2}}{4} \quad \text{et} \quad \tan(\alpha) = 2 - \sqrt{3}
\]
}

