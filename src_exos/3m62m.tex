\titre{}
\theme{dérivées}
\auteur{Nathan Scheinmann}
\niveau{3M}
\source{analysis}
\type{serie}
\piments{2}
\pts{}
\annee{2425}

\contenu{
\tcblower
Montrer que l’équation \(6x^4 - 7x + 1 = 0\) n’a pas plus de deux racines réelles. (Utiliser le théorème de Rolle.)
}
\correction{
\tcblower
\textit{Generated by AI}

Nous allons procéder par l'absurde en supposant que l'équation $6x^4 - 7x + 1 = 0$ possède au moins trois racines réelles distinctes.

Soit $f(x) = 6x^4 - 7x + 1$.

\textbf{Hypothèse par l'absurde :} Supposons que $f$ possède au moins trois racines réelles distinctes, notées $a < b < c$.

Alors $f(a) = f(b) = f(c) = 0$.

\textbf{Application du théorème de Rolle :}

Le théorème de Rolle stipule que si une fonction est continue sur $[a;b]$, dérivable sur $]a;b[$, et que $f(a) = f(b)$, alors il existe au moins un point $c \in ]a;b[$ tel que $f'(c) = 0$.

\begin{itemize}
\item Entre $a$ et $b$ : Comme $f(a) = f(b) = 0$ et $f$ est un polynôme (donc continue et dérivable partout), il existe par le théorème de Rolle au moins un point $\alpha \in ]a;b[$ tel que $f'(\alpha) = 0$.

\item Entre $b$ et $c$ : De même, il existe au moins un point $\beta \in ]b;c[$ tel que $f'(\beta) = 0$.
\end{itemize}

Donc $f'(x)$ possède au moins deux racines réelles distinctes $\alpha$ et $\beta$ avec $\alpha < \beta$.

\textbf{Calcul de $f'(x)$ et nouvelle application de Rolle :}

\[f'(x) = 24x^3 - 7\]

Mais si $f'$ possède deux racines distinctes $\alpha$ et $\beta$, alors par le théorème de Rolle appliqué à $f'$ sur $[\alpha;\beta]$, il existe $\gamma \in ]\alpha;\beta[$ tel que $f''(\gamma) = 0$.

Calculons $f''(x)$ :
\[f''(x) = 72x^2\]

Or, $f''(x) = 72x^2 = 0$ seulement si $x = 0$.

Donc $f''$ n'a qu'une seule racine : $x = 0$.

\textbf{Contradiction :}

Nous avons supposé que $f$ avait au moins trois racines, ce qui nous a conduit à conclure que $f'$ devait avoir au moins deux racines distinctes. Mais $f'(x) = 24x^3 - 7$ est une fonction strictement croissante (car $f''(x) = 72x^2 \geq 0$), donc elle ne peut avoir qu'une seule racine réelle.

Ceci contredit notre hypothèse.

\textbf{Conclusion :}

Par l'absurde, l'équation $6x^4 - 7x + 1 = 0$ ne peut pas avoir plus de deux racines réelles.

\[\boxed{\text{L'équation } 6x^4 - 7x + 1 = 0 \text{ a au plus deux racines réelles.}}\]
}

