\titre{}
\theme{dérivées}
\auteur{Nathan Scheinmann}
\niveau{3M}
\source{fundamentum}
\type{serie}
\piments{2}
\pts{}
\annee{2425}

\contenu{
\tcblower
On dispose de 288 m de clôture grillagée pour construire 6 enclos identiques pour un zoo selon le plan ci-contre. Quelles dimensions donner à ces enclos de manière à maximiser leur surface au sol ?
\begin{center}
\begin{tikzpicture}
    % Définir les coins du rectangle extérieur
    \tkzDefPoint(0,0){A}
    \tkzDefPoint(6,0){B}
    \tkzDefPoint(6,4){C}
    \tkzDefPoint(0,4){D}
    
    % Définir les points pour les lignes de la grille interne
    \tkzDefPoint(0,2){H_gauche}
    \tkzDefPoint(6,2){H_droite}
    \tkzDefPoint(2,0){V1_bas}
    \tkzDefPoint(2,4){V1_haut}
    \tkzDefPoint(4,0){V2_bas}
    \tkzDefPoint(4,4){V2_haut}
    
    % Dessiner le rectangle extérieur
    \tkzDrawPolygon[thick](A,B,C,D)
    
    % Dessiner les lignes de la grille interne
    \tkzDrawSegment[color=gray](H_gauche,H_droite)
    \tkzDrawSegment[color=gray](V1_bas,V1_haut)
    \tkzDrawSegment[color=gray](V2_bas,V2_haut)
\end{tikzpicture}
\end{center}
}
\correction{
\tcblower
\textit{Generated by AI}
{\scriptsize \textit{Correction générée par IA}}

Soit $x$ la largeur d'un enclos et $y$ sa hauteur. Les 6 enclos forment un rectangle de dimensions $3x$ sur $2y$.

La clôture grillagée comprend :
\begin{itemize}
\item Le périmètre extérieur : $2(3x) + 2(2y) = 6x + 4y$
\item Les 2 séparations horizontales intérieures : $2 \times 3x = 6x$
\item Les 2 séparations verticales intérieures : $2 \times y = 2y$
\end{itemize}

La longueur totale de clôture est donc :
\[L = 6x + 4y + 6x + 2y = 12x + 6y = 288\]

D'où la contrainte :
\[12x + 6y = 288 \quad \Rightarrow \quad y = \frac{288 - 12x}{6} = 48 - 2x\]

La surface totale des 6 enclos est :
\[S = 3x \times 2y = 6xy = 6x(48 - 2x) = 288x - 12x^2\]

Pour maximiser la surface, nous dérivons par rapport à $x$ :
\[S'(x) = 288 - 24x\]

L'annulation de la dérivée donne :
\[288 - 24x = 0 \quad \Rightarrow \quad x = 12 \text{ m}\]

D'où :
\[y = 48 - 2(12) = 48 - 24 = 24 \text{ m}\]

Vérifions qu'il s'agit bien d'un maximum :
\[S''(x) = -24 < 0\]

donc il s'agit bien d'un maximum.

Les dimensions optimales sont donc : chaque enclos doit avoir une largeur de $\boxed{12 \text{ m}}$ et une hauteur de $\boxed{24 \text{ m}}$.

La surface totale maximale est $S = 6 \times 12 \times 24 = 1728$ m$^2$.
}

