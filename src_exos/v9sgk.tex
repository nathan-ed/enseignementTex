\titre{}
\theme{dérivées}
\auteur{Nathan Scheinmann}
\niveau{3M}
\source{analysis}
\type{cote}
\piments{2}
\pts{}
\annee{2425}

\contenu{
\tcblower
Prouver que, si $(c, f(c))$ est un point d'inflexion, alors
$$\text{soit } \quad f''(c) = 0 \quad \text{ou} \quad f''(c) \text{ n'existe pas}.$$
}
\correction{
\tcblower
\textit{Generated by AI}

Pour prouver cette propriété, nous devons montrer que si $(c, f(c))$ est un point d'inflexion, alors nécessairement soit $f''(c) = 0$ soit $f''(c)$ n'existe pas.

\textbf{Rappel :} Un point $(c, f(c))$ est un point d'inflexion si la concavité de $f$ change en $c$, c'est-à-dire si $f''$ change de signe en $c$.

\textbf{Démonstration par contraposée :}

Supposons que $f''(c)$ existe et que $f''(c) \neq 0$. Nous allons montrer que $(c, f(c))$ n'est pas un point d'inflexion.

Si $f''(c)$ existe et $f''(c) \neq 0$, alors par continuité de $f''$ au voisinage de $c$ (ou par le théorème de conservation du signe), il existe un intervalle ouvert $I$ contenant $c$ tel que $f''(x)$ garde un signe constant sur $I$.

\begin{itemize}
\item Si $f''(c) > 0$, alors $f''(x) > 0$ pour tout $x \in I$ (intervalle autour de $c$)
\item Si $f''(c) < 0$, alors $f''(x) < 0$ pour tout $x \in I$
\end{itemize}

Dans les deux cas, $f''$ ne change pas de signe en $c$, donc la concavité ne change pas, et par conséquent $(c, f(c))$ n'est \textbf{pas} un point d'inflexion.

\textbf{Conclusion :}

Par contraposée, nous avons montré que si $(c, f(c))$ est un point d'inflexion, alors il est impossible que $f''(c)$ existe et soit différente de zéro. Donc :
\[\text{soit } \quad f''(c) = 0 \quad \text{ou} \quad f''(c) \text{ n'existe pas.}\]

\textbf{Remarque importante :} La réciproque n'est pas vraie. Le fait que $f''(c) = 0$ ne garantit pas l'existence d'un point d'inflexion en $c$. Par exemple, pour $f(x) = x^4$, on a $f''(0) = 0$ mais il n'y a pas de point d'inflexion en $x = 0$ car $f''(x) = 12x^2 \geq 0$ pour tout $x$.
}

