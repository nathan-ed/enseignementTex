\titre{}
\theme{derivation}
\auteur{Nathan Scheinmann}
\niveau{3M}
\source{sesamath3e}
\type{serie}
\piments{2}
\pts{}
\annee{2425}

\contenu{
	\tcblower
Vrai ou faux ? Justifier soigneusement :
\begin{tasks}(1)
  \task Si une droite est tangente à une courbe en un point, alors elle ne rencontre la courbe qu’à ce point.
  \task Si une droite rencontre une courbe en un seul point, alors elle est tangente à la courbe en ce point.
\end{tasks}
}
\correction{
\tcblower

\textit{Generated by AI}

\begin{tasks}(1)
\task \textbf{Faux.} Une droite tangente peut rencontrer la courbe en plusieurs points.

Par exemple, considérons la fonction $f(x) = x^3$ et la droite $y = 0$ (l'axe des abscisses). Cette droite est tangente à la courbe en $x = 0$ (car $f(0) = 0$ et $f'(0) = 0$), mais elle rencontre aussi la courbe uniquement en ce point pour cette fonction particulière.

Cependant, pour la fonction $f(x) = \sin(x)$, la droite $y = 0$ est tangente en $x = 0$ mais rencontre la courbe en une infinité de points ($x = k\pi$ pour tout $k \in \mathbb{Z}$).

\task \textbf{Faux.} Une droite qui ne rencontre une courbe qu'en un seul point n'est pas nécessairement tangente.

Contre-exemple : Considérons la fonction $f(x) = x^3$ et la droite $y = 1$. Cette droite ne rencontre la courbe qu'en un seul point ($x = 1$), mais elle n'est pas tangente à la courbe en ce point.

Pour qu'une droite soit tangente à une courbe en un point, il faut que la pente de la droite soit égale à la dérivée de la fonction en ce point, c'est-à-dire que la droite « épouse » localement la courbe.
\end{tasks}

}
