\titre{}
\theme{derivation}
\auteur{Nathan Scheinmann}
\niveau{3M}
\source{sesamath3e}
\type{serie}
\piments{2}
\pts{}
\annee{2425}

\contenu{
	\tcblower
Vrai ou faux ? Justifier soigneusement :
\begin{tasks}(1)
  \task Si une droite est tangente à une courbe en un point, alors elle ne rencontre la courbe qu’à ce point.
  \task Si une droite rencontre une courbe en un seul point, alors elle est tangente à la courbe en ce point.
\end{tasks}
}
\correction{
\tcblower
\begin{tasks}
    \task \textit{Si une droite est tangente à une courbe en un point, alors elle ne rencontre la courbe qu'à ce point.} \\
    \textbf{FAUX.} Contre-exemple : Pour $f(x)=x^3$, la tangente en $x=2$ ($y=12x-16$) rencontre la courbe en un second point $(-4; -64)$, comme vu à l'exercice 36. De même, pour $f(x)=\sin(x)$, les tangentes horizontales recoupent la courbe une infinité de fois.
    
    \task \textit{Si une droite rencontre une courbe en un seul point, alors elle est tangente à la courbe en ce point.} \\
    \textbf{FAUX.} Contre-exemple : La droite $y=1$ rencontre la courbe $f(x)=x^3$ en un unique point $(1;1)$. Pourtant, elle n'est pas tangente en ce point (la pente de la tangente est $f'(1)=3$, alors que la droite a une pente nulle). La droite "intersecte" simplement la courbe.
\end{tasks}
}
