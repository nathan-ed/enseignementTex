\titre{3}
\theme{trigo}
\auteur{Nathan Scheinmann}
\niveau{1M}
\source{sesamath-1M-trigo}
\type{serie}
\piments{2}
\pts{}
\annee{2425}

\contenu{
	\tcblower
Dans chaque cas, calculer la mesure de l'angle $\widehat{MNO}$; donner la valeur arrondie au degré.
\begin{tasks}(2)
	\task \includegraphics[scale=1]{../medias/1M/trigo/1M-exo-3-1}
	\task \includegraphics[scale=1]{../medias/1M/trigo/1M-exo-3-2}
	\task \includegraphics[scale=1]{../medias/1M/trigo/1M-exo-3-3}
	\task \includegraphics[scale=1]{../medias/1M/trigo/1M-exo-3-4}
\end{tasks}
}
\correction{
\tcblower

\textit{Generated by AI}

Dans chaque cas, nous devons calculer la mesure de l'angle $\widehat{MNO}$ en utilisant les rapports trigonométriques.

\begin{tasks}(1)
\task \textbf{Triangle 1 : Triangle MNO rectangle en M}

Nous connaissons $MO = 2$ cm (côté opposé à l'angle $\widehat{MNO}$) et $MN = 5$ cm (côté adjacent).

\[\tan(\widehat{MNO}) = \frac{MO}{MN} = \frac{2}{5} = 0{,}4\]

\[\widehat{MNO} = \arctan(0{,}4) \approx 21{,}8° \approx \boxed{22°}\]

\task \textbf{Triangle 2 : Triangle MNO rectangle en M}

Nous connaissons $NO = 2$ cm (hypoténuse), $OM = 1{,}2$ cm et $NM = 1{,}6$ cm.

Vérifions : $1{,}2^2 + 1{,}6^2 = 1{,}44 + 2{,}56 = 4 = 2^2$ \checkmark

Le côté opposé à $\widehat{MNO}$ est $OM = 1{,}2$ cm.

\[\sin(\widehat{MNO}) = \frac{OM}{NO} = \frac{1{,}2}{2} = 0{,}6\]

\[\widehat{MNO} = \arcsin(0{,}6) \approx 36{,}87° \approx \boxed{37°}\]

\task \textbf{Triangle 3 : Triangle MON rectangle en O}

Nous connaissons $MO = 5$ cm et $ON = 7$ cm. L'angle recherché est $\widehat{MNO}$.

Dans ce triangle, le côté opposé à $\widehat{MNO}$ est $MO = 5$ cm et le côté adjacent est $ON = 7$ cm.

\[\tan(\widehat{MNO}) = \frac{MO}{ON} = \frac{5}{7} \approx 0{,}714\]

\[\widehat{MNO} = \arctan\left(\frac{5}{7}\right) \approx 35{,}54° \approx \boxed{36°}\]

\task \textbf{Triangle 4 : Quadrilatère PMNO}

Ce n'est pas un simple triangle. Nous avons un quadrilatère avec des angles droits en N et M. Il contient un triangle rectangle $\triangle MON$ avec :
\begin{itemize}
\item L'angle en O : $55°$
\item $MO = 2$ cm (côté adjacent à l'angle de $55°$ dans le triangle MON)
\end{itemize}

Dans le triangle rectangle $\triangle MON$ (rectangle en M), nous connaissons l'angle $\widehat{MON} = 55°$ et $MO = 2$ cm.

L'angle $\widehat{MNO}$ dans ce triangle est :
\[\widehat{MNO} = 90° - 55° = \boxed{35°}\]

(Car dans un triangle rectangle, les deux angles aigus sont complémentaires.)
\end{tasks}

}

