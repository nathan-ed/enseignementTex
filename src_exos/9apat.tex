\titre{26}
\theme{trigo}
\auteur{Nathan Scheinmann}
\niveau{1M}
\source{sesamath-1M-trigo}
\type{serie}
\piments{2}
\pts{}
\annee{2425}

\contenu{
	\tcblower
\begin{tasks}(1)
	\task Calculer $\sin(\alpha)$ et $\tan(\alpha)$ sachant que $\cos(\alpha)=\dfrac{4}{7}$. 
	\task Calculer $\cos(\alpha)$ et $\tan(\alpha)$ sachant que $\sin(\alpha)=\dfrac{3}{8}$. 
	\task Calculer $\cos(\alpha)$ et $\sin(\alpha)$ sachant que $\tan(\alpha)=6$. 
	\task Calculer $\cos(\alpha)$ et $\tan(\alpha)$ sachant que $\sin(\alpha)=\dfrac{4}{3}$. 
\end{tasks}
}
\correction{
\tcblower
\textit{Generated by AI}

\begin{tasks}(1)
\task $\cos(\alpha) = \dfrac{4}{7}$

En utilisant l'identité fondamentale $\sin^2(\alpha) + \cos^2(\alpha) = 1$ :

\[\sin^2(\alpha) = 1 - \cos^2(\alpha) = 1 - \left(\frac{4}{7}\right)^2 = 1 - \frac{16}{49} = \frac{33}{49}\]

\[\boxed{\sin(\alpha) = \pm\frac{\sqrt{33}}{7}}\]

\[\boxed{\tan(\alpha) = \frac{\sin(\alpha)}{\cos(\alpha)} = \pm\frac{\sqrt{33}/7}{4/7} = \pm\frac{\sqrt{33}}{4}}\]

\task $\sin(\alpha) = \dfrac{3}{8}$

\[\cos^2(\alpha) = 1 - \sin^2(\alpha) = 1 - \left(\frac{3}{8}\right)^2 = 1 - \frac{9}{64} = \frac{55}{64}\]

\[\boxed{\cos(\alpha) = \pm\frac{\sqrt{55}}{8}}\]

\[\boxed{\tan(\alpha) = \frac{\sin(\alpha)}{\cos(\alpha)} = \pm\frac{3/8}{\sqrt{55}/8} = \pm\frac{3}{\sqrt{55}} = \pm\frac{3\sqrt{55}}{55}}\]

\task $\tan(\alpha) = 6$

On sait que $\tan(\alpha) = \dfrac{\sin(\alpha)}{\cos(\alpha)} = 6$, donc $\sin(\alpha) = 6\cos(\alpha)$.

En utilisant $\sin^2(\alpha) + \cos^2(\alpha) = 1$ :

\[(6\cos(\alpha))^2 + \cos^2(\alpha) = 1\]
\[36\cos^2(\alpha) + \cos^2(\alpha) = 1\]
\[37\cos^2(\alpha) = 1\]

\[\boxed{\cos(\alpha) = \pm\frac{1}{\sqrt{37}} = \pm\frac{\sqrt{37}}{37}}\]

\[\boxed{\sin(\alpha) = 6\cos(\alpha) = \pm\frac{6}{\sqrt{37}} = \pm\frac{6\sqrt{37}}{37}}\]

\task $\sin(\alpha) = \dfrac{4}{3}$

\textbf{Attention :} Cette valeur est impossible car $|\sin(\alpha)| \leq 1$ toujours.

\[\boxed{\text{Aucune solution (valeur impossible)}}\]
\end{tasks}
}

