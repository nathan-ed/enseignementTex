\titre{}
\theme{dérivées}
\auteur{Nathan Scheinmann}
\niveau{3M}
\source{analysis}
\type{serie}
\piments{2}
\pts{}
\annee{2425}

\contenu{
\tcblower

\begin{tasks}(1)
  \task Tracer le graphe de la fonction \(f(x) = |2x - 1| - 3\) et calculer sa dérivée.
  \task Vérifier que \(f(-1) = 0 = f(2)\) et pourtant \(f'(x)\) n’est jamais nul.
  \task Expliquer pourquoi cela ne contredit pas le théorème de Rolle.
\end{tasks}
}
\correction{
\tcblower
\textit{Generated by AI}

\textbf{Question a) : Tracer le graphe et calculer la dérivée}

La fonction $f(x) = |2x - 1| - 3$ peut s'écrire :
\[
f(x) = \begin{cases}
(2x - 1) - 3 = 2x - 4 & \text{si } x \geq \frac{1}{2} \\
-(2x - 1) - 3 = -2x - 2 & \text{si } x < \frac{1}{2}
\end{cases}
\]

La dérivée est :
\[
f'(x) = \begin{cases}
2 & \text{si } x > \frac{1}{2} \\
-2 & \text{si } x < \frac{1}{2} \\
\text{n'existe pas} & \text{si } x = \frac{1}{2}
\end{cases}
\]

Le graphe est en forme de V avec un sommet en $\left(\frac{1}{2}, -3\right)$.

\textbf{Question b) : Vérification}

Calculons :
\begin{align*}
f(-1) &= |2(-1) - 1| - 3 = |-3| - 3 = 3 - 3 = 0 \\
f(2) &= |2(2) - 1| - 3 = |3| - 3 = 3 - 3 = 0
\end{align*}

Donc effectivement $f(-1) = 0 = f(2)$.

Cependant, $f'(x)$ vaut soit $2$ soit $-2$ (quand elle existe), donc $f'(x) \neq 0$ partout où elle est définie.

\textbf{Question c) : Pourquoi cela ne contredit pas le théorème de Rolle}

Le théorème de Rolle stipule que si $f$ est :
\begin{itemize}
\item continue sur $[a;b]$
\item dérivable sur $]a;b[$
\item $f(a) = f(b)$
\end{itemize}
alors il existe $c \in ]a;b[$ tel que $f'(c) = 0$.

Dans notre cas, bien que $f$ soit continue sur $[-1;2]$ et que $f(-1) = f(2)$, la fonction $f$ n'est pas dérivable en $x = \frac{1}{2}$ (qui appartient à $]-1;2[$).

La condition de dérivabilité sur l'intervalle ouvert n'est donc pas satisfaite, ce qui explique pourquoi le théorème de Rolle ne s'applique pas.

\textbf{Conclusion :} L'absence de dérivabilité en $x = \frac{1}{2}$ fait que les hypothèses du théorème de Rolle ne sont pas remplies, donc il n'y a pas de contradiction.
}

