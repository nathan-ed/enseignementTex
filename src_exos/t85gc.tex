\titre{}
\theme{fonctions}
\auteur{Nathan Scheinmann}
\niveau{1M}
\source{}
\type{serie}
\piments{2}
\pts{}
\annee{2425}

\contenu{
	\tcblower
Déterminer dans chaque cas la (ou une) fonction affine $f$ dont la représentation graphique
\begin{tasks}
\task passe par le point $(3 ; 2)$ et dont la pente vaut $4$ ;
\task est parallèle à la droite d'équation $y=-x+7$ et passe par le point $(-6 ; 8)$;
\task passe par les points $(5 ; 6)$ et $(-9 ; 5)$;
\task est perpendiculaire à la droite d'équation $y=3 x-4$;
\task est perpendiculaire à la droite d'équation $y=3 x-4$ et passe par le point $(2 ; 0)$;
\task est de pente $-\dfrac{1}{2}$ et telle que $f\left(\dfrac{1}{2}\right)=-2$.
\end{tasks}
}
\correction{
	\tcblower
\begin{tasks}(3)
\task  $y=4x-10$
\task $y=-x+2$.
\task  $y=\dfrac{1}{14}x+\dfrac{79}{14}$.
\task $y=\dfrac{-1}{3}x-4$.
\task $y=-\dfrac{1}{3}x+\dfrac{2}{3}$.
\task  $y=-\dfrac{1}{2}x-\dfrac{7}{8}$.
\end{tasks}
}
