\titre{31}
\theme{trigo}
\auteur{Nathan Scheinmann}
\niveau{1M}
\source{sesamath-1M-trigo}
\type{serie}
\piments{2}
\pts{}
\annee{2425}

\contenu{
	\tcblower
	Soit un cercle de centre $C$ et de rayon $6~\text{cm}$. De $A$, un point extérieur au cercle, part une demi-droite $d_{AF}$ issue de $A$ et tangente au cercle au point $D$.
	On a un triangle $\triangle CAD$, un triangle $\triangle CAE$ avec $CE$ perpendiculare à $CA$, et un triangle $\triangle AGF$ avec $G$ un point commun à $CA$ et à la tangente $GF$. 

	\begin{center}
		\includegraphics[scale=1]{../medias/1M/trigo/1M-exo-31}
	\end{center}

	Calculer les longueurs des trois côtés de ces trois triangles. 
}
\correction{
\tcblower
\textit{Generated by AI}

D'après la figure, nous avons besoin de calculer les côtés de trois triangles. Sans voir exactement tous les détails numériques de la figure, voici la méthode générale pour ce type d'exercice :

\textbf{Triangle $\triangle CAD$ (rayon et tangente) :}

Le triangle $CAD$ est rectangle en $D$ (car la tangente est perpendiculaire au rayon).
\begin{itemize}
\item $CD = 6~\text{cm}$ (rayon)
\item Si l'on connaît un angle, par exemple $\widehat{CAD}$, on peut calculer :
\item $AD = CA \cdot \sin(\widehat{CAD})$ ou utiliser Pythagore si $CA$ est donné
\item $CA = \sqrt{CD^2 + AD^2}$ (Pythagore)
\end{itemize}

\textbf{Triangle $\triangle CAE$ (avec $CE \perp CA$) :}

Ce triangle est rectangle en $A$.
\begin{itemize}
\item $CA$ déjà calculé ci-dessus
\item $CE = 6~\text{cm}$ (rayon)
\item $AE = \sqrt{CE^2 - CA^2}$ ou selon les angles donnés
\end{itemize}

\textbf{Triangle $\triangle AGF$ :}

$G$ est sur $CA$ et $GF$ est tangente au cercle.
\begin{itemize}
\item Utiliser la propriété que $GF \perp CG$ au point de tangence
\item $CG = 6~\text{cm}$ (rayon)
\item Calculer selon la position de $G$ sur $CA$
\end{itemize}

\textbf{Note :} Pour obtenir les valeurs numériques exactes, il faudrait les mesures ou angles spécifiques indiqués sur la figure.
}

