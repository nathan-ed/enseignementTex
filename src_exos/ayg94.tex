\titre{}
\theme{derivation}
\auteur{Nathan Scheinmann}
\niveau{3M}
\source{sesamath3e}
\type{serie}
\piments{2}
\pts{}
\annee{2425}

\contenu{
\tcblower
Trouver l’équation de la droite tangente à la parabole \(y=x^2\) au point \((-1;1)\).
}
\correction{
\tcblower
\textit{Generated by AI}

Pour trouver l'équation de la tangente à $y = x^2$ au point $(-1, 1)$, nous avons besoin de la pente de la tangente en ce point.

La pente de la tangente est donnée par la dérivée :
\[y' = 2x\]

Au point $x = -1$ :
\[m = y'(-1) = 2(-1) = -2\]

L'équation de la tangente passant par le point $(-1, 1)$ avec une pente $m = -2$ est :
\begin{align*}
y - y_0 &= m(x - x_0) \\
y - 1 &= -2(x - (-1)) \\
y - 1 &= -2(x + 1) \\
y - 1 &= -2x - 2 \\
y &= -2x - 1
\end{align*}

\textbf{Réponse :} \boxed{y = -2x - 1}
}

