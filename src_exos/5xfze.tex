\titre{}
\theme{dérivées}
\auteur{Nathan Scheinmann}
\niveau{3M}
\source{fundamentum}
\type{serie}
\piments{2}
\pts{}
\annee{2425}

\contenu{
\tcblower
Deux usines situées à 10 km l'une de l'autre émettent des fumées polluantes. On suppose que la pollution provoquée par chaque usine en un endroit est proportionnelle à la quantité de fumée, et inversement proportionnelle au cube de la distance à l'usine. Si la première usine rejette trois fois plus de fumée que la seconde, quel est l'endroit le moins pollué situé entre les deux usines ?
}
\correction{

\tcblower
\textit{Generated by AI}
{\scriptsize \textit{Correction générée par IA}}

Plaçons les deux usines sur un axe : la première en $x = 0$ et la deuxième en $x = 10$.

Pour un point situé en position $x$ sur l'intervalle $[0, 10]$ :
\begin{itemize}
\item La distance à la première usine est $d_1 = x$
\item La distance à la deuxième usine est $d_2 = 10 - x$
\end{itemize}

La pollution provenant de chaque usine est proportionnelle à la quantité de fumée et inversement proportionnelle au cube de la distance :
\[P_1 = \dfrac{kQ_1}{x^3} \quad \text{et} \quad P_2 = \dfrac{kQ_2}{(10-x)^3}\]

où $k$ est une constante de proportionnalité, et $Q_1, Q_2$ les quantités de fumée.

Sachant que $Q_1 = 3Q_2$, la pollution totale est :
\[P(x) = \dfrac{kQ_2 \cdot 3}{x^3} + \dfrac{kQ_2}{(10-x)^3} = kQ_2\left(\dfrac{3}{x^3} + \dfrac{1}{(10-x)^3}\right)\]

Pour minimiser $P(x)$, nous dérivons par rapport à $x$ :
\[P'(x) = kQ_2\left(-\dfrac{9}{x^4} + \dfrac{3}{(10-x)^4}\right)\]

L'annulation de la dérivée donne :
\[-\dfrac{9}{x^4} + \dfrac{3}{(10-x)^4} = 0\]
\[\dfrac{3}{(10-x)^4} = \dfrac{9}{x^4}\]
\[\dfrac{1}{(10-x)^4} = \dfrac{3}{x^4}\]
\[x^4 = 3(10-x)^4\]
\[x = \sqrt[4]{3} \cdot (10-x)\]
\[x = 10\sqrt[4]{3} - x\sqrt[4]{3}\]
\[x(1 + \sqrt[4]{3}) = 10\sqrt[4]{3}\]
\[x = \dfrac{10\sqrt[4]{3}}{1 + \sqrt[4]{3}}\]

Simplifions en multipliant numérateur et dénominateur par $\dfrac{1}{\sqrt[4]{3}}$ :
\[x = \dfrac{10}{1/\sqrt[4]{3} + 1} = \dfrac{10\sqrt[4]{3}}{\sqrt[4]{3} + 1}\]

Numériquement : $\sqrt[4]{3} \approx 1{,}316$, donc :
\[x \approx \dfrac{10 \times 1{,}316}{1{,}316 + 1} = \dfrac{13{,}16}{2{,}316} \approx 5{,}68 \text{ km}\]

L'endroit le moins pollué est donc situé à environ $\boxed{5{,}68 \text{ km}}$ de la première usine (celle qui rejette le plus de fumée).

Remarque : Le point est plus proche de l'usine la plus polluante (environ 5,68 km de la première) que de la seconde (environ 4,32 km de la deuxième), car l'effet de la distance (au cube) compense l'effet de la quantité de fumée.

}

