\titre{}
\theme{derivation}
\auteur{Nathan Scheinmann}
\niveau{3M}
\source{sesamath3e}
\type{serie}
\piments{2}
\pts{}
\annee{2425}

\contenu{
\tcblower
Déterminer les asymptotes horizontales des fonctions suivantes~: 

\begin{tasks}(2)
\task $f(x) = \dfrac{3x^4 - 5x}{x^2 + 1}$
\task $f(x) = \dfrac{3x^2 - 5x}{x^2 + 1}$
\task $f(x) = \dfrac{x^2 - 3x + 2}{2x^2 - 4x}$
\task $f(x) = \dfrac{x^2 + 2x + 3}{4 - x^2}$
\end{tasks}
{\itshape Pourquoi parle-t-on d'asymptotes horizontales dans ces cas~?} 
}
\correction{
\tcblower
\begin{tasks}(4)
	\task $\emptyset$
	\task $y=3$ 
	\task $y=\dfrac{1}{2}$ 
	\task $y=-1$
\end{tasks}
}

