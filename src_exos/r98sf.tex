\titre{}
\theme{derivation}
\auteur{Nathan Scheinmann}
\niveau{3M}
\source{sesamath3e}
\type{serie}
\piments{2}
\pts{}
\annee{2425}

\contenu{
\tcblower
Esquisser une représentation graphique d'une fonction $f$ dont voici le tableau des signes de la dérivée :

\begin{tabular}{|c|c|c|c|c|c|c|c|}
\hline
$x$ & & $-1$ & & $1$ & & $2$ & \\
\hline
$f'(x)$ & $+$ & $0$ & $-$ & $0$ & $-$ & $0$ & $+$ \\
\hline
\end{tabular}
}
\correction{
\tcblower
\textit{Generated by AI}

D'après le tableau des signes de $f'(x)$ :

\begin{itemize}
\item Pour $x < -1$ : $f'(x) > 0$, donc $f$ est strictement croissante
\item En $x = -1$ : $f'(-1) = 0$, donc $x = -1$ est un point critique (maximum local)
\item Pour $-1 < x < 1$ : $f'(x) < 0$, donc $f$ est strictement décroissante
\item En $x = 1$ : $f'(1) = 0$, donc $x = 1$ est un point critique (minimum local, car $f'$ ne change pas de signe)
\item Pour $1 < x < 2$ : $f'(x) < 0$, donc $f$ continue à être strictement décroissante
\item En $x = 2$ : $f'(2) = 0$, donc $x = 2$ est un point critique (minimum local)
\item Pour $x > 2$ : $f'(x) > 0$, donc $f$ est strictement croissante
\end{itemize}

\textbf{Caractéristiques de la courbe :}

\begin{itemize}
\item Maximum local en $x = -1$
\item La fonction décroît de $x = -1$ à $x = 2$
\item Point d'inflexion ou palier en $x = 1$ (la dérivée s'annule mais ne change pas de signe)
\item Minimum local en $x = 2$
\item Croissance après $x = 2$
\end{itemize}

\textbf{Esquisse graphique :}

La courbe monte jusqu'à $x = -1$, atteint un maximum, puis descend en passant par $x = 1$ (où la tangente est horizontale mais la fonction continue de décroître), atteint un minimum en $x = 2$, puis remonte.

\textit{Pour tracer : placer les points critiques en $x = -1, 1, 2$ avec des tangentes horizontales, puis relier en respectant la monotonie décrite ci-dessus.}
}

