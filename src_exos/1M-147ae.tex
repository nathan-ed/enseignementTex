\titre{}
\theme{dérivées}
\auteur{Nathan Scheinmann}
\niveau{3M}
\source{analysis}
\type{serie}
\piments{2}
\pts{}
\annee{2425}

\contenu{
\tcblower
Après avoir vérifié que la fonction satisfait les conditions du théorème des accroissements finis sur l'intervalle indiqué, trouver les valeurs prédite pour $c$ par le théorème sur l'intervalle $[a;b]$ donné.

\begin{tasks}(1)
  \task \(f(x) = x^2\quad [1;2]\)
  \task \(f(x) = x^2\quad [1;3]\)
  \task \(f(x) = x^3\quad [0;1]\)
  \task \(f(x) = \sqrt{x^2+4x}\quad [1;4]\)
  \task \(f(x) = \sqrt[3]{x^2+1}\quad [1;3]\)
  \task \(f(x) = x^{2/3} \quad [1;8]\)
\end{tasks}
}
\correction{
\tcblower
}

