\titre{}
\theme{derivation}
\auteur{Nathan Scheinmann}
\niveau{3M}
\source{musy}
\type{serie}
\piments{1}
\pts{}
\annee{2526}

\contenu{
\tcblower
Déterminer l'équation de la tangente à la fonction $f(x)=3 x^2-5 x+7$ qui est parallèle à la droite d'équation $x-2 y-3=0$. (lllustrer graphiquement)
}
\correction{
\tcblower
{\scriptsize \textit{Correction générée par IA}}

Réécrivons d'abord la droite sous la forme $y=mx+p$ :
\[
x-2y-3=0 \quad \Rightarrow \quad 2y=x-3 \quad \Rightarrow \quad y=\dfrac{1}{2}x-\dfrac{3}{2}
\]

La pente de cette droite est $m=\dfrac{1}{2}$. Une tangente parallèle à cette droite aura donc la même pente.

Calculons la dérivée de $f$ :
\[
f'(x) = 6x-5
\]

Cherchons le point où $f'(x)=\dfrac{1}{2}$ :
\[
6x-5=\dfrac{1}{2} \quad \Rightarrow \quad 6x=\dfrac{11}{2} \quad \Rightarrow \quad x=\dfrac{11}{12}
\]

Pour $x=\dfrac{11}{12}$, on a :
\[
f\left(\dfrac{11}{12}\right)=3\left(\dfrac{11}{12}\right)^2-5\cdot\dfrac{11}{12}+7=3\cdot\dfrac{121}{144}-\dfrac{55}{12}+7=\dfrac{121}{48}-\dfrac{220}{48}+\dfrac{336}{48}=\dfrac{237}{48}=\dfrac{79}{16}
\]

L'équation de la tangente est :
\[
y=f'\left(\dfrac{11}{12}\right)\left(x-\dfrac{11}{12}\right)+f\left(\dfrac{11}{12}\right)=\dfrac{1}{2}\left(x-\dfrac{11}{12}\right)+\dfrac{79}{16}=\dfrac{1}{2}x-\dfrac{11}{24}+\dfrac{79}{16}
\]

En réduisant au même dénominateur :
\[
y=\dfrac{1}{2}x-\dfrac{22}{48}+\dfrac{237}{48}=\dfrac{1}{2}x+\dfrac{215}{48}
\]

La tangente a pour équation $y=\dfrac{1}{2}x+\dfrac{215}{48}$.
}

