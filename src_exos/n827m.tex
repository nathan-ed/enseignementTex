\titre{21}
\theme{trigo}
\auteur{Nathan Scheinmann}
\niveau{1M}
\source{sesamath-1M-trigo}
\type{serie}
\piments{2}
\pts{}
\annee{2425}

\contenu{
	\tcblower
	Quelle est la longueur de l'ombre projetée par un arbre de $12~\text{m}$ de haut lorsque le soleil est élevé de $52^\circ$ au-dessus de l'horizon~?
}
\correction{
\tcblower
\textit{Generated by AI}

L'arbre de hauteur $h = 12~\text{m}$ projette une ombre au sol. Le soleil forme un angle de $52°$ au-dessus de l'horizon.

\textbf{Configuration :}

Nous avons un triangle rectangle où :
\begin{itemize}
\item La hauteur verticale de l'arbre est $h = 12~\text{m}$
\item L'ombre au sol est la longueur $L$ que nous cherchons
\item L'angle entre le rayon du soleil et l'horizontale est $52°$
\end{itemize}

\textbf{Relation trigonométrique :}

Dans ce triangle rectangle, on a :
\[
\tan(52°) = \frac{h}{L} = \frac{12}{L}
\]

D'où :
\[
L = \frac{12}{\tan(52°)}
\]

\textbf{Calcul numérique :}

\[
L = \frac{12}{\tan(52°)} \approx \frac{12}{1{,}2799} \approx 9{,}38~\text{m}
\]

\textbf{Réponse :} La longueur de l'ombre est d'environ $9{,}4~\text{m}$.
}

