\titre{8}
\theme{trigo}
\auteur{Nathan Scheinmann}
\niveau{1M}
\source{sesamath-1M-trigo}
\type{serie}
\piments{2}
\pts{}
\annee{2425}

\contenu{
	\tcblower
Caluler l'aide de ce trapèze.
\begin{center}
\includegraphics[scale=1]{../medias/1M/trigo/1M-exo-8}	
\end{center}
}
\correction{
\tcblower
\textit{Generated by AI}

Pour calculer l'aire du trapèze, nous devons utiliser la formule :
\[
\mathcal{A} = \frac{(b + B) \times h}{2}
\]

où $b$ et $B$ sont les bases parallèles et $h$ est la hauteur.

D'après la figure, nous devons déterminer la hauteur du trapèze en utilisant la trigonométrie.

Si l'on trace la hauteur depuis le sommet, on forme un triangle rectangle. En utilisant les données de la figure (qu'il faudrait spécifier), on peut calculer :

\textbf{Méthode générale :}

Si le trapèze a :
\begin{itemize}
\item Une petite base $b$
\item Une grande base $B$
\item Un côté oblique de longueur $c$ formant un angle $\alpha$ avec la base
\end{itemize}

Alors la hauteur est : $h = c \sin(\alpha)$

Et l'aire est :
\[
\mathcal{A} = \frac{(b + B) \times c \sin(\alpha)}{2}
\]

\textit{Note : Pour donner une réponse numérique précise, il faudrait les dimensions exactes de la figure.}
}

