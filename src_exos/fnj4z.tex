\titre{}
\theme{derivation}
\auteur{Nathan Scheinmann}
\niveau{3M}
\source{sesamath3e}
\type{serie}
\piments{2}
\pts{}
\annee{2425}

\contenu{
\tcblower
Déterminer les coordonnées des points de la courbe d'équation $y = 1 - x^2$ les plus proches de l'origine.
}
\correction{
\tcblower
\textit{Generated by AI}

Nous cherchons le point $(x, y)$ sur la courbe $y = 1 - x^2$ le plus proche de l'origine $(0, 0)$.

La distance au carré de $(x, y)$ à l'origine est :
\[D^2(x) = x^2 + y^2 = x^2 + (1 - x^2)^2\]

Développons :
\[D^2(x) = x^2 + 1 - 2x^2 + x^4 = x^4 - x^2 + 1\]

Pour minimiser la distance, minimisons $D^2$ (c'est équivalent et plus simple).

Dérivons :
\[\frac{d(D^2)}{dx} = 4x^3 - 2x = 2x(2x^2 - 1)\]

Les points critiques sont : $x = 0$ ou $x^2 = \frac{1}{2}$, donc $x = \pm\frac{1}{\sqrt{2}} = \pm\frac{\sqrt{2}}{2}$

Vérifions avec la dérivée seconde :
\[\frac{d^2(D^2)}{dx^2} = 12x^2 - 2\]

Pour $x = 0$ : $\frac{d^2(D^2)}{dx^2} = -2 < 0$ (maximum local)

Pour $x = \pm\frac{\sqrt{2}}{2}$ : $\frac{d^2(D^2)}{dx^2} = 12 \cdot \frac{1}{2} - 2 = 4 > 0$ (minimum local)

Calculons $y$ pour $x = \pm\frac{\sqrt{2}}{2}$ :
\[y = 1 - \left(\frac{\sqrt{2}}{2}\right)^2 = 1 - \frac{1}{2} = \frac{1}{2}\]

\textbf{Réponse :} Les points les plus proches de l'origine sont \boxed{\left(\pm\frac{\sqrt{2}}{2}, \frac{1}{2}\right)}
}

