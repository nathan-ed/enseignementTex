\titre{}
\theme{derivation}
\auteur{Nathan Scheinmann}
\niveau{3M}
\source{sesamath3e}
\type{serie}
\piments{2}
\pts{}
\annee{2425}

\contenu{
\tcblower
Pour chacune des fonctions représentées  ci-dessous, représenter graphiquement la fonction dérivée dans le même repère :
GRAPHIC
}
\correction{
\tcblower
\textit{Generated by AI}

Pour représenter graphiquement la dérivée $f'$ à partir du graphe de $f$, il faut analyser :

\textbf{Principes généraux :}
\begin{itemize}
\item Quand $f$ est croissante, $f' > 0$ (au-dessus de l'axe des $x$)
\item Quand $f$ est décroissante, $f' < 0$ (en-dessous de l'axe des $x$)
\item Aux extrema de $f$ (maximum ou minimum), $f' = 0$ (croise l'axe des $x$)
\item Aux points d'inflexion de $f$ (changement de concavité), $f'$ a un extremum
\item Plus la pente de $f$ est raide, plus $|f'|$ est grand
\end{itemize}

\textbf{Méthode :}
\begin{enumerate}
\item Identifier les points où $f$ a des extrema → $f'$ croise l'axe des $x$
\item Déterminer les intervalles où $f$ est croissante → $f' > 0$
\item Déterminer les intervalles où $f$ est décroissante → $f' < 0$
\item Identifier les points d'inflexion de $f$ → extrema de $f'$
\item Estimer la pente de $f$ en différents points → valeur de $f'$ en ces points
\end{enumerate}

\textbf{Note :} Sans le graphique spécifique sous les yeux, il n'est pas possible de tracer la dérivée exacte, mais cette méthode s'applique à tout graphique donné.
}

