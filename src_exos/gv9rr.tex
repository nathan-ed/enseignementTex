\titre{}
\theme{derivation}
\auteur{Nathan Scheinmann}
\niveau{3M}
\source{sesamath3e}
\type{serie}
\piments{2}
\pts{}
\annee{2425}

\contenu{
\tcblower
Déterminer l'expression algébrique d'une fonction qui admet des points d'inflexion en $x = 1$ et en $x = 2$.
}
\correction{
\tcblower
\textit{Generated by AI}

Un point d'inflexion correspond à un changement de concavité de la fonction, c'est-à-dire à un point où la dérivée seconde s'annule et change de signe.

Pour qu'une fonction admette des points d'inflexion en $x = 1$ et en $x = 2$, il faut que $f''(1) = 0$ et $f''(2) = 0$.

\textbf{Construction d'une fonction polynomiale :}

Considérons un polynôme de degré 3 de la forme :
\[
f(x) = ax^3 + bx^2 + cx + d
\]

Calculons les dérivées :
\begin{align*}
f'(x) &= 3ax^2 + 2bx + c \\
f''(x) &= 6ax + 2b
\end{align*}

Pour avoir des points d'inflexion en $x = 1$ et $x = 2$, il faut :
\begin{align*}
f''(1) = 0 &\Rightarrow 6a + 2b = 0 \\
f''(2) = 0 &\Rightarrow 12a + 2b = 0
\end{align*}

De la première équation : $b = -3a$

De la seconde équation : $b = -6a$

Ces deux équations sont contradictoires, ce qui signifie qu'une fonction polynomiale de degré 3 ne peut pas avoir deux points d'inflexion.

\textbf{Solution avec un polynôme de degré 4 :}

Prenons $f(x) = a(x-1)(x-2)^2 + b(x-1)^2(x-2)$ ou plus simplement :
\[
f(x) = (x-1)(x-2)g(x)
\]

où $g(x)$ est choisi tel que $f''(1) = f''(2) = 0$.

Une solution simple est :
\[
f(x) = (x-1)(x-2)(x-1.5)
\]

ou toute fonction de la forme $f(x) = k(x-1)(x-2)(x-\alpha)$ avec $k \neq 0$ et $\alpha \in \mathbb{R}$.
}

