\titre{}
\theme{numerique}
\auteur{Nathan Scheinmann}
\niveau{1M}
\source{musy}
\type{serie}
\piments{2}
\pts{}
\annee{2425}

\contenu{
\tcblower
Quelle est la millième décimale de chacun des nombres rationnels suivants~?
\begin{inlineumerate}
\item $\dfrac{1}{7}$
\item $\dfrac{17}{41}$
\end{inlineumerate}
}
\correction{
\tcblower
	\begin{tasks}
		\task On a $1\div 7=0,\overline{142857}$, donc une période de six chiffres. On divise $1000$ par $6$ et on obtient $166$ reste $4$. Le millième chiffre après la virgule est le quatrième chiffre de la période soit $8$. 
		\task On a $17\div 41=0,\overline{41463}$, donc une période de cinq chiffres. On divise $1000$ par $5$ et on obtient $200$ reste $0$. Le millième chiffre après la virgule est le cinquième chiffre de la période soit $3$. 
	\end{tasks}

}

