\titre{}
\theme{numerique}
\auteur{Nathan Scheinmann}
\niveau{1M}
\source{musy}
\type{serie}
\piments{2}
\pts{}
\annee{2425}

\contenu{
\tcblower
On considère les fractions 
\[\dfrac{1}{7}, \quad \dfrac{2}{7}, \quad \dfrac{3}{7}, \quad \dfrac{4}{7}, \quad \dfrac{5}{7}, \quad \dfrac{6}{7}.\]
\begin{enumerate}
\item Trouver l'écriture décimale exacte de ces nombres à l'aide d'une calculatrice.
\item Remarquer qu'en plus d'avoir les même chiffres $1,4,2,8,5,7$, ceux-ci sont toujours dans cet ordre de gauche à droite. Par exemple, pour 2, on commence par lire 2, 8, 5, 7, puis on revient au début avec 1, 4. (On dit que les chiffres de la période sont cycliques.)
\item Les fractions dont le dénominateur est 23 ont les mêmes propriétés. Au lieu d'avoir une période cyclique de 6 chiffres, elles en ont 22 .
A l'aide d'une calculatrice uniquement (sans poser la division), trouver les 22 décimales de la période de $\dfrac{22}{23}$.
\end{enumerate}
}
\correction{
\tcblower
\begin{tasks}(2)
		\task* $\dfrac{1}{7}=0,\overline{142857};\,\dfrac{2}{7}=0,\overline{285714};\,\dfrac{3}{7}=0,\overline{428571};\,\dfrac{4}{7}=0,\overline{571428};\,\dfrac{5}{7}=0,\overline{714285};\,\dfrac{6}{7}=0,\overline{857142}$.
		\task À remarquer.
		\task $\dfrac{22}{23}=0,\overline{9565217391304347826086}$
	\end{tasks}
}

