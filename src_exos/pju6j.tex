\titre{}
\theme{derivation}
\auteur{Nathan Scheinmann}
\niveau{3M}
\source{crm}
\type{serie}
\piments{1}
\pts{}
\annee{2526}

\contenu{
\tcblower
Prouver que $\tan'(x)=\dfrac{1}{\cos^2(x)}=1+\tan^2(x)$
}
\correction{
\tcblower
{\scriptsize \textit{Correction générée par IA}}

$\tan(x) = \dfrac{\sin(x)}{\cos(x)}$. on utilise l'identité trigonométrique $\cos^2(x) + \sin^2(x) = 1$~:

\( \begin{aligned}
1 + \tan^2(x) &= 1 + \dfrac{\sin^2(x)}{\cos^2(x)} \\
&= \dfrac{\cos^2(x)}{\cos^2(x)} + \dfrac{\sin^2(x)}{\cos^2(x)} \\
&= \dfrac{\cos^2(x) + \sin^2(x)}{\cos^2(x)} \\
&= \dfrac{1}{\cos^2(x)}
\end{aligned} \)

Donc $\tan'(x) = \dfrac{1}{\cos^2(x)} = 1 + \tan^2(x)$.
}

