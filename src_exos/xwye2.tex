\titre{}
\theme{fonctions}
\auteur{Nathan Scheinmann}
\niveau{1M}
\source{}
\type{serie}
\piments{2}
\pts{}
\annee{2425}

\contenu{
	\tcblower
Déterminer la pente et l'ordonnée à l'origine de la droite $f$ d'équation : $2 x-\frac{3}{2} y+3=0$.
\begin{tasks}
\task Représenter cette droite graphiquement.
\task Calculer ses intersections avec les axes du repère.
\task Trouver l'équation de la droite $g$, parallèle à $f$ et passant par le point $(3 ;-4)$.
\task Trouver l'équation de la droite $h$, perpendiculaire à $f$ et passant par le point $(2;-1)$.
\end{tasks}
}
\correction{
	\tcblower
\begin{tasks}(4)
	\task pente $\dfrac{4}{3}$ et $f(0)=2$
	\task $(0;2)$ et $\left(-\dfrac{2}{3};0\right)$
	\task $g(x)=\dfrac{4}{3}x-8$
	\task $h(x)=-\dfrac{3}{4}x+\dfrac{1}{2}$
\end{tasks}
}

