\titre{}
\theme{derivation}
\auteur{Nathan Scheinmann}
\niveau{3M}
\source{musy}
\type{serie}
\piments{1}
\pts{}
\annee{2526}

\contenu{
\tcblower
\phantom{test}
\begin{tasks}
	\task Donner un exemple d'une fonction continue en $x=2$, mais pas dérivable en $x=2$. Justifier. 
	\task Démontrer qu'une fonction $f$ définie dans un voisinage de $x\in\mathbb{R}$ qui est dérivable en $x$ est continue en $x$. 
	\task Donner un critère visuel sur le graphe d'une fonction continue pour déterminer si elle est dérivable.
\end{tasks}
}
\correction{
\tcblower
\begin{tasks}
	\task Par exemple, $|x-2|$ (pourquoi?)
	\task Démonstration faite en classe.
	\task Par exemple, si la représentation graphique contient un point anguleux en $(x;f(x))$.
\end{tasks}
}

