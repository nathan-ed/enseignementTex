\titre{}
\theme{derivation}
\auteur{Nathan Scheinmann}
\niveau{3M}
\source{musy}
\type{serie}
\piments{1}
\pts{}
\annee{2526}

\contenu{
\tcblower
On considère la fonction $f(x)=\sqrt{x}$. A l'aide de la définition de la dérivée :
\begin{tasks}
\task montrer que $f^{\prime}(0)$ n'existe pas;
\task calculer $f^{\prime}(a)$, pour $a$, un réel strictement positif ($a>0$).
\end{tasks}
}
\correction{
\tcblower
\begin{tasks}
	\task Par définition de la fonction racine, la limite à gauche en $0$ n'est pas calculable, ainsi la dérivée en $0$ n'existe pas. 
	
\task $f(x)=\sqrt{x}$, $x>0$. 
\[
f'(x)=\displaystyle\lim_{h\to 0}\dfrac{\sqrt{x+h}-\sqrt{x}}{h}
=\displaystyle\lim_{h\to 0}\dfrac{1}{\sqrt{x+h}+\sqrt{x}}\cdot
\dfrac{(\sqrt{x+h}-\sqrt{x})(\sqrt{x+h}+\sqrt{x})}{h}
=\dfrac{1}{2\sqrt{x}}.
\]


\end{tasks}
}

