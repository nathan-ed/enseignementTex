\titre{}
\theme{derivation}
\auteur{Nathan Scheinmann}
\niveau{3M}
\source{sesamath3e}
\type{serie}
\piments{2}
\pts{}
\annee{2425}

\contenu{
\tcblower
Soit \(f\) définie par \(f(x)=x^3 - x\).
\begin{tasks}(1)
  \task Déterminer l’équation de la tangente sachant que 2 est l’abscisse du point de tangence.
  \task Déterminer l’équation de la tangente sachant que la pente vaut 2.
  \task Déterminer les \(x\) pour lesquels les tangentes sont horizontales.
\end{tasks}
}
\correction{
\tcblower

\textit{Generated by AI}

Soit $f(x) = x^3 - x$. Calculons d'abord la dérivée :
\[f'(x) = 3x^2 - 1\]

\begin{tasks}(1)
\task \textbf{Équation de la tangente au point d'abscisse $x = 2$}

Calculons $f(2)$ et $f'(2)$ :
\begin{align*}
f(2) &= 2^3 - 2 = 8 - 2 = 6 \\
f'(2) &= 3 \cdot 2^2 - 1 = 12 - 1 = 11
\end{align*}

L'équation de la tangente au point $(2, f(2))$ est :
\[y - f(2) = f'(2)(x - 2)\]
\[y - 6 = 11(x - 2)\]
\[y = 11x - 22 + 6\]
\[y = 11x - 16\]

\task \textbf{Équation de la tangente de pente 2}

La pente de la tangente est donnée par $f'(x) = 2$. Résolvons :
\[3x^2 - 1 = 2\]
\[3x^2 = 3\]
\[x^2 = 1\]
\[x = \pm 1\]

\textbf{Pour $x = 1$ :}
\begin{align*}
f(1) &= 1^3 - 1 = 0 \\
y - 0 &= 2(x - 1) \\
y &= 2x - 2
\end{align*}

\textbf{Pour $x = -1$ :}
\begin{align*}
f(-1) &= (-1)^3 - (-1) = -1 + 1 = 0 \\
y - 0 &= 2(x - (-1)) \\
y &= 2x + 2
\end{align*}

Les tangentes de pente 2 sont : $y = 2x - 2$ et $y = 2x + 2$.

\task \textbf{Valeurs de $x$ pour lesquelles les tangentes sont horizontales}

Une tangente est horizontale lorsque sa pente est nulle, donc $f'(x) = 0$ :
\[3x^2 - 1 = 0\]
\[x^2 = \dfrac{1}{3}\]
\[x = \pm \dfrac{1}{\sqrt{3}} = \pm \dfrac{\sqrt{3}}{3}\]

Les tangentes sont horizontales en $x = \dfrac{\sqrt{3}}{3}$ et $x = -\dfrac{\sqrt{3}}{3}$.
\end{tasks}

}

