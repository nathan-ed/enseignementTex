\titre{}
\theme{derivation}
\auteur{Nathan Scheinmann}
\niveau{3M}
\source{sesamath3e}
\type{serie}
\piments{2}
\pts{}
\annee{2425}

\contenu{
\tcblower
On considère la fonction \(f\) définie par \(f(x)=x(3-x)(x-1)^2\) pour \(x\in[0;3]\). La courbe représentative de cette fonction modélise la vue en coupe de deux sommets des Alpes, qu’on désire relier par un téléphérique.
\begin{tasks}(1)
  \task Déterminer l’équation de la tangente en deux points distincts de cette courbe (les deux points de tangence représentent les stations du téléphérique et le segment de droite compris entre ces deux points représente le câble).
  \task Calculer la longueur du câble.
\end{tasks}
Indication : utiliser un outil de calcul formel pour résoudre le système !
}
\correction{

}

