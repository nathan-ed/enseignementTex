\titre{}
\theme{dérivées}
\auteur{Nathan Scheinmann}
\niveau{3M}
\source{fundamentum}
\type{serie}
\piments{2}
\pts{}
\annee{2425}

\contenu{
\tcblower
On considère la parabole $\gamma$ d'équation $y = 1-x^2$ ainsi qu'un point $M$ de $\gamma$ situé dans le premier quadrant. La tangente à $\gamma$ en $M$ coupe l'axe $Ox$ au point A et l'axe $Oy$ au point B. Déterminer les coordonnées du point M pour que l'aire du triangle OAB soit minimale.
}
\correction{
\tcblower
%% GENERATED BY AI %%
{\scriptsize \textit{Correction générée par IA}}

Soit $M(x, 1-x^2)$ un point de la parabole $\gamma$ dans le premier quadrant, avec $0 < x < 1$.

L'équation de la tangente à $\gamma$ en $M$ est :
\[y - (1-x^2) = -2x(t - x)\]
où $-2x$ est la dérivée de $f(t) = 1-t^2$ évaluée en $x$.

Simplifions :
\[y = -2xt + 2x^2 + 1 - x^2 = -2xt + x^2 + 1\]

Point $A$ (intersection avec l'axe $Ox$, où $y=0$) :
\[0 = -2xt + x^2 + 1 \implies t = \frac{x^2 + 1}{2x}\]
Donc $A\left(\frac{x^2 + 1}{2x}, 0\right)$.

Point $B$ (intersection avec l'axe $Oy$, où $t=0$) :
\[y = x^2 + 1\]
Donc $B(0, x^2 + 1)$.

L'aire du triangle $OAB$ est :
\[S(x) = \frac{1}{2} \cdot \frac{x^2 + 1}{2x} \cdot (x^2 + 1) = \frac{(x^2 + 1)^2}{4x}\]

Pour minimiser $S(x)$, dérivons :
\[S'(x) = \frac{4x \cdot 2(x^2 + 1) \cdot 2x - (x^2 + 1)^2 \cdot 4}{16x^2} = \frac{16x^2(x^2 + 1) - 4(x^2 + 1)^2}{16x^2}\]
\[= \frac{4(x^2 + 1)[4x^2 - (x^2 + 1)]}{16x^2} = \frac{(x^2 + 1)(3x^2 - 1)}{4x^2}\]

L'annulation de $S'(x)$ donne $3x^2 - 1 = 0$, soit $x = \frac{1}{\sqrt{3}}$ (puisque $x > 0$).

Donc $\boxed{M\left(\frac{1}{\sqrt{3}}, \frac{2}{3}\right)}$ minimise l'aire du triangle $OAB$.
}

