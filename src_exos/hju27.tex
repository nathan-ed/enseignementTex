\titre{}
\theme{derivation2}
\auteur{Nathan Scheinmann}
\niveau{3M}
\source{sesamath3e}
\type{serie}
\piments{2}
\pts{}
\annee{2425}

\contenu{
\tcblower
Trouver la ou les valeurs prévues par le théorème des accroissements finis pour la fonction \(f\) définie par
\[
  f(x)=\dfrac{1}{x-7}
\]
sur l’intervalle \([7{,}1;7{,}2]\).
}
\correction{
\tcblower
\textit{Generated by AI}

Le théorème des accroissements finis stipule que si $f$ est continue sur $[a;b]$ et dérivable sur $]a;b[$, alors il existe au moins un point $c \in ]a;b[$ tel que :
\[
f'(c) = \frac{f(b) - f(a)}{b - a}
\]

\textbf{Application à la fonction donnée :}

Ici, $f(x) = \dfrac{1}{x-7}$ sur l'intervalle $[7{,}1 ; 7{,}2]$.

Vérifions les conditions :
\begin{itemize}
\item $f$ est continue sur $[7{,}1 ; 7{,}2]$ (car $x > 7$ sur cet intervalle)
\item $f$ est dérivable sur $]7{,}1 ; 7{,}2[$
\end{itemize}

Calculons $f'(x)$ :
\[
f'(x) = -\frac{1}{(x-7)^2}
\]

Calculons le taux d'accroissement :
\begin{align*}
\frac{f(7{,}2) - f(7{,}1)}{7{,}2 - 7{,}1} &= \frac{\dfrac{1}{0{,}2} - \dfrac{1}{0{,}1}}{0{,}1} \\
&= \frac{5 - 10}{0{,}1} \\
&= \frac{-5}{0{,}1} \\
&= -50
\end{align*}

D'après le théorème, il existe $c \in ]7{,}1 ; 7{,}2[$ tel que :
\[
f'(c) = -50
\]

Résolvons :
\begin{align*}
-\frac{1}{(c-7)^2} &= -50 \\
\frac{1}{(c-7)^2} &= 50 \\
(c-7)^2 &= \frac{1}{50} \\
c - 7 &= \pm \frac{1}{\sqrt{50}} = \pm \frac{1}{5\sqrt{2}} = \pm \frac{\sqrt{2}}{10}
\end{align*}

Comme $c \in ]7{,}1 ; 7{,}2[$, on a $c - 7 \in ]0{,}1 ; 0{,}2[$, donc :
\[
c = 7 + \frac{\sqrt{2}}{10} = 7 + \frac{1}{5\sqrt{2}} \approx 7{,}141
\]

\textbf{Réponse :} La valeur prévue par le théorème des accroissements finis est $c = 7 + \dfrac{\sqrt{2}}{10} \approx 7{,}141$.
}

