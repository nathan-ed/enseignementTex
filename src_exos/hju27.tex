\titre{}
\theme{derivation2}
\auteur{Nathan Scheinmann}
\niveau{3M}
\source{sesamath3e}
\type{serie}
\piments{2}
\pts{}
\annee{2425}

\contenu{
\tcblower
Trouver la ou les valeurs prévues par le théorème des accroissements finis pour la fonction \(f\) définie par
\[
  f(x)=\dfrac{1}{x-7}
\]
sur l’intervalle \([7{,}1;7{,}2]\).
}
\correction{
\tcblower
$f(x) = \dfrac{1}{x-7}, [7,1; 7,2]$. $f'(c) = \dfrac{5-10}{0,1} = -50$.
$-\dfrac{1}{(c-7)^2} = -50 \implies (c-7)^2 = \dfrac{1}{50} \implies c = 7 + \dfrac{1}{\sqrt{50}} \approx 7,1414$.
}
