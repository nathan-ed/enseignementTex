\titre{12}
\theme{trigo}
\auteur{Nathan Scheinmann}
\niveau{1M}
\source{sesamath-1M-trigo}
\type{serie}
\piments{2}
\pts{}
\annee{2425}

\contenu{
	\tcblower
Au centre d'un bassin carré de $12$ mètres de côté se trouve un jet d'eau dont l'extrémité apparaît, depuis l'un des sommets du carré, sous un angle d'élévation de $52^\circ$. Quelle est sa hauteur~?
}
\correction{
\tcblower

\textit{Generated by AI}

\textbf{Situation :} Nous avons un bassin carré de $12$ mètres de côté. Depuis un sommet du carré, on observe le jet d'eau situé au centre avec un angle d'élévation de $52°$.

\textbf{Configuration géométrique :}

\begin{center}
\begin{tikzpicture}[scale=0.6]
% Dessiner le carré (vue de dessus simplifiée)
\draw[thick] (0,0) rectangle (6,6);
% Centre du carré
\filldraw (3,3) circle (2pt) node[below left] {C};
% Sommet
\filldraw (0,0) circle (2pt) node[below left] {A};
% Distance au centre
\draw[dashed] (0,0) -- (3,3) node[midway, above left] {$d$};
% Triangle vertical (vue de côté)
\begin{scope}[shift={(9,0)}]
\draw[thick] (0,0) -- (4,0) node[midway, below] {$d$};
\draw[thick] (4,0) -- (4,3) node[midway, right] {$h$};
\draw[thick, dashed] (0,0) -- (4,3);
\draw (0.8,0) arc (0:36.87:0.8) node[midway, right] {\small $52°$};
\filldraw (0,0) circle (2pt) node[below] {A};
\filldraw (4,3) circle (2pt) node[above] {J};
\end{scope}
\end{tikzpicture}
\end{center}

\textbf{Étape 1 : Calculer la distance du sommet au centre du carré}

Le centre d'un carré de côté $12$ m est situé à égale distance de tous les sommets. La distance $d$ du sommet $A$ au centre $C$ est la moitié de la diagonale du carré.

La diagonale d'un carré de côté $a$ vaut $a\sqrt{2}$.

\[d = \frac{12\sqrt{2}}{2} = 6\sqrt{2} \text{ m}\]

\textbf{Étape 2 : Calculer la hauteur du jet d'eau}

Nous avons un triangle rectangle où :
\begin{itemize}
\item La base (distance horizontale) est $d = 6\sqrt{2}$ m
\item L'angle d'élévation est $52°$
\item La hauteur $h$ est le côté opposé à l'angle
\end{itemize}

Utilisons la tangente :
\[\tan(52°) = \frac{h}{d} = \frac{h}{6\sqrt{2}}\]

\[h = 6\sqrt{2} \times \tan(52°)\]

Calculons numériquement :
\[h = 6\sqrt{2} \times 1{,}2799... = 8{,}485 \times 1{,}2799 \approx 10{,}86 \text{ m}\]

\textbf{Réponse :} La hauteur du jet d'eau est d'environ $\boxed{10{,}9 \text{ m}}$ (ou $\boxed{6\sqrt{2} \times \tan(52°) \text{ m}}$ en valeur exacte).

}

