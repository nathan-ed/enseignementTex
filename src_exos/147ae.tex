\titre{}
\theme{dérivées}
\auteur{Nathan Scheinmann}
\niveau{3M}
\source{analysis}
\type{serie}
\piments{2}
\pts{}
\annee{2425}

\contenu{
\tcblower
Après avoir vérifié que la fonction satisfait les conditions du théorème des accroissements finis sur l'intervalle indiqué, trouver les valeurs prédite pour $c$ par le théorème sur l'intervalle $[a;b]$ donné.

\begin{tasks}(2)
  \task \(f(x) = x^2\quad [1;2]\)
  \task \(f(x) = x^2\quad [1;3]\)
  \task \(f(x) = x^3\quad [0;1]\)
  \task \(f(x) = \sqrt{x^2+4x}\quad [1;4]\)
  \task \(f(x) = \sqrt[3]{x^2+1}\quad [1;3]\)
  \task \(f(x) = x^{2/3} \quad [1;8]\)
\end{tasks}
}
\correction{
\tcblower
\textit{Generated by AI}

Le théorème des accroissements finis stipule que si $f$ est continue sur $[a;b]$ et dérivable sur $]a;b[$, alors il existe au moins un point $c \in ]a;b[$ tel que :
\[f'(c) = \frac{f(b) - f(a)}{b - a}\]

\begin{tasks}(1)
\task $f(x) = x^2$ sur $[1;2]$ :

$f$ est un polynôme, donc continue sur $[1;2]$ et dérivable sur $]1;2[$.

$f'(x) = 2x$, $f(1) = 1$, $f(2) = 4$.

\[f'(c) = \frac{4-1}{2-1} = 3 \implies 2c = 3 \implies \boxed{c = \frac{3}{2}}\]

\task $f(x) = x^2$ sur $[1;3]$ :

$f$ est continue sur $[1;3]$ et dérivable sur $]1;3[$.

\[f'(c) = \frac{9-1}{3-1} = 4 \implies 2c = 4 \implies \boxed{c = 2}\]

\task $f(x) = x^3$ sur $[0;1]$ :

$f$ est continue sur $[0;1]$ et dérivable sur $]0;1[$.

$f'(x) = 3x^2$, $f(0) = 0$, $f(1) = 1$.

\[f'(c) = \frac{1-0}{1-0} = 1 \implies 3c^2 = 1 \implies c^2 = \frac{1}{3} \implies \boxed{c = \frac{1}{\sqrt{3}} = \frac{\sqrt{3}}{3}}\]

\task $f(x) = \sqrt{x^2+4x}$ sur $[1;4]$ :

Sur $[1;4]$, on a $x^2+4x = x(x+4) > 0$, donc $f$ est bien définie, continue et dérivable.

$f'(x) = \frac{2x+4}{2\sqrt{x^2+4x}} = \frac{x+2}{\sqrt{x^2+4x}}$, $f(1) = \sqrt{5}$, $f(4) = \sqrt{32} = 4\sqrt{2}$.

\[f'(c) = \frac{4\sqrt{2}-\sqrt{5}}{3} \implies \frac{c+2}{\sqrt{c^2+4c}} = \frac{4\sqrt{2}-\sqrt{5}}{3}\]

En résolvant numériquement : $\boxed{c \approx 2{,}30}$

\task $f(x) = \sqrt[3]{x^2+1}$ sur $[1;3]$ :

$f$ est continue et dérivable sur tout $\mathbb{R}$.

$f'(x) = \frac{2x}{3(x^2+1)^{2/3}}$, $f(1) = \sqrt[3]{2}$, $f(3) = \sqrt[3]{10}$.

\[f'(c) = \frac{\sqrt[3]{10}-\sqrt[3]{2}}{2} \implies \frac{2c}{3(c^2+1)^{2/3}} = \frac{\sqrt[3]{10}-\sqrt[3]{2}}{2}\]

En résolvant numériquement : $\boxed{c \approx 2{,}03}$

\task $f(x) = x^{2/3}$ sur $[1;8]$ :

$f$ est continue sur $[1;8]$ et dérivable sur $]1;8[$ (car $x > 0$).

$f'(x) = \frac{2}{3}x^{-1/3} = \frac{2}{3\sqrt[3]{x}}$, $f(1) = 1$, $f(8) = 4$.

\[f'(c) = \frac{4-1}{8-1} = \frac{3}{7} \implies \frac{2}{3\sqrt[3]{c}} = \frac{3}{7} \implies \sqrt[3]{c} = \frac{14}{9} \implies \boxed{c = \left(\frac{14}{9}\right)^3 = \frac{2744}{729} \approx 3{,}76}\]
\end{tasks}
}

