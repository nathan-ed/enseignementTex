\titre{}
\theme{derivation}
\auteur{Nathan Scheinmann}
\niveau{3M}
\source{sesamath3e}
\type{serie}
\piments{2}
\pts{}
\annee{2425}

\contenu{
\tcblower
Trouver la ou les valeurs prévues par le
théorème des accroissements finis pour la
fonction $f(x) = x^2 - x$ sur
l'intervalle $[-3;4]$
}
\correction{
\tcblower

\textit{Generated by AI}

Le théorème des accroissements finis énonce que si $f$ est continue sur $[a,b]$ et dérivable sur $]a,b[$, alors il existe au moins un point $c \in ]a,b[$ tel que :
\[f'(c) = \dfrac{f(b) - f(a)}{b - a}\]

\textbf{Application à $f(x) = x^2 - x$ sur $[-3, 4]$ :}

\textbf{Étape 1 :} Calculons $f(-3)$ et $f(4)$ :
\begin{align*}
f(-3) &= (-3)^2 - (-3) = 9 + 3 = 12 \\
f(4) &= 4^2 - 4 = 16 - 4 = 12
\end{align*}

\textbf{Étape 2 :} Calculons le taux d'accroissement :
\[\dfrac{f(4) - f(-3)}{4 - (-3)} = \dfrac{12 - 12}{7} = 0\]

\textbf{Étape 3 :} Calculons la dérivée de $f$ :
\[f'(x) = 2x - 1\]

\textbf{Étape 4 :} Trouvons $c$ tel que $f'(c) = 0$ :
\[2c - 1 = 0 \implies c = \dfrac{1}{2}\]

\textbf{Vérification :} $c = \dfrac{1}{2} \in ]-3, 4[$, donc la condition est vérifiée.

\textbf{Réponse :} La valeur prévue par le théorème des accroissements finis est $c = \dfrac{1}{2}$.

}

