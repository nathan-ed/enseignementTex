\titre{}
\theme{derivation}
\auteur{Nathan Scheinmann}
\niveau{3M}
\source{sesamath3e}
\type{serie}
\piments{2}
\pts{}
\annee{2425}

\contenu{
\tcblower
Trouver l'angle d'intersection des courbes représentatives des fonctions $f$ et $g$ définies par $f(x) = x^2$ et $g(x) = \dfrac{1}{3}x + 1$.
}
\correction{
\tcblower
\textit{Generated by AI}

Pour trouver l'angle d'intersection des courbes $f(x) = x^2$ et $g(x) = \frac{1}{3}x + 1$, nous devons :

\textbf{1. Trouver les points d'intersection :}

Résolvons $f(x) = g(x)$ :
\[x^2 = \frac{1}{3}x + 1\]
\[3x^2 = x + 3\]
\[3x^2 - x - 3 = 0\]

En utilisant la formule quadratique :
\[x = \frac{1 \pm \sqrt{1 + 36}}{6} = \frac{1 \pm \sqrt{37}}{6}\]

Prenons $x_0 = \frac{1 + \sqrt{37}}{6} \approx 1{,}18$

\textbf{2. Calculer les pentes des tangentes :}

Pour $f$ : $f'(x) = 2x$, donc $f'(x_0) = 2 \cdot \frac{1 + \sqrt{37}}{6} = \frac{1 + \sqrt{37}}{3}$

Pour $g$ : $g'(x) = \frac{1}{3}$ (constante)

\textbf{3. Calculer l'angle entre les tangentes :}

L'angle $\theta$ entre deux droites de pentes $m_1$ et $m_2$ est donné par :
\[\tan(\theta) = \left|\frac{m_1 - m_2}{1 + m_1 m_2}\right|\]

\[\tan(\theta) = \left|\frac{\frac{1+\sqrt{37}}{3} - \frac{1}{3}}{1 + \frac{1+\sqrt{37}}{3} \cdot \frac{1}{3}}\right| = \left|\frac{\frac{\sqrt{37}}{3}}{1 + \frac{1+\sqrt{37}}{9}}\right| = \left|\frac{\frac{\sqrt{37}}{3}}{\frac{10+\sqrt{37}}{9}}\right| = \left|\frac{3\sqrt{37}}{10+\sqrt{37}}\right|\]

Numériquement : $\tan(\theta) \approx 1{,}17 \implies \boxed{\theta \approx 49{,}5°}$
}

