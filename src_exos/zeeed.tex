\titre{}
\theme{derivation}
\auteur{Nathan Scheinmann}
\niveau{3M}
\source{sesamath3e}
\type{serie}
\piments{2}
\pts{}
\annee{2425}

\contenu{
\tcblower
Pour chacune des affirmations suivantes, indiquer si elle est vraie ou fausse. Justifier très clairement chaque réponse en vous appuyant sur les définitions, propriétés et théorèmes vus au cours ou sur un contre-exemple précis :

\begin{tasks}(1)
\task Si $f'(a) = 0$ et $f''(a) = 0$, alors $f$ admet en $a$ un point d'inflexion à tangente horizontale (un "palier").
\task Soit $f : \mathbb{R} \to \mathbb{R}$ une fonction continue. Si $f$ admet un maximum local en $a$ et un minimum local en $d$, alors $f$ admet un point d'inflexion dans $\left]c ; d\right[$.
\task Il existe une fonction $f : \mathbb{R} \to \mathbb{R}$ qui admet un maximum en $a$ et telle que $f''(a) = 0$.
\task Si $f$ est une fonction polynomiale de degré 3 et si $f''(a) = 0$, alors $f$ admet un point d'inflexion en $a$.
\end{tasks}
}
\correction{
\tcblower
\textit{Generated by AI}

\begin{tasks}(1)
\task \textbf{Affirmation :} Si $f'(a) = 0$ et $f''(a) = 0$, alors $f$ admet en $a$ un point d'inflexion à tangente horizontale (un "palier").

\textbf{Faux.} Contre-exemple : Considérons $f(x) = x^4$.

Calculons : $f'(x) = 4x^3$ donc $f'(0) = 0$.

$f''(x) = 12x^2$ donc $f''(0) = 0$.

Cependant, $f''(x) = 12x^2 \geq 0$ pour tout $x \in \mathbb{R}$, donc $f''$ ne change pas de signe en $x = 0$. Il n'y a donc \textbf{pas} de point d'inflexion en $a = 0$. La fonction a un minimum local en $0$.

\textbf{Conclusion :} L'affirmation est fausse. Les conditions $f'(a) = 0$ et $f''(a) = 0$ ne suffisent pas pour garantir un point d'inflexion.

\task \textbf{Affirmation :} Soit $f : \mathbb{R} \to \mathbb{R}$ une fonction continue. Si $f$ admet un maximum local en $a$ et un minimum local en $d$, alors $f$ admet un point d'inflexion dans $\left]a ; d\right[$.

\textbf{Faux.} Contre-exemple : Considérons la fonction $f(x) = x^4 - 2x^2$.

Calculons : $f'(x) = 4x^3 - 4x = 4x(x^2 - 1) = 4x(x-1)(x+1)$.

Les points critiques sont $x = -1, 0, 1$.

Analysons : $f(-1) = 1 - 2 = -1$ (minimum local), $f(0) = 0$ (maximum local), $f(1) = 1 - 2 = -1$ (minimum local).

Donc $f$ a un maximum en $a = 0$ et un minimum en $d = 1$. Vérifions s'il y a un point d'inflexion dans $]0;1[$ :

$f''(x) = 12x^2 - 4$. Pour qu'il y ait un point d'inflexion, $f''$ doit changer de signe.

$f''(x) = 0 \implies x^2 = \frac{1}{3} \implies x = \pm \frac{1}{\sqrt{3}} \approx \pm 0{,}577$.

Le point $x = \frac{1}{\sqrt{3}} \in ]0;1[$ est bien un point d'inflexion.

\textbf{Attention :} Ce contre-exemple ne fonctionne pas ! Cherchons un meilleur exemple.

Considérons plutôt $f(x) = -x^4 + x^2$ sur un intervalle restreint, ou une fonction par morceaux. En réalité, pour des fonctions polynomiales, l'affirmation tend à être vraie.

\textbf{Correction :} L'affirmation est \textbf{généralement vraie} pour les fonctions suffisamment régulières (polynômes, fonctions $C^2$), mais elle n'est pas universellement vraie pour toutes les fonctions continues. Il faudrait des hypothèses supplémentaires (comme $f \in C^2$).

\task \textbf{Affirmation :} Il existe une fonction $f : \mathbb{R} \to \mathbb{R}$ qui admet un maximum en $a$ et telle que $f''(a) = 0$.

\textbf{Vrai.} Exemple : $f(x) = -x^4$.

$f'(x) = -4x^3$, donc $f'(0) = 0$.

$f''(x) = -12x^2$, donc $f''(0) = 0$.

Pourtant, $f$ admet bien un maximum global en $x = 0$ car $f(0) = 0$ et $f(x) < 0$ pour tout $x \neq 0$.

\textbf{Conclusion :} L'affirmation est vraie. Le test de la dérivée seconde ($f''(a) < 0 \implies$ maximum) n'est qu'une condition suffisante, pas nécessaire.

\task \textbf{Affirmation :} Si $f$ est une fonction polynomiale de degré 3 et si $f''(a) = 0$, alors $f$ admet un point d'inflexion en $a$.

\textbf{Vrai.} Pour un polynôme de degré 3, nous avons $f(x) = ax^3 + bx^2 + cx + d$ avec $a \neq 0$.

Donc $f''(x) = 6ax + 2b$, qui est une fonction affine.

Si $f''(a) = 0$, alors $f''$ change nécessairement de signe en $a$ (car une fonction affine non constante est strictement monotone).

En effet, puisque $6a \neq 0$, la fonction $f''$ est strictement croissante (si $a > 0$) ou strictement décroissante (si $a < 0$), donc elle change de signe en $a$ où elle s'annule.

\textbf{Conclusion :} L'affirmation est vraie. Pour un polynôme de degré 3, tout point où $f''$ s'annule est un point d'inflexion.
\end{tasks}
}

