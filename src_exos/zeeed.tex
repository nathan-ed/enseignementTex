\titre{}
\theme{derivation}
\auteur{Nathan Scheinmann}
\niveau{3M}
\source{sesamath3e}
\type{serie}
\piments{2}
\pts{}
\annee{2425}

\contenu{
\tcblower
Pour chacune des affirmations suivantes, indiquer si elle est vraie ou fausse. Justifier très clairement chaque réponse en vous appuyant sur les définitions, propriétés et théorèmes vus au cours ou sur un contre-exemple précis :

\begin{tasks}(1)
\task Si $f'(a) = 0$ et $f''(a) = 0$, alors $f$ admet en $a$ un point d'inflexion à tangente horizontale (un "palier").
\task Soit $f : \mathbb{R} \to \mathbb{R}$ une fonction continue. Si $f$ admet un maximum local en $a$ et un minimum local en $d$, alors $f$ admet un point d'inflexion dans $\left]c ; d\right[$.
\task Il existe une fonction $f : \mathbb{R} \to \mathbb{R}$ qui admet un maximum en $a$ et telle que $f''(a) = 0$.
\task Si $f$ est une fonction polynomiale de degré 3 et si $f''(a) = 0$, alors $f$ admet un point d'inflexion en $a$.
\end{tasks}
}
\correction{

}

