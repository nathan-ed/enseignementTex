\titre{}
\theme{derivation}
\auteur{Nathan Scheinmann}
\niveau{3M}
\source{sesamath3e}
\type{serie}
\piments{2}
\pts{}
\annee{2425}

\contenu{
\tcblower
Déterminer la valeur du nombre $k$ pour que la fonction définie par $f(x) = \dfrac{x^2}{x + k}$ admette un minimum égal à 8.
}
\correction{
\tcblower
\textit{Generated by AI}

Pour la fonction $f(x) = \dfrac{x^2}{x + k}$, nous cherchons $k$ tel que $f$ admette un minimum égal à 8.

\textbf{Domaine :} $\mathbb{R} \setminus \{-k\}$

\textbf{Calcul de la dérivée :}

En utilisant la règle du quotient :
\[
f'(x) = \frac{2x(x + k) - x^2}{(x + k)^2} = \frac{2x^2 + 2kx - x^2}{(x + k)^2} = \frac{x^2 + 2kx}{(x + k)^2} = \frac{x(x + 2k)}{(x + k)^2}
\]

\textbf{Points critiques :}

$f'(x) = 0 \Leftrightarrow x = 0$ ou $x = -2k$

Pour avoir un minimum (et non un maximum), analysons le signe de $f'$ :

\begin{itemize}
\item Si $k > 0$ :
  \begin{itemize}
  \item $x = 0$ donne un minimum local
  \item $x = -2k < 0$ donne un maximum local
  \end{itemize}
\item Si $k < 0$ :
  \begin{itemize}
  \item $x = 0$ donne un maximum local
  \item $x = -2k > 0$ donne un minimum local
  \end{itemize}
\end{itemize}

\textbf{Condition : minimum égal à 8}

Pour $k > 0$, le minimum est en $x = 0$ :
\[
f(0) = \frac{0}{k} = 0 \neq 8
\]

Donc $k < 0$ et le minimum est en $x = -2k$ :
\[
f(-2k) = \frac{(-2k)^2}{-2k + k} = \frac{4k^2}{-k} = -4k = 8
\]

D'où :
\[
-4k = 8 \quad \Rightarrow \quad k = -2
\]

\textbf{Vérification :}

Pour $k = -2$, le minimum est en $x = -2(-2) = 4$ :
\[
f(4) = \frac{16}{4 - 2} = \frac{16}{2} = 8 \quad \checkmark
\]

\textbf{Réponse :} $k = -2$
}

