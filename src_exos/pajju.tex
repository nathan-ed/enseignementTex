\titre{}
\theme{derivation}
\auteur{Nathan Scheinmann}
\niveau{3M}
\source{musy}
\type{serie}
\piments{1}
\pts{}
\annee{2526}

\contenu{
	{\bfseries Entraînement individuel}
\tcblower
En utilisant la définition de la dérivée en un point, calculer les nombres suivants :
\begin{tasks}(2)
\task $f^{\prime}(2)$, si $f(x)=-3 x^2+1$;
\task $f^{\prime}(1)$, si $f(x)=\sqrt{3 x+1}$;
\task $f^{\prime}(0)$, si $f(x)=\frac{x-5}{2 x+1}$;
\task $f^{\prime}(-2)$, si $f(x)=x^3-2 x+3$;
\task $f^{\prime}(0)$, si $f(x)=\sin (x)$;
%\task $f^{\prime}(0)$, si $f(x)=\cos (x)$.
\end{tasks}
}
\correction{
\tcblower
\begin{tasks}(6)
	\task $-12$
	\task $\dfrac{3}{4}$
	\task $11$
	\task $10$
	\task $1$
%	\task $0$
\end{tasks}
}

