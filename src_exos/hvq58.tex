\titre{}
\theme{derivation}
\auteur{Nathan Scheinmann}
\niveau{3M}
\source{sesamath3e}
\type{serie}
\piments{2}
\pts{}
\annee{2425}

\contenu{
\tcblower
On inscrit un cylindre droit dans un cône droit de rayon $r$ et de hauteur $h$.

\begin{tasks}(1)
\task Quelles sont ses dimensions pour que son volume soit minimal ?
\task Quelles sont ses dimensions pour que son aire latérale soit minimale ?
\end{tasks}
}
\correction{
\tcblower
\textit{Generated by AI}

Notons $R$ le rayon du cylindre et $H$ sa hauteur. Par similitude des triangles, on a la relation :
\[
\frac{H}{h} = \frac{r - R}{r} \quad \Rightarrow \quad H = h\left(1 - \frac{R}{r}\right) = h - \frac{hR}{r}
\]

\textbf{Question 1 : Volume minimal du cylindre}

Le volume du cylindre est :
\[
V(R) = \pi R^2 H = \pi R^2 \left(h - \frac{hR}{r}\right) = \pi h R^2 - \pi \frac{h}{r} R^3
\]

Calculons la dérivée :
\[
V'(R) = 2\pi h R - 3\pi \frac{h}{r} R^2 = \pi h R \left(2 - \frac{3R}{r}\right)
\]

Étudions le signe de $V'(R)$ pour $R \in [0;r]$ :
\begin{itemize}
\item $V'(R) = 0 \Leftrightarrow R = 0$ ou $R = \frac{2r}{3}$
\item Si $0 < R < \frac{2r}{3}$ : $V'(R) > 0$ donc $V$ est croissante
\item Si $R > \frac{2r}{3}$ : $V'(R) < 0$ donc $V$ est décroissante
\end{itemize}

Le volume est maximal en $R = \frac{2r}{3}$ et minimal aux extrémités. Comme $V(0) = 0$ et $V(r) = 0$, le volume minimal est $0$ (cylindre dégénéré).

Si on cherche un cylindre non dégénéré de volume minimal, la question n'a pas de sens car le volume peut être arbitrairement petit.

\textbf{Question 2 : Aire latérale minimale}

L'aire latérale du cylindre est :
\[
A(R) = 2\pi R H = 2\pi R \left(h - \frac{hR}{r}\right) = 2\pi h R - 2\pi \frac{h}{r} R^2
\]

Calculons la dérivée :
\[
A'(R) = 2\pi h - 4\pi \frac{h}{r} R = 2\pi h \left(1 - \frac{2R}{r}\right)
\]

$A'(R) = 0 \Leftrightarrow R = \frac{r}{2}$

Étudions le signe :
\begin{itemize}
\item Si $0 < R < \frac{r}{2}$ : $A'(R) > 0$ donc $A$ est croissante
\item Si $R > \frac{r}{2}$ : $A'(R) < 0$ donc $A$ est décroissante
\end{itemize}

L'aire est maximale en $R = \frac{r}{2}$ et minimale aux extrémités où $A(0) = A(r) = 0$.

Pour $R = \frac{r}{2}$, on a $H = h - \frac{h}{2} = \frac{h}{2}$.

\textbf{Réponses :}
\begin{itemize}
\item Le volume minimal (non nul) n'existe pas (il tend vers 0)
\item L'aire latérale minimale (non nulle) n'existe pas (elle tend vers 0)
\end{itemize}

Si l'on cherche les extrema non triviaux : $R = \frac{2r}{3}$, $H = \frac{h}{3}$ pour le volume maximal et $R = \frac{r}{2}$, $H = \frac{h}{2}$ pour l'aire maximale.
}

