\titre{}
\theme{remediation}
\auteur{Nathan Scheinmann}
\niveau{1M}
\source{co}
\type{serie}
\piments{1}
\pts{}
\annee{2425}

\contenu{
\qrcode{https://coopmaths.fr/alea?uuid=622b9&id=11FA6-7&n=9&d=10&s=1-2-3-4-5-6-7-8-9&s2=false&cd=0&cols=1&es=2211001&lang=fr-CH&title=&v=eleve}
\tcblower
\begin{tasks}
		\task Une équipe de basket a marqué 101 points lors d'un match. Au cours de ce match, elle a marqué 24 points sur lancers francs.\\L'équipe a marqué 11 paniers à trois points de moins que de paniers à deux points.\\Combien a-t-elle marqué de paniers à deux points ?
	\task Une équipe de basket a marqué 118 points lors d'un match. Au cours de ce match, elle a marqué 27 points sur lancers francs.\\L'équipe a marqué 8 paniers à deux points de plus que de paniers à trois points.\\Combien a-t-elle marqué de paniers à trois points ?
	\task Le club de ski d'un village propose deux tarifs à ses pratiquants.\\Le tarif A propose de payer CHF $5{,}70$\~à chaque séance.\\Le tarif B propose de payer un abonnement annuel de CHF $30$\,~puis de payer CHF $3{,}20$\,~par séance.\\Pour quel nombre de séances le tarif B devient-il plus avantageux que le tarif A ?
	\task Manon et Karole choisissent un même nombre.\\ Manon lui ajoute 7 puis multiplie le résultat par 3 alors que Karole lui ajoute 5 puis multiplie le résultat par 2.\\Manon et Karole obtiennent le même résultat.\\Quel nombre commun ont choisi Manon et Karole ?
	\task Un triangle possède un côté de longueur $4{,}2$ cm et tous ses autres côtés ont même longueur.\\Son périmètre est $11{,}8$ cm.\\Quelle est la longueur des côtés de même longueur ?
	\task Dans une salle de spectacle de $2\,510$ places, le prix d'entrée pour un adulte est CHF $19{,}90$\,~et, pour un enfant, il est de CHF $6{,}30$\,.\\Le spectacle de ce soir s'est déroulé devant une salle pleine et la recette est de CHF $42\,754{,}60$.\\Combien d'adultes y avait-il dans la salle ?

\end{tasks}
}
\correction{
\tcblower
\begin{tasks}
	\task L'équipe a marqué 22 paniers à deux points.
\task L'équipe a marqué 15 paniers à trois points.
\task C'est à partir de $12$ séances que le tarif B devient plus avantageux que le tarif A (pour $12$ séances, les deux tarifs sont équivalents).
\task Manon et Karole ont donc choisi au départ le nombre $-11$.
\task Les côtés de même longueur mesurent donc $3{,}8$ cm.
\task Il y a eu $1\,981$ adultes au spectacle.

\end{tasks}
}

