\titre{}
\theme{derivation}
\auteur{Nathan Scheinmann}
\niveau{3M}
\source{sesamath3e}
\type{serie}
\piments{2}
\pts{}
\annee{2425}

\contenu{
\tcblower
La somme de deux nombres positifs est 20. Déterminer ces deux nombres de telle façon que :

\begin{tasks}(1)
\task le produit soit maximal
\task la somme des carrés soit minimale
\task le produit du carré de l'un par le cube de l'autre soit maximal.
\end{tasks}
}
\correction{
\tcblower
\textit{Generated by AI}

Notons $x$ et $y$ les deux nombres positifs tels que $x + y = 20$.

\textbf{Question a) : Produit maximal}

Le produit est $P(x) = xy = x(20 - x) = 20x - x^2$ pour $x \in [0;20]$.

Dérivée : $P'(x) = 20 - 2x$

$P'(x) = 0 \Leftrightarrow x = 10$

Comme $P''(x) = -2 < 0$, il s'agit d'un maximum.

Pour $x = 10$, on a $y = 10$.

\textbf{Réponse :} Les deux nombres sont $x = 10$ et $y = 10$, et le produit maximal est $P = 100$.

\textbf{Question b) : Somme des carrés minimale}

La somme des carrés est $S(x) = x^2 + y^2 = x^2 + (20 - x)^2$.

Développons :
\[
S(x) = x^2 + 400 - 40x + x^2 = 2x^2 - 40x + 400
\]

Dérivée : $S'(x) = 4x - 40$

$S'(x) = 0 \Leftrightarrow x = 10$

Comme $S''(x) = 4 > 0$, il s'agit d'un minimum.

Pour $x = 10$, on a $y = 10$.

\textbf{Réponse :} Les deux nombres sont $x = 10$ et $y = 10$, et la somme minimale des carrés est $S = 200$.

\textbf{Question c) : Produit du carré de l'un par le cube de l'autre maximal}

Considérons $F(x) = x^2 \cdot y^3 = x^2(20 - x)^3$ pour $x \in [0;20]$.

Dérivée :
\begin{align*}
F'(x) &= 2x(20 - x)^3 + x^2 \cdot 3(20 - x)^2 \cdot (-1) \\
&= (20 - x)^2[2x(20 - x) - 3x^2] \\
&= (20 - x)^2[40x - 2x^2 - 3x^2] \\
&= (20 - x)^2[40x - 5x^2] \\
&= x(20 - x)^2(40 - 5x)
\end{align*}

$F'(x) = 0 \Leftrightarrow x = 0$ ou $x = 20$ ou $x = 8$

Pour $x = 8$ : $y = 12$

Vérifions qu'il s'agit d'un maximum :
\begin{itemize}
\item $F(0) = 0$
\item $F(8) = 64 \times 1728 = 110592$
\item $F(20) = 0$
\end{itemize}

\textbf{Réponse :} Les deux nombres sont $x = 8$ et $y = 12$, et le produit maximal est $F = 8^2 \times 12^3 = 110592$.

Alternativement, si on considère $x^3 y^2$, par symétrie on trouverait $x = 12$ et $y = 8$.
}

