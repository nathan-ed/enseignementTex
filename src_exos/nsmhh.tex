\titre{}
\theme{dérivées}
\auteur{Nathan Scheinmann}
\niveau{3M}
\source{fundamentum}
\type{serie}
\piments{2}
\pts{}
\annee{2425}

\contenu{
\tcblower
Un mur de 2 m de haut, situé à 1 m d'une façade, interdit l'accès à celle-ci. Calculer la longueur de l'échelle la plus courte qui s'appuie contre la façade et dont le pied est sur le sol, devant le mur.
}
\correction{

\tcblower
\textit{Generated by AI}
{\scriptsize \textit{Correction générée par IA}}

Plaçons le mur le long de l'axe vertical, la façade à $x = 1$ m, et le sol en $y = 0$.

Le mur s'élève jusqu'à $y = 2$ m. L'échelle doit passer au-dessus du coin supérieur du mur situé en $(0, 2)$.

Soit $L$ la longueur de l'échelle qui touche le sol en $(a, 0)$ avec $a < 0$ et la façade en $(1, b)$ avec $b > 0$.

L'équation de l'échelle est une droite passant par ces deux points :
\[\dfrac{y - 0}{x - a} = \dfrac{b - 0}{1 - a}\]
\[y = \dfrac{b}{1 - a}(x - a)\]

Pour que l'échelle passe par le point $(0, 2)$ :
\[2 = \dfrac{b}{1 - a}(0 - a) = \dfrac{-ab}{1 - a}\]
\[2(1 - a) = -ab \implies 2 - 2a = -ab \implies ab = 2a - 2 \implies b = 2 - \dfrac{2}{a}\]

La longueur de l'échelle est :
\[L^2 = (1 - a)^2 + b^2\]

Posons $x = -a > 0$ (distance du pied de l'échelle à la façade du côté opposé au mur). Alors $a = -x$ et :
\[b = 2 - \dfrac{2}{-x} = 2 + \dfrac{2}{x}\]

La longueur devient :
\[L(x) = \sqrt{(1 + x)^2 + \left(2 + \dfrac{2}{x}\right)^2}\]

Pour minimiser $L$, il est équivalent de minimiser $L^2$ :
\[f(x) = (1 + x)^2 + \left(2 + \dfrac{2}{x}\right)^2 = 1 + 2x + x^2 + 4 + \dfrac{8}{x} + \dfrac{4}{x^2}\]

Dérivons :
\[f'(x) = 2 + 2x - \dfrac{8}{x^2} - \dfrac{8}{x^3}\]

Annulation : $2 + 2x = \dfrac{8}{x^2} + \dfrac{8}{x^3}$

Cette équation est complexe. Simplifions en multipliant par $x^3$ :
\[2x^3 + 2x^4 = 8x + 8 \implies 2x^4 + 2x^3 - 8x - 8 = 0 \implies x^4 + x^3 - 4x - 4 = 0\]

Par approximation numérique, $x \approx 1{,}565$ m.

Donc $b = 2 + \dfrac{2}{1{,}565} \approx 3{,}28$ m et $L = \sqrt{(2{,}565)^2 + (3{,}28)^2} \approx \boxed{4{,}16 \text{ m}}$.

}

