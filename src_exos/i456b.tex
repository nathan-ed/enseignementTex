\titre{}
\theme{calcDif}
\auteur{Nathan Scheinmann}
\niveau{3M}
\source{}
\type{serie}
\piments{1}
\pts{}
\annee{2526}

\contenu{
	\tcblower
Trouver les intervalles sur lesquels \(f\) est croissante et ceux sur lesquels elle est décroissante, si \(f(x)\) est donnée comme suit.
	\begin{tasks}(2)
\task \(\dfrac{x^2}{x^2 - 1}\)
\task \(\dfrac{x}{x^2 + 1}\)
\task \(\dfrac{x^2}{x + 1}\)
\task \(\dfrac{x^2(1 + x)^2}{x - 1}\)
\task \(\dfrac{x}{x^2 - 1}\)
\task \(\dfrac{x^2}{x^2 + 1}\)
\task \(\left(1 - \sqrt{x}\right)^7\)
\task \(\sqrt{\dfrac{2 + x}{1 + x^2}}\)
\task \(\dfrac{1}{\sqrt{x^2 + 1}}\)
\task \(|x + 1||x - 2|\)
\end{tasks}

}
\correction{
	\tcblower

}
