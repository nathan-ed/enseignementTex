\titre{}
\theme{derivation}
\auteur{Nathan Scheinmann}
\niveau{3M}
\source{sesamath3e}
\type{serie}
\piments{2}
\pts{}
\annee{2425}

\contenu{
\tcblower
On construit une boîte rectangulaire en pliant une feuille de carton mesurant 30 cm sur 20 cm. Déterminez la hauteur $x$ de la boîte pour qu'elle ait la plus grande capacité possible.
}
\correction{
\tcblower
\textit{Generated by AI}

Pour construire la boîte, on découpe des carrés de côté $x$ aux quatre coins de la feuille, puis on plie les côtés vers le haut.

\textbf{Dimensions de la boîte :}

Après découpage et pliage :
\begin{itemize}
\item Longueur : $L = 30 - 2x$ cm
\item Largeur : $\ell = 20 - 2x$ cm
\item Hauteur : $h = x$ cm
\end{itemize}

avec $0 < x < 10$ (car $x$ doit être inférieur à la moitié de la plus petite dimension).

\textbf{Volume de la boîte :}

Le volume est :
\[
V(x) = L \times \ell \times h = (30 - 2x)(20 - 2x) \cdot x
\]

Développons :
\begin{align*}
V(x) &= x(30 - 2x)(20 - 2x) \\
&= x(600 - 60x - 40x + 4x^2) \\
&= x(600 - 100x + 4x^2) \\
&= 4x^3 - 100x^2 + 600x
\end{align*}

\textbf{Maximisation du volume :}

Calculons la dérivée :
\[
V'(x) = 12x^2 - 200x + 600 = 4(3x^2 - 50x + 150)
\]

Cherchons les points critiques en résolvant $V'(x) = 0$ :
\[
3x^2 - 50x + 150 = 0
\]

En utilisant la formule quadratique :
\[
x = \frac{50 \pm \sqrt{2500 - 1800}}{6} = \frac{50 \pm \sqrt{700}}{6} = \frac{50 \pm 10\sqrt{7}}{6}
\]

Ce qui donne :
\[
x_1 = \frac{50 - 10\sqrt{7}}{6} \approx 3{,}92~\text{cm} \quad \text{et} \quad x_2 = \frac{50 + 10\sqrt{7}}{6} \approx 12{,}74~\text{cm}
\]

Comme $x < 10$, seule la solution $x_1 \approx 3{,}92~\text{cm}$ est valable.

Vérifions qu'il s'agit d'un maximum avec la dérivée seconde :
\[
V''(x) = 24x - 200
\]
\[
V''(3{,}92) = 24(3{,}92) - 200 \approx -106 < 0
\]

Donc il s'agit bien d'un maximum.

\textbf{Réponse :} La hauteur optimale est $x = \dfrac{50 - 10\sqrt{7}}{6} \approx 3{,}9~\text{cm}$.
}

