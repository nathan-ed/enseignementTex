\titre{}
\theme{limites}
\auteur{Nathan Scheinmann}
\niveau{3M}
\source{musy}
\type{serie}
\piments{2}
\pts{}
\annee{2425}

\contenu{
\tcblower
Les fonctions 
\[f(x)=\dfrac{x^2-1}{x+1} \text{ et } g(x)=\dfrac{x^2-1}{(x+1)^2}\]
ne sont pas définies en $x=-1$. Peut-on attribuer une valeur à l'image de $-1$ pour que ces fonctions ainsi prolongées soient continue en $x=-1$. Esquissez les graphes de $f$ et $g$. 
}
\correction{
	\tcblower
\begin{itemize}
  \item On factorise le numérateur : \( x^2 - 1 = (x - 1)(x + 1) \)
  \item Donc :
    \[
    \begin{aligned}
    f(x) &= \dfrac{(x - 1)(x + 1)}{x + 1} = x - 1 \quad \text{pour } x \ne -1 \\
    g(x) &= \dfrac{(x - 1)(x + 1)}{(x + 1)^2} = \dfrac{x - 1}{x + 1} \quad \text{pour } x \ne -1
    \end{aligned}
    \]
  \item Pour \( f \) : \( \displaystyle \lim_{x \to -1} f(x) = (-1) - 1 = -2 \), donc on peut définir \( f(-1) = -2 \) pour la rendre continue.
  \item Pour \( g \) : \( \displaystyle \lim_{x \to -1^-} g(x) = «\,\dfrac{-2}{0^-}\,»= +\infty \), \quad \( \displaystyle \lim_{x \to -1^+} g(x) = «\,~\dfrac{-2}{0^+}\,»= -\infty \), donc la limite n'existe pas.
  \item Ainsi, seule \( f \) peut être prolongée continûment en \( x = -1 \). Si on appelle cette nouvelle fonction $\widetilde f$ on doit poser $\widetilde f(-1)=-2$.
\end{itemize}
}

