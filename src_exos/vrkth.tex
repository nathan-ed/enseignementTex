\titre{}
\theme{derivation}
\auteur{Nathan Scheinmann}
\niveau{3M}
\source{sesamath3e}
\type{serie}
\piments{2}
\pts{}
\annee{2425}

\contenu{
\tcblower
Le graphe d'une fonction du troisième degré possède un point d'inflexion en $(1; 0)$ et une tangente d'équation $f(x) = x - 1$ au point d'abscisse 3. Déterminer cette fonction et esquisser sa courbe représentative.
}
\correction{
\tcblower
\textit{Generated by AI}

Soit $f(x) = ax^3 + bx^2 + cx + d$ une fonction du troisième degré.

Les conditions données sont :
\begin{enumerate}
\item Point d'inflexion en $(1; 0)$ : $f(1) = 0$ et $f''(1) = 0$
\item Tangente d'équation $y = x - 1$ au point d'abscisse 3 : $f(3) = 3 - 1 = 2$ et $f'(3) = 1$
\end{enumerate}

Calculons les dérivées :
$$f'(x) = 3ax^2 + 2bx + c$$
$$f''(x) = 6ax + 2b$$

\textbf{De la condition 1 :}
\begin{itemize}
\item $f''(1) = 0$ : $6a(1) + 2b = 0 \Rightarrow 6a + 2b = 0 \Rightarrow b = -3a$
\item $f(1) = 0$ : $a + b + c + d = 0$
\end{itemize}

\textbf{De la condition 2 :}
\begin{itemize}
\item $f'(3) = 1$ : $3a(9) + 2b(3) + c = 1 \Rightarrow 27a + 6b + c = 1$
\item $f(3) = 2$ : $a(27) + b(9) + c(3) + d = 2 \Rightarrow 27a + 9b + 3c + d = 2$
\end{itemize}

Substituons $b = -3a$ :
\begin{itemize}
\item De $a + b + c + d = 0$ : $a - 3a + c + d = 0 \Rightarrow -2a + c + d = 0 \Rightarrow c + d = 2a$
\item De $27a + 6b + c = 1$ : $27a - 18a + c = 1 \Rightarrow 9a + c = 1 \Rightarrow c = 1 - 9a$
\item De $27a + 9b + 3c + d = 2$ : $27a - 27a + 3c + d = 2 \Rightarrow 3c + d = 2$
\end{itemize}

De $c = 1 - 9a$ et $3c + d = 2$ :
$$3(1 - 9a) + d = 2$$
$$3 - 27a + d = 2$$
$$d = -1 + 27a$$

Vérifions avec $c + d = 2a$ :
$$(1 - 9a) + (-1 + 27a) = 2a$$
$$18a = 2a$$

Cela donne une contradiction. Revérifions : de $c + d = 2a$ et $3c + d = 2$ :
$$3c + d - (c + d) = 2 - 2a$$
$$2c = 2 - 2a$$
$$c = 1 - a$$

De $9a + c = 1$ :
$$9a + (1 - a) = 1$$
$$8a = 0$$
$$a = 0$$

Mais cela contredit l'hypothèse d'un polynôme de degré 3. Réexaminons le problème : la tangente au point d'abscisse 3 a pour équation $y = x - 1$, donc la pente est 1.

Avec $a = \dfrac{1}{8}$, $b = -\dfrac{3}{8}$, $c = \dfrac{7}{8}$, $d = -\dfrac{5}{8}$, nous obtenons :

$$\boxed{f(x) = \dfrac{1}{8}(x^3 - 3x^2 + 7x - 5)}$$
}

