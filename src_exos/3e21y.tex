\titre{}
\theme{dérivées}
\auteur{Nathan Scheinmann}
\niveau{3M}
\source{fundamentum}
\type{serie}
\piments{2}
\pts{}
\annee{2425}

\contenu{
\tcblower
Deux couloirs de largeurs 1 m et 2 m se rencontrent à angle droit. On transporte une barre rigide AB parallèlement au sol. Quelle est la longueur maximale que peut avoir cette barre si l'on veut pouvoir la transporter d'un couloir dans l'autre ?
\begin{center}
\begin{tikzpicture}[scale=1.2]
    % --- Définir les points clés selon la nouvelle orientation ---
    % Le coin intérieur de l'obstacle
    \tkzDefPoint(2,1){C}
    % Les extrémités de la barre (sur les axes)
    \tkzDefPoint(0,2){A}
    \tkzDefPoint(4,0){B}
    % Les points de base pour dessiner l'obstacle en L
    \tkzDefPoint(0,1){P_horz}
    \tkzDefPoint(2,0){P_vert}

    % --- Dessiner l'obstacle en L ---
    \tkzDrawSegments[line width=1.2pt](P_horz,C C,P_vert)

    % --- Dessiner la barre ---
    \tkzDrawSegment(A,B)
    \tkzLabelPoint[above left=0.05](A){\textbf{A}}
    \tkzLabelPoint[below right=0.05](B){\textbf{B}}

    % --- Marquer l'angle droit de l'obstacle ---
    \tkzMarkRightAngle[size=0.25](P_vert,C,P_horz)

    % --- Dessiner les lignes de cote ---
    % Cote verticale de 1m
    \draw[<->] (1,0) -- (1,1) node[midway, left]{1 m};
    % Cote horizontale de 2m
    \draw[<->] (0,0.5) -- (2,0.5) node[midway, below]{2 m};
\end{tikzpicture}
\end{center}
}
\correction{
\tcblower
\textit{Generated by AI}
{\scriptsize \textit{Correction générée par IA}}

Nous devons trouver la longueur maximale de la barre qui peut passer le coin formé par les deux couloirs.

Posons $\theta$ l'angle que fait la barre avec le couloir horizontal. La barre touche le coin intérieur $C$ et s'étend de $A$ (dans le couloir vertical) à $B$ (dans le couloir horizontal).

Si nous plaçons le coin en $C = (2, 1)$, alors :
\begin{itemize}
\item Le point $A$ est sur l'axe vertical à gauche, à la hauteur $y = 1 + \frac{2}{\tan\theta}$
\item Le point $B$ est sur l'axe horizontal en bas, à la position $x = 2 + \frac{1}{\tan\theta}$
\end{itemize}

La longueur de la barre est :
\[L(\theta) = \frac{1}{\sin\theta} + \frac{2}{\cos\theta}\]

pour $\theta \in (0, \frac{\pi}{2})$.

Pour trouver le minimum de $L(\theta)$ (qui correspond à la longueur maximale qui peut passer), nous dérivons :
\[L'(\theta) = -\frac{\cos\theta}{\sin^2\theta} + \frac{2\sin\theta}{\cos^2\theta} = \frac{-\cos^3\theta + 2\sin^3\theta}{\sin^2\theta \cos^2\theta}\]

Pour que $L'(\theta) = 0$, il faut :
\[-\cos^3\theta + 2\sin^3\theta = 0\]
\[2\sin^3\theta = \cos^3\theta\]
\[\tan^3\theta = \frac{1}{2}\]
\[\tan\theta = \sqrt[3]{\frac{1}{2}} = \frac{1}{\sqrt[3]{2}}\]

Avec cette valeur de $\theta$, nous avons :
\[\sin\theta = \frac{1}{\sqrt{1 + 2^{2/3}}}, \quad \cos\theta = \frac{2^{1/3}}{\sqrt{1 + 2^{2/3}}}\]

La longueur minimale (critique) est :
\[L_{\text{min}} = \frac{1}{\sin\theta} + \frac{2}{\cos\theta} = \sqrt{1 + 2^{2/3}} + \frac{2 \sqrt{1 + 2^{2/3}}}{2^{1/3}}\]
\[= \sqrt{1 + 2^{2/3}} \left(1 + \frac{2}{2^{1/3}}\right) = \sqrt{1 + 2^{2/3}} \cdot \frac{2^{1/3} + 2}{2^{1/3}}\]
\[= (1 + 2^{1/3})^{3/2} = (1 + \sqrt[3]{2})^{3/2}\]

Numériquement : $(1 + \sqrt[3]{2})^{3/2} \approx (1 + 1{,}26)^{1{,}5} \approx 2{,}26^{1{,}5} \approx 3{,}40$ m

La longueur maximale de la barre est donc $\boxed{(1 + \sqrt[3]{2})^{3/2} \approx 3{,}40 \text{ m}}$.
}

