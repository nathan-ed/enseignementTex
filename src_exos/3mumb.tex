\titre{Dérivabilité de fonctions définies par morceaux}
\theme{derivabilite}
\auteur{Nathan Scheinmann}
\niveau{3M}
\source{}
\type{serie}
\piments{2}
\pts{}
\annee{2425}

\contenu{
	\tcblower
	Soit la fonction $f$ définie par morceaux ci-dessous.
	\[
	f(x) = \begin{cases}
		2x - 1, & \text{si } x < 2 \\
		3, & \text{si } x = 2 \\
		-x^2 + 3x + 1, & \text{si } x > 2
	\end{cases}
	\]

	\begin{enumerate}[label=\alph*)]
		\item En utilisant la définition de la dérivée, déterminer si $f$ est dérivable en $a = 2$.
		\item Même question pour $g$ pour $a = -1$ :
		\[
		g(x) = \begin{cases}
			3x^2 + 1, & \text{si } x < -1 \\
			4, & \text{si } x = -1 \\
			x^2 - 4x - 1, & \text{si } x > -1
		\end{cases}
		\]
	\end{enumerate}
}
\correction{
	\tcblower
	\begin{enumerate}[label=\alph*)]
		\item Pour que $f$ soit dérivable en $a = 2$, il faut que les dérivées à gauche et à droite existent et soient égales.

		Calculons la dérivée à gauche :
		\[
		f'_g(2) = \lim_{h \to 0^-} \frac{f(2+h) - f(2)}{h} = \lim_{h \to 0^-} \frac{(2(2+h) - 1) - 3}{h} = \lim_{h \to 0^-} \frac{4 + 2h - 1 - 3}{h} = \lim_{h \to 0^-} \frac{2h}{h} = 2
		\]

		Calculons la dérivée à droite :
		\[
		f'_d(2) = \lim_{h \to 0^+} \frac{f(2+h) - f(2)}{h} = \lim_{h \to 0^+} \frac{(-(2+h)^2 + 3(2+h) + 1) - 3}{h}
		\]
		\[
		= \lim_{h \to 0^+} \frac{-(4 + 4h + h^2) + 6 + 3h + 1 - 3}{h} = \lim_{h \to 0^+} \frac{-4 - 4h - h^2 + 6 + 3h + 1 - 3}{h}
		\]
		\[
		= \lim_{h \to 0^+} \frac{-h^2 - h}{h} = \lim_{h \to 0^+} (-h - 1) = -1
		\]

		Puisque $f'_g(2) = 2 \neq -1 = f'_d(2)$, la fonction $f$ n'est pas dérivable en $a = 2$.

		\item Pour $g$ en $a = -1$ :

		Dérivée à gauche :
		\[
		g'_g(-1) = \lim_{h \to 0^-} \frac{g(-1+h) - g(-1)}{h} = \lim_{h \to 0^-} \frac{(3(-1+h)^2 + 1) - 4}{h}
		\]
		\[
		= \lim_{h \to 0^-} \frac{3(1 - 2h + h^2) + 1 - 4}{h} = \lim_{h \to 0^-} \frac{3 - 6h + 3h^2 + 1 - 4}{h} = \lim_{h \to 0^-} \frac{3h^2 - 6h}{h} = \lim_{h \to 0^-} (3h - 6) = -6
		\]

		Dérivée à droite :
		\[
		g'_d(-1) = \lim_{h \to 0^+} \frac{g(-1+h) - g(-1)}{h} = \lim_{h \to 0^+} \frac{((-1+h)^2 - 4(-1+h) - 1) - 4}{h}
		\]
		\[
		= \lim_{h \to 0^+} \frac{(1 - 2h + h^2) + 4 - 4h - 1 - 4}{h} = \lim_{h \to 0^+} \frac{h^2 - 6h}{h} = \lim_{h \to 0^+} (h - 6) = -6
		\]

		Puisque $g'_g(-1) = -6 = g'_d(-1)$, la fonction $g$ est dérivable en $a = -1$ et $g'(-1) = -6$.
	\end{enumerate}
}

