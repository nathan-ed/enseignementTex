\titre{}
\theme{remediation}
\auteur{Nathan Scheinmann}
\niveau{1M}
\source{co}
\type{serie}
\piments{1}
\pts{}
\annee{2425}

\contenu{
\tcblower
% @see : 

\begin{tasks}
	\task Un triangle isocèle a pour périmètre $172$ mm. Sa base est plus courte de $17$ mm que chacun des côtés égaux.\\Quelle est la mesure de sa base ? (La figure n'est pas en vraie grandeur.)

	\begin{tikzpicture}[baseline,scale = 0.8]

    \tikzset{
      point/.style={
        thick,
        draw,
        cross out,
        inner sep=0pt,
        minimum width=5pt,
        minimum height=5pt,
      },
    }
    \clip (-1,-1) rectangle (7,4);
    	\draw[color={black}] (6,1.5)--(0,3)--(0,0)--cycle;
	\draw [color={black}] (3,2.25) node[anchor = center,scale=1, rotate = -14] {//};
\draw [color={black}] (3,0.75) node[anchor = center,scale=1, rotate = -346] {//};


\end{tikzpicture}\\

	\task Quynh a acheté $3{,}2$ kg de fraises avec un billet de CHF $20$\,. Le marchand lui a rendu CHF$5{,}28$.\\Quel est le prix d'un kilogramme de fraises ?
	\task Nathalie et Marie choisissent un même nombre.\\ Nathalie lui ajoute 1 puis multiplie le résultat par 7 alors que Marie le multiplie par 10 puis ajoute au résultat 10.\\Nathalie et Marie obtiennent le même résultat.\\Quel nombre commun ont choisi Nathalie et Marie ?
\end{tasks}

}
\correction{
\tcblower
\begin{tasks}
\task La base de ce triangle isocèle mesure $46$ mm.
\task Le prix d'un kilogramme de fraises est donc de $4{,}6$.
\task Nathalie et Marie ont donc choisi au départ le nombre -1.
\end{tasks}


}

