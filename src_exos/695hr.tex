\titre{}
\theme{fonctions}
\auteur{Nathan Scheinmann}
\niveau{1M}
\source{}
\type{serie}
\piments{5}
\pts{}
\annee{2425}

\contenu{
	\tcblower
Déterminer le domaine de définition de la fonction $x\mapsto \dfrac{1}{x-4+\sqrt{x^2-2}}$
}
\correction{
\tcblower
{\scriptsize \textit{Correction générée par IA}}

\textbf{Contrainte 1~:} $x^2-2 \geq 0 \implies |x| \geq \sqrt{2}$, soit $x \in ]-\infty; -\sqrt{2}] \cup [\sqrt{2}; +\infty[$.

\textbf{Contrainte 2~:} Le dénominateur $x-4+\sqrt{x^2-2} \neq 0$.

Résolvons $\sqrt{x^2-2} = 4-x$ (nécessite $x \leq 4$)~:
\( \begin{aligned}
x^2-2 &= (4-x)^2 = 16-8x+x^2 \\
-2 &= 16-8x \\
x &= \dfrac{9}{4}
\end{aligned} \)

Vérification~: $\dfrac{9}{4}-4+\sqrt{\dfrac{81}{16}-2} = \dfrac{9}{4}-4+\dfrac{7}{4} = 0$ ✓

\textbf{Domaine~:} $\mathcal{D}_f = ]-\infty; -\sqrt{2}] \cup [\sqrt{2}; \dfrac{9}{4}[ \cup ]\dfrac{9}{4}; +\infty[$
}

