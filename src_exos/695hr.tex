\titre{}
\theme{fonctions}
\auteur{Nathan Scheinmann}
\niveau{1M}
\source{}
\type{serie}
\piments{5}
\pts{}
\annee{2425}

\contenu{
	\tcblower
Déterminer le domaine de définition de la fonction $x\mapsto \dfrac{1}{x-4+\sqrt{x^2-2}}$
}
\correction{
	\tcblower
{\scriptsize \textit{Correction générée par IA}}

Pour déterminer le domaine de définition de $f(x) = \dfrac{1}{x-4+\sqrt{x^2-2}}$, nous devons identifier les valeurs de $x$ pour lesquelles la fonction est définie.

\textbf{Contraintes :}

1. Le dénominateur ne doit pas être nul : $x-4+\sqrt{x^2-2} \neq 0$

2. L'expression sous la racine doit être positive ou nulle : $x^2-2 \geq 0$

\textbf{Étude de la contrainte 2 :}

\[
x^2-2 \geq 0 \quad \Rightarrow \quad x^2 \geq 2 \quad \Rightarrow \quad |x| \geq \sqrt{2}
\]

Donc $x \in ]-\infty; -\sqrt{2}] \cup [\sqrt{2}; +\infty[$.

\textbf{Étude de la contrainte 1 :}

Il faut trouver les valeurs de $x$ pour lesquelles :
\[
x-4+\sqrt{x^2-2} = 0 \quad \Rightarrow \quad \sqrt{x^2-2} = 4-x
\]

Pour que cette équation ait un sens, il faut que $4-x \geq 0$, donc $x \leq 4$.

En élevant au carré (valide car les deux membres doivent être positifs) :
\[
x^2-2 = (4-x)^2 = 16-8x+x^2
\]

\[
-2 = 16-8x
\]

\[
8x = 18
\]

\[
x = \dfrac{18}{8} = \dfrac{9}{4}
\]

Vérifions : pour $x = \dfrac{9}{4}$ :
\[
\sqrt{\left(\dfrac{9}{4}\right)^2-2} = \sqrt{\dfrac{81}{16}-\dfrac{32}{16}} = \sqrt{\dfrac{49}{16}} = \dfrac{7}{4}
\]

Et :
\[
\dfrac{9}{4}-4+\dfrac{7}{4} = \dfrac{9+7-16}{4} = 0
\]

La valeur $x = \dfrac{9}{4}$ annule bien le dénominateur.

\textbf{Domaine de définition :}

En combinant les contraintes, le domaine est :
\[
\mathcal{D}_f = \left]-\infty; -\sqrt{2}\right] \cup \left[\sqrt{2}; \dfrac{9}{4}\right[ \cup \left]\dfrac{9}{4}; +\infty\right[
\]
}

