\titre{}
\theme{dérivées}
\auteur{Nathan Scheinmann}
\niveau{3M}
\source{analysis}
\type{serie}
\piments{2}
\pts{}
\annee{2425}

\contenu{
\tcblower
Soit \(f\) deux fois dérivable. Montrer que, si l’équation \(f(x) = 0\) a \(n\) racines réelles, alors l’équation \(f'(x) = 0\) a au moins \(n - 1\) racines réelles et l’équation \(f''(x) = 0\) a au moins \(n - 2\) racines réelles.
}
\correction{
\tcblower
\textit{Generated by AI}

Soit $f$ une fonction deux fois dérivable ayant $n$ racines réelles distinctes $x_1 < x_2 < \cdots < x_n$.

\textbf{Partie 1 : $f'$ a au moins $n-1$ racines}

Entre deux racines consécutives $x_i$ et $x_{i+1}$ de $f$, on a $f(x_i) = f(x_{i+1}) = 0$.

D'après le théorème de Rolle, puisque $f$ est continue sur $[x_i, x_{i+1}]$ et dérivable sur $]x_i, x_{i+1}[$, il existe au moins un point $c_i \in ]x_i, x_{i+1}[$ tel que $f'(c_i) = 0$.

Ceci est vrai pour chaque paire de racines consécutives, donc :
\begin{itemize}
\item Entre $x_1$ et $x_2$ : au moins une racine de $f'$
\item Entre $x_2$ et $x_3$ : au moins une racine de $f'$
\item $\vdots$
\item Entre $x_{n-1}$ et $x_n$ : au moins une racine de $f'$
\end{itemize}

Donc $f'$ possède au moins $n-1$ racines réelles (toutes distinctes car situées dans des intervalles disjoints).

\textbf{Partie 2 : $f''$ a au moins $n-2$ racines}

Nous venons de montrer que $f'$ a au moins $n-1$ racines. Notons-les $y_1 < y_2 < \cdots < y_{n-1}$.

En appliquant le même raisonnement avec le théorème de Rolle à $f'$ :

Entre deux racines consécutives $y_i$ et $y_{i+1}$ de $f'$, on a $f'(y_i) = f'(y_{i+1}) = 0$.

Donc il existe au moins un point où $f''$ s'annule entre $y_i$ et $y_{i+1}$.

Puisque $f'$ a $n-1$ racines, on obtient $n-2$ intervalles entre racines consécutives, donc $f''$ a au moins $n-2$ racines réelles.
}

