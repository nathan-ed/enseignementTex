\titre{}
\theme{derivation}
\auteur{Nathan Scheinmann}
\niveau{3M}
\source{sesamath3e}
\type{serie}
\piments{2}
\pts{}
\annee{2425}

\contenu{
\tcblower
Déterminer l'expression algébrique d'une fonction qui admet un maximum en $x = 1$, un minimum en $x = 3$ et un point d'inflexion en $x = 2$ puis esquisser sa courbe représentative.
}
\correction{
\tcblower
\textit{Generated by AI}

Pour qu'une fonction ait un maximum en $x=1$ et un minimum en $x=3$, il faut que $f'(1) = 0$ et $f'(3) = 0$.

Pour qu'elle ait un point d'inflexion en $x=2$, il faut que $f''(2) = 0$.

Le polynôme le plus simple satisfaisant ces conditions est un polynôme de degré 3 :
\[f(x) = ax^3 + bx^2 + cx + d\]

Calculons les dérivées :
\begin{align*}
f'(x) &= 3ax^2 + 2bx + c \\
f''(x) &= 6ax + 2b
\end{align*}

\textbf{Conditions à satisfaire :}
\begin{itemize}
\item $f'(1) = 0$ : $3a + 2b + c = 0$
\item $f'(3) = 0$ : $27a + 6b + c = 0$
\item $f''(2) = 0$ : $12a + 2b = 0$
\end{itemize}

De la troisième équation : $b = -6a$

En substituant dans la première : $3a + 2(-6a) + c = 0 \implies c = 9a$

Vérifions avec la deuxième : $27a + 6(-6a) + 9a = 27a - 36a + 9a = 0$ \checkmark

Prenons $a = 1$ (choix arbitraire pour simplifier), on obtient :
\[\boxed{f(x) = x^3 - 6x^2 + 9x + d}\]

où $d$ est une constante arbitraire (on peut prendre $d = 0$ par exemple).

\textbf{Vérification :}
\begin{itemize}
\item $f'(x) = 3x^2 - 12x + 9 = 3(x-1)(x-3)$ s'annule bien en $x=1$ et $x=3$
\item $f''(x) = 6x - 12 = 6(x-2)$ s'annule bien en $x=2$
\item $f''(1) = -6 < 0$ donc maximum en $x=1$
\item $f''(3) = 6 > 0$ donc minimum en $x=3$
\end{itemize}
}

