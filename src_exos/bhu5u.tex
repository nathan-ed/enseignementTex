\titre{}
\theme{derivation}
\auteur{Nathan Scheinmann}
\niveau{3M}
\source{sesamath3e}
\type{serie}
\piments{2}
\pts{}
\annee{2425}

\contenu{
\tcblower
Quelles sont les dimensions du rectangle grisé d'aire maximale inscrit sous cette parabole de sommet $(0,4)$ et de zéros 5 et 5 ?
}
\correction{
\tcblower
\textit{Generated by AI}

La parabole a pour sommets $(0, 4)$ et pour zéros $-5$ et $5$. Son équation est donc de la forme :
\[y = a(x - 5)(x + 5) = a(x^2 - 25)\]

Puisque le sommet est en $(0, 4)$ :
\[4 = a(0 - 25) = -25a \implies a = -\frac{4}{25}\]

Donc : $y = -\dfrac{4}{25}(x^2 - 25) = -\dfrac{4}{25}x^2 + 4$

Le rectangle inscrit sous la parabole, symétrique par rapport à l'axe $y$, a :
- Base : $2x$ (de $-x$ à $x$)
- Hauteur : $y = -\dfrac{4}{25}x^2 + 4$

L'aire du rectangle est :
\[A(x) = 2x \cdot y = 2x\left(-\frac{4}{25}x^2 + 4\right) = -\frac{8}{25}x^3 + 8x\]

Pour maximiser l'aire, dérivons :
\[A'(x) = -\frac{24}{25}x^2 + 8\]

\[A'(x) = 0 \implies -\frac{24}{25}x^2 + 8 = 0 \implies x^2 = \frac{8 \cdot 25}{24} = \frac{200}{24} = \frac{25}{3}\]

\[x = \frac{5}{\sqrt{3}} = \frac{5\sqrt{3}}{3}\]

La base est : $2x = \dfrac{10\sqrt{3}}{3}$

La hauteur est : $y = -\dfrac{4}{25} \cdot \dfrac{25}{3} + 4 = -\dfrac{4}{3} + 4 = \dfrac{8}{3}$

\textbf{Réponse :} Base = $\boxed{\dfrac{10\sqrt{3}}{3} \approx 5{,}77}$ et Hauteur = $\boxed{\dfrac{8}{3} \approx 2{,}67}$
}

