\titre{}
\theme{geometrie}
\auteur{Nathan Scheinmann}
\niveau{1M}
\source{adbou}
\type{serie}
\piments{2}
\pts{}
\annee{2425}

\contenu{
	\tcblower
Démontrer le théorème suivant~:
Si un quadrilatère est inscrit dans un cerlce, alors la somme des angles opposés (dans le quadrilatère) est de $180^\circ$.
}
\correction{

	\tcblower
	\textit{Generated by AI}
	{\scriptsize \textit{Correction générée par IA}}

	Soit $ABCD$ un quadrilatère inscrit dans un cercle de centre $O$. Nous devons démontrer que la somme des angles opposés est égale à $180^\circ$.

	\textbf{Démonstration :}

	Considérons les angles $\widehat{ABC}$ et $\widehat{ADC}$ qui sont opposés dans le quadrilatère.

	L'angle $\widehat{ABC}$ est un angle inscrit qui intercepte l'arc $\widehat{ADC}$ (l'arc allant de $A$ à $C$ en passant par $D$).

	L'angle $\widehat{ADC}$ est un angle inscrit qui intercepte l'arc $\widehat{ABC}$ (l'arc allant de $A$ à $C$ en passant par $B$).

	D'après le théorème de l'angle inscrit, un angle inscrit est égal à la moitié de l'angle au centre qui intercepte le même arc.

	Notons $\alpha$ la mesure de l'arc $\widehat{ABC}$ et $\beta$ la mesure de l'arc $\widehat{ADC}$.

	Alors :
	\[\widehat{ABC} = \dfrac{\beta}{2} \quad \text{et} \quad \widehat{ADC} = \dfrac{\alpha}{2}\]

	Comme le quadrilatère est inscrit dans le cercle, les deux arcs $\widehat{ABC}$ et $\widehat{ADC}$ forment le cercle complet :
	\[\alpha + \beta = 360^\circ\]

	Par conséquent :
	\[\widehat{ABC} + \widehat{ADC} = \dfrac{\beta}{2} + \dfrac{\alpha}{2} = \dfrac{\alpha + \beta}{2} = \dfrac{360^\circ}{2} = 180^\circ\]

	De même, on peut démontrer que $\widehat{BAD} + \widehat{BCD} = 180^\circ$.

	\textbf{Conclusion :} Si un quadrilatère est inscrit dans un cercle, alors la somme des angles opposés est égale à $180^\circ$. $\square$

}

