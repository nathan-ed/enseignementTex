\titre{}
\theme{derivation}
\auteur{Nathan Scheinmann}
\niveau{3M}
\source{sesamath3e}
\type{serie}
\piments{2}
\pts{}
\annee{2425}

\contenu{
\tcblower
Tracer le graphe d'une fonction réelle $f$ satisfaisant simultanément toutes les conditions suivantes :

\begin{tasks}(1)
\task $f(2) = -4$
\task l'ensemble des préimages de 1 est $\{-5; 8\}$
\task l'ensemble $Z_f$ des zéros de $f$ est $\{-4; 7\}$
\task $\displaystyle\lim_{x \to -1} f(x) = -\infty$ et $\displaystyle\lim_{x \to -1^+} f(x) = -\infty$
\task $f$ n'est pas définie en $x = -3$ et $\displaystyle\lim_{x \to -3} f(x) = -2$
\task $f(1) = -5$ et $\displaystyle\lim_{x \to +\infty} f(x)$ n'existe pas
\task $\displaystyle\lim_{x \to -\infty} f(x) = 3$ et $\displaystyle\lim_{x \to +\infty} f(x) = 3$
\task $f$ n'est pas dérivable en $x = 4$
\end{tasks}
}
\correction{
\tcblower
\textit{Generated by AI}

Pour tracer le graphe d'une fonction $f$ satisfaisant toutes ces conditions, analysons chaque contrainte :

\textbf{Points et valeurs spécifiques :}
\begin{itemize}
\item $f(2) = -4$ : point $(2, -4)$ sur la courbe
\item Préimages de 1 : $f(-5) = 1$ et $f(8) = 1$
\item Zéros : $f(-4) = 0$ et $f(7) = 0$
\item $f(1) = -5$ : point $(1, -5)$ sur la courbe
\end{itemize}

\textbf{Comportement asymptotique :}
\begin{itemize}
\item En $x = -1$ : asymptote verticale avec $\lim_{x \to -1^-} f(x) = -\infty$ et $\lim_{x \to -1^+} f(x) = -\infty$
\item En $x = -3$ : trou (discontinuité amovible) car $f$ n'est pas définie mais $\lim_{x \to -3} f(x) = -2$
\item $\lim_{x \to -\infty} f(x) = 3$ et $\lim_{x \to +\infty} f(x) = 3$ : asymptote horizontale $y = 3$
\end{itemize}

\textbf{Conditions supplémentaires :}
\begin{itemize}
\item Non dérivable en $x = 4$ : point anguleux ou cuspide
\item Condition contradictoire : on nous dit que $\lim_{x \to +\infty} f(x)$ n'existe pas ET $\lim_{x \to +\infty} f(x) = 3$. Nous allons privilégier $\lim_{x \to +\infty} f(x) = 3$ (condition 7).
\end{itemize}

\textbf{Esquisse du graphe :}

Le graphe doit montrer :
\begin{enumerate}
\item Une asymptote horizontale en $y = 3$ pour $x \to \pm\infty$
\item Une asymptote verticale en $x = -1$
\item Un trou en $(-3, -2)$
\item Les points : $(-5, 1)$, $(-4, 0)$, $(1, -5)$, $(2, -4)$, $(7, 0)$, $(8, 1)$
\item Un point anguleux en $x = 4$
\end{enumerate}
}

