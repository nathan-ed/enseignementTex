\titre{}
\theme{derivation}
\auteur{Nathan Scheinmann}
\niveau{3M}
\source{sesamath3e}
\type{serie}
\piments{2}
\pts{}
\annee{2425}

\contenu{
\tcblower
La somme de deux nombres positifs est 16. Déterminer ces deux nombres de telle façon que :

\begin{tasks}(1)
\task la somme des cubes soit minimale; quelle est alors la valeur de ce minimum ?
\task la somme des cubes soit maximale; quelle est alors la valeur de ce maximum
\end{tasks}
}
\correction{
\tcblower
\textit{Generated by AI}

Notons $x$ et $y$ les deux nombres positifs tels que $x + y = 16$.

\textbf{Question 1 : Minimum de la somme des cubes}

On cherche à minimiser $f(x) = x^3 + y^3 = x^3 + (16-x)^3$ pour $x \in [0;16]$.

Calculons la dérivée :
\[
f'(x) = 3x^2 + 3(16-x)^2 \cdot (-1) = 3x^2 - 3(16-x)^2
\]

On cherche les points critiques en résolvant $f'(x) = 0$ :
\begin{align*}
3x^2 - 3(16-x)^2 &= 0 \\
x^2 &= (16-x)^2 \\
x^2 &= 256 - 32x + x^2 \\
32x &= 256 \\
x &= 8
\end{align*}

Étudions le signe de $f'(x)$ :
\begin{itemize}
\item Si $x < 8$ : $f'(x) < 0$ donc $f$ est décroissante
\item Si $x > 8$ : $f'(x) > 0$ donc $f$ est croissante
\end{itemize}

Donc $f$ admet un minimum en $x = 8$, ce qui donne $y = 8$.

La valeur minimale est : $f(8) = 8^3 + 8^3 = 2 \times 512 = 1024$

\textbf{Réponse :} Les deux nombres sont $x = 8$ et $y = 8$, et le minimum de la somme des cubes est $1024$.

\textbf{Question 2 : Maximum de la somme des cubes}

La fonction $f$ est continue sur $[0;16]$ et admet son minimum en $x=8$. Le maximum est donc atteint aux extrémités de l'intervalle.

Calculons :
\begin{itemize}
\item $f(0) = 0^3 + 16^3 = 4096$
\item $f(16) = 16^3 + 0^3 = 4096$
\end{itemize}

\textbf{Réponse :} Les deux nombres sont $(0;16)$ ou $(16;0)$, et le maximum de la somme des cubes est $4096$.
}

