\titre{undefined}
\theme{equations}
\auteur{Nathan Scheinmann}
\niveau{1M}
\source{undefined}
\type{serie}
\piments{1}
\pts{}
\annee{2425}

\contenu{
	\tcblower
Pour organiser une sortie de fin d’année, un collège loue des cars. Il y a des grands cars de
56 places et des petits cars de 44 places. Il y a quatre grands cars de plus que de petits.
624 élèves participent à la sortie et tous les cars sont remplis.
Combien le collège a-t-il loué de cars de chaque catégorie?
}
\correction{
	\tcblower
On pose les inconnues
\begin{align*}
	&x=\text{Nombre de cars à 44 places} &&y=\text{Nombre de cars à 56 places}
\end{align*}
On obtient le système
$
\begin{cases}
y-x=4\\
56y+44x=624
\end{cases}
$
On résout le système (par exemple par substitution) et on obtient que $x=4$ et $y=8$. Le cycle a donc loué 4 petits cars et 8 grands cars. 
}

