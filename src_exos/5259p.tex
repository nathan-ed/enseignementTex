\titre{}
\theme{geometrie}
\auteur{Nathan Scheinmann}
\niveau{1M}
\source{}
\type{serie}
\piments{2}
\pts{}
\annee{2425}

\contenu{
	\tcblower

	\begin{minipage}[t]{0.6\textwidth}{
	\vspace{0pt}
	Déterminer la longueur de $h$ sachant que le diamètre des disques est de $5cm$ et que les deux disques sont inscrits dans un rectangle. 
	
	Indice : utiliser les centres des cercles. 
	}
	\end{minipage}
	\hfill
	\begin{minipage}[t]{0.3\textwidth}{
	\vspace{0pt}
\begin{tikzpicture}[scale=1]
    % Define center of the first circle
    \tkzDefPoint(2,3){A}
    
    % Define center of the second circle using slope -4
    \tkzDefPoint(2+1, 3-2){B}

    \tkzDefPointBy[homothety=center B ratio 0.5](A) \tkzGetPoint{C}
    \tkzCalcLength(A,C)\tkzGetLength{radius}
    \tkzDrawCircle(A,C)
    \tkzDrawCircle(B,C)

    \tkzDefPoint(2-\radius,3){D}
    \tkzDefPoint(2,3+\radius){E}
    \tkzDefPoint(3,1-\radius){F}
    \tkzDefPoint(3+\radius,1){G}
    \tkzDefPoint(2-\radius,1-\radius){H}
    \tkzDefPoint(2-\radius,3+\radius){I}
    \tkzDefPoint(3+\radius,1-\radius){J}
    \tkzDefPoint(3+\radius,3+\radius){K}
    % Draw the centers
    \tkzDrawPoints(A,B)
    \tkzDrawSegments(K,J J,H)
    \tkzDrawSegment[dim={$h$,6pt,}](H,I)
    \tkzDrawSegment[dim={$8cm$,6pt,}](J,H)
    \tkzDrawSegment[dashed](I,K)

    % Draw the line connecting the centers (optional)
    \tkzDrawLine[add=0.70 and 0.70, dashed](A,B)


\end{tikzpicture}
	}
	\end{minipage}
}
\correction{
	\tcblower
	$h=9cm$

}

