\titre{}
\theme{cercle}
\auteur{Nathan Scheinmann}
\niveau{1M}
\source{abdou}
\type{serie}
\piments{2}
\pts{}
\annee{2425}

\contenu{
	\tcblower
\begin{minipage}[t]{0.4\textwidth}{
\vspace{0pt}
Soit $O$ le centre du cercle $(C)$. 

$ABCDE$ est un pentagone régulier inscrit dans $(C)$. 

Déterminer la valeur de l'angle $\alpha$. 
}
\end{minipage}
\begin{minipage}[t]{0.4\textwidth}{
\vspace{0pt}
\begin{center}
\begin{tikzpicture}[scale=0.7]
	\tkzSetUpPoint[shape=cross]
% Define the center and a point on the circle
  \tkzDefPoint(0,0){O}
  \tkzDefPoint(3,0){A}
  
\tkzLabelCircle[above right](O,A)(30){$(C)$}
  % Draw the circle with center O passing through A
  \tkzDrawCircle(O,A)
    % Compute the other vertices by successive rotations of 72°

  \tkzDefPointBy[rotation= center O angle 72](A) \tkzGetPoint{B}
  \tkzDefPointBy[rotation= center O angle 72](B) \tkzGetPoint{C}
  \tkzDefPointBy[rotation= center O angle 72](C) \tkzGetPoint{D}
  \tkzDefPointBy[rotation= center O angle 72](D) \tkzGetPoint{E}
  
  % Draw the inscribed pentagon
  \tkzDrawPolygon(A,B,C,D,E)
  
  \tkzInterLL(A,D)(C,E)\tkzGetPoint{F}
  % Mark the vertices with crosses
  \tkzDrawPoints(A,B,C,D,E,O,F)
  
  \tkzDrawSegment(A,D)
  \tkzDrawSegment(C,E)
  % Label the vertices
  \tkzLabelPoints[right](A)
  \tkzLabelPoints[above](B)
  \tkzLabelPoints[above right](F, O)
  \tkzLabelPoints[above left](C)
  \tkzLabelPoints[left](D)
  \tkzLabelPoints[below](E)
  \tkzMarkAngle[size=0.8cm, mark=none](E,F,A)
  \tkzLabelAngle[pos=1.2](E,F,A){$\alpha$}

\end{tikzpicture}	
\end{center}
}
\end{minipage}
}
\correction{
	\tcblower
	$\widehat{DOE}=72^\circ$. Par le thm de l'angle inscrit et de l'angle au centre, $\widehat{ADE}=36^\circ$. Par ailleurs, $\widehat{AOC}=72\cdot 2=144^\circ$ et par le thm de l'angle inscrit et de l'angle au centre, $\widehat{AEC}=72^\circ$. Ainsi par la somme des angles dans un triangle, $\alpha=180-72-36=72^\circ$. 
}

