\titre{}
\theme{derivation}
\auteur{Nathan Scheinmann}
\niveau{3M}
\source{crm}
\type{serie}
\piments{1}
\pts{}
\annee{2526}

\contenu{
\tcblower
Soit la fonction $f$ définie par: $f(x)=\left\{\begin{array}{l}x^2+a x-1, \text { si } x \leq 1 \\ 3 x-1, \text { si } x>1\end{array}\right.$
\begin{tasks}
\task Lorsque $a$ vaut $1$, la fonction $f$ est-elle continue en 1 ? (Illustrer graphiquement.)
\task Pour quelles valeurs du paramètre $a$ cette fonction sera-t-elle continue en $1$ ? (Idem.)
\task Pour la valeur de $a$ trouvée en b), la fonction $f$ est-elle dérivable en $1$ ?
\end{tasks}
}
\correction{
\tcblower
\begin{tasks}(3)
	\task Non;
	\task $2$;
	\task limite à gauche vaut $4$ et limite à droite vaut $3$.
\end{tasks}
}

