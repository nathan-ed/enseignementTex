\titre{}
\theme{limites}
\auteur{Nathan Scheinmann}
\niveau{3M}
\source{sesamath3e}
\type{serie}
\piments{4}
\pts{}
\annee{2425}

\contenu{
\tcblower
\noindent Dans chaque cas, déterminer l'expression algébrique d'une fonction \(f\), sur \(\mathbb{R}\), satisfaisant aux conditions données :  
\begin{tasks}
\task \(f\) est continue partout sauf pour \(x=2\) et \(\displaystyle\lim_{x\to 2}f(x)\) n'existe pas.
\task \(f\) admet une limite à gauche et une limite à droite au point \(x=2\) mais elle est non continue en ce point.
\end{tasks}
}
\correction{
\tcblower
\begin{tasks}(1)
\task \( f(x) = \begin{cases}
1 & \text{si } x < 2 \\
-1 & \text{si } x > 2
\end{cases} \). La fonction n’est pas définie en \( x = 2 \), et la limite en \( x = 2 \) n’existe pas car les limites à gauche et à droite sont différentes.

\task \( f(x) = \begin{cases}
x & \text{si } x < 2 \\
5 & \text{si } x = 2 \\
x & \text{si } x > 2
\end{cases} \). La limite en \( x = 2 \) existe et vaut 2, mais \( f(2) = 5 \ne 2 \), donc \( f \) est discontinue en ce point.
\end{tasks}
}

