\titre{}
\theme{dérivées}
\auteur{Nathan Scheinmann}
\niveau{3M}
\source{analysis}
\type{serie}
\piments{2}
\pts{}
\annee{2425}

\contenu{
\tcblower
Trouver $d$ sachant que $(d, f(d))$ est un point d'inflexion de
$$f(x) = (x - a)(x - b)(x - c).$$
}
\correction{
\tcblower
%% GENERATED BY AI %%
Pour trouver le point d'inflexion, nous devons calculer $f''(x)$ et utiliser la condition qu'il s'annule en $x = d$.

Développons $f(x)$ :
\begin{align*}
f(x) &= (x-a)(x-b)(x-c) \\
&= (x-a)[(x-b)(x-c)] \\
&= (x-a)[x^2 - (b+c)x + bc] \\
&= x^3 - (b+c)x^2 + bcx - ax^2 + a(b+c)x - abc \\
&= x^3 - (a+b+c)x^2 + (ab+ac+bc)x - abc
\end{align*}

Calculons les dérivées :
\begin{align*}
f'(x) &= 3x^2 - 2(a+b+c)x + (ab+ac+bc) \\
f''(x) &= 6x - 2(a+b+c)
\end{align*}

Pour un point d'inflexion en $x = d$, nous devons avoir $f''(d) = 0$ :
\begin{align*}
6d - 2(a+b+c) &= 0 \\
6d &= 2(a+b+c) \\
d &= \dfrac{a+b+c}{3}
\end{align*}

Le point d'inflexion se trouve donc à l'abscisse moyenne des trois racines.

$$\boxed{d = \dfrac{a+b+c}{3}}$$
}

