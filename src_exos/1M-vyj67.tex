\titre{}
\theme{limites}
\auteur{Nathan Scheinmann}
\niveau{3M}
\source{sesamath3e}
\type{serie}
\piments{1}
\pts{}
\annee{2425}

\contenu{
\tcblower
Soit une fonction \(f\) définie par  
\[
f(x)=\frac{x^{4}+x^{3}-x^{2}-x}{1-x}.
\]

\begin{tasks}
\task Déterminer son domaine de définition et ses zéros.
\task Calculer les images de \(1,9 ; 1,99 ; 1,999\) puis de \(2,1 ; 2,01\) et \(2,001\).
\task Que penser de \(\displaystyle\lim_{x\to 2} f(x)\)~?
\task Calculer les images de \(0,9 ; 0,99 ; 0,999\) puis de \(1,1 ; 1,01\) et \(1,001\).
\task Que penser de \(\displaystyle\lim_{x\to 1} f(x)\)~?
\task Interpréter graphiquement avec un grapheur.
\end{tasks}
}
\correction{
\tcblower
Utiliser la calculatrice.
}

