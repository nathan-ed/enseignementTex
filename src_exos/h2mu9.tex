\titre{}
\theme{derivation}
\auteur{Nathan Scheinmann}
\niveau{3M}
\source{sesamath3e}
\type{serie}
\piments{2}
\pts{}
\annee{2425}

\contenu{
\tcblower
Expliquer intuitivement pourquoi si une fonction est constante sur un intervalle, alors sa dérivée est nulle sur tout l’intervalle.
}
\correction{
	\tcblower
La dérivée en un point correspond à la pente de la tangente en ce point. Si la droite est constante, la pente en tout point est nulle. 
}

