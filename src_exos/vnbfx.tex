\titre{}
\theme{derivation}
\auteur{Nathan Scheinmann}
\niveau{3M}
\source{sesamath3e}
\type{serie}
\piments{2}
\pts{}
\annee{2425}

\contenu{
\tcblower
Déterminer les constantes $a$ et $b$ pour que la fonction définie par $f(x) = x^3 + ax^2 - 6x + b$ admette au point $(1;1)$ une tangente horizontale.
}
\correction{
\tcblower
\textit{Generated by AI}

Pour que la fonction admette une tangente horizontale au point $(1;1)$, deux conditions doivent être satisfaites :
\begin{enumerate}
\item Le point $(1;1)$ appartient à la courbe : $f(1) = 1$
\item La tangente est horizontale en ce point : $f'(1) = 0$
\end{enumerate}

\textbf{Condition 1 :} $f(1) = 1$

$$f(1) = 1^3 + a(1)^2 - 6(1) + b = 1$$
$$1 + a - 6 + b = 1$$
$$a + b = 6$$

\textbf{Condition 2 :} $f'(1) = 0$

Calculons d'abord la dérivée :
$$f'(x) = 3x^2 + 2ax - 6$$

Donc :
$$f'(1) = 3(1)^2 + 2a(1) - 6 = 0$$
$$3 + 2a - 6 = 0$$
$$2a = 3$$
$$a = \dfrac{3}{2}$$

À partir de $a + b = 6$ :
$$b = 6 - a = 6 - \dfrac{3}{2} = \dfrac{12 - 3}{2} = \dfrac{9}{2}$$

Les constantes sont : $\boxed{a = \dfrac{3}{2}}$ et $\boxed{b = \dfrac{9}{2}}$.
}

