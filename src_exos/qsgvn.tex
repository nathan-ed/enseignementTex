\titre{}
\theme{derivation}
\auteur{Nathan Scheinmann}
\niveau{3M}
\source{sesamath3e}
\type{serie}
\piments{2}
\pts{}
\annee{2425}

\contenu{
\tcblower
Déterminer les asymptotes des fonctions suivantes et interpréter graphiquement :

\begin{tasks}(1)
\task $f(x) = x - \sqrt{x - 1}$
\task $f(x) = x + \sqrt{1 - x^2}$
\task $f(x) = 2x - \sqrt{4x^2 + 2x + 1}$
\task $f(x) = \sqrt{x^2 + 2x} - \sqrt{x + 1}$
\end{tasks}
}
\correction{
\tcblower
\textit{Generated by AI}

\textbf{Question a) : $f(x) = x - \sqrt{x - 1}$}

\textit{Domaine :} $[1; +\infty[$

Asymptote oblique quand $x \to +\infty$ :
\[
\lim_{x \to +\infty} [f(x) - x] = \lim_{x \to +\infty} (-\sqrt{x - 1}) = -\infty
\]

Donc pas d'asymptote oblique de la forme $y = ax + b$.

Étudions $\lim_{x \to +\infty} \frac{f(x)}{x} = \lim_{x \to +\infty} \left(1 - \frac{\sqrt{x-1}}{x}\right) = 1$

Comme $f(x) - x = -\sqrt{x-1} \to -\infty$, il n'y a pas d'asymptote.

\textbf{Question b) : $f(x) = x + \sqrt{1 - x^2}$}

\textit{Domaine :} $[-1; 1]$

Comme le domaine est borné, pas d'asymptote à l'infini.

\textbf{Question c) : $f(x) = 2x - \sqrt{4x^2 + 2x + 1}$}

Asymptote quand $x \to +\infty$ :
\[
f(x) = 2x - \sqrt{4x^2 + 2x + 1} = 2x - 2x\sqrt{1 + \frac{1}{2x} + \frac{1}{4x^2}}
\]

Développement limité :
\[
\sqrt{1 + u} \approx 1 + \frac{u}{2} \text{ pour } u \to 0
\]

Donc :
\begin{align*}
f(x) &\approx 2x - 2x\left(1 + \frac{1}{4x} + O\left(\frac{1}{x^2}\right)\right) \\
&\approx 2x - 2x - \frac{1}{2} \\
&\approx -\frac{1}{2}
\end{align*}

Asymptote horizontale : $y = -\frac{1}{2}$ quand $x \to +\infty$.

Pour $x \to -\infty$ :
\[
f(x) = 2x - \sqrt{4x^2 + 2x + 1} = 2x - |2x|\sqrt{1 + \frac{1}{2x} + \frac{1}{4x^2}}
\]

Comme $x < 0$, $|2x| = -2x$, donc :
\[
f(x) = 2x + 2x\sqrt{1 + \frac{1}{2x}} \approx 2x + 2x + \frac{1}{2} = 4x + \frac{1}{2}
\]

Asymptote oblique : $y = 4x + \frac{1}{2}$ quand $x \to -\infty$.

\textbf{Question d) : $f(x) = \sqrt{x^2 + 2x} - \sqrt{x + 1}$}

Pour $x \to +\infty$ :
\[
f(x) = \sqrt{x^2(1 + \frac{2}{x})} - \sqrt{x + 1} = x\sqrt{1 + \frac{2}{x}} - \sqrt{x}\sqrt{1 + \frac{1}{x}}
\]
\[
\approx x\left(1 + \frac{1}{x}\right) - \sqrt{x} = x + 1 - \sqrt{x}
\]

Comme $\sqrt{x} \to +\infty$, $f(x) \sim x$ quand $x \to +\infty$.

Asymptote oblique : $y = x + 1$ quand $x \to +\infty$.
}

