\titre{}
\theme{dérivées}
\auteur{Nathan Scheinmann}
\niveau{3M}
\source{fundamentum}
\type{serie}
\piments{2}
\pts{}
\annee{2425}

\contenu{
\tcblower
Sur l'axe $Ox$ on fixe une première source de lumière en $x=0$ et une deuxième en $x=10$. On note $I_1$ et $I_2$ les intensités lumineuses des deux sources, $L_1$ et $L_2$ les intensités des flux lumineux en un point provenant respectivement de la première source et de la deuxième source. Sachant que $I_2 = 4I_1$, et que l'intensité du flux lumineux en un point est proportionnelle à l'intensité lumineuse de la source considérée et inversement proportionnelle au carré de la distance séparant ce point de la source, déterminer le point de l'intervalle $[0;10]$ qui reçoit un flux total minimal des deux sources. Calculer le rapport des distances séparant ce point aux deux sources.
}
\correction{

\tcblower
\textit{Generated by AI}
{\scriptsize \textit{Correction générée par IA}}

Plaçons les deux sources sur l'axe $Ox$ : la première en $x = 0$ et la deuxième en $x = 10$.

Pour un point situé en position $x$ sur l'intervalle $[0, 10]$ :
\begin{itemize}
\item La distance à la première source est $d_1 = x$
\item La distance à la deuxième source est $d_2 = 10 - x$
\end{itemize}

Le flux lumineux provenant de chaque source est proportionnel à l'intensité lumineuse et inversement proportionnel au carré de la distance :
\[L_1 = \dfrac{kI_1}{x^2} \quad \text{et} \quad L_2 = \dfrac{kI_2}{(10-x)^2}\]

où $k$ est une constante de proportionnalité.

Sachant que $I_2 = 4I_1$, le flux total est :
\[L(x) = \dfrac{kI_1}{x^2} + \dfrac{4kI_1}{(10-x)^2} = kI_1\left(\dfrac{1}{x^2} + \dfrac{4}{(10-x)^2}\right)\]

Pour minimiser $L(x)$, nous dérivons par rapport à $x$ :
\[L'(x) = kI_1\left(-\dfrac{2}{x^3} + \dfrac{8}{(10-x)^3}\right)\]

L'annulation de la dérivée donne :
\[-\dfrac{2}{x^3} + \dfrac{8}{(10-x)^3} = 0\]
\[\dfrac{8}{(10-x)^3} = \dfrac{2}{x^3}\]
\[\dfrac{4}{(10-x)^3} = \dfrac{1}{x^3}\]
\[4x^3 = (10-x)^3\]
\[\sqrt[3]{4} \cdot x = 10 - x\]
\[x(\sqrt[3]{4} + 1) = 10\]
\[x = \dfrac{10}{\sqrt[3]{4} + 1} = \dfrac{10}{1 + \sqrt[3]{4}}\]

Rationalisons en multipliant par $\dfrac{\sqrt[3]{2}}{\sqrt[3]{2}}$ :
\[x = \dfrac{10}{1 + 2^{2/3}} \cdot \dfrac{2^{1/3}}{2^{1/3}} = \dfrac{10 \cdot 2^{1/3}}{2^{1/3} + 2}\]

Numériquement : $\sqrt[3]{4} \approx 1{,}587$, donc $x \approx \dfrac{10}{2{,}587} \approx 3{,}87$.

\textbf{Rapport des distances :}
\[\dfrac{d_1}{d_2} = \dfrac{x}{10-x} = \dfrac{\dfrac{10}{1 + \sqrt[3]{4}}}{\dfrac{10\sqrt[3]{4}}{1 + \sqrt[3]{4}}} = \dfrac{1}{\sqrt[3]{4}} = \dfrac{1}{2^{2/3}} = 2^{-2/3}\]

Le point qui reçoit un flux minimal est situé à $\boxed{x = \dfrac{10}{1 + \sqrt[3]{4}} \approx 3{,}87}$ de la première source.

Le rapport des distances aux deux sources est $\boxed{\dfrac{d_1}{d_2} = 2^{-2/3} = \dfrac{1}{\sqrt[3]{4}} \approx 0{,}63}$.

}

