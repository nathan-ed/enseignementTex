\titre{}
\theme{dérivées}
\auteur{Nathan Scheinmann}
\niveau{3M}
\source{fundamentum}
\type{serie}
\piments{2}
\pts{}
\annee{2425}

\contenu{
\tcblower
Sur l'axe $Ox$ on fixe une première source de lumière en $x=0$ et une deuxième en $x=10$. On note $I_1$ et $I_2$ les intensités lumineuses des deux sources, $L_1$ et $L_2$ les intensités des flux lumineux en un point provenant respectivement de la première source et de la deuxième source. Sachant que $I_2 = 4I_1$, et que l'intensité du flux lumineux en un point est proportionnelle à l'intensité lumineuse de la source considérée et inversement proportionnelle au carré de la distance séparant ce point de la source, déterminer le point de l'intervalle $[0;10]$ qui reçoit un flux total minimal des deux sources. Calculer le rapport des distances séparant ce point aux deux sources.
}
\correction{
\tcblower
}

