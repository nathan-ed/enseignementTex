\titre{}
\theme{fonctions}
\auteur{Nathan Scheinmann}
\niveau{1M}
\source{}
\type{serie}
\piments{2}
\pts{}
\annee{2425}

\contenu{
	\tcblower
Décrire par des inéquations les intervalles suivants:
\begin{tasks}(2)
	\task $]-\infty;-3]$
	\task $]-2;+\infty[$
	\task $[0;2]$ 
	\task $]-3;3[$
	\task $]-5; -4[$
	\task $]-2;-1[\cup[0;+\infty[$
	\task $]-\infty;0[\cup[1;3]$
	\task $]-\infty;4] \cup [7;+\infty[$
\end{tasks}
}
\correction{
	\tcblower
\begin{tasks}(4)
	\task $x\leq -3$
	\task $x> -2$
	\task $0\leq x\leq 2$
	\task $-3<x<3$
	\task $-5<x<-4$
	\task*(2) $-2<x<-1$ ou $0\leq x$
	\task*(2) $x<0$ ou $1\leq x\leq 3$
	\task*(2) $x\leq 4$ ou $x\geq 7$
\end{tasks}	

}

