\titre{}
\theme{derivation}
\auteur{Nathan Scheinmann}
\niveau{3M}
\source{sesamath3e}
\type{serie}
\piments{2}
\pts{}
\annee{2425}

\contenu{
\tcblower
Déterminer l'expression algébrique d'une fonction $f$ telle que $x = -2$ et $x = 4$ soient des asymptotes verticales, et que $y = -1$ soit une asymptote horizontale à $+\infty$.
Justifier que la fonction donnée remplie bien les conditions demandées.
}
\correction{
	\tcblower
Par exemple 
$f(x)=\dfrac{-x^2+1}{(x+2)(x-4)}$
}

