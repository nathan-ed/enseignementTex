\titre{}
\theme{dérivées}
\auteur{Nathan Scheinmann}
\niveau{3M}
\source{analysis}
\type{serie}
\piments{2}
\pts{}
\annee{2425}

\contenu{
\tcblower
Trouver $c$ sachant que le graphe de $f(x) = cx^2 + x^{-2}$ a un point d'inflexion en $(1, f(1))$.
}
\correction{
\tcblower
\textit{Generated by AI}
Pour qu'il y ait un point d'inflexion en $(1, f(1))$, nous devons avoir $f''(1) = 0$.

Calculons les dérivées de $f(x) = cx^2 + x^{-2}$ :
\begin{align*}
f'(x) &= 2cx - 2x^{-3} \\
f''(x) &= 2c + 6x^{-4} = 2c + \dfrac{6}{x^4}
\end{align*}

La condition $f''(1) = 0$ donne :
\begin{align*}
2c + \dfrac{6}{1^4} &= 0 \\
2c + 6 &= 0 \\
c &= -3
\end{align*}

Vérifions que $(1, f(1))$ est bien un point d'inflexion : avec $c = -3$, nous avons $f''(x) = -6 + \dfrac{6}{x^4}$.

Pour $x$ proche de 1 :
\begin{itemize}
\item Si $x < 1$ : $x^4 < 1$ donc $\dfrac{6}{x^4} > 6$ et $f''(x) > 0$
\item Si $x > 1$ : $x^4 > 1$ donc $\dfrac{6}{x^4} < 6$ et $f''(x) < 0$
\end{itemize}

La dérivée seconde change bien de signe en $x = 1$.

$$\boxed{c = -3}$$
}

