\titre{}
\theme{derivation}
\auteur{Nathan Scheinmann}
\niveau{3M}
\source{musy}
\type{serie}
\piments{1}
\pts{}
\annee{2526}

\contenu{
\tcblower
Soit la fonction $f(x)=\frac{x^2}{5}$.
\begin{tasks}
\task Déterminer l'équation des tangentes à $f$ qui passent par le point $A=(-1 ;-3)$.
\task Représenter la situation proposée sur un repère.
\end{tasks}
}
\correction{
\tcblower
{\scriptsize \textit{Correction générée par IA}}

\begin{tasks}
\task
La tangente au graphe de $f$ en un point $(a;f(a))$ a pour équation :
\[
y=f'(a)(x-a)+f(a)
\]

Le point $A=(-1;-3)$ appartient à la tangente, donc :
\[
-3=f'(a)((-1)-a)+f(a)
\]

On a $f(x)=\dfrac{x^2}{5}$, donc $f'(x)=\dfrac{2x}{5}$.

Ainsi $f(a)=\dfrac{a^2}{5}$ et $f'(a)=\dfrac{2a}{5}$.

L'équation devient :
\[
-3=\dfrac{2a}{5}(-1-a)+\dfrac{a^2}{5}
\]

Multiplions par 5 :
\[
-15=2a(-1-a)+a^2=-2a-2a^2+a^2=-2a-a^2
\]

D'où :
\[
a^2+2a-15=0
\]

En résolvant cette équation du second degré :
\[
\Delta=4+60=64 \quad \Rightarrow \quad a=\dfrac{-2\pm 8}{2}
\]

On obtient $a_1=3$ ou $a_2=-5$.

\textbf{Pour $a=3$ :}
\begin{itemize}
\item $f(3)=\dfrac{9}{5}$
\item $f'(3)=\dfrac{6}{5}$
\end{itemize}

L'équation de la tangente est :
\[
y=\dfrac{6}{5}(x-3)+\dfrac{9}{5}=\dfrac{6}{5}x-\dfrac{18}{5}+\dfrac{9}{5}=\dfrac{6}{5}x-\dfrac{9}{5}
\]

\textbf{Pour $a=-5$ :}
\begin{itemize}
\item $f(-5)=\dfrac{25}{5}=5$
\item $f'(-5)=\dfrac{-10}{5}=-2$
\end{itemize}

L'équation de la tangente est :
\[
y=-2(x-(-5))+5=-2(x+5)+5=-2x-10+5=-2x-5
\]

Les deux tangentes ont pour équations $y=\dfrac{6}{5}x-\dfrac{9}{5}$ et $y=-2x-5$.

\task La représentation graphique montrerait la parabole $f(x)=\dfrac{x^2}{5}$, le point $A=(-1;-3)$, et les deux tangentes passant par $A$ et touchant la parabole aux points $(3;\dfrac{9}{5})$ et $(-5;5)$.
\end{tasks}
}

