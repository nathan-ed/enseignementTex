\titre{}
\theme{derivation}
\auteur{Nathan Scheinmann}
\niveau{3M}
\source{musy}
\type{serie}
\piments{1}
\pts{}
\annee{2526}

\contenu{
\tcblower
Soit la fonction $f(x)=\frac{x^2}{5}$.
\begin{tasks}
\task Déterminer l'équation des tangentes à $f$ qui passent par le point $A=(-1 ;-3)$.
\task Représenter la situation proposée sur un repère.
\end{tasks}
}
\correction{
\tcblower
{\scriptsize \textit{Correction générée par IA}}

\begin{tasks}
\task
La tangente en $(a;f(a))$ passant par $A=(-1;-3)$ vérifie~:
\[
-3=f'(a)(-1-a)+f(a)
\]

Avec $f(x)=\dfrac{x^2}{5}$ et $f'(x)=\dfrac{2x}{5}$~:
\( \begin{aligned}
-3 &= \dfrac{2a}{5}(-1-a)+\dfrac{a^2}{5} \\
-15 &= -2a-2a^2+a^2 \\
a^2+2a-15 &= 0
\end{aligned} \)

$\Delta=64 \implies a=3$ ou $a=-5$.

\textbf{Pour $a=3$~:} $f(3)=\dfrac{9}{5}$, $f'(3)=\dfrac{6}{5}$, donc $y=\dfrac{6}{5}x-\dfrac{9}{5}$.

\textbf{Pour $a=-5$~:} $f(-5)=5$, $f'(-5)=-2$, donc $y=-2x-5$.

\task Le graphe montre la parabole, le point $A=(-1;-3)$, et les deux tangentes touchant aux points $(3;\dfrac{9}{5})$ et $(-5;5)$.
\end{tasks}
}

