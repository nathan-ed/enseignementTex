\titre{}
\theme{derivation}
\auteur{Nathan Scheinmann}
\niveau{3M}
\source{crm}
\type{serie}
\piments{2}
\pts{}
\annee{2526}

\contenu{
\tcblower
Soit la fonction $f(x)=\dfrac{1}{x^2}$.
\begin{tasks}
\task Déterminer l'équation de la tangente au graphe de $f$ au point d'abscisse $2$.
\task En quel point la tangente au graphe de $f$ a-t-elle une pente de $\dfrac{1}{4}$ ?
\task Déterminer les points de tangence des tangentes au graphe de $f$ qui passent par le point $\left(\dfrac{1}{2};2\right)$.
\end{tasks}
}
\correction{
\tcblower
On a $f(x)=\dfrac{1}{x^2}=x^{-2}$, donc $f'(x)=-2x^{-3}=-\dfrac{2}{x^3}$.

\begin{tasks}
\task Pour $x=2$, on a $f(2)=\dfrac{1}{4}$ et $f'(2)=-\dfrac{2}{8}=-\dfrac{1}{4}$.

L'équation de la tangente est :
\[
y=-\dfrac{1}{4}(x-2)+\dfrac{1}{4}\iff y=-\dfrac{1}{4}x+\dfrac{3}{4}
\]

\task On cherche $a$ tel que $f'(a)=\dfrac{1}{4}$.
Pour $a\neq 0$~:
\[
-\dfrac{2}{a^3}=\dfrac{1}{4} \quad\implies \quad a^3=-8 \quad\Rightarrow\quad a=-2
\]

Le point de tangence est $\left(-2;\dfrac{1}{4}\right)$.

\task Soit $a$ l'abscisse du point de tangence. L'équation de la tangente en ce point est :
\[
y=-\dfrac{2}{a^3}(x-a)+\dfrac{1}{a^2}
\]

Cette tangente passe par $\left(\dfrac{1}{2};2\right)$, donc :
\[
2=-\dfrac{2}{a^3}\left(\dfrac{1}{2}-a\right)+\dfrac{1}{a^2}\iff 2=-\dfrac{1}{a^3}+\dfrac{3}{a^2}
\]

En multipliant par $a^3$ $(a\neq0)$:
\[
2a^3=-1+3a \quad\Rightarrow\quad 2a^3-3a+1=0
\]

Par factorisation : $(a-1)(2a^2+2a-1)=0$.

Donc $a=1$ ou $a=\dfrac{-2\pm\sqrt{4+8}}{4}=\dfrac{-2\pm 2\sqrt{3}}{4}=\dfrac{-1\pm\sqrt{3}}{2}$.

Les points de tangence sont : $\left(1,1\right)$, $\left(\dfrac{-1+\sqrt{3}}{2};\dfrac{4}{4-2\sqrt{3}}\right)$ et $\left(\dfrac{-1-\sqrt{3}}{2};\dfrac{4}{4+2\sqrt{3}}\right)$.
\end{tasks}
}

