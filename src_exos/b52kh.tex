\titre{}
\theme{dérivées}
\auteur{Nathan Scheinmann}
\niveau{3M}
\source{fundamentum}
\type{serie}
\piments{2}
\pts{}
\annee{2425}

\contenu{
\tcblower
On fabrique un cornet de forme conique en rejoignant les bords rectilignes d'un secteur circulaire de rayon $r$. Quel est le volume du plus grand cornet possible ?
\begin{center}
\begin{tikzpicture}
    \tkzDefPoint(0,0){O}
    \tkzDefPoint(2,0){A}
    \tkzDefPoint(90:2){B}
    \tkzDefPoint(330:2){C}
    \tkzDrawCircle(O,A)
    \tkzDrawSegment[dashed](O,B)
    \tkzDrawSegment[dashed](O,C)
    \tkzFillSector[gray!20](O,B)(C)
\end{tikzpicture}
\end{center}
}
\correction{
\tcblower
\textit{Generated by AI}
{\scriptsize \textit{Correction générée par IA}}

Considérons un secteur circulaire de rayon $r$ et d'angle au centre $\theta$. La longueur de l'arc est $L = r\theta$.

Lorsqu'on forme le cône, cet arc devient le périmètre de la base du cône : $2\pi R = r\theta$, où $R$ est le rayon de la base du cône.

Donc :
\[R = \frac{r\theta}{2\pi}\]

La hauteur du cône se calcule par Pythagore (l'apothème du cône vaut $r$) :
\[h = \sqrt{r^2 - R^2} = \sqrt{r^2 - \frac{r^2\theta^2}{4\pi^2}} = r\sqrt{1 - \frac{\theta^2}{4\pi^2}}\]

Le volume du cône est :
\[V = \frac{1}{3}\pi R^2 h = \frac{1}{3}\pi \left(\frac{r\theta}{2\pi}\right)^2 \cdot r\sqrt{1 - \frac{\theta^2}{4\pi^2}} = \frac{r^3\theta^2}{12\pi}\sqrt{1 - \frac{\theta^2}{4\pi^2}}\]

Pour maximiser $V$, posons $u = \theta^2$ (avec $0 < \theta < 2\pi$, donc $0 < u < 4\pi^2$).

\[V(u) = \frac{r^3 u}{12\pi}\sqrt{1 - \frac{u}{4\pi^2}} = \frac{r^3}{12\pi}\sqrt{u^2\left(1 - \frac{u}{4\pi^2}\right)} = \frac{r^3}{12\pi}\sqrt{u^2 - \frac{u^3}{4\pi^2}}\]

Maximiser $V$ équivaut à maximiser $f(u) = u^2 - \frac{u^3}{4\pi^2}$.

Dérivons :
\[f'(u) = 2u - \frac{3u^2}{4\pi^2} = u\left(2 - \frac{3u}{4\pi^2}\right)\]

Annulation : $2 - \frac{3u}{4\pi^2} = 0 \implies u = \frac{8\pi^2}{3}$

Donc $\theta = \sqrt{\frac{8\pi^2}{3}} = \frac{2\sqrt{2}\pi}{\sqrt{3}} = \frac{2\sqrt{6}\pi}{3}$.

Le volume maximal est :
\[V_{\max} = \frac{r^3}{12\pi} \cdot \frac{8\pi^2}{3}\sqrt{1 - \frac{2}{3}} = \frac{r^3}{12\pi} \cdot \frac{8\pi^2}{3} \cdot \frac{1}{\sqrt{3}} = \boxed{\frac{2\sqrt{3}\pi r^3}{27}}\]
}

