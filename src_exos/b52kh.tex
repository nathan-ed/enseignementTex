\titre{}
\theme{dérivées}
\auteur{Nathan Scheinmann}
\niveau{3M}
\source{fundamentum}
\type{serie}
\piments{2}
\pts{}
\annee{2425}

\contenu{
\tcblower
On fabrique un cornet de forme conique en rejoignant les bords rectilignes d'un secteur circulaire de rayon $r$. Quel est le volume du plus grand cornet possible ?
\begin{center}
\begin{tikzpicture}
    \tkzDefPoint(0,0){O}
    \tkzDefPoint(2,0){A}
    \tkzDefPoint(90:2){B}
    \tkzDefPoint(330:2){C}
    \tkzDrawCircle(O,A)
    \tkzDrawSegment[dashed](O,B)
    \tkzDrawSegment[dashed](O,C)
    \tkzFillSector[gray!20](O,B)(C)
\end{tikzpicture}
\end{center}
}
\correction{
\tcblower
}

