\titre{}
\theme{fonctions}
\auteur{Nathan Scheinmann}
\niveau{1M}
\source{}
\type{serie}
\piments{2}
\pts{}
\annee{2425}

\contenu{
	\tcblower
	Déterminer les intervalles suivants où $A=]-2;3]$, $B=[0;4[$ et $C=]-\infty;2]$:
\begin{tasks}(4)
	\task $A\cup B$
	\task $A\cap B$
	\task $A\setminus B$
	\task $B\setminus A$
	\task $A\cup C$
	\task $A\cap C$
	\task $A\setminus C$
	\task $C\setminus A$
	\task $B\cup C$
	\task $B\cap C$
	\task $B\setminus C$
	\task $C\setminus B$
\end{tasks}
}
\correction{
	\tcblower
\begin{tasks}(4)
\task $A\cup B=\ointerval{-2}{4}$
\task $A\cap B=\interval{0}{3}$
\task $A\setminus B=\ointerval{-2}{0}$
\task $B\setminus A=\ointerval{3}{4}$
\task $A\cup C=\linterval{-\infty}{3}$
\task $A\cap C=\linterval{-2}{2}$
\task $A\setminus C=\linterval{2}{3}$
\task $C\setminus A=\linterval{-\infty}{-2}$
\task $B\cup C=\ointerval{-\infty}{4}$
\task $B\cap C=\interval{0}{2}$
\task $B\setminus C=\ointerval{2}{4}$
\task $C\setminus B=\ointerval{-\infty}{0}$
\end{tasks}

}

