\titre{}
\theme{dérivées}
\auteur{Nathan Scheinmann}
\niveau{3M}
\source{fundamentum}
\type{serie}
\piments{2}
\pts{}
\annee{2425}

\contenu{
\tcblower
Parmi tous les triangles rectangles d'hypoténuse donnée $h$, quel est celui dont le périmètre est le plus grand ? Quel est ce périmètre maximal ?
}
\correction{
\tcblower
%% GENERATED BY AI %%
{\scriptsize \textit{Correction générée par IA}}

Soit $a$ et $b$ les cathètes du triangle rectangle. Par le théorème de Pythagore :
\[a^2 + b^2 = h^2\]

Le périmètre est :
\[P = a + b + h\]

De la contrainte, nous tirons $b = \sqrt{h^2 - a^2}$. Substituons :
\[P(a) = a + \sqrt{h^2 - a^2} + h\]

Pour maximiser le périmètre, dérivons :
\[P'(a) = 1 + \frac{-2a}{2\sqrt{h^2 - a^2}} = 1 - \frac{a}{\sqrt{h^2 - a^2}}\]

Annulation :
\[1 - \frac{a}{\sqrt{h^2 - a^2}} = 0 \implies \sqrt{h^2 - a^2} = a \implies h^2 - a^2 = a^2 \implies h^2 = 2a^2\]

Donc $a = \frac{h}{\sqrt{2}}$ et, par symétrie, $b = \frac{h}{\sqrt{2}}$.

Le triangle rectangle de périmètre maximal est donc isocèle avec $\boxed{a = b = \frac{h}{\sqrt{2}} = \frac{h\sqrt{2}}{2}}$.

Le périmètre maximal est :
\[\boxed{P_{\max} = \frac{h}{\sqrt{2}} + \frac{h}{\sqrt{2}} + h = \frac{2h}{\sqrt{2}} + h = h\sqrt{2} + h = h(1 + \sqrt{2})}\]
}

