\titre{106-musy17}
\theme{equations}
\auteur{Nathan Scheinmann}
\niveau{1M}
\source{muzy-2017}
\type{serie}
\piments{1}
\pts{}
\annee{2425}

\contenu{
	\tcblower
Un problème de Leonhard Euler (1707 - 1783).

Un père mourut en laissant quatre filles. Celles-ci se partagèrent ses biens de la manière suivante : la première prit la moitié de la fortune, moins 3000 livres; la deuxième en prit le tiers moins 1000 livres; la troisième prit exactement le quart des biens; la quatrième prit 600 livres plus le cinquième des biens. Quelle était la fortune totale, et quelle somme reçut chacun des enfants?
}
\correction{
	\tcblower
Fortune totale $12000$ livres et chaque fille reçoit $3000$ livres.
}

