\titre{}
\theme{derivation}
\auteur{Nathan Scheinmann}
\niveau{3M}
\source{sesamath3e}
\type{serie}
\piments{2}
\pts{}
\annee{2425}

\contenu{
\tcblower
Utiliser les formules de dérivation pour  déterminer (donner des réponses sans exposant négatif ou fractionnaire) :
\begin{tasks}(3)
\task $\displaystyle \left(\dfrac{4}{x}\right)'$
\task $\displaystyle \left(\dfrac{-18}{x}\right)'$
\task $\displaystyle \left(\dfrac{1}{x^{2}}\right)'$
\task $\displaystyle \left(\dfrac{1}{3x^{3}}\right)'$
\task $\displaystyle \left(\dfrac{24}{x^{2}}\right)'$
\task $\displaystyle \left(\dfrac{1}{\sqrt{x}}\right)'$
\task $\displaystyle \left(\dfrac{1}{x\sqrt{x}}\right)'$
\task $\displaystyle \left(x^{2}-\dfrac{1}{2x}\right)'$
\task $\displaystyle \left(\dfrac{x^{2}}{x^{3}+1}\right)'$
\task $\displaystyle \left(\dfrac{1+2u^{3}}{2u}\right)'$
\task $\displaystyle ((2x+3)(3x-7))'$
\task $\displaystyle \bigl(x^{2}(1+\sqrt{x})\bigr)'$
\task $\displaystyle ((x^{3}-x)(x^{2}-9))'$
\task $\displaystyle \left(\dfrac{x^{2}+1}{4x}\right)'$
\end{tasks}
}
\correction{

}

