\titre{}
\theme{dérivées}
\auteur{Nathan Scheinmann}
\niveau{3M}
\source{ldonna}
\type{serie}
\piments{2}
\pts{}
\annee{2425}

\contenu{
\tcblower
Montrer que si $f$ est dérivable en $x$, alors $\left(f(x)^2\right)'=2f(x)f'(x)$.
}
\correction{
\tcblower
Puisque $f$ est dérivable $f^2$ l'est également (car un produit de fonction dérivable l'est). On applique la formule de dérivation d'un produit.  

$
\begin{aligned}
	\left(f(x)^2\right)'&=(f(x)\cdot f(x))'\\
			    &=f'(x)f(x)+f(x)f'(x)\\
			    &=2f'(x)f(x).
\end{aligned}
$
}

