\titre{15}
\theme{trigo}
\auteur{Nathan Scheinmann}
\niveau{1M}
\source{sesamath-1M-trigo}
\type{serie}
\piments{2}
\pts{}
\annee{2425}

\contenu{
\tcblower
Soit $\triangle MOT$ un triangle rectangle en $M$. 
\begin{tasks}(1)
	\task Que peut-on dire des angles $\widehat{MTO}$ et $\widehat{TOM}$~?
	\task Exprimer le sinus, le cosinus et la tangent des angles $\widehat{MTO}$ et $\widehat{TOM}$ en fontion des côtés $\overline{MO}, \overline{OT}$ et $\overline{MT}$.
	\task Utiliser la question précédente pour écrire trois égalités. 
	\task Déduire de ces égalités deux propriétés sur les angles complémentaires d'un triangle rectangle.
\end{tasks}
}
\correction{
\tcblower
\textit{Generated by AI}

Soit $\triangle MOT$ un triangle rectangle en $M$.

\begin{tasks}(1)
\task \textbf{Que peut-on dire des angles $\widehat{MTO}$ et $\widehat{TOM}$ ?}

Dans un triangle, la somme des angles vaut $180°$.

Dans le triangle rectangle $\triangle MOT$, nous avons :
\[\widehat{TMO} + \widehat{MTO} + \widehat{TOM} = 180°\]

Puisque $\widehat{TMO} = 90°$ (triangle rectangle en $M$), nous obtenons :
\[90° + \widehat{MTO} + \widehat{TOM} = 180°\]

Donc :
\[\widehat{MTO} + \widehat{TOM} = 90°\]

\textbf{Conclusion :} Les angles $\widehat{MTO}$ et $\widehat{TOM}$ sont \textbf{complémentaires} (leur somme vaut $90°$).

\task \textbf{Rapports trigonométriques :}

Notons $\alpha = \widehat{MTO}$ et $\beta = \widehat{TOM}$.

Pour l'angle $\alpha = \widehat{MTO}$ :
\begin{itemize}
\item $\sin(\alpha) = \dfrac{\overline{MO}}{\overline{OT}}$ (opposé/hypoténuse)
\item $\cos(\alpha) = \dfrac{\overline{MT}}{\overline{OT}}$ (adjacent/hypoténuse)
\item $\tan(\alpha) = \dfrac{\overline{MO}}{\overline{MT}}$ (opposé/adjacent)
\end{itemize}

Pour l'angle $\beta = \widehat{TOM}$ :
\begin{itemize}
\item $\sin(\beta) = \dfrac{\overline{MT}}{\overline{OT}}$ (opposé/hypoténuse)
\item $\cos(\beta) = \dfrac{\overline{MO}}{\overline{OT}}$ (adjacent/hypoténuse)
\item $\tan(\beta) = \dfrac{\overline{MT}}{\overline{MO}}$ (opposé/adjacent)
\end{itemize}

\task \textbf{Trois égalités :}

En comparant les expressions ci-dessus, nous obtenons :
\begin{align*}
\sin(\alpha) &= \cos(\beta) \\
\cos(\alpha) &= \sin(\beta) \\
\tan(\alpha) &= \frac{1}{\tan(\beta)} \quad \text{ou} \quad \tan(\alpha) \times \tan(\beta) = 1
\end{align*}

\task \textbf{Propriétés sur les angles complémentaires :}

Puisque $\alpha + \beta = 90°$ (angles complémentaires), nous déduisons les propriétés suivantes :

\textbf{Propriété 1 :} $\sin(\alpha) = \cos(90° - \alpha)$

\textbf{Propriété 2 :} $\cos(\alpha) = \sin(90° - \alpha)$

Ces propriétés montrent que le sinus d'un angle est égal au cosinus de son complément, et réciproquement.

\textbf{Propriété 3 :} $\tan(\alpha) \times \tan(90° - \alpha) = 1$

Autrement dit, la tangente d'un angle est l'inverse de la tangente de son complément.
\end{tasks}
}

