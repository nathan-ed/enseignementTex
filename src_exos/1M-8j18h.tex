\titre{}
\theme{limites}
\auteur{Nathan Scheinmann}
\niveau{3M}
\source{sesamath3e}
\type{serie}
\piments{4}
\pts{}
\annee{2425}

\contenu{
\tcblower
\noindent Calculer les limites suivantes.  
\begin{tasks}(2)
\task \(\displaystyle\lim_{x\to1}\frac{x^3 - x^2 + x - 1}{x - 1}\)
\task \(\displaystyle\lim_{x\to1}\frac{x^4 + x^3 - x - 3}{\sqrt{x - 1}}\)
\end{tasks}
}
\correction{
\tcblower

\begin{tasks}(2)

\task Type $« \dfrac{0}{0} »$ :  

\(
\begin{aligned}
&\lim_{x \to 1} \dfrac{x^3 - x^2 + x - 1}{x - 1}\\
&= \lim_{x\to 1} \dfrac{(x - 1)(x^2 + 1)}{x - 1} \\
&\stackrel{x \ne 1}{=} \lim_{x\to 1} x^2 + 1 = 2
\end{aligned}
\)

\task Type $« \dfrac{0}{0} »$ :  

\(
\begin{aligned}
&\lim_{x \to 1^+} \dfrac{x^4 + x^3 + x - 3}{\sqrt{x - 1}}\\
&= \lim_{x\to 1^+} \dfrac{\sqrt{x - 1}(x^4 + x^3 +  x - 3)}{x - 1} \\
&= \lim_{x\to 1^+} \dfrac{\sqrt{x - 1}(x-1)(x^3 + 2x^2 + 2x + 3)}{x - 1} \\
&\stackrel{x\neq 1^+}{=} \lim_{x\to 1} \sqrt{x - 1}(x^3+2x^2+2x+3) =  0
\end{aligned}
\)

\end{tasks}
}

