\titre{6}
\theme{trigo}
\auteur{Nathan Scheinmann}
\niveau{1M}
\source{sesamath-1M-trigo}
\type{serie}
\piments{2}
\pts{}
\annee{2425}

\contenu{
	\tcblower
	Soit le $\triangle HIJ$ rectangle en $H$. On a $\overline{IJ}=4{,}75~\text{cm}$ et $\widehat{IJH}=65{,}8^\circ$. Calculer $\overline{IH}, \overline{JH}$ et $\widehat{JIH}$.
}
\correction{
\tcblower
\textit{Generated by AI}

Dans le triangle $HIJ$ rectangle en $H$, on connaît l'hypoténuse $\overline{IJ} = 4{,}75~\text{cm}$ et l'angle $\widehat{IJH} = 65{,}8°$.

\textbf{Calcul de $\overline{IH}$ :}

Le côté $IH$ est opposé à l'angle $\widehat{IJH}$, donc :
\[\sin(\widehat{IJH}) = \frac{\overline{IH}}{\overline{IJ}}\]

\[\overline{IH} = \overline{IJ} \times \sin(65{,}8°) = 4{,}75 \times \sin(65{,}8°) \approx 4{,}75 \times 0{,}912 \approx \boxed{4{,}33~\text{cm}}\]

\textbf{Calcul de $\overline{JH}$ :}

Le côté $JH$ est adjacent à l'angle $\widehat{IJH}$, donc :
\[\cos(\widehat{IJH}) = \frac{\overline{JH}}{\overline{IJ}}\]

\[\overline{JH} = \overline{IJ} \times \cos(65{,}8°) = 4{,}75 \times \cos(65{,}8°) \approx 4{,}75 \times 0{,}410 \approx \boxed{1{,}95~\text{cm}}\]

\textbf{Calcul de $\widehat{JIH}$ :}

La somme des angles dans un triangle vaut $180°$. Puisque le triangle est rectangle en $H$ :

\[\widehat{JIH} + \widehat{IJH} + 90° = 180°\]

\[\widehat{JIH} = 180° - 90° - 65{,}8° = \boxed{24{,}2°}\]
}

