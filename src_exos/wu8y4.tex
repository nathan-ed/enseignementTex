\titre{}
\theme{derivation}
\auteur{Nathan Scheinmann}
\niveau{3M}
\source{sesamath3e}
\type{serie}
\piments{2}
\pts{}
\annee{2425}

\contenu{
\tcblower
Quelles dimensions faut-il donner à une boîte cylindrique fermée en aluminium de 5dl pour utiliser le minimum d'aluminium ? Que vaut alors ce minimum ?
}
\correction{
\tcblower
\textit{Generated by AI}

Soit $r$ le rayon et $h$ la hauteur du cylindre. Le volume est fixé à 5 dl = 0,5 L = 500 cm³.

\textbf{Contrainte de volume :}
$$V = \pi r^2 h = 500$$
$$h = \dfrac{500}{\pi r^2}$$

\textbf{Surface d'aluminium :}
La surface totale comprend deux disques (haut et bas) et la surface latérale :
$$S(r) = 2\pi r^2 + 2\pi r h = 2\pi r^2 + 2\pi r \cdot \dfrac{500}{\pi r^2} = 2\pi r^2 + \dfrac{1000}{r}$$

\textbf{Minimisation :}
Calculons la dérivée :
$$S'(r) = 4\pi r - \dfrac{1000}{r^2}$$

Résolvons $S'(r) = 0$ :
$$4\pi r = \dfrac{1000}{r^2}$$
$$4\pi r^3 = 1000$$
$$r^3 = \dfrac{1000}{4\pi} = \dfrac{250}{\pi}$$
$$r = \sqrt[3]{\dfrac{250}{\pi}} \approx 4{,}30 \text{ cm}$$

La hauteur correspondante :
$$h = \dfrac{500}{\pi r^2} = \dfrac{500}{\pi \cdot \left(\dfrac{250}{\pi}\right)^{2/3}} = \dfrac{500}{\pi} \cdot \left(\dfrac{\pi}{250}\right)^{2/3} = \dfrac{500}{\pi} \cdot \dfrac{\pi^{2/3}}{250^{2/3}}$$
$$h = 2 \cdot \left(\dfrac{250}{\pi}\right)^{1/3} = 2r \approx 8{,}60 \text{ cm}$$

Vérifions : $S''(r) = 4\pi + \dfrac{2000}{r^3} > 0$, donc c'est bien un minimum.

La surface minimale est :
$$S_{\min} = 2\pi r^2 + \dfrac{1000}{r} = 3 \cdot 2\pi r^2 = 6\pi \left(\dfrac{250}{\pi}\right)^{2/3} \approx 348{,}7 \text{ cm}^2$$

\textbf{Dimensions optimales :} $\boxed{r \approx 4{,}30 \text{ cm}}$ et $\boxed{h \approx 8{,}60 \text{ cm}}$ (soit $h = 2r$).

\textbf{Surface minimale :} $\boxed{S_{\min} \approx 348{,}7 \text{ cm}^2}$
}

