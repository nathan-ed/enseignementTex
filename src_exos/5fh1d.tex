\titre{}
\theme{geometrie}
\auteur{Nathan Scheinmann}
\niveau{1M}
\source{}
\type{serie}
\piments{2}
\pts{}
\annee{2425}

\contenu{
	\tcblower
	\begin{minipage}[t]{0.45\textwidth}{
		\vspace{0pt}
		On a $BD=25 cm, ED= 15 cm$ et $EC=35 cm$.  
Calculer l'aire et le périmètre du quadrilatère $ADEC$. 
		}
		\end{minipage}
		\begin{minipage}[t]{0.4\textwidth}{
		\vspace{0pt}
\begin{tikzpicture}[scale=.75]
\tkzDefPoint(5,3.5){A}
\tkzDefPoint(0,0){B}
\tkzDefPoint(7,0){C}
\tkzDefPointBy[homothety=center B ratio 0.3](A) \tkzGetPoint{D}
\tkzDefPointBy[projection=onto B--C](D) \tkzGetPoint{E}
\tkzMarkRightAngle(C,A,B)
\tkzMarkRightAngle(D,E,C)
\tkzDrawPoints(A,B,C,D,E)
\tkzDrawSegment(D,E)
\tkzDrawPolygon(A,B,C)
\tkzLabelPoints[below](B,C,E)
\tkzLabelPoints[above](A,D)
\end{tikzpicture}
		}
		\end{minipage}

}
\correction{
	\tcblower
	\textit{Generated by AI}
	{\scriptsize \textit{Correction générée par IA}}

	Le triangle $ABC$ est rectangle en $A$. Le triangle $DEC$ est rectangle en $E$.

	Comme $D$ appartient au segment $[AB]$ et $E$ appartient au segment $[BC]$, et que $DE$ est parallèle à $AC$ (les deux angles droits le montrent), les triangles $BDE$ et $BAC$ sont semblables.

	Calculons d'abord les dimensions du triangle $ABC$. Comme $BD = 25$ cm et que $D$ est sur $[AB]$, nous savons que $B$, $D$ et $A$ sont alignés.

	De plus, $DE = 15$ cm et $EC = 35$ cm, donc $BC = BE + EC$.

	Par le théorème de Thalès, puisque les triangles sont semblables, nous avons :
	\[\frac{BD}{BA} = \frac{BE}{BC} = \frac{DE}{AC}\]

	Nous savons que $DE = 15$ cm. Pour trouver $BE$, nous utilisons le fait que le triangle $DEB$ est rectangle en $E$.

	Par Pythagore dans le triangle $BDE$ rectangle en $E$ :
	\[BD^2 = BE^2 + DE^2\]
	\[25^2 = BE^2 + 15^2\]
	\[625 = BE^2 + 225\]
	\[BE^2 = 400\]
	\[BE = 20 \text{ cm}\]

	Donc $BC = BE + EC = 20 + 35 = 55$ cm.

	Le rapport de similitude est $k = \frac{BD}{BA}$. Nous avons aussi $\frac{BE}{BC} = \frac{20}{55} = \frac{4}{11}$.

	Donc $BD = \frac{4}{11} BA$, ce qui donne $25 = \frac{4}{11} BA$, soit $BA = \frac{25 \times 11}{4} = \frac{275}{4} = 68{,}75$ cm.

	De même, $DE = \frac{4}{11} AC$, donc $15 = \frac{4}{11} AC$, soit $AC = \frac{15 \times 11}{4} = \frac{165}{4} = 41{,}25$ cm.

	Maintenant, calculons $DA = BA - BD = 68{,}75 - 25 = 43{,}75$ cm.

	\textbf{Périmètre du quadrilatère $ADEC$ :}
	\[\mathcal{P}_{ADEC} = AD + DE + EC + CA = 43{,}75 + 15 + 35 + 41{,}25 = 135 \text{ cm}\]

	\textbf{Aire du quadrilatère $ADEC$ :}

	L'aire de $ADEC$ est l'aire du triangle $ABC$ moins l'aire du triangle $BDE$.

	\[\mathcal{A}_{ABC} = \frac{1}{2} \times BA \times AC = \frac{1}{2} \times 68{,}75 \times 41{,}25 = 1417{,}97 \text{ cm}^2\]

	\[\mathcal{A}_{BDE} = \frac{1}{2} \times BE \times DE = \frac{1}{2} \times 20 \times 15 = 150 \text{ cm}^2\]

	\[\mathcal{A}_{ADEC} = 1417{,}97 - 150 = 1267{,}97 \text{ cm}^2\]

	Le périmètre du quadrilatère $ADEC$ est $\boxed{135 \text{ cm}}$ et son aire est $\boxed{1267{,}97 \text{ cm}^2}$.
}

