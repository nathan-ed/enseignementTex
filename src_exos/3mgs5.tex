\titre{}
\theme{dérivées}
\auteur{Nathan Scheinmann}
\niveau{3M}
\source{nathan}
\type{serie}
\piments{4}
\pts{}
\annee{2425}

\contenu{
\tcblower
Montrer que si $f$ est dérivable, alors 
$\left(f(x)^n\right)'=nf(x)^{n-1}f'(x)$.
}
\correction{
\tcblower
On procède par récurrence. Le base de l'induction pour $n=1$, on a $(f(x))'=f'(x)$, qui est évidemment vrai. On suppose que $f(x)^k$ est dérivable avec la formule $(f(x)^k)'=kf(x)^{k-1}f'(x)$ et on le montre pour $k+1$. Si $f(x)^k$ est dérivable, alors $f(x)^{k+1}$ l'est aussi, car $f(x)^{k+1}=f(x)^k\cdot f(x)$. On utilise la formule du produit pour démontrer la formule pour $(f(x)^{k+1})'$.
On a

$\begin{aligned}
	(f(x)^{k+1})'&=(f(x)\cdot f(x)^k)'\\
		     &=f(x)\cdot (f(x)^k)'+f'(x)f(x)^k\\
		     &=f(x)\cdot kf(x)^{k-1}f'(x)+f'(x)f(x)^k \quad \text{(en utilisant l'hypothèse de récurrence)}\\
		     &=kf(x)^{k}f'(x)+f'(x)f(x)^k\\
		     &=(k+1)f(x)^kf'(x)
\end{aligned}
$
}

