\titre{}
\theme{dérivées}
\auteur{Nathan Scheinmann}
\niveau{3M}
\source{analysis}
\type{serie}
\piments{2}
\pts{}
\annee{2425}

\contenu{
\tcblower
Soit \(P\) un polynôme non constant
\[
P(x) = a_n x^n + \cdots + a_1 x + a_0.
\]
Montrer que, entre deux racines consécutives de l’équation \(P(x) = 0\), il existe au moins une racine de l’équation \(P'(x) = 0\).
}
\correction{
\tcblower
\textit{Generated by AI}

Supposons par l'absurde qu'il existe deux racines distinctes $\alpha$ et $\beta$ de $P'(x) = 0$ avec $\alpha < \beta$ situées entre deux racines consécutives $a$ et $b$ de $P(x) = 0$ (avec $a < b$).

Puisque $a$ et $b$ sont des racines consécutives de $P$, on a :
\[P(a) = P(b) = 0\]

Le théorème de Rolle affirme que si $P$ est continue sur $[a, b]$, dérivable sur $]a, b[$, et si $P(a) = P(b)$, alors il existe au moins un point $c \in ]a, b[$ tel que $P'(c) = 0$.

Donc il existe au moins une racine de $P'$ entre $a$ et $b$.

Maintenant, si $\alpha$ et $\beta$ sont deux racines distinctes de $P'$ avec $a < \alpha < \beta < b$, alors :
\[P'(\alpha) = P'(\beta) = 0\]

En appliquant le théorème de Rolle à $P'$ sur $[\alpha, \beta]$, il existerait un point $d \in ]\alpha, \beta[$ tel que $P''(d) = 0$.

Mais alors, en appliquant à nouveau Rolle entre $a$ et $\alpha$ (où $P'$ change de signe car il existe une racine de $P'$ en $\alpha$), et entre $\beta$ et $b$, nous obtiendrions des contradictions sur la structure monotone des racines.

Plus rigoureusement : entre deux racines consécutives de $P$, la fonction $P$ est strictement monotone (croissante ou décroissante). Donc $P'$ garde un signe constant, et ne peut s'annuler qu'une seule fois (au point d'inflexion si elle change de signe aux bornes).

\textbf{Conclusion :} Entre deux racines consécutives de $P(x) = 0$, il existe \textbf{au plus une racine} de $P'(x) = 0$.
}

