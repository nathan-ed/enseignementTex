\titre{musy-172}
\theme{fonctions}
\auteur{Nathan Scheinmann}
\niveau{1M}
\source{}
\type{serie}
\piments{2}
\pts{}
\annee{2425}

\contenu{
	\tcblower
Questions diverses, à traiter sans l'aide d'une représentation graphique :
\begin{tasks}
\task Déterminer la fonction affine $f$ dont le graphe passe par les points $A=(-7 ; 2)$ et $B=(-1 ; 3)$.
Calculer l'abscisse du point $C$ du graphe de $f$ ayant -12 comme ordonnée.
\task Déterminer la fonction affine $g$ dont le graphe coupe l'axe $O x$ en $I=(-5 ; 0)$ et dont la pente vaut $-\dfrac{5}{8}$. En déduire l'ordonnée du point $J=(0 ; ?)$ appartenant au graphe de $g$.
\task Déterminer la fonction affine $h$ telle que $h(-6)=9$ et dont le graphe passe par l'origine. Calculer $x$ dans le cas où $h(x)=-7$.
\task Trouver l'abscisse du point $R=(? ; 12)$ et l'ordonnée du point $S=(-7 ; ?)$ sachant que les points $R$ et $S$ sont alignés avec les points $T=(2 ;-3)$ et $U=(5 ;-1)$.
\end{tasks}
}
\correction{
	\tcblower
\begin{tasks}(2)
\task $f(x)=\dfrac{1}{6} x+\dfrac{19}{6} ;-91$
\task $f(x)=-\dfrac{5}{8} x-\dfrac{25}{8} ; y=-\dfrac{25}{8}$
\task $h(x)=-\dfrac{3}{2} x ; x=\dfrac{14}{3}$
\task $x=\dfrac{49}{2} ; y=-9$
\end{tasks}
}

