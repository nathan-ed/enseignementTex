\titre{}
\theme{}
\auteur{Nathan Scheinmann}
\niveau{}
\source{}
\type{serie}
\piments{1}
\pts{}
\annee{2526}

\contenu{

  \qrcode{https://coopmaths.fr/alea/?uuid=e37e2&id=1mF2-13&n=2&d=10&s=3&s2=false&s3=1&cd=1&uuid=e37e2&id=1mF2-13&n=2&d=10&s=3&s2=false&s3=2&cd=1&uuid=e37e2&id=1mF2-13&n=3&d=10&s=3&s2=true&s3=4&cd=1&lang=fr-CH&v=eleve&es=1211001}
	\tcblower
\begin{tasks}
	\task Soit la droite $d_1$ d'équation $y=9 x-7 $ et la droite $d_2$ d'équation $y=-5 x-2 $. \\Déterminer l'ensemble des points d'intersection de $d_1$ et $d_2$.
	\task Soit la droite $d$ d'équation $y=6 x-9 $ et la parabole $\mathcal{C}$ d'équation $y=8 x^2-8 $.\\ Déterminer l'ensemble des points d'intersection de $d$ et $\mathcal{C}$.
	\task Soit la parabole $\mathcal{C_1}$ d'équation $y=8 x^2-5 x + 5 $ et la parabole $\mathcal{C_2}$ d'équation $y=10 x^2-4 x + 4 $. \\Déterminer l'ensemble des points d'intersection de $\mathcal{C_1}$ et $\mathcal{C_2}$.
\end{tasks}


}
\correction{
	\tcblower
\begin{tasks}
	\task Afin de déterminer les points d'intersection de $d_1$ et $d_2$, on cherche les solutions de l'équation \[9 x-7 =-5 x-2  \iff 14 x=5 \]L'équation admet une solution $x_1=\dfrac{5}{14}$.\\
        On détermine la deuxième coordonnée du point d'intersection en évaluant l'équation d'une des droites en $x=x_1$.\[y_1=9\cdot \dfrac{5}{14}-7 =-\dfrac{53}{14}\]Les points d'intersection de ${d_1\cap d_2=\left\{\left(\dfrac{5}{14}\,;\,-\dfrac{53}{14}\right)\right\}}$.\\
	\task Afin de déterminer les points d'intersection de $d$ et $\mathcal{C}$, on cherche les solutions de l'équation \[8 x^2-8 =6 x-9 \]
    c'est-à-dire
    \[8x^2-6x+1=0\]
    On résout cette équation en utilisant la méthode de résolution du deuxième degré. On calcule le discriminant $\Delta=4$.\\Le discriminant étant positif, l'équation admet deux solutions réelles. Elles valent $x_1=\dfrac{1}{4}$ et $x_2=\dfrac{1}{2}$.\\
        On détermine la deuxième coordonnée des points d'intersection en évaluant l'expression de la droite en $x=x_1$ et $x=x_2$. On obtient \[y_1=6\cdot \dfrac{1}{4}-9 =-\dfrac{15}{2}\quad\text{ et }\quad y_2=6\cdot \dfrac{1}{2}-9 =-6\]On a $d\cap \mathcal{C}=\left\{\left(\dfrac{1}{4}\,;\,-\dfrac{15}{2}\right)\,;\,\left(\dfrac{1}{2}\,;\,-6\right)\right\}$.\\
	\task Afin de déterminer les points d'intersection de $\mathcal{C_1}$ et $\mathcal{C_2}$, on cherche les solutions de l'équation \[8 x^2-5 x + 5 =10 x^2-4 x + 4 \]
    c'est-à-dire
    \[-2x^2-x+1=0\]
    On résout cette équation en utilisant la méthode de résolution du deuxième degré. On calcule le discriminant $\Delta=9$.\\Le discriminant étant positif, l'équation admet deux solutions réelles. Elles valent $x_1=\dfrac{1}{2}$ et $x_2=-1$.\\
        On détermine la deuxième coordonnée des points d'intersection en évaluant l'expression d'une des paraboles en $x=x_1$ et $x=x_2$. On obtient \[y_1=10\cdot \left(\dfrac{1}{2}\right)^{2}-4\cdot \dfrac{1}{2} + 4 =\dfrac{9}{2}\quad\text{ et }\quad y_2=10\cdot \left(-1\right)^{2}-4\cdot \left(-1\right) + 4 =18\]On a $\mathcal{C_1} \cap \mathcal{C_2}=\left\{\left(\dfrac{1}{2}\,;\,\dfrac{9}{2}\right)\,;\,\left(-1\,;\,18\right)\right\}$.\\
\end{tasks}

}
