\titre{}
\theme{derivation}
\auteur{Nathan Scheinmann}
\niveau{3M}
\source{sesamath3e}
\type{serie}
\piments{2}
\pts{}
\annee{2425}

\contenu{
\tcblower
Une ficelle de longueur $L$ est coupée en deux morceaux : avec l'un des deux on obtient un carré et avec l'autre un triangle équilatéral. À quel endroit doit-on couper la ficelle pour que la somme des aires des domaines obtenus soit maximale ?
}
\correction{
\tcblower
\textit{Generated by AI}

Notons $x$ la longueur de ficelle utilisée pour le carré, où $0 \leq x \leq L$. La longueur restante pour le triangle équilatéral est $L - x$.

\textbf{Aire du carré :} Le périmètre est $x$, donc le côté vaut $\dfrac{x}{4}$. L'aire est :
$$A_{\text{carré}} = \left(\dfrac{x}{4}\right)^2 = \dfrac{x^2}{16}$$

\textbf{Aire du triangle équilatéral :} Le périmètre est $L - x$, donc le côté vaut $\dfrac{L - x}{3}$. L'aire d'un triangle équilatéral de côté $c$ est $\dfrac{c^2\sqrt{3}}{4}$, donc :
$$A_{\text{triangle}} = \dfrac{\sqrt{3}}{4} \left(\dfrac{L - x}{3}\right)^2 = \dfrac{\sqrt{3}(L - x)^2}{36}$$

\textbf{Somme des aires :}
$$S(x) = \dfrac{x^2}{16} + \dfrac{\sqrt{3}(L - x)^2}{36}$$

Pour maximiser $S(x)$, nous calculons sa dérivée :
$$S'(x) = \dfrac{2x}{16} + \dfrac{\sqrt{3} \cdot 2(L - x) \cdot (-1)}{36} = \dfrac{x}{8} - \dfrac{\sqrt{3}(L - x)}{18}$$

Nous résolvons $S'(x) = 0$ :
$$\dfrac{x}{8} = \dfrac{\sqrt{3}(L - x)}{18}$$
$$18x = 8\sqrt{3}(L - x)$$
$$18x = 8\sqrt{3}L - 8\sqrt{3}x$$
$$x(18 + 8\sqrt{3}) = 8\sqrt{3}L$$
$$x = \dfrac{8\sqrt{3}L}{18 + 8\sqrt{3}} = \dfrac{8\sqrt{3}L}{18 + 8\sqrt{3}} \cdot \dfrac{18 - 8\sqrt{3}}{18 - 8\sqrt{3}} = \dfrac{8\sqrt{3}L(18 - 8\sqrt{3})}{324 - 192} = \dfrac{8\sqrt{3}L(18 - 8\sqrt{3})}{132}$$

Simplifions : $x = \dfrac{4\sqrt{3}L(9 - 4\sqrt{3})}{66} = \dfrac{2\sqrt{3}L(9 - 4\sqrt{3})}{33}$

Vérifions qu'il s'agit d'un maximum : $S''(x) = \dfrac{1}{8} + \dfrac{\sqrt{3}}{18} > 0$, donc $x$ correspond à un minimum.

La somme des aires est donc \textbf{maximale aux bornes} : soit $x = 0$ (toute la ficelle pour le triangle), soit $x = L$ (toute la ficelle pour le carré).

Comparons : $S(0) = \dfrac{\sqrt{3}L^2}{36}$ et $S(L) = \dfrac{L^2}{16}$.

Puisque $\dfrac{1}{16} \approx 0{,}0625 > \dfrac{\sqrt{3}}{36} \approx 0{,}048$, la somme des aires est maximale quand \boxed{\text{toute la ficelle est utilisée pour le carré}}.
}

