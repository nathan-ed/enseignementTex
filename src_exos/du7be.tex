\titre{}
\theme{limites}
\auteur{Nathan Scheinmann}
\niveau{3M}
\source{sesamath3e}
\type{serie}
\piments{4}
\pts{}
\annee{2425}

\contenu{
\tcblower
\noindent Pour chacune des propositions suivantes, dire si elle est vraie ou fausse. Justifier :  
\begin{tasks}
\task Si \(\displaystyle\lim_{x\to a}f(x)=-\infty\) et \(\displaystyle\lim_{x\to a}g(x)=+\infty\), alors \(\displaystyle\lim_{x\to a}\bigl(f(x)+g(x)\bigr)=0\).
\task \(\displaystyle\lim_{x\to2}f(x)=\lim_{h\to0}f(2+h)\).
\task Si \(\displaystyle\lim_{x\to a}f(x)=+\infty\) et \(\displaystyle\lim_{x\to a}g(x)=0\) avec \(g(x)>0\), alors \(\displaystyle\lim_{x\to a}\frac{f(x)}{g(x)}=~+\infty\).
\end{tasks}
}
\correction{
	\tcblower
\begin{tasks}
	\task Faux. Par exemple, si $f(x)=-\dfrac{1}{x^2}$ et $g(x)=\dfrac{1}{x^4}$ alors 

	$
	\begin{aligned}\lim_{x\to 0}-\dfrac{1}{x^2}+\dfrac{1}{x^4}&=\lim_{x\to 0} -\dfrac{x^2}{x^4}+\dfrac{1}{x^4}=\lim_{x\to 0}\dfrac{-x^2+1}{x^4}=\lim_{x\to 0} \dfrac{(1-x)(1+x)}{x^4}=\dfrac{1}{\lim_{x\to 0} x^4}=+\infty
	\end{aligned}$.
	\task Vrai. Par le changement de variable $x=h+2$, on a bien l'égalité désirée.
	\task Vrai. On a déjà vu que si une limite du type $«~\dfrac{1}{0}~»$ existe, alors elle vaut $+\infty$ ou $-\infty$. Dans notre cas, le fait que $g(x)>0$ garantit que la limite du quotient existe et donc que $\dfrac{f(x)}{g(x)}$ tend bien vers $+\infty$.  
\end{tasks}
}

