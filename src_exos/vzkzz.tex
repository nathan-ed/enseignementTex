\titre{19}
\theme{trigo}
\auteur{Nathan Scheinmann}
\niveau{1M}
\source{sesamath-1M-trigo}
\type{serie}
\piments{2}
\pts{}
\annee{2425}

\contenu{
\tcblower
Pour chacune des figures ci-dessous, déterminer $x$. Dans les deux cas, il s'agit d'un demi-cercle. 
\begin{tasks}(2)
	\task \includegraphics[scale=1]{../medias/1M/trigo/1M-exo-19-1}
	\task \includegraphics[scale=1]{../medias/1M/trigo/1M-exo-19-2}
\end{tasks}
}
\correction{
\tcblower
\textit{Generated by AI}

Dans un demi-cercle, tout triangle inscrit avec le diamètre comme base est rectangle (théorème de Thalès).

\begin{tasks}
\task Le segment vertical de longueur $x$ est perpendiculaire au diamètre $ab$. Si le rayon du demi-cercle est $r$, et que la distance du point $a$ au pied de la perpendiculaire est $d$, alors par le théorème de Pythagore dans le triangle rectangle :
$$x^2 + d^2 = (2r - d)^2$$
ou, en utilisant la propriété de la hauteur dans un triangle rectangle inscrit dans un demi-cercle, si le demi-cercle a un diamètre $ab$, alors $x$ est la hauteur issue de l'angle droit.

\task Le segment $x$ relie le point $a$ à un point sur le demi-cercle. Dans ce cas, $x$ est une corde du demi-cercle. Si l'angle au centre correspondant est connu ou si d'autres mesures sont données, on peut utiliser les propriétés trigonométriques ou le théorème de Pythagore pour déterminer $x$.

\textit{Note : Les valeurs numériques spécifiques dépendent des dimensions données dans les figures.}
\end{tasks}
}

