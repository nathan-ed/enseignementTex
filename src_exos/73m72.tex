\titre{28}
\theme{trigo}
\auteur{Nathan Scheinmann}
\niveau{1M}
\source{sesamath-1M-trigo}
\type{serie}
\piments{2}
\pts{}
\annee{2425}

\contenu{
\tcblower 
On doit percer un tunner pour une nouvelle autoroute à travers une montagne de $3225~\text{m}$ de haut. À une distande de $2000~\text{m}$ de la base de la montagne l'angle d'élévation est de $36^\circ$. Sur l'autre face, l'angle d'élévation à une distance de $1500~\text{m}$ est de $60^\circ$. 
\begin{center}
	\includegraphics[scale=1]{../medias/1M/trigo/1M-exo-28}
\end{center}
Calculer la longueur du tunnel.

}
\correction{
\tcblower

\textit{Generated by AI}

Pour calculer la longueur du tunnel, nous devons déterminer les positions horizontales des points d'entrée du tunnel de chaque côté de la montagne.

\textbf{Côté 1 : Distance de 2000 m, angle d'élévation de $36^\circ$}

Soit $h_1$ la hauteur atteinte depuis ce côté. Dans le triangle rectangle formé :
\[\tan(36^\circ) = \dfrac{h_1}{2000}\]
\[h_1 = 2000 \times \tan(36^\circ) \approx 2000 \times 0{,}727 \approx 1454~\text{m}\]

Cette hauteur correspond à la hauteur du point d'entrée du tunnel de ce côté.

\textbf{Côté 2 : Distance de 1500 m, angle d'élévation de $60^\circ$}

Soit $h_2$ la hauteur atteinte depuis ce côté :
\[\tan(60^\circ) = \dfrac{h_2}{1500}\]
\[h_2 = 1500 \times \tan(60^\circ) = 1500 \times \sqrt{3} \approx 1500 \times 1{,}732 \approx 2598~\text{m}\]

\textbf{Calcul de la longueur du tunnel :}

Si le tunnel est horizontal et passe à une certaine hauteur, nous devons considérer la géométrie du problème. En supposant que le tunnel traverse la montagne horizontalement à une hauteur donnée, la longueur du tunnel est approximativement la somme des distances horizontales où le tunnel perce la montagne.

Si nous supposons que le tunnel est au niveau le plus bas possible, la longueur approximative du tunnel est :
\[\text{Longueur} \approx 2000 + 1500 = 3500~\text{m}\]

Cependant, selon le schéma et la géométrie exacte, la longueur du tunnel peut être calculée en utilisant les coordonnées des points d'entrée et de sortie.

\textbf{Réponse approximative :} La longueur du tunnel est d'environ $\mathbf{3500~\text{m}}$ ou $\mathbf{3{,}5~\text{km}}$.

(Note : Une solution précise nécessite l'analyse détaillée du schéma fourni.)

}

