\titre{}
\theme{derivation}
\auteur{Nathan Scheinmann}
\niveau{3M}
\source{sesamath3e}
\type{serie}
\piments{2}
\pts{}
\annee{2425}

\contenu{
\tcblower
Les courbes \(f(x)=x^2\) et \(g(x)=\dfrac{1}{x}\) ont-elles une tangente commune ? Si oui, déterminer son équation. Si non, le prouver.
}
\correction{
\tcblower
\textit{Generated by AI}

Pour que les courbes $f(x) = x^2$ et $g(x) = \dfrac{1}{x}$ aient une tangente commune, il faut qu'il existe deux points $A(a; a^2)$ sur $f$ et $B(b; \frac{1}{b})$ sur $g$ tels que les tangentes en ces points soient identiques.

\textbf{Conditions pour une tangente commune :}

La tangente à $f$ en $A(a; a^2)$ a pour équation :
\[
y = f'(a)(x - a) + a^2 = 2a(x - a) + a^2 = 2ax - a^2
\]

La tangente à $g$ en $B(b; \frac{1}{b})$ a pour équation :
\[
y = g'(b)(x - b) + \frac{1}{b} = -\frac{1}{b^2}(x - b) + \frac{1}{b} = -\frac{1}{b^2}x + \frac{1}{b} + \frac{1}{b} = -\frac{1}{b^2}x + \frac{2}{b}
\]

Pour que ces deux tangentes soient identiques, il faut :
\begin{align*}
2a &= -\frac{1}{b^2} \quad \text{(égalité des pentes)} \\
-a^2 &= \frac{2}{b} \quad \text{(égalité des ordonnées à l'origine)}
\end{align*}

De la première équation :
\[
b^2 = -\frac{1}{2a}
\]

Pour que $b^2$ soit positif, il faut $a < 0$.

De la seconde équation :
\[
b = -\frac{2}{a^2}
\]

En substituant dans $b^2 = -\frac{1}{2a}$ :
\[
\left(-\frac{2}{a^2}\right)^2 = -\frac{1}{2a}
\]
\[
\frac{4}{a^4} = -\frac{1}{2a}
\]
\[
8a = -a^4
\]
\[
a^4 + 8a = 0
\]
\[
a(a^3 + 8) = 0
\]

Donc $a = 0$ (impossible car on aurait $b$ indéfini) ou $a^3 = -8$, soit $a = -2$.

Pour $a = -2$ :
\[
b = -\frac{2}{(-2)^2} = -\frac{2}{4} = -\frac{1}{2}
\]

Vérifions :
\begin{itemize}
\item Pente : $2a = 2(-2) = -4$ et $-\frac{1}{b^2} = -\frac{1}{1/4} = -4$ ✓
\item Ordonnée à l'origine : $-a^2 = -4$ et $\frac{2}{b} = \frac{2}{-1/2} = -4$ ✓
\end{itemize}

\textbf{Réponse :} Oui, les courbes ont une tangente commune d'équation $y = -4x - 4$.
}

