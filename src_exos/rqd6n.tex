\titre{}
\theme{derivation}
\auteur{Nathan Scheinmann}
\niveau{3M}
\source{sesamath3e}
\type{serie}
\piments{2}
\pts{}
\annee{2425}

\contenu{
\tcblower
À partir de la définition de la dérivée de \(f\) en \(a\), calculer les dérivées \(f'(a)\) et interpréter graphiquement :
\begin{tasks}(2)
  \task \(f(x)=x^2\) avec \(a=1\) puis \(a=~3\).
  \task \(f(x)=x^3\) avec \(a=2\).
  \task \(f(x)=x\) avec \(a=2\) puis \(a=5\).
  \task \(f(x)=3\) avec \(a=2\) puis \(a=7\).
\end{tasks}
}
\correction{
\tcblower
{\scriptsize \textit{Correction générée par IA}}

\begin{tasks}
\task Pour $f(x)=x^2$~:

$a=1$~: $f'(1) = \lim_{h \to 0} \dfrac{(1+h)^2 - 1}{h} = \lim_{h \to 0} (2+h) = 2$. Pente de $2$ en $(1;1)$.

$a=3$~: $f'(3) = \lim_{h \to 0} \dfrac{(3+h)^2 - 9}{h} = \lim_{h \to 0} (6+h) = 6$. Pente de $6$ en $(3;9)$.

\task Pour $f(x)=x^3$ avec $a=2$~:
\[
f'(2) = \lim_{h \to 0} \dfrac{(2+h)^3 - 8}{h} = \lim_{h \to 0} (12+6h+h^2) = 12
\]
Pente de $12$ en $(2;8)$.

\task Pour $f(x)=x$~: $f'(a) = 1$ pour tout $a$. Pente constante de $1$.

\task Pour $f(x)=3$~: $f'(a) = 0$ pour tout $a$. Tangente horizontale.
\end{tasks}
}

