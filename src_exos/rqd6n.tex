\titre{}
\theme{derivation}
\auteur{Nathan Scheinmann}
\niveau{3M}
\source{sesamath3e}
\type{serie}
\piments{2}
\pts{}
\annee{2425}

\contenu{
\tcblower
À partir de la définition de la dérivée de \(f\) en \(a\), calculer les dérivées \(f'(a)\) et interpréter graphiquement :
\begin{tasks}(2)
  \task \(f(x)=x^2\) avec \(a=1\) puis \(a=~3\).
  \task \(f(x)=x^3\) avec \(a=2\).
  \task \(f(x)=x\) avec \(a=2\) puis \(a=5\).
  \task \(f(x)=3\) avec \(a=2\) puis \(a=7\).
\end{tasks}
}
\correction{
\tcblower
{\scriptsize \textit{Correction générée par IA}}

Rappelons que la dérivée de $f$ en $a$ est définie par :
\[
f'(a) = \lim_{h \to 0} \dfrac{f(a+h) - f(a)}{h}
\]

\begin{tasks}
\task Pour $f(x)=x^2$ :

\textbf{Pour $a=1$ :}
\( \begin{aligned}
f'(1) &= \lim_{h \to 0} \dfrac{(1+h)^2 - 1^2}{h} \\
&= \lim_{h \to 0} \dfrac{1+2h+h^2 - 1}{h} \\
&= \lim_{h \to 0} \dfrac{2h+h^2}{h} \\
&= \lim_{h \to 0} (2+h) \\
&= 2
\end{aligned} \)

\textbf{Interprétation :} La tangente au graphe de $f$ au point $(1;1)$ a une pente de $2$.

\textbf{Pour $a=3$ :}
\( \begin{aligned}
f'(3) &= \lim_{h \to 0} \dfrac{(3+h)^2 - 3^2}{h} \\
&= \lim_{h \to 0} \dfrac{9+6h+h^2 - 9}{h} \\
&= \lim_{h \to 0} \dfrac{6h+h^2}{h} \\
&= \lim_{h \to 0} (6+h) \\
&= 6
\end{aligned} \)

\textbf{Interprétation :} La tangente au graphe de $f$ au point $(3;9)$ a une pente de $6$.

\task Pour $f(x)=x^3$ avec $a=2$ :

\( \begin{aligned}
f'(2) &= \lim_{h \to 0} \dfrac{(2+h)^3 - 2^3}{h} \\
&= \lim_{h \to 0} \dfrac{8+12h+6h^2+h^3 - 8}{h} \\
&= \lim_{h \to 0} \dfrac{12h+6h^2+h^3}{h} \\
&= \lim_{h \to 0} (12+6h+h^2) \\
&= 12
\end{aligned} \)

\textbf{Interprétation :} La tangente au graphe de $f$ au point $(2;8)$ a une pente de $12$.

\task Pour $f(x)=x$ :

\textbf{Pour $a=2$ :}
\( \begin{aligned}
f'(2) &= \lim_{h \to 0} \dfrac{(2+h) - 2}{h} \\
&= \lim_{h \to 0} \dfrac{h}{h} \\
&= 1
\end{aligned} \)

\textbf{Pour $a=5$ :}
\( \begin{aligned}
f'(5) &= \lim_{h \to 0} \dfrac{(5+h) - 5}{h} \\
&= \lim_{h \to 0} \dfrac{h}{h} \\
&= 1
\end{aligned} \)

\textbf{Interprétation :} La fonction $f(x)=x$ est une droite de pente $1$, donc la tangente en tout point a une pente de $1$.

\task Pour $f(x)=3$ (fonction constante) :

\textbf{Pour $a=2$ :}
\( \begin{aligned}
f'(2) &= \lim_{h \to 0} \dfrac{3 - 3}{h} \\
&= \lim_{h \to 0} \dfrac{0}{h} \\
&= 0
\end{aligned} \)

\textbf{Pour $a=7$ :}
\( \begin{aligned}
f'(7) &= \lim_{h \to 0} \dfrac{3 - 3}{h} \\
&= 0
\end{aligned} \)

\textbf{Interprétation :} La fonction constante a une dérivée nulle en tout point, ce qui signifie que sa tangente est horizontale partout.
\end{tasks}
}

