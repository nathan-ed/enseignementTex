\titre{}
\theme{derivation}
\auteur{Nathan Scheinmann}
\niveau{3M}
\source{sesamath3e}
\type{serie}
\piments{2}
\pts{}
\annee{2425}

\contenu{
\tcblower
À partir de la définition de la dérivée de \(f\) en \(a\), calculer les dérivées \(f'(a)\) et interpréter graphiquement :
\begin{tasks}(2)
  \task \(f(x)=x^2\) avec \(a=1\) puis \(a=~3\).
  \task \(f(x)=x^3\) avec \(a=2\).
  \task \(f(x)=x\) avec \(a=2\) puis \(a=5\).
  \task \(f(x)=3\) avec \(a=2\) puis \(a=7\).
\end{tasks}
}
\correction{

}

