\titre{}
\theme{equations}
\auteur{Nathan Scheinmann}
\niveau{1M}
\source{MER11e}
\type{serie}
\piments{1}
\pts{}
\annee{2425}

\contenu{
	\tcblower
	Traduire chacune des ces situations par un système de deux équations et déterminer les solutions.
\begin{enumerate}
	\item La somme de deux nombrres est $100$. La différence de ces deux nombres est $68$. Quels sont ces nombres?
	\item Entendu de bon matin à la terrasse d'un café:
		\begin{itemize}
			\item "Deux chocolats et trois croissants: Fr. $8,90$."
			\item "Trois chocolats et cinq croissants: Fr. $13,80$."
			\end{itemize}
			Quel est le prix d'un chocolat? Et celui d'un croissant?
		\item 350 spectateurs ont assisté à un spectacle. 
			Au parterre, la place revient à Fr. $20.-$; à la galerie, elle revient à Fr. $30.-$.

			Le montant de la recette des entrées est de Fr. $7850.-$.

			Combien y avait-il de spectateurs au parterre? Et à la galerie?
\end{enumerate}
}
\correction{
	\tcblower
	\begin{tasks}
	\task On pose les inconnues
\begin{align*}
	&x=\text{le premier nombre} &&y=\text{le deuxième nombre}
\end{align*}
On obtient le système
$\begin{cases}
x+y=100\\
x-y=68
\end{cases}$
On résout le système (par exemple par combinaison) et on obtient $x=84$ et $y=16$. Le premier nombre vaut $84$ et le deuxième $16$. 
\task On pose les inconnues
\begin{align*}
	&x=\text{Le prix d'un chocolat} &&y=\text{Le prix d'un croissant}
\end{align*}
On obtient le système
$\begin{cases}
2x+3y=8,90\\
3x+5y=13,80
\end{cases}$
On résout le système (par exemple par combinaison) et on obtient que $x=3,10$ et $y=0,90$. Donc un chocolat coûte CHF 3,10 et un croissant coûte CHF 0,90. 
\task On pose les inconnues
\begin{align*}
	&x=\text{Le nombre de spectateurs au parterre} &&y=\text{Le nombre de spectateurs à la gallerie}
\end{align*}
On obtient le système
$\begin{cases}
x+y=350\\
20x+30y=7850
\end{cases}$
On résout le système (par exemple par combinaison) et on obtient que $x=265$ et $y=85$. Il y avait donc 265 spectateurs au parterre et 85 dans la gallerie.
\end{tasks}

}

