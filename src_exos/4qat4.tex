\titre{}
\theme{dérivées}
\auteur{Nathan Scheinmann}
\niveau{3M}
\source{analysis}
\type{serie}
\piments{2}
\pts{}
\annee{2425}

\contenu{
\tcblower
Soit \(f(x) = x^{-1},\ a = -1,\ b = 1\). Vérifier qu’il n’existe aucun nombre \(c\) tel que
\[
f'(c) = \dfrac{f(b) - f(a)}{b - a}.
\]

Expliquer en quoi cela ne contredit pas le théorème des accroissements finis. 
}
\correction{
\tcblower
\textit{Generated by AI}

Soit $f(x) = x^{-1} = \dfrac{1}{x}$, avec $a = -1$ et $b = 1$.

\textbf{Calcul de la dérivée :}
\[f'(x) = (x^{-1})' = -x^{-2} = -\dfrac{1}{x^{2}}\]

\textbf{Calcul du taux d'accroissement moyen :}
\begin{align*}
\dfrac{f(b) - f(a)}{b - a} &= \dfrac{f(1) - f(-1)}{1 - (-1)}\\
&= \dfrac{\frac{1}{1} - \frac{1}{-1}}{2}\\
&= \dfrac{1 - (-1)}{2}\\
&= \dfrac{2}{2} = 1
\end{align*}

\textbf{Recherche d'un $c$ tel que $f'(c) = 1$ :}

Il faudrait trouver $c$ tel que :
\[-\dfrac{1}{c^{2}} = 1\]

Cela donnerait :
\[c^{2} = -1\]

Or, cette équation n'a pas de solution réelle.

\textbf{Conclusion :} Il n'existe aucun nombre réel $c$ tel que $f'(c) = \dfrac{f(b) - f(a)}{b - a}$.

\textbf{Explication - Pourquoi cela ne contredit pas le théorème des accroissements finis :}

Le théorème des accroissements finis stipule que si $f$ est :
\begin{itemize}
\item Continue sur $[a;b]$
\item Dérivable sur $]a;b[$
\end{itemize}

alors il existe $c \in ]a;b[$ tel que $f'(c) = \dfrac{f(b) - f(a)}{b - a}$.

\textbf{Dans notre cas :}

La fonction $f(x) = \dfrac{1}{x}$ n'est \textbf{PAS continue} sur $[-1;1]$ car elle n'est pas définie en $x = 0$ (qui appartient à l'intervalle $[-1;1]$).

Plus précisément :
\begin{itemize}
\item $f$ a une asymptote verticale en $x = 0$
\item $\lim_{x \to 0^-} f(x) = -\infty$ et $\lim_{x \to 0^+} f(x) = +\infty$
\item Il y a une discontinuité (non-définie) en $x = 0$
\end{itemize}

Donc l'hypothèse de continuité sur $[a;b]$ du théorème des accroissements finis n'est \textbf{pas satisfaite}.

\[\boxed{\text{Cela ne contredit pas le TAF car $f$ n'est pas continue sur } [-1;1].}\]
}

