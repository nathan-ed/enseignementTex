\titre{}
\theme{calcLit}
\auteur{Nathan Scheinmann}
\niveau{1M}
\source{musy}
\type{serie}
\piments{2}
\pts{}
\annee{2425}

\contenu{
\tcblower
Un élève a développé tous les produits de trois des binômes $(x+1),(x-1),(x+2)$ et $(x-2)$, de toutes les manières possibles, sans répétition d'un binôme. Il a noté les résultats suivants :
$$
x^3-x^2-4 x+4, x^3-2 x^2-x+2, x^3+2 x^2-x-2 \text { et } x^3+x^2-4 x-4 \text {. }
$$
Malheureusement, cet élève ne se souvient pas dans quel ordre il a effectué ses calculs.
Comment peut-on l'aider à s'y retrouver immédiatement, par une simple observation~?
}
\correction{
\tcblower
	On utilise le terme constant (de degré 0) qui est différent pour toutes les expressions. Ainsi, il suffit de multiplier les termes de degré 0 de chaque expression pour retrouver les trois polynômes. 
}

