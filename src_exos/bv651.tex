\titre{}
\theme{derivation}
\auteur{Nathan Scheinmann}
\niveau{3M}
\source{sesamath3e}
\type{serie}
\piments{2}
\pts{}
\annee{2425}

\contenu{
\tcblower
Déterminer les zéros de la dérivée de la fonction / définie par $f(x) = x^3 - 2x^2 + 1$

\begin{tasks}
\task Peux-t-on en déduire les extrema de $f$ ?
\task Comment déterminer ces extrema ?
\end{tasks}
}
\correction{
\tcblower
\textit{Generated by AI}

Soit $f(x) = x^3 - 2x^2 + 1$.

Calculons la dérivée :
\[f'(x) = 3x^2 - 4x = x(3x - 4)\]

Les zéros de la dérivée sont obtenus en résolvant $f'(x) = 0$ :
\[x(3x - 4) = 0\]

\[\boxed{x = 0 \quad \text{ou} \quad x = \frac{4}{3}}\]

\begin{tasks}(1)
\task Peut-on en déduire les extrema de $f$ ?

Pas directement. Les zéros de $f'$ nous indiquent les \textit{points critiques} où $f$ pourrait avoir des extrema, mais il faut vérifier leur nature (maximum, minimum, ou point d'inflexion).

\task Comment déterminer ces extrema ?

Il faut étudier le signe de $f'(x)$ autour de ces points critiques, ou utiliser le test de la dérivée seconde.

\textbf{Méthode 1 : Étude du signe de $f'(x) = x(3x-4)$}

\begin{itemize}
\item Pour $x < 0$ : $f'(x) > 0$ (fonction croissante)
\item Pour $0 < x < \frac{4}{3}$ : $f'(x) < 0$ (fonction décroissante)
\item Pour $x > \frac{4}{3}$ : $f'(x) > 0$ (fonction croissante)
\end{itemize}

Donc : $x = 0$ est un \textbf{maximum local} avec $f(0) = 1$

$x = \frac{4}{3}$ est un \textbf{minimum local} avec $f\left(\frac{4}{3}\right) = \left(\frac{4}{3}\right)^3 - 2\left(\frac{4}{3}\right)^2 + 1 = \frac{64}{27} - \frac{32}{9} + 1 = \frac{1}{27}$

\textbf{Méthode 2 : Test de la dérivée seconde}

$f''(x) = 6x - 4$

$f''(0) = -4 < 0$ donc maximum en $x=0$

$f''\left(\frac{4}{3}\right) = 8 - 4 = 4 > 0$ donc minimum en $x = \frac{4}{3}$
\end{tasks}
}

