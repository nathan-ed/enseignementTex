\titre{}
\theme{derivation}
\auteur{Nathan Scheinmann}
\niveau{3M}
\source{musy}
\type{serie}
\piments{1}
\pts{}
\annee{2526}

\contenu{
\tcblower
Dériver la fonction $h$, donnée par $h(x)=\left(x^2+1\right)^3$, de deux façons différentes :
\begin{tasks}
\task après avoir d'abord distribué et réduit;
\task directement à l'aide de la formule pour la dérivée d'une fonction composée.
\end{tasks}
}
\correction{
\tcblower
}

