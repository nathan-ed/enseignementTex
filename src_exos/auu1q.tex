\titre{}
\theme{derivation}
\auteur{Nathan Scheinmann}
\niveau{3M}
\source{musy}
\type{serie}
\piments{1}
\pts{}
\annee{2526}

\contenu{
\tcblower
Dériver la fonction $h$, donnée par $h(x)=\left(x^2+1\right)^3$, de deux façons différentes :
\begin{tasks}
\task après avoir d'abord distribué et réduit;
\task directement à l'aide de la formule pour la dérivée d'une fonction composée.
\end{tasks}
}
\correction{
\tcblower
\textbf{Première méthode~: après distribution et réduction}

Développons d'abord $(x^2+1)^3$ :

\( \begin{aligned}
(x^2+1)^3 &= (x^2+1)(x^2+1)^2 \\
&= (x^2+1)(x^4+2x^2+1) \\
&= x^6+2x^4+x^2+x^4+2x^2+1 \\
&= x^6+3x^4+3x^2+1
\end{aligned} \)

Donc $h(x) = x^6+3x^4+3x^2+1$.

En dérivant terme à terme :
\[
h'(x) = 6x^5+12x^3+6x
\]

\textbf{Deuxième méthode~: avec la formule de la fonction composée}

Posons $u(x)=x^2+1$ et $v(u)=u^3$, de sorte que $h(x)=v(u(x))$.

On a :
\begin{itemize}
\item $u'(x)=2x$
\item $v'(u)=3u^2$
\end{itemize}

D'après la formule de dérivation d'une fonction composée :
\[
h'(x) = v'(u(x)) \cdot u'(x) = 3(x^2+1)^2 \cdot 2x = 6x(x^2+1)^2
\]

On peut vérifier que les deux expressions sont équivalentes en développant la deuxième forme :
\[
6x(x^2+1)^2 = 6x(x^4+2x^2+1) = 6x^5+12x^3+6x
\]
}

