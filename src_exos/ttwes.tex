\titre{}
\theme{derivation}
\auteur{Nathan Scheinmann}
\niveau{3M}
\source{sesamath3e}
\type{serie}
\piments{2}
\pts{}
\annee{2425}

\contenu{
\tcblower
Vrai ou faux ? Justifier.

\begin{tasks}
\task Il n'existe pas de fonction à la fois croissante et décroissante sur un intervalle $I$.
\task Si $f$ est nulle sur un intervalle ouvert $I$, alors $f'(x) > 0$ sur $I$.
\task Si $f$ est strictement croissante sur un intervalle ouvert $I$, alors $f'(x) > 0$ sur $I$.
\task Si $f$ est strictement décroissante et dérivable sur un intervalle ouvert $I$, alors $f'(x) < 0$ sur $I$.
\end{tasks}
}
\correction{
\tcblower
\textit{Generated by AI}

\begin{tasks}
\task \textbf{Vrai.} Une fonction ne peut pas être à la fois strictement croissante et strictement décroissante sur le même intervalle. Si elle était croissante, pour $x_1 < x_2$ on aurait $f(x_1) < f(x_2)$. Si elle était décroissante, on aurait $f(x_1) > f(x_2)$. Les deux conditions sont contradictoires.

\task \textbf{Faux.} Si $f$ est nulle sur un intervalle $I$, alors $f(x) = 0$ pour tout $x \in I$. Une fonction constante a une dérivée nulle, donc $f'(x) = 0$ sur $I$, et non $f'(x) > 0$.

\task \textbf{Faux.} Le contre-exemple classique est $f(x) = x^3$ sur $\mathbb{R}$. Cette fonction est strictement croissante sur $\mathbb{R}$, mais $f'(0) = 0$. Le théorème correct est : si $f'(x) > 0$ sur $I$, alors $f$ est strictement croissante. La réciproque n'est pas vraie.

\task \textbf{Vrai.} Si $f$ est strictement décroissante et dérivable sur un intervalle ouvert $I$, alors $f'(x) \leq 0$ sur $I$. De plus, $f'$ ne peut pas être nulle sur un sous-intervalle de $I$ car sinon $f$ serait constante sur ce sous-intervalle. Donc $f'(x) < 0$ sur $I$.
\end{tasks}
}

