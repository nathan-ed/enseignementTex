\titre{23}
\theme{trigo}
\auteur{Nathan Scheinmann}
\niveau{1M}
\source{sesamath-1M-trigo}
\type{serie}
\piments{2}
\pts{}
\annee{2425}

\contenu{
	\tcblower
	Elsa joue au cerf-volant sur la plage. La ficelle est déroulée au maximum et est tendue. L'angle de la ficelle avec l'horizontale est de $48^\circ$. Elle tient son dévidoir à $60~\text{cm}$ du sol. Le cerf-volant vole à $12~\text{m}$ du sol. 
	\begin{tasks}(1)
		\task Faire un schéma de la situation.
		\task Calculer la longueur de la ficelle déroulée. Donner la valeur arrondie au décimètre. 
	\end{tasks}
}
\correction{
\tcblower
\textit{Generated by AI}

\begin{tasks}(1)
\task \textbf{Schéma de la situation :}

On peut représenter la situation par un triangle rectangle où :
\begin{itemize}
\item Le côté horizontal représente la distance au sol
\item Le côté vertical représente la hauteur du cerf-volant par rapport au dévidoir
\item L'hypoténuse représente la ficelle tendue
\item L'angle avec l'horizontale est de $48°$
\end{itemize}

\task \textbf{Calcul de la longueur de la ficelle :}

Le cerf-volant vole à $12~\text{m}$ du sol et le dévidoir est à $0{,}6~\text{m}$ du sol.

La différence de hauteur est :
\[h = 12 - 0{,}6 = 11{,}4~\text{m}\]

Dans le triangle rectangle formé, cette hauteur correspond au côté opposé à l'angle de $48°$.

Si on note $L$ la longueur de la ficelle (hypoténuse) :
\[\sin(48°) = \frac{h}{L} = \frac{11{,}4}{L}\]

\[L = \frac{11{,}4}{\sin(48°)} \approx \frac{11{,}4}{0{,}743} \approx 15{,}34~\text{m}\]

\textbf{Réponse :} La longueur de la ficelle déroulée est d'environ \boxed{15{,}3~\text{m}} (arrondie au décimètre).
\end{tasks}
}

