\titre{}
\theme{numerique}
\auteur{Nathan Scheinmann}
\niveau{1M}
\source{musy}
\type{serie}
\piments{2}
\pts{}
\annee{2425}

\contenu{
\tcblower
Quel est le chiffre des unités du nombre $8^{2024}$~? 

(*) ...Et celui des dizaines~?!
}
\correction{
\tcblower
{\scriptsize \textit{Correction générée par IA}}

Pour trouver le chiffre des unités de $8^{2024}$, observons les puissances de $8$ :

\begin{itemize}
\item $8^1 = 8$ (chiffre des unités : 8)
\item $8^2 = 64$ (chiffre des unités : 4)
\item $8^3 = 512$ (chiffre des unités : 2)
\item $8^4 = 4096$ (chiffre des unités : 6)
\item $8^5 = 32768$ (chiffre des unités : 8)
\end{itemize}

Le motif des chiffres des unités se répète tous les 4 termes : $8, 4, 2, 6, 8, 4, 2, 6, \ldots$

Pour déterminer où se situe $8^{2024}$ dans ce cycle, calculons le reste de la division de $2024$ par $4$ :
\[
2024 = 4 \times 506 + 0
\]

Puisque le reste est $0$, cela correspond à la quatrième position du cycle (comme $8^4$).

Le chiffre des unités de $8^{2024}$ est donc $\textbf{6}$.

\medskip

\textbf{(*) Pour le chiffre des dizaines :}

Pour trouver le chiffre des dizaines, il faut étudier les deux derniers chiffres de $8^n$ (modulo 100).

\begin{itemize}
\item $8^1 = 08$
\item $8^2 = 64$
\item $8^3 = 512 \equiv 12 \pmod{100}$
\item $8^4 = 4096 \equiv 96 \pmod{100}$
\item $8^5 = 32768 \equiv 68 \pmod{100}$
\item $8^6 \equiv 44 \pmod{100}$
\item $8^7 \equiv 52 \pmod{100}$
\item $8^8 \equiv 16 \pmod{100}$
\item $8^9 \equiv 28 \pmod{100}$
\item $8^{10} \equiv 24 \pmod{100}$
\end{itemize}

Le motif se répète avec une période de $20$. En effet, $8^{20} \equiv 76 \pmod{100}$ et le cycle reprend.

Calculons : $2024 = 20 \times 101 + 4$

Donc $8^{2024} \equiv 8^4 \equiv 96 \pmod{100}$.

Le chiffre des dizaines est $\textbf{9}$.
}

