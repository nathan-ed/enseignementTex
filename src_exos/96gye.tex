\titre{}
\theme{fonctions}
\auteur{Nathan Scheinmann}
\niveau{1M}
\source{}
\type{serie}
\piments{2}
\pts{}
\annee{2425}

\contenu{
	\tcblower
Déterminer l'équation $\ldots$ 
    \begin{tasks}
        \task d'une droite parallèle à la droite d'équation $y=\dfrac{2}{3}x+1$.
        \task d'une droite perpendiculaire à la droite d'équation $y=\dfrac{-3}{4}x-2$.
        \task de la droite parallèle à la droite d'équation $y=2x+2$ et passant par le point $(0;4)$.
        \task de la droite parallèle à la droite d'équation $y=-\dfrac{1}{2}x-\dfrac{5}{4}$ et passant par le point $(0;0)$. 
    \end{tasks}

}
\correction{
	\tcblower
	\begin{tasks}(2)
\task  Par exemple $y=\dfrac{2}{3}x+2$
\task  Par exemple $y=\dfrac{4}{3}x+5$, car $\dfrac{-3}{4}\cdot \dfrac{4}{3}=-1$
\task  $y=2x+4$ 
\task $y=-\dfrac{4}{7}x$
\end{tasks}
}

