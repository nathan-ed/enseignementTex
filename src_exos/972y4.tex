\titre{}
\theme{dérivées}
\auteur{Nathan Scheinmann}
\niveau{3M}
\source{nathan}
\type{serie}
\piments{3}
\pts{}
\annee{2425}

\contenu{
\tcblower
Soient $f$ et $g$ des fonctions dérivables en $x$ et $g(x)\neq 0$, alors $\dfrac{f}{g}$ est dérivable en $x$ et $\left(\dfrac{f}{g}\right)'(x)=\dfrac{f'(x)g(x)-f(x)g'(x)}{g(x)^2}$. 

{\itshape Indication:  $f/g = f \cdot 1/g$.}
}
\correction{
\tcblower

Nous devons démontrer que $\left(\dfrac{f}{g}\right)'(x) = \dfrac{f'(x)g(x) - f(x)g'(x)}{g(x)^2}$.

L'indication nous suggère d'écrire $\dfrac{f}{g} = f \cdot \dfrac{1}{g}$.

Par la formule de dérivation de l'inverse, on a $\left(\dfrac{1}{g}\right)'(x) = -\dfrac{g'(x)}{g(x)^2}$.

Puisque $\dfrac{f}{g} = f \cdot \dfrac{1}{g}$, en appliquant la règle du produit :

\begin{align*}
\left(\dfrac{f}{g}\right)'(x) &= \left(f \cdot \dfrac{1}{g}\right)'(x) \\
&= f'(x) \cdot \dfrac{1}{g(x)} + f(x) \cdot \left(\dfrac{1}{g}\right)'(x) \\
&= \dfrac{f'(x)}{g(x)} + f(x) \cdot \left(-\dfrac{g'(x)}{g(x)^2}\right) \\
&= \dfrac{f'(x)}{g(x)} - \dfrac{f(x)g'(x)}{g(x)^2} \\
&= \dfrac{f'(x)g(x)}{g(x)^2} - \dfrac{f(x)g'(x)}{g(x)^2} \\
&= \boxed{\dfrac{f'(x)g(x) - f(x)g'(x)}{g(x)^2}}
\end{align*}
}

