\titre{}
\theme{dérivées}
\auteur{Nathan Scheinmann}
\niveau{3M}
\source{analysis}
\type{serie}
\piments{2}
\pts{}
\annee{2425}

\contenu{
\tcblower
Trouver les intervalles sur lesquels \(f\) est croissante et ceux sur lesquels elle est décroissante, si \(f(x)\) est donnée comme suit.

\begin{tasks}(2)
\task \(x^3 - 3x + 2\)
\task \(x^3 - 3x^2 + 6\)
\task \(3x + \dfrac{1}{x}\)
\task \(x^3(1 + x)\)
\task \(\dfrac{x}{x^2 + 1}\)
\task \((x + 1)^4\)
\task \(\sqrt{x - 2}\)
\task \(\dfrac{1}{x - 5}\)
\end{tasks}
}
\correction{
\tcblower
\textit{Generated by AI}

Pour chaque fonction, nous calculons la dérivée $f'(x)$, étudions son signe, puis déterminons les intervalles de croissance et décroissance.

\begin{tasks}(2)
\task $f(x) = x^3 - 3x + 2$

$f'(x) = 3x^2 - 3 = 3(x^2 - 1) = 3(x-1)(x+1)$

$f'(x) \geq 0$ pour $x \leq -1$ ou $x \geq 1$

Croissante sur $]-\infty ; -1] \cup [1 ; +\infty[$, décroissante sur $[-1 ; 1]$

\task $f(x) = x^3 - 3x^2 + 6$

$f'(x) = 3x^2 - 6x = 3x(x - 2)$

$f'(x) \geq 0$ pour $x \leq 0$ ou $x \geq 2$

Croissante sur $]-\infty ; 0] \cup [2 ; +\infty[$, décroissante sur $[0 ; 2]$

\task $f(x) = 3x + \dfrac{1}{x}$, $x \neq 0$

$f'(x) = 3 - \dfrac{1}{x^2} = \dfrac{3x^2 - 1}{x^2}$

$f'(x) \geq 0$ pour $|x| \geq \dfrac{1}{\sqrt{3}}$

Croissante sur $\left]-\infty ; -\dfrac{1}{\sqrt{3}}\right] \cup \left[\dfrac{1}{\sqrt{3}} ; +\infty\right[$, décroissante sur $\left[-\dfrac{1}{\sqrt{3}} ; 0\right[ \cup \left]0 ; \dfrac{1}{\sqrt{3}}\right]$

\task $f(x) = x^3(1 + x) = x^4 + x^3$

$f'(x) = 4x^3 + 3x^2 = x^2(4x + 3)$

$f'(x) \geq 0$ pour $x \geq -\dfrac{3}{4}$

Croissante sur $\left[-\dfrac{3}{4} ; +\infty\right[$, décroissante sur $\left]-\infty ; -\dfrac{3}{4}\right]$

\task $f(x) = \dfrac{x}{x^2 + 1}$

$f'(x) = \dfrac{x^2 + 1 - 2x^2}{(x^2 + 1)^2} = \dfrac{1 - x^2}{(x^2 + 1)^2}$

$f'(x) \geq 0$ pour $-1 \leq x \leq 1$

Croissante sur $[-1 ; 1]$, décroissante sur $]-\infty ; -1] \cup [1 ; +\infty[$

\task $f(x) = (x + 1)^4$

$f'(x) = 4(x + 1)^3$

$f'(x) \geq 0$ pour $x \geq -1$

Croissante sur $[-1 ; +\infty[$, décroissante sur $]-\infty ; -1]$

\task $f(x) = \sqrt{x - 2}$, $x \geq 2$

$f'(x) = \dfrac{1}{2\sqrt{x - 2}} > 0$ pour $x > 2$

Croissante sur $[2 ; +\infty[$

\task $f(x) = \dfrac{1}{x - 5}$, $x \neq 5$

$f'(x) = -\dfrac{1}{(x - 5)^2} < 0$ pour tout $x \neq 5$

Décroissante sur $]-\infty ; 5[$ et sur $]5 ; +\infty[$

\task $f(x) = \dfrac{x^2}{x^2 - 1}$, $x \neq \pm 1$

$f'(x) = \dfrac{2x(x^2 - 1) - x^2 \cdot 2x}{(x^2 - 1)^2} = \dfrac{-2x}{(x^2 - 1)^2}$

$f'(x) \geq 0$ pour $x \leq 0$ (hors $x = -1$)

Croissante sur $]-\infty ; -1[$ et sur $]-1 ; 0]$, décroissante sur $[0 ; 1[$ et sur $]1 ; +\infty[$

\task (Identique à la tâche 5)

\task $f(x) = \dfrac{x^2}{x + 1}$, $x \neq -1$

$f'(x) = \dfrac{2x(x + 1) - x^2}{(x + 1)^2} = \dfrac{x^2 + 2x}{(x + 1)^2} = \dfrac{x(x + 2)}{(x + 1)^2}$

$f'(x) \geq 0$ pour $x \leq -2$ ou $x \geq 0$

Croissante sur $]-\infty ; -2] \cup [0 ; +\infty[$, décroissante sur $[-2 ; -1[$ et sur $]-1 ; 0]$

\task $f(x) = \dfrac{x^2(1 + x)^2}{x - 1}$, $x \neq 1$

Dérivée complexe, étude du signe nécessaire après calcul.

\task $f(x) = \dfrac{x}{x^2 - 1}$, $x \neq \pm 1$

$f'(x) = \dfrac{(x^2 - 1) - x \cdot 2x}{(x^2 - 1)^2} = \dfrac{-x^2 - 1}{(x^2 - 1)^2} < 0$ pour tout $x \neq \pm 1$

Décroissante sur $]-\infty ; -1[$, sur $]-1 ; 1[$ et sur $]1 ; +\infty[$

\task $f(x) = \dfrac{x^2}{x^2 + 1}$

$f'(x) = \dfrac{2x(x^2 + 1) - x^2 \cdot 2x}{(x^2 + 1)^2} = \dfrac{2x}{(x^2 + 1)^2}$

$f'(x) \geq 0$ pour $x \geq 0$

Croissante sur $[0 ; +\infty[$, décroissante sur $]-\infty ; 0]$

\task $f(x) = (1 - \sqrt{x})^7$, $x \geq 0$

$f'(x) = 7(1 - \sqrt{x})^6 \cdot \left(-\dfrac{1}{2\sqrt{x}}\right) < 0$ pour $x > 0$

Décroissante sur $]0 ; +\infty[$

\task $f(x) = \sqrt{\dfrac{2 + x}{1 + x^2}}$

Dérivée complexe, nécessite l'étude du quotient sous la racine.

\task $f(x) = \dfrac{1}{\sqrt{x^2 + 1}}$

$f'(x) = -\dfrac{x}{(x^2 + 1)^{3/2}}$

$f'(x) \geq 0$ pour $x \leq 0$

Croissante sur $]-\infty ; 0]$, décroissante sur $[0 ; +\infty[$

\task $f(x) = |x + 1||x - 2|$

Fonction non dérivable en $x = -1$ et $x = 2$. Étude par intervalles nécessaire.
\end{tasks}
}

