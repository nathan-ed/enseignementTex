\titre{}
\theme{limites}
\auteur{Nathan Scheinmann}
\niveau{3M}
\source{sesamath3e}
\type{serie}
\piments{4}
\pts{}
\annee{2425}

\contenu{
\tcblower
\noindent Déterminer si la fonction \(f\) définie par  
\[
	f(x)=\begin{cases}[c]
\dfrac{x^2-4}{x-2}, & x\neq 2,\\
4, & x=2
\end{cases}
\]
est continue en \(a=2\). Justifier. 
}
\correction{
	\tcblower
\begin{itemize}
  \item Pour \( x \ne 2 \), on a \( f(x) = \dfrac{x^2 - 4}{x - 2} = \dfrac{(x - 2)(x + 2)}{x - 2} = x + 2 \)
  \item Donc \( \lim_{x \to 2} f(x) = \lim_{x \to 2} (x + 2) = 4 \)
  \item Or \( f(2) = 4 \), donc \( \lim_{x \to 2} f(x) = f(2) \)
  \item Ainsi, \( f \) est continue en \( a = 2 \)
\end{itemize}

}

