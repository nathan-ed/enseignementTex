\titre{}
\theme{derivation}
\auteur{Nathan Scheinmann}
\niveau{3M}
\source{sesamath3e}
\type{serie}
\piments{2}
\pts{}
\annee{2425}

\contenu{
\tcblower
Déterminer, dans chacun des cas suivants, l'expression algébrique d'une fonction rationnelle avec une asymptote :

\begin{tasks}(1)
\task oblique d'équation $y = 3x - 5$
\task horizontale d'équation $y = -2$
\task verticale d'équation $x = 7$.
\task horizontale d'équation $y = 0$, deux verticales d'équations $x = 3$ et $x = -10$.
\task Une asymptote verticale d'équation $x = 5$ et une asymptote oblique d'équation $y = 2x + 5$
\end{tasks}
Justifier vos réponses en montrant que les conditions sont bien vérifiées. 
}
\correction{
	\tcblower
	Sans justification.
\begin{tasks}(3)
	\task $f(x)=\dfrac{3x^3-5x^2+1}{x^2}$ 
	\task $f(x)=\dfrac{-2x^2+1}{x^2}$
	\task $f(x)=\dfrac{x}{x-7}$
	\task $f(x)=\dfrac{x}{(x-3)(x+10)}$
	\task $f(x)=\dfrac{2x^2-5x+1}{x-5}$
\end{tasks}
}

