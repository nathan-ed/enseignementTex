\titre{}
\theme{derivation}
\auteur{Nathan Scheinmann}
\niveau{3M}
\source{sesamath3e}
\type{serie}
\piments{2}
\pts{}
\annee{2425}

\contenu{
\tcblower
On s'intéresse à des fonctions telles que $f(3)$ n'existe pas et $\displaystyle\lim_{x \to 3} f(x) = 5$.

\begin{tasks}(1)
\task Représenter graphiquement trois fonctions distinctes qui vérifient ces deux conditions.
\task Donner l'expression $f(x)$ d'une fonction $f$ qui vérifie ces deux conditions.
\end{tasks}
}
\correction{
	\tcblower
\begin{tasks}(2)
	\task À vérifier avec l'enseignant.
	\task par exemple $x\mapsto \dfrac{5(x-2)(x-3)}{(x-3)}$. 
\end{tasks}
}

