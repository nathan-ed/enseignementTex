\titre{}
\theme{derivation2}
\auteur{Nathan Scheinmann}
\niveau{3M}
\source{sesamath3e}
\type{serie}
\piments{2}
\pts{}
\annee{2425}

\contenu{
\tcblower
Étudier les fonctions suivantes :
\begin{tasks}(2)
\task $f(x) = 2\cos(x) - \cos(2x)$
\task $f(x) = \sin^2(x) - \sin(x)$
\task $f(x) = 2\cos^3(x) - 3$
\task $f(x) = \dfrac{\sin(x)}{\cos(2x)}$
\task $f(x) = \dfrac{\tan(2x)}{\tan^2(x)}$
\task $f(x) = \dfrac{2}{\sqrt{1-2\sin^2(x)}}$
\end{tasks}
}
\correction{
\tcblower
{\scriptsize \textit{Correction générée par IA}}

Pour étudier ces fonctions, il faut déterminer leur domaine de définition, puis calculer leur dérivée pour analyser les variations.

\begin{tasks}
\task $f(x) = 2\cos(x) - \cos(2x)$

\textbf{Domaine :} $\mathbb{R}$

\textbf{Dérivée :}
\[
f'(x) = -2\sin(x) - (-\sin(2x)) \cdot 2 = -2\sin(x) + 2\sin(2x)
\]

donc $2x \neq \dfrac{\pi}{2} + k\pi$ avec $k \in \mathbb{Z}$, soit $x \neq \dfrac{\pi}{4} + \dfrac{k\pi}{2}$

\textbf{Dérivée :} donc $x \neq \dfrac{\pi}{4} + \dfrac{k\pi}{2}$, $x \neq \dfrac{\pi}{2} + k\pi$ et $x \neq k\pi$ avec $k \in \mathbb{Z}$

\textbf{Dérivée :} Cette dérivée nécessite la formule du quotient et est assez complexe à développer.

\task $f(x) = \dfrac{2}{\sqrt{1-2\sin^2(x)}}$

\textbf{Domaine :} Il faut $1-2\sin^2(x) > 0$, donc $\sin^2(x) < \dfrac{1}{2}$, soit $|\sin(x)| < \dfrac{1}{\sqrt{2}}$

\textbf{Dérivée :} En posant $u(x)=1-2\sin^2(x)$, on a $u'(x)=-4\sin(x)\cos(x)=-2\sin(2x)$.

Avec $f(x)=2u(x)^{-\dfrac{1}{2}}$ :
\[
f'(x) = 2 \cdot \left(-\dfrac{1}{2}\right)u(x)^{-\dfrac{3}{2}} \cdot u'(x) = -u(x)^{-\dfrac{3}{2}} \cdot (-2\sin(2x)) = \dfrac{2\sin(2x)}{(1-2\sin^2(x))^{\dfrac{3}{2}}}
\]
\end{tasks}
}

