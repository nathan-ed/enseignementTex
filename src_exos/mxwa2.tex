\titre{}
\theme{derivation}
\auteur{Nathan Scheinmann}
\niveau{3M}
\source{crm}
\type{serie}
\piments{1}
\pts{}
\annee{2526}

\contenu{
\tcblower
Déterminer l'équation de la droite $d$ qui passe par l'origine et qui coupe orthogonalement la courbe $\gamma$ d'équation $y=\dfrac{x^2+1}{x}$. Montrer que cette droite est une bissectrice des asymptotes de la courbe $\gamma$.
}
\correction{
\tcblower

On a 
\[
y = \frac{x^{2}+1}{x} = x+\frac{1}{x}, \qquad y'(x)=1-\frac{1}{x^{2}}.
\]
La droite $d:y=mx$ coupe la courbe en $M(a,\;a+\tfrac{1}{a})$ et passe par l'origine $(0\,;\,0)$, donc 
\[
	m=\frac{a+\tfrac{1}{a}-0}{a-0}=1+\frac{1}{a^{2}} \quad \text{(formule de la pente)}.
\]
Condition d’orthogonalité : 
\[
\Bigl(1+\tfrac{1}{a^{2}}\Bigr)\Bigl(1-\tfrac{1}{a^{2}}\Bigr)=-1 
\;\;\Longrightarrow\;\; a^{4}=\tfrac{1}{2}.
\]
Ainsi $m=1+\sqrt{2}$ et
\[
d:\; y=(1+\sqrt{2})x.
\]

Les asymptotes sont $y=x$ et $x=0$, d’angles $45^\circ$ et $90^\circ$ avec l'axe des ordonnées à l'origine.  
Or l'angle de $d$ avec l'axe des ordonnées à l'origine vaut $\arctan(1+\sqrt{2})=67{,}5^\circ=\tfrac{45^\circ+90^\circ}{2}$, donc $d$ est leur bissectrice.
}

