\titre{}
\theme{fonctions}
\auteur{Nathan Scheinmann}
\niveau{1M}
\source{}
\type{serie}
\piments{2}
\pts{}
\annee{2425}

\contenu{
	\tcblower
Une agence de voyage organise une excursion. Le prix du billet a été fixé à 60 CHF, mais la compagnie a consenti, dans le cas où plus de 100 personnes feraient le voyage, à baisser le prix de chaque billet de 25 cts par personne additionnelle. Sachant qu'il en coûte 1000 CHF à l'agence pour transporter les 100 premiers passagers et 15 CHF par passager additionnel, trouver le nombre de passagers pour lequel le bénéfice net de la compagnie est maximal. Interpréter graphiquement.
}
\correction{
	\tcblower
\[
\text{Pour } N < 100 :
\quad \text{Bénéfice} = 60\,N \;-\; 1000.
\]
Cette fonction est linéaire croissante en $N$. Donc son maximum dans cette plage 
est atteint en $N=100$. À $N=100$, on obtient 
\[
B(100) = 60\times100 - 1000 = 5000.
\]

\[
\text{Pour } N \ge 100 :
\quad x = N - 100,\quad \text{Prix par billet} = 60 - 0,25\,x,
\]
\[
\text{Coût total} = 1000 + 15\,x, \quad
\text{Revenu} = (100 + x)\bigl(60 - 0,25\,x\bigr).
\]
\[
\text{Bénéfice} = (100 + x)\bigl(60 - 0,25\,x\bigr) - \bigl(1000 + 15\,x\bigr)
= 5000 + 20\,x - 0,25\,x^2.
\]
Le sommet de cette parabole (coefficient dominant $-0,25<0$) est :
\[
x_{\max} = -\frac{b}{2a}
= -\frac{20}{2 \cdot (-0,25)}
= 40.
\]
Soit $N=100+40=140$. Le bénéfice maximum alors :
\[
B(140) = 5000 + 20\times 40 - 0,25\times 40^2 
= 5400.
\]
Comme $5400 > 5000$, le bénéfice maximal \textbf{n'est pas atteint en dessous de 100 passagers} 
mais bien en $N=140$.
}

