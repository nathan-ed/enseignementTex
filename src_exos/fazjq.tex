\titre{}
\theme{limites}
\auteur{Nathan Scheinmann}
\niveau{3M}
\source{sesamath3e}
\type{serie}
\piments{2}
\pts{}
\annee{2425}

\contenu{
\tcblower
\noindent Montrer avec la définition la propriété P1 : « Si les nouvelles limites existent, la limite d'une somme est égale à la somme des limites ».  
}
\correction{
\tcblower
\textit{Generated by AI}

\textbf{Propriété P1 :} Si $\displaystyle\lim_{x \to a} f(x) = L$ et $\displaystyle\lim_{x \to a} g(x) = M$, alors $\displaystyle\lim_{x \to a} [f(x) + g(x)] = L + M$.

\textbf{Démonstration :}

Soit $\varepsilon > 0$ quelconque. Nous devons montrer qu'il existe $\delta > 0$ tel que :
\[0 < |x - a| < \delta \implies |[f(x) + g(x)] - (L + M)| < \varepsilon\]

Puisque $\displaystyle\lim_{x \to a} f(x) = L$, il existe $\delta_1 > 0$ tel que :
\[0 < |x - a| < \delta_1 \implies |f(x) - L| < \frac{\varepsilon}{2}\]

Puisque $\displaystyle\lim_{x \to a} g(x) = M$, il existe $\delta_2 > 0$ tel que :
\[0 < |x - a| < \delta_2 \implies |g(x) - M| < \frac{\varepsilon}{2}\]

Prenons $\delta = \min(\delta_1, \delta_2)$. Alors, pour $0 < |x - a| < \delta$ :

\begin{align*}
|[f(x) + g(x)] - (L + M)| &= |[f(x) - L] + [g(x) - M]| \\
&\leq |f(x) - L| + |g(x) - M| \quad \text{(inégalité triangulaire)} \\
&< \frac{\varepsilon}{2} + \frac{\varepsilon}{2} \\
&= \varepsilon
\end{align*}

Donc $\displaystyle\lim_{x \to a} [f(x) + g(x)] = L + M$.
}

