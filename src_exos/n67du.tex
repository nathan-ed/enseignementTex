\titre{}
\theme{derivation}
\auteur{Nathan Scheinmann}
\niveau{3M}
\source{sesamath3e}
\type{serie}
\piments{2}
\pts{}
\annee{2425}

\contenu{
\tcblower
Soient $p$ et $q$ deux réels et $n$ un entier
naturel supérieur ou égal à 2. Montrer que la
fonction polynomiale $P$ définie par
$P(x) = x^n + px + q$ admet au plus trois racines
réelles si $n$ est impair et au plus deux racines
réelles si $n$ est pair.
}
\correction{

}

