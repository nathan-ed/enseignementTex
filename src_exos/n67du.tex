\titre{}
\theme{derivation}
\auteur{Nathan Scheinmann}
\niveau{3M}
\source{sesamath3e}
\type{serie}
\piments{2}
\pts{}
\annee{2425}

\contenu{
\tcblower
Soient $p$ et $q$ deux réels et $n$ un entier
naturel supérieur ou égal à 2. Montrer que la
fonction polynomiale $P$ définie par
$P(x) = x^n + px + q$ admet au plus trois racines
réelles si $n$ est impair et au plus deux racines
réelles si $n$ est pair.
}
\correction{
\tcblower
\textit{Generated by AI}

Considérons la fonction $P(x) = x^n + px + q$ avec $n \geq 2$.

\textbf{Étude de la dérivée :}

La dérivée de $P$ est :
\[
P'(x) = nx^{n-1} + p
\]

\textbf{Cas 1 : $n$ impair}

Si $n$ est impair, alors $n-1$ est pair, donc $x^{n-1} \geq 0$ pour tout $x \in \mathbb{R}$.

\textit{Sous-cas 1a :} Si $p \geq 0$, alors $P'(x) > 0$ pour tout $x \neq 0$, donc $P$ est strictement croissante sur $\mathbb{R}$ et admet au plus une racine.

\textit{Sous-cas 1b :} Si $p < 0$, alors $P'(x) = 0 \Leftrightarrow x^{n-1} = -\frac{p}{n}$, ce qui donne deux solutions $x = \pm \left(-\frac{p}{n}\right)^{1/(n-1)}$.

La fonction $P$ admet donc au plus deux extrema locaux. Entre deux extrema locaux, il peut y avoir au plus une racine. Comme $P(x) \to +\infty$ quand $x \to +\infty$ et $P(x) \to -\infty$ quand $x \to -\infty$ (car $n$ est impair), la fonction peut avoir :
\begin{itemize}
\item Une racine avant le minimum local
\item Une racine entre le minimum et le maximum
\item Une racine après le maximum local
\end{itemize}

Donc au plus 3 racines.

\textbf{Cas 2 : $n$ pair}

Si $n$ est pair, alors $n-1$ est impair, donc $P'(x) = 0$ admet une unique solution :
\[
x = \left(-\frac{p}{n}\right)^{1/(n-1)}
\]

si $p < 0$, ou aucune solution si $p \geq 0$.

Comme $P(x) \to +\infty$ quand $x \to \pm \infty$ (car $n$ est pair), la fonction admet au plus un minimum. Elle peut donc avoir :
\begin{itemize}
\item Aucune racine si le minimum est positif
\item Une racine si le minimum est nul
\item Deux racines si le minimum est négatif (une de chaque côté)
\end{itemize}

Donc au plus 2 racines.

\textbf{Conclusion :} La fonction $P$ admet au plus 3 racines réelles si $n$ est impair et au plus 2 racines réelles si $n$ est pair.
}

