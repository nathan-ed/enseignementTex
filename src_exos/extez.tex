\titre{}
\theme{limites}
\auteur{Nathan Scheinmann}
\niveau{3M}
\source{sesamath3e}
\type{serie}
\piments{2}
\pts{}
\annee{2425}

\contenu{
\tcblower
\noindent Montrer avec la définition que :  
\begin{tasks}
\task \(\displaystyle\lim_{x\to -1}2x=-2\).
\task \(\displaystyle\lim_{x\to 2}(3x-1)=5\).
\task \(\displaystyle\lim_{x\to a}x=a\).
\task \(\displaystyle\lim_{x\to a}k=k\).
\end{tasks}
}
\correction{
\tcblower
\textit{Generated by AI}

La définition de la limite dit : $\displaystyle\lim_{x \to a} f(x) = L$ si et seulement si, pour tout $\varepsilon > 0$, il existe $\delta > 0$ tel que :
\[0 < |x - a| < \delta \implies |f(x) - L| < \varepsilon\]

\begin{tasks}(1)
\task $\displaystyle\lim_{x\to -1}2x=-2$

Il faut montrer que pour tout $\varepsilon > 0$, il existe $\delta > 0$ tel que :
\[0 < |x - (-1)| < \delta \implies |2x - (-2)| < \varepsilon\]

Simplifions : $|2x + 2| = 2|x + 1| < \varepsilon$

Donc si $|x + 1| < \frac{\varepsilon}{2}$, alors $|2x + 2| < \varepsilon$.

Prenons $\delta = \frac{\varepsilon}{2}$. Alors :
\[|x + 1| < \delta = \frac{\varepsilon}{2} \implies |2x + 2| = 2|x + 1| < 2 \cdot \frac{\varepsilon}{2} = \varepsilon\]

\task $\displaystyle\lim_{x\to 2}(3x-1)=5$

Il faut montrer : $|3x - 1 - 5| < \varepsilon$ quand $|x - 2|$ est assez petit.

$|3x - 6| = 3|x - 2| < \varepsilon$

Prenons $\delta = \frac{\varepsilon}{3}$. Alors :
\[|x - 2| < \delta \implies |3x - 6| = 3|x - 2| < 3 \cdot \frac{\varepsilon}{3} = \varepsilon\]

\task $\displaystyle\lim_{x\to a}x=a$

Il faut montrer : $|x - a| < \varepsilon$ quand $|x - a|$ est assez petit.

C'est immédiat ! Prenons $\delta = \varepsilon$ :
\[|x - a| < \delta = \varepsilon \implies |x - a| < \varepsilon\]

\task $\displaystyle\lim_{x\to a}k=k$

Il faut montrer : $|k - k| < \varepsilon$, c'est-à-dire $0 < \varepsilon$.

Ceci est toujours vrai pour tout $\delta > 0$ (on peut prendre n'importe quel $\delta$) car $|k - k| = 0 < \varepsilon$ pour tout $\varepsilon > 0$.
\end{tasks}
}

