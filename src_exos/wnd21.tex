\titre{}
\theme{limites}
\auteur{Nathan Scheinmann}
\niveau{3M}
\source{sesamath3e}
\type{serie}
\piments{2}
\pts{}
\annee{2425}

\contenu{
\tcblower
Calculer les limites suivantes en  justifiant chaque étape.
\begin{tasks}(3)
\task $\displaystyle\lim_{x\to -1}\dfrac{x^3 - 5x + 1}{1}$
\task $\displaystyle\lim_{x\to 1}\dfrac{x^2 - 2}{x^2 + 3x + 5}$
\task \(\displaystyle\lim_{x\to 2}\frac{2}{2x+4}\)
\task \(\displaystyle\lim_{x\to 1}\frac{2-x}{x^2-16}\)
\task \(\displaystyle\lim_{x\to 3}\frac{5x}{(x+3)^2}\)
\task \(\displaystyle\lim_{x\to 0}\frac{x^2}{x+2}\)
\task \(\displaystyle\lim_{x\to 1}\frac{x^2-1}{x^2+3x+5}\)
\task \(\displaystyle\lim_{x\to0}\sqrt{x^2 - 2x + 4 + 2x}\)
\task*(2) \(\displaystyle\lim_{x\to2}\sqrt{x^2 - 2x + 4 + x}\)
\task $\displaystyle\lim_{x\to 0}\dfrac{x + \sqrt{x + 6}}{x + \sqrt{2 - x}}$
\end{tasks}
}
\correction{
	\tcblower
\begin{tasks}(1)
\task \( \displaystyle \lim_{x \to -1} \dfrac{x^3 - 5x + 1}{1}
\stackrel{\text{P4}}{=} \dfrac{\lim_{x \to -1} (x^3 - 5x + 1)}{1}
\stackrel{\text{cor. \ref{thm:poly}}}{=} \dfrac{(-1)^3 - 5(-1) + 1}{1}
= \dfrac{-1 + 5 + 1}{1} = \dfrac{5}{1} = 5 \)

\task \( \displaystyle \lim_{x \to 1} \dfrac{x^2 - 2}{x^2 + 3x + 5}
\stackrel{\text{P4}}{=} \dfrac{\lim_{x \to 1} (x^2 - 2)}{\lim_{x \to 1} (x^2 + 3x + 5)}
\stackrel{\text{cor. \ref{thm:poly}}}{=} \dfrac{1 - 2}{1 + 3 + 5} = \dfrac{-1}{9} = -\dfrac{1}{9} \)
  \task \( \displaystyle \lim_{x \to 2} \dfrac{2}{2x + 4}
  \stackrel{\text{P4}}{=} \dfrac{2}{\lim_{x \to 2} (2x + 4)}
  \stackrel{\text{cor. \ref{thm:poly}}}{=} \dfrac{2}{2\cdot2 + 4}
  = \dfrac{2}{8} = \dfrac{1}{4} \)
  \task \( \displaystyle \lim_{x \to 1} \dfrac{2 - x}{x^2 - 16}
  = \dfrac{2 - 1}{1^2 - 16}
  = \dfrac{1}{-15} = -\dfrac{1}{15} \quad \text{(idem)} \)

 \task \( \displaystyle \lim_{x \to 3} \dfrac{5x}{(x + 3)^2}
 = \dfrac{15}{36} = \dfrac{5}{12} \quad \text{P4 + P3 + cor. \ref{thm:poly}} \)

 \task $\displaystyle\lim_{x\to 0}\dfrac{x^2}{x+2}=\dfrac{0}{2}=0 \quad \text{P4+cor. \ref{thm:poly}}$ 
\task \( \displaystyle \lim_{x \to 1} \dfrac{x^2 - 1}{x^2 + 3x + 5}
\stackrel{\text{P4+cor. \ref{thm:poly}}}{=} \dfrac{1 - 1}{1 + 3 + 5}
= \dfrac{0}{9} = 0 \)
\task 
\(
\begin{aligned}
\lim_{x \to 0} \sqrt{x^2 - 2x + 4 + 2x}
&= \lim_{x \to 0} \sqrt{x^2 + 2x + 4} = \sqrt{4} = 2 \quad \text{thm. \ref{thm:compo}+L4+ cor. \ref{thm:poly}}
\end{aligned}
\)

\task 
\(
\begin{aligned}
\lim_{x \to 2} \sqrt{x^2 - 2x + 4 + x}
&= \lim_{x \to 2} \sqrt{x^2 - x + 4} = \sqrt{6}\quad \text{P4 + thm. \ref{thm:compo}+L4+ cor. \ref{thm:poly}}
\end{aligned}
\)
\task   
\(
\begin{aligned}
	\lim_{x \to 0} \dfrac{x + \sqrt{x + 6}}{x + \sqrt{2 - x}}
= \lim_{x \to 0} \dfrac{x + \sqrt{x + 6}}{\sqrt{2}}
= \dfrac{\sqrt{6}}{\sqrt{2}} = \sqrt{3}
\text{P4 + P1 + thm. \ref{thm:compo}+L4+ cor. \ref{thm:poly}}
\end{aligned}
\)
\end{tasks}
}

