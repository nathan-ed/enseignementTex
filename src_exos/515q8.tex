\titre{}
\theme{derivation}
\auteur{Nathan Scheinmann}
\niveau{3M}
\source{sesamath3e}
\type{serie}
\piments{2}
\pts{}
\annee{2425}

\contenu{
\tcblower
Étudier les fonctions définies par (sans deuxième dérivée) :

\begin{tasks}(2)
\task $f(x) = 3x^5 - 5x^3$
\task $f(x) = x^4 - 2x^3$
\task $f(x) = \dfrac{x^2 - 4x + 6}{(x - 2)^2}$
\task $f(x) = \dfrac{2x^2 - 3}{x - 1}$
\task $f(x) = \dfrac{x - x^2}{x^2 - 6x + 5x}$
\end{tasks}
}
\correction{
\tcblower

\textit{Generated by AI}

Pour étudier chaque fonction, nous allons calculer sa dérivée $f'(x)$, étudier son signe, puis déterminer les variations de $f$.

\begin{tasks}
\task $f(x) = 3x^5 - 5x^3$

\textbf{Dérivée :} $f'(x) = 15x^4 - 15x^2 = 15x^2(x^2 - 1) = 15x^2(x-1)(x+1)$

\textbf{Signe de $f'(x)$ :}
\begin{itemize}
\item $15x^2 \geq 0$ pour tout $x$
\item $(x-1)(x+1) > 0$ pour $x < -1$ ou $x > 1$
\item $(x-1)(x+1) < 0$ pour $-1 < x < 1$
\end{itemize}

Ainsi $f'(x) \leq 0$ sur $[-1 ; 1]$ et $f'(x) \geq 0$ sur $]-\infty ; -1] \cup [1 ; +\infty[$.

\textbf{Variations :} $f$ est croissante sur $]-\infty ; -1]$, décroissante sur $[-1 ; 1]$, puis croissante sur $[1 ; +\infty[$.

$f$ admet un maximum local en $x = -1$ avec $f(-1) = 2$, et un minimum local en $x = 1$ avec $f(1) = -2$.

\task $f(x) = x^4 - 2x^3$

\textbf{Dérivée :} $f'(x) = 4x^3 - 6x^2 = 2x^2(2x - 3)$

\textbf{Signe de $f'(x)$ :}
$f'(x) \leq 0$ pour $x \leq \dfrac{3}{2}$ et $f'(x) \geq 0$ pour $x \geq \dfrac{3}{2}$.

\textbf{Variations :} $f$ est décroissante sur $\left]-\infty ; \dfrac{3}{2}\right]$ et croissante sur $\left[\dfrac{3}{2} ; +\infty\right[$.

$f$ admet un minimum en $x = \dfrac{3}{2}$ avec $f\left(\dfrac{3}{2}\right) = -\dfrac{27}{16}$.

\task $f(x) = \dfrac{x^2 - 4x + 6}{(x - 2)^2}$

\textbf{Dérivée :} En utilisant la formule $\left(\dfrac{u}{v}\right)' = \dfrac{u'v - uv'}{v^2}$ :

$f'(x) = \dfrac{(2x-4)(x-2)^2 - (x^2-4x+6) \cdot 2(x-2)}{(x-2)^4} = \dfrac{2(x-2)[(x-2)^2 - (x^2-4x+6)]}{(x-2)^4}$

$= \dfrac{2(x^2-4x+4 - x^2+4x-6)}{(x-2)^3} = \dfrac{-4}{(x-2)^3}$

\textbf{Signe de $f'(x)$ :} $f'(x) < 0$ pour $x > 2$ et $f'(x) > 0$ pour $x < 2$.

\textbf{Variations :} $f$ est croissante sur $]-\infty ; 2[$ et décroissante sur $]2 ; +\infty[$ (discontinuité en $x = 2$).

\task $f(x) = \dfrac{2x^2 - 3}{x - 1}$

\textbf{Dérivée :} $f'(x) = \dfrac{4x(x-1) - (2x^2-3)}{(x-1)^2} = \dfrac{4x^2-4x - 2x^2+3}{(x-1)^2} = \dfrac{2x^2-4x+3}{(x-1)^2}$

Le discriminant de $2x^2-4x+3$ est $\Delta = 16 - 24 = -8 < 0$, donc $2x^2-4x+3 > 0$ pour tout $x$.

\textbf{Signe de $f'(x)$ :} $f'(x) > 0$ pour tout $x \neq 1$.

\textbf{Variations :} $f$ est croissante sur $]-\infty ; 1[$ et sur $]1 ; +\infty[$ (discontinuité en $x = 1$).

\task $f(x) = \dfrac{x - x^2}{x^2 - 6x + 5}$ (en supposant qu'il y a une erreur dans l'énoncé : $5x$ devrait être $5$)

Domaine : $x^2 - 6x + 5 = (x-1)(x-5) \neq 0$, donc $\mathcal{D}_f = \mathbb{R} \setminus \{1 ; 5\}$.

\textbf{Dérivée :} $f'(x) = \dfrac{(1-2x)(x^2-6x+5) - (x-x^2)(2x-6)}{(x^2-6x+5)^2}$

Après développement et simplification, on étudie le signe du numérateur pour déterminer les variations.

\end{tasks}

}

