\titre{}
\theme{derivation2}
\auteur{Nathan Scheinmann}
\niveau{3M}
\source{sesamath3e}
\type{serie}
\piments{2}
\pts{}
\annee{2425}

\contenu{
\tcblower
Soit la fonction \(f(x)=m x^2 + k x + q\) (\(m\neq0\)). On considère \(f\) dans l’intervalle \([0;3]\).
\begin{tasks}(1)
  \task Poser \(m=-1\), \(k=2\) et \(q=3\) et calculer le point \(c\in]0;3[\) annoncé par le théorème des accroissements finis.
  \task Représenter graphiquement la situation du point \(c\).
  \task Considérer \(f\) dans l’intervalle \([1;5]\) et calculer le point \(c\in]1;5[\) annoncé par le théorème des accroissements finis.
  \task Considérer \(f\) dans l’intervalle \([2;8]\) et calculer le point \(c\in]2;8[\) annoncé par le théorème des accroissements finis.
  \task Considérer \(f\) dans l’intervalle \([a;b]\). Formuler et démontrer une conjecture sur le point \(c\in]a;b[\) annoncé par le théorème des accroissements finis.
\end{tasks}
}
\correction{
\tcblower
Pour $f(x)=mx^2+kx+q$, on a toujours $c = \dfrac{a+b}{2}$.
\begin{tasks}(4)
  \task $c=1,5$
  \task Représenter à l'aide d'un grapheur
  \task $c=3$
  \task $c=5$
  \end{tasks}
}
