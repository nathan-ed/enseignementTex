\titre{}
\theme{derivation2}
\auteur{Nathan Scheinmann}
\niveau{3M}
\source{sesamath3e}
\type{serie}
\piments{2}
\pts{}
\annee{2425}

\contenu{
\tcblower
Soit la fonction \(f(x)=m x^2 + k x + q\) (\(m\neq0\)). On considère \(f\) dans l’intervalle \([0;3]\).
\begin{tasks}(1)
  \task Poser \(m=-1\), \(k=2\) et \(q=3\) et calculer le point \(c\in]0;3[\) annoncé par le théorème des accroissements finis.
  \task Représenter graphiquement la situation du point \(c\).
  \task Considérer \(f\) dans l’intervalle \([1;5]\) et calculer le point \(c\in]1;5[\) annoncé par le théorème des accroissements finis.
  \task Considérer \(f\) dans l’intervalle \([2;8]\) et calculer le point \(c\in]2;8[\) annoncé par le théorème des accroissements finis.
  \task Considérer \(f\) dans l’intervalle \([a;b]\). Formuler et démontrer une conjecture sur le point \(c\in]a;b[\) annoncé par le théorème des accroissements finis.
\end{tasks}
}
\correction{
\tcblower
\textit{Generated by AI}

Rappelons le théorème des accroissements finis (TAF) : si $f$ est continue sur $[a;b]$ et dérivable sur $]a;b[$, alors il existe $c \in ]a;b[$ tel que :
\[f'(c) = \frac{f(b) - f(a)}{b - a}\]

\begin{tasks}(1)
\task Avec $m=-1$, $k=2$ et $q=3$, on a $f(x) = -x^2 + 2x + 3$ sur $[0;3]$.

$f$ est un polynôme, donc continue sur $[0;3]$ et dérivable sur $]0;3[$.

Calculons : $f'(x) = -2x + 2$, $f(0) = 3$, $f(3) = -9 + 6 + 3 = 0$.

D'après le TAF :
\[f'(c) = \frac{f(3) - f(0)}{3 - 0} = \frac{0 - 3}{3} = -1\]

Résolvons :
\[-2c + 2 = -1 \implies -2c = -3 \implies \boxed{c = \frac{3}{2}}\]

\task Représentation graphique :

Le point $c = \frac{3}{2}$ est l'abscisse du point de la courbe où la tangente est parallèle à la droite $(A,B)$ reliant les points $A(0, f(0)) = (0, 3)$ et $B(3, f(3)) = (3, 0)$.

La pente de la droite $(AB)$ est $\frac{0-3}{3-0} = -1$.

Au point $c = \frac{3}{2}$, on a $f\left(\frac{3}{2}\right) = -\frac{9}{4} + 3 + 3 = \frac{15}{4}$ et $f'\left(\frac{3}{2}\right) = -1$.

\task Sur $[1;5]$ avec $f(x) = -x^2 + 2x + 3$ :

$f(1) = -1 + 2 + 3 = 4$ et $f(5) = -25 + 10 + 3 = -12$.

\[f'(c) = \frac{f(5) - f(1)}{5 - 1} = \frac{-12 - 4}{4} = \frac{-16}{4} = -4\]

\[-2c + 2 = -4 \implies -2c = -6 \implies \boxed{c = 3}\]

\task Sur $[2;8]$ :

$f(2) = -4 + 4 + 3 = 3$ et $f(8) = -64 + 16 + 3 = -45$.

\[f'(c) = \frac{f(8) - f(2)}{8 - 2} = \frac{-45 - 3}{6} = \frac{-48}{6} = -8\]

\[-2c + 2 = -8 \implies -2c = -10 \implies \boxed{c = 5}\]

\task Sur $[a;b]$ avec $f(x) = mx^2 + kx + q$ :

$f'(x) = 2mx + k$, $f(a) = ma^2 + ka + q$, $f(b) = mb^2 + kb + q$.

\[f'(c) = \frac{f(b) - f(a)}{b - a} = \frac{(mb^2 + kb + q) - (ma^2 + ka + q)}{b - a} = \frac{m(b^2 - a^2) + k(b - a)}{b - a}\]

\[= \frac{m(b-a)(b+a) + k(b-a)}{b - a} = m(b+a) + k\]

Résolvons :
\[2mc + k = m(a+b) + k \implies 2mc = m(a+b) \implies \boxed{c = \frac{a+b}{2}}\]

\textbf{Conjecture :} Pour une fonction quadratique $f(x) = mx^2 + kx + q$ ($m \neq 0$), le point $c$ annoncé par le théorème des accroissements finis sur l'intervalle $[a;b]$ est toujours le milieu de l'intervalle : $c = \frac{a+b}{2}$.

Cela s'explique par la symétrie de la parabole : la pente moyenne sur l'intervalle est égale à la dérivée au point milieu.
\end{tasks}
}

