\titre{}
\theme{dérivées}
\auteur{Nathan Scheinmann}
\niveau{3M}
\source{analysis}
\type{serie}
\piments{2}
\pts{}
\annee{2425}

\contenu{
\tcblower
Décrire la concavité du graphe des fonctions suivantes et déterminer les points d'inflections s'il y en a~:
\begin{tasks}(3)
\task $\dfrac{1}{x}$
\task $x + \dfrac{1}{x}$
\task $x^3 - 3x + 2$
\task $2x^2 - 5x + 2$
\task $\dfrac{1}{4}x^4 - \dfrac{1}{2}x^2$
\task $x^3(1 - x)$
\task $\dfrac{x}{x^2 - 1}$
\task $\dfrac{x + 2}{x - 2}$
\task $(1 - x)^2(1 + x)^2$
\task $\dfrac{6x}{x^2 + 1}$
\task $\dfrac{1 - \sqrt{x}}{1 + \sqrt{x}}$
\task $(x - 3)^{1/5}$
\end{tasks}
}
\correction{
\tcblower
%% GENERATED BY AI %%
Pour chaque fonction, nous calculons $f''(x)$ pour déterminer la concavité et les points d'inflexion.

\begin{tasks}(3)
\task $f(x) = \dfrac{1}{x}$, $D_f = \mathbb{R}^*$

$f'(x) = -\dfrac{1}{x^2}$, $f''(x) = \dfrac{2}{x^3}$

$f''(x) > 0$ pour $x > 0$ : concave vers le haut sur $]0; +\infty[$

$f''(x) < 0$ pour $x < 0$ : concave vers le bas sur $]-\infty; 0[$

Pas de point d'inflexion ($f$ n'est pas continue en $x = 0$).

\task $f(x) = x + \dfrac{1}{x}$, $D_f = \mathbb{R}^*$

$f'(x) = 1 - \dfrac{1}{x^2}$, $f''(x) = \dfrac{2}{x^3}$

Même concavité que ci-dessus : concave vers le haut pour $x > 0$, vers le bas pour $x < 0$.

Pas de point d'inflexion.

\task $f(x) = x^3 - 3x + 2$

$f'(x) = 3x^2 - 3$, $f''(x) = 6x = 0 \Rightarrow x = 0$

$f''(x) < 0$ pour $x < 0$ : concave vers le bas

$f''(x) > 0$ pour $x > 0$ : concave vers le haut

Point d'inflexion en $(0, f(0)) = (0, 2)$.

\task $f(x) = 2x^2 - 5x + 2$

$f'(x) = 4x - 5$, $f''(x) = 4 > 0$ pour tout $x$

Toujours concave vers le haut. Pas de point d'inflexion.

\task $f(x) = \dfrac{1}{4}x^4 - \dfrac{1}{2}x^2$

$f'(x) = x^3 - x$, $f''(x) = 3x^2 - 1 = 0 \Rightarrow x = \pm \dfrac{1}{\sqrt{3}}$

Changement de concavité en $x = -\dfrac{1}{\sqrt{3}}$ et $x = \dfrac{1}{\sqrt{3}}$.

Points d'inflexion : $\left(\pm \dfrac{1}{\sqrt{3}}, -\dfrac{1}{6}\right)$.

\task $f(x) = x^3(1 - x) = x^3 - x^4$

$f'(x) = 3x^2 - 4x^3$, $f''(x) = 6x - 12x^2 = 6x(1 - 2x) = 0$

$x = 0$ ou $x = \dfrac{1}{2}$

Points d'inflexion en $(0, 0)$ et $\left(\dfrac{1}{2}, \dfrac{1}{16}\right)$.

\task $f(x) = \dfrac{x}{x^2 - 1}$, $D_f = \mathbb{R} \setminus \{-1, 1\}$

$f'(x) = \dfrac{(x^2-1) - x \cdot 2x}{(x^2-1)^2} = \dfrac{-x^2-1}{(x^2-1)^2}$

$f''(x) = \dfrac{-2x(x^2-1)^2 - (-x^2-1) \cdot 2(x^2-1) \cdot 2x}{(x^2-1)^4}$

$= \dfrac{-2x(x^2-1) + 4x(x^2+1)}{(x^2-1)^3} = \dfrac{2x(x^2+3)}{(x^2-1)^3} = 0 \Rightarrow x = 0$

Point d'inflexion en $(0, 0)$.

\task $f(x) = \dfrac{x + 2}{x - 2}$, $D_f = \mathbb{R} \setminus \{2\}$

$f'(x) = \dfrac{(x-2) - (x+2)}{(x-2)^2} = \dfrac{-4}{(x-2)^2}$

$f''(x) = \dfrac{8}{(x-2)^3}$

$f''(x) > 0$ pour $x > 2$, $f''(x) < 0$ pour $x < 2$

Pas de point d'inflexion (discontinuité en $x = 2$).

\task $f(x) = (1 - x)^2(1 + x)^2 = [(1-x)(1+x)]^2 = (1-x^2)^2$

$f(x) = 1 - 2x^2 + x^4$

$f'(x) = -4x + 4x^3$, $f''(x) = -4 + 12x^2 = 0 \Rightarrow x^2 = \dfrac{1}{3} \Rightarrow x = \pm \dfrac{1}{\sqrt{3}}$

Points d'inflexion en $\left(\pm \dfrac{1}{\sqrt{3}}, \dfrac{4}{9}\right)$.

\task $f(x) = \dfrac{6x}{x^2 + 1}$

$f'(x) = \dfrac{6(x^2+1) - 6x \cdot 2x}{(x^2+1)^2} = \dfrac{6 - 6x^2}{(x^2+1)^2}$

$f''(x) = \dfrac{-12x(x^2+1)^2 - (6-6x^2) \cdot 2(x^2+1) \cdot 2x}{(x^2+1)^4}$

$= \dfrac{-12x(x^2+1) - 4x(6-6x^2)}{(x^2+1)^3} = \dfrac{-12x^3 - 12x - 24x + 24x^3}{(x^2+1)^3} = \dfrac{12x^3 - 36x}{(x^2+1)^3} = \dfrac{12x(x^2-3)}{(x^2+1)^3} = 0$

$x = 0$ ou $x = \pm\sqrt{3}$

Points d'inflexion en $(0, 0)$, $\left(\sqrt{3}, \dfrac{3\sqrt{3}}{2}\right)$ et $\left(-\sqrt{3}, -\dfrac{3\sqrt{3}}{2}\right)$.

\task $f(x) = \dfrac{1 - \sqrt{x}}{1 + \sqrt{x}}$, $D_f = [0; +\infty[$

Posons $u = \sqrt{x}$, alors $f(u) = \dfrac{1-u}{1+u}$

$\dfrac{df}{du} = \dfrac{-(1+u) - (1-u)}{(1+u)^2} = \dfrac{-2}{(1+u)^2}$

Par règle de chaîne : $f'(x) = \dfrac{-2}{(1+\sqrt{x})^2} \cdot \dfrac{1}{2\sqrt{x}} = \dfrac{-1}{\sqrt{x}(1+\sqrt{x})^2}$

Calcul de $f''(x)$ est complexe mais donne : changement de concavité possible.

Pour simplifier : $f''(x) = \dfrac{3 + 2\sqrt{x}}{4x^{3/2}(1+\sqrt{x})^3} > 0$ pour $x > 0$

Toujours concave vers le haut, pas de point d'inflexion.

\task $f(x) = (x - 3)^{1/5}$

$f'(x) = \dfrac{1}{5}(x-3)^{-4/5}$

$f''(x) = -\dfrac{4}{25}(x-3)^{-9/5}$

$f''(x) > 0$ pour $x < 3$ (concave vers le haut)

$f''(x) < 0$ pour $x > 3$ (concave vers le bas)

Point d'inflexion en $(3, 0)$.
\end{tasks}
}

