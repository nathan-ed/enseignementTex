\titre{}
\theme{derivation}
\auteur{Nathan Scheinmann}
\niveau{3M}
\source{sesamath3e}
\type{serie}
\piments{2}
\pts{}
\annee{2425}

\contenu{
\tcblower
Déterminer l'expression algébrique d'une fonction qui s'annule en $x = 0$, qui est décroissante pour $x < -2$ et croissante sinon.
}
\correction{
\tcblower
\textit{Generated by AI}

Nous cherchons une fonction qui :
\begin{itemize}
\item S'annule en $x = 0$ : $f(0) = 0$
\item Est décroissante pour $x < -2$ : $f'(x) < 0$ pour $x < -2$
\item Est croissante pour $x > -2$ : $f'(x) > 0$ pour $x > -2$
\end{itemize}

Cela indique que $f$ a un minimum en $x = -2$.

\textbf{Proposition :} Une fonction polynomiale du second degré convient :
$$f(x) = a(x + 2)^2 + b$$

\textbf{Condition 1 :} $f(0) = 0$
$$a(0 + 2)^2 + b = 0$$
$$4a + b = 0$$
$$b = -4a$$

Donc :
$$f(x) = a(x + 2)^2 - 4a = a[(x + 2)^2 - 4] = a(x^2 + 4x + 4 - 4) = a(x^2 + 4x)$$
$$f(x) = ax(x + 4)$$

\textbf{Vérification :}
\begin{itemize}
\item $f(0) = 0$ : vrai
\item $f'(x) = a(2x + 4) = 2a(x + 2)$
\item Pour $a > 0$ : $f'(x) < 0$ si $x < -2$ et $f'(x) > 0$ si $x > -2$ : vrai
\end{itemize}

\textbf{Exemple de fonction :} En prenant $a = 1$ :
$$\boxed{f(x) = x(x + 4) = x^2 + 4x}$$

Ou plus généralement : $\boxed{f(x) = ax(x + 4)}$ avec $a > 0$.
}

