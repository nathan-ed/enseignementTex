\titre{}
\theme{derivation}
\auteur{Nathan Scheinmann}
\niveau{3M}
\source{sesamath3e}
\type{serie}
\piments{2}
\pts{}
\annee{2425}

\contenu{
\tcblower
Déterminer les points critiques de la fonction $f$ déterminée par 
$f(x) = x^4 - 2x^3 - 12x^2 + 8x + 6$
}
\correction{
\tcblower
\textit{Generated by AI}

Les points critiques d'une fonction sont les points où la dérivée s'annule ou n'existe pas.

Calculons la dérivée de $f(x) = x^4 - 2x^3 - 12x^2 + 8x + 6$ :
\[
f'(x) = 4x^3 - 6x^2 - 24x + 8
\]

Cherchons les valeurs de $x$ pour lesquelles $f'(x) = 0$ :
\[
4x^3 - 6x^2 - 24x + 8 = 0
\]

Simplifions en divisant par 2 :
\[
2x^3 - 3x^2 - 12x + 4 = 0
\]

Cherchons des racines rationnelles possibles parmi les diviseurs de $\frac{4}{2} = 2$ : $\pm 1, \pm 2, \pm \frac{1}{2}$.

Testons $x = 2$ :
\[
2(2)^3 - 3(2)^2 - 12(2) + 4 = 16 - 12 - 24 + 4 = -16 \neq 0
\]

Testons $x = -2$ :
\[
2(-2)^3 - 3(-2)^2 - 12(-2) + 4 = -16 - 12 + 24 + 4 = 0
\]

Donc $x = -2$ est une racine. Effectuons la division polynomiale :
\[
2x^3 - 3x^2 - 12x + 4 = (x + 2)(2x^2 - 7x + 2)
\]

Résolvons $2x^2 - 7x + 2 = 0$ :
\[
x = \frac{7 \pm \sqrt{49 - 16}}{4} = \frac{7 \pm \sqrt{33}}{4}
\]

Donc :
\[
x_1 = \frac{7 - \sqrt{33}}{4} \approx 0{,}31 \quad \text{et} \quad x_2 = \frac{7 + \sqrt{33}}{4} \approx 3{,}19
\]

\textbf{Réponse :} Les points critiques sont :
\[
x = -2, \quad x = \frac{7 - \sqrt{33}}{4}, \quad x = \frac{7 + \sqrt{33}}{4}
\]
}

