\titre{4}
\theme{trigo}
\auteur{Nathan Scheinmann}
\niveau{1M}
\source{sesamath-1M-trigo}
\type{serie}
\piments{2}
\pts{}
\annee{2425}

\contenu{
	\tcblower
À l'aide des informations de la figure, calculer la mesure arrondie au degré de l'angle $\widehat{AIO}$.
\begin{center}
	\includegraphics[scale=1]{../medias/1M/trigo/1M-exo-4}
\end{center}
}
\correction{
\tcblower
\textit{Generated by AI}

Pour calculer l'angle $\widehat{AIO}$, il faut utiliser les informations données dans la figure.

\textbf{Méthode générale :}

Sans voir les détails exacts de la figure, voici les approches possibles selon les informations disponibles :

\textbf{Si on connaît les longueurs des côtés du triangle $AIO$ :}
\begin{itemize}
\item Utiliser la loi des cosinus : $\cos(\widehat{AIO}) = \dfrac{\overline{AI}^2 + \overline{IO}^2 - \overline{AO}^2}{2 \cdot \overline{AI} \cdot \overline{IO}}$
\item Puis calculer l'angle avec $\widehat{AIO} = \arccos(\text{valeur})$
\end{itemize}

\textbf{Si le triangle est rectangle :}
\begin{itemize}
\item Identifier quel côté est l'hypoténuse
\item Utiliser $\sin$, $\cos$ ou $\tan$ selon les côtés connus
\end{itemize}

\textbf{Si on connaît d'autres angles :}
\begin{itemize}
\item Utiliser le fait que la somme des angles d'un triangle vaut $180°$
\end{itemize}

\textbf{Note :} Pour obtenir la valeur numérique exacte, il faudrait les mesures spécifiques indiquées sur la figure 1M-exo-4.
}

