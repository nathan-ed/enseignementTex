\titre{16}
\theme{trigo}
\auteur{Nathan Scheinmann}
\niveau{1M}
\source{sesamath-1M-trigo}
\type{serie}
\piments{2}
\pts{}
\annee{2425}

\contenu{
\tcblower
Calculer la valeur exacte de $\sin(\beta)$ et de $\tan(\beta)$ sachant que $\beta$ est un anlge aigu tel que $\cos(\beta)=\dfrac{\sqrt{2}}{3}$. 
}
\correction{
\tcblower
\textit{Generated by AI}

Sachant que $\cos(\beta) = \dfrac{\sqrt{2}}{3}$ et que $\beta$ est un angle aigu, calculons $\sin(\beta)$ et $\tan(\beta)$.

\textbf{Calcul de $\sin(\beta)$ :}

Utilisons l'identité fondamentale $\sin^2(\beta) + \cos^2(\beta) = 1$ :

\[\sin^2(\beta) = 1 - \cos^2(\beta) = 1 - \left(\frac{\sqrt{2}}{3}\right)^2 = 1 - \frac{2}{9} = \frac{7}{9}\]

Puisque $\beta$ est aigu, $\sin(\beta) > 0$ :

\[\boxed{\sin(\beta) = \frac{\sqrt{7}}{3}}\]

\textbf{Calcul de $\tan(\beta)$ :}

\[\tan(\beta) = \frac{\sin(\beta)}{\cos(\beta)} = \frac{\sqrt{7}/3}{\sqrt{2}/3} = \frac{\sqrt{7}}{\sqrt{2}} = \frac{\sqrt{7} \cdot \sqrt{2}}{2} = \frac{\sqrt{14}}{2}\]

\[\boxed{\tan(\beta) = \frac{\sqrt{14}}{2}}\]
}

