\titre{}
\theme{derivation}
\auteur{Nathan Scheinmann}
\niveau{3M}
\source{sesamath3e}
\type{serie}
\piments{2}
\pts{}
\annee{2425}

\contenu{
\tcblower
Sur chacun des graphiques suivants, on a représenté graphiquement deux fonctions. Dire si l’une peut être la dérivée de l’autre.
GRAPHIC
}
\correction{
\tcblower
\textit{Generated by AI}

Pour déterminer si une fonction peut être la dérivée de l'autre, nous devons vérifier les critères suivants :

\begin{itemize}
\item Si $f$ est la dérivée de $g$, alors :
\begin{itemize}
\item Quand $f > 0$, $g$ est croissante
\item Quand $f < 0$, $g$ est décroissante
\item Quand $f = 0$ (et change de signe), $g$ a un extremum local
\item Quand $f$ a un extremum, $g$ a un point d'inflexion
\end{itemize}
\end{itemize}

\textbf{Analyse graphique :}

Sans voir les graphiques spécifiques, voici la méthode générale :

\begin{enumerate}
\item Identifier les zones où une fonction est croissante/décroissante
\item Vérifier si l'autre fonction est positive/négative dans ces mêmes zones
\item Vérifier si les extrema de l'une correspondent aux zéros de l'autre
\item Vérifier si les points d'inflexion de l'une correspondent aux extrema de l'autre
\end{enumerate}

\textit{Note : L'analyse détaillée nécessite de voir les graphiques mentionnés dans l'énoncé.}
}

