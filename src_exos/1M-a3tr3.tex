\titre{}
\theme{calcLit}
\auteur{Nathan Scheinmann}
\niveau{1M}
\source{musy}
\type{serie}
\piments{2}
\pts{}
\annee{2425}

\contenu{
\tcblower
En utilisant la lettre $n$ pour désigner un entier quelconque, exprimer sous forme littérale :
\begin{tasks}
\task trois entiers consécutifs;
\task le carré d'un entier impair quelconque;
\task un nombre positif, différence des carrés de deux nombres entiers consécutifs;
\task un multiple de 7;
\task un entier qui laisse un reste de 2 lorsqu'on le divise par 3 ;
\task un entier qui précède immédiatement un multiple de 4 ;
\task trois carrés parfaits consécutifs;
\task un nombre pair.
\end{tasks}
}
\correction{
\tcblower
\begin{tasks}(4)
	\task $n;n+1;n+2$
	\task $(2n+1)^2$
	\task $(n+1)^2-n^2$
	\task $7n$
	\task $3n+2$
	\task $4n-1$
	\task $n^2;(n+1)^2;(n+2)^2$
	\task $2n$
\end{tasks}
}

