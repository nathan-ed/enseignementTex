\titre{}
\theme{dérivées}
\auteur{Nathan Scheinmann}
\niveau{3M}
\source{fundamentum}
\type{serie}
\piments{2}
\pts{}
\annee{2425}

\contenu{
\tcblower
À 10 kilomètres de votre maison, vous vous rappelez avoir oublié de fermer un robinet, ce qui vous coûte 40 centimes par heure. En roulant à une vitesse constante de $s$ kilomètres par heure, le coût du carburant est de $8+\dfrac{s}{20}$ centimes par kilomètre. À quelle vitesse devez-vous faire l'aller et retour pour minimiser les frais totaux ?
}
\correction{
\tcblower
%% GENERATED BY AI %%
{\scriptsize \textit{Correction générée par IA}}

Notons $s$ la vitesse en km/h. La distance totale de l'aller-retour est $2 \times 10 = 20$ km.

Le temps nécessaire pour l'aller-retour est :
\[t = \frac{20}{s} \text{ heures}\]

\textbf{Coût du robinet :} Pendant ce temps, le robinet ouvert coûte $40$ centimes par heure, donc :
\[\text{Coût robinet} = 40t = \frac{800}{s} \text{ centimes}\]

\textbf{Coût du carburant :} Le coût par kilomètre est $8 + \frac{s}{20}$ centimes, donc pour 20 km :
\[\text{Coût carburant} = 20\left(8 + \frac{s}{20}\right) = 160 + s \text{ centimes}\]

\textbf{Coût total :}
\[C(s) = \frac{800}{s} + 160 + s\]

Pour minimiser le coût, nous dérivons par rapport à $s$ :
\[C'(s) = -\frac{800}{s^2} + 1\]

L'annulation de la dérivée donne :
\[-\frac{800}{s^2} + 1 = 0\]
\[\frac{800}{s^2} = 1\]
\[s^2 = 800\]
\[s = \sqrt{800} = 20\sqrt{2} \approx 28{,}28 \text{ km/h}\]

Vérifions qu'il s'agit bien d'un minimum :
\[C''(s) = \frac{1600}{s^3} > 0\]

pour $s > 0$, donc nous avons bien un minimum.

La vitesse optimale est $\boxed{20\sqrt{2} \approx 28{,}28 \text{ km/h}}$.

Le coût minimal est alors :
\[C(20\sqrt{2}) = \frac{800}{20\sqrt{2}} + 160 + 20\sqrt{2} = \frac{40}{\sqrt{2}} + 160 + 20\sqrt{2} = 20\sqrt{2} + 160 + 20\sqrt{2} = 40\sqrt{2} + 160 \approx 216{,}57 \text{ centimes}\]
}

