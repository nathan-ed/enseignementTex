\titre{}
\theme{derivation}
\auteur{Nathan Scheinmann}
\niveau{3M}
\source{sesamath3e}
\type{serie}
\piments{2}
\pts{}
\annee{2425}

\contenu{
\tcblower
On considère la fonction définie par $f(x) = ax^3 + bx^2 + cx + d$.

\begin{tasks}(1)
\task Déterminer les constantes $a$, $b$, $c$ et $d$ afin que le graphe de la fonction possède un extremum en $(1; 16)$ et un autre en $(3; -16)$.
\task $f$ admet-elle un point d'inflexion ? Si oui, calculer ses coordonnées.
\task Esquisser la courbe représentative. de $f$.
\end{tasks}
}
\correction{
\tcblower
\textit{Generated by AI}

Soit $f(x) = ax^3 + bx^2 + cx + d$.

\begin{tasks}(1)
\task \textbf{Déterminer les constantes :}

La dérivée est $f'(x) = 3ax^2 + 2bx + c$.

\textbf{Conditions :}
\begin{itemize}
\item Extremum en $(1, 16)$ : $f(1) = 16$ et $f'(1) = 0$
\item Extremum en $(3, -16)$ : $f(3) = -16$ et $f'(3) = 0$
\end{itemize}

Cela donne le système :
\begin{align*}
a + b + c + d &= 16 \quad (1) \\
3a + 2b + c &= 0 \quad (2) \\
27a + 9b + 3c + d &= -16 \quad (3) \\
27a + 6b + c &= 0 \quad (4)
\end{align*}

De (2) et (4) : $24a + 4b = 0 \implies b = -6a$

Substituant dans (2) : $3a - 12a + c = 0 \implies c = 9a$

De (1) : $a - 6a + 9a + d = 16 \implies d = 16 - 4a$

De (3) : $27a - 54a + 27a + 16 - 4a = -16 \implies -4a = -32 \implies a = 8$

Donc : $a = 8$, $b = -48$, $c = 72$, $d = -32$

\[\boxed{f(x) = 8x^3 - 48x^2 + 72x - 32}\]

\task \textbf{Point d'inflexion :}

$f''(x) = 48x - 96 = 0 \implies x = 2$

$f(2) = 8(8) - 48(4) + 72(2) - 32 = 64 - 192 + 144 - 32 = -16 + 16 = 0$

Point d'inflexion : \boxed{(2, 0)}

\task L'esquisse montrera une cubique avec un maximum local en $(1, 16)$, un minimum local en $(3, -16)$, et un point d'inflexion en $(2, 0)$.
\end{tasks}
}

