\titre{}
\theme{numerique}
\auteur{Nathan Scheinmann}
\niveau{1M}
\source{musy}
\type{serie}
\piments{1}
\pts{}
\annee{2425}

\contenu{
\tcblower
Décomposer les nombres suivants en produit de facteurs premiers (sans calculatrice)~:
\begin{tasks}(5)
\task 10
\task $10^2$
\task 100000
\task $24 \cdot 1000$
\task $38 \cdot 10^5$
\task 25000
\task 28000
\task 66000
\task 16000
\task 3600000
\end{tasks}
}
\correction{
\tcblower
	On utilise surtout la décomposition de $10=2\cdot 5$ et donc que $10^n=2^n\cdot 5^n$.
%corriger cet exercice
	\begin{tasks}(3)
\task $10=2\cdot 5$
\task $10^2=2^2\cdot 5^2$
\task $100000=2^5\cdot 5^5$
\task $24 \cdot 1000=2^6\cdot 3\cdot 5^3$
\task $38 \cdot 10^5=2^6\cdot 5^5\cdot 19$
\task $25000=5^5\cdot 2^3$
\task $28000= 2^5\cdot 5^3\cdot 7$
\task $66000=2^4\cdot 3\cdot 5^3\cdot 11$
\task $16000=2^7\cdot 5^3$
\task $3600000=2^7\cdot 3^2\cdot 5^5$
	\end{tasks}
}

