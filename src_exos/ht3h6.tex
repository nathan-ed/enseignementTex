\titre{}
\theme{intervalle}
\auteur{Nathan Scheinmann}
\niveau{1M}
\source{crm}
\type{serie}
\piments{2}
\pts{}
\annee{2425}

\contenu{
\tcblower
On donne trois intervalles $I$, $J$ et $K$ de $\mathbb{R}$. Déterminer $I \cap J$, $I \cap K$, $I \setminus (J \cup K)$, $(I \setminus J) \cup (I \setminus K)$ dans les cas suivants :
\begin{tasks}
\task $I = \interval[open left]{-3}{4}$ \quad $J = \interval[open]{-2}{0}$ \quad $K = \interval[open right]{-5}{3}$
\task $I = \interval[open right]{-4}{2}$ \quad $J = \interval[open left]{-2}{3}$ \quad $K = \interval[open right]{-3}{1}$
\task $I = \interval[open right]{-5}{3}$ \quad $J = \interval[open right]{-1}{5}$ \quad $K = \interval[open left]{-3}{4}$
\end{tasks}
}
\correction{
\tcblower
\begin{tasks}(3)
\task
  \begin{enumerate}
    \item $I\cap J=\ointerval{-2}{0}$
    \item $I\cap K=\ointerval{-3}{3}$
    \item $I\setminus(J\cup K)=\interval{3}{4}$
    \item $(I\setminus J)\cup(I\setminus K)=\linterval{-3}{-2}\cup\interval{0}{4}$
  \end{enumerate}

\task
  \begin{enumerate}
    \item $I\cap J=\ointerval{-2}{2}$
    \item $I\cap K=\rinterval{-3}{1}$
    \item $I\setminus(J\cup K)=\rinterval{-4}{-3}$
    \item $(I\setminus J)\cup(I\setminus K)=\interval{-4}{-2}\cup\rinterval{1}{2}$
  \end{enumerate}

\task
  \begin{enumerate}
    \item $I\cap J=\rinterval{-1}{3}$
    \item $I\cap K=\ointerval{-3}{3}$
    \item $I\setminus(J\cup K)=\interval{-5}{-3}$
    \item $(I\setminus J)\cup(I\setminus K)=\rinterval{-5}{-1}$
  \end{enumerate}
\end{tasks}
}

