\titre{}
\theme{derivation}
\auteur{Nathan Scheinmann}
\niveau{3M}
\source{sesamath3e}
\type{serie}
\piments{2}
\pts{}
\annee{2425}

\contenu{
\tcblower
Montrer que la pente de la tangente à la fonction \(f(x)=\dfrac{1}{x}\) en \(A(a;f(a))\) est \(-\dfrac{1}{a^2}\), puis calculer l’équation de la tangente au point \(A(4;f(4))\) et la représenter graphiquement avec \(f\).
}
\correction{
\tcblower
\textit{Generated by AI}

\textbf{Partie 1 : Calcul de la pente de la tangente}

Pour la fonction $f(x) = \dfrac{1}{x}$, calculons la dérivée :
\[
f'(x) = -\frac{1}{x^2}
\]

La pente de la tangente à la courbe au point $A(a;f(a))$ est donnée par $f'(a)$ :
\[
f'(a) = -\frac{1}{a^2}
\]

Ceci démontre que la pente de la tangente est bien $-\dfrac{1}{a^2}$.

\textbf{Partie 2 : Équation de la tangente en $A(4;f(4))$}

Calculons d'abord $f(4)$ :
\[
f(4) = \frac{1}{4}
\]

La pente de la tangente en $A\left(4;\frac{1}{4}\right)$ est :
\[
m = f'(4) = -\frac{1}{16}
\]

L'équation de la tangente est de la forme :
\[
y - f(4) = f'(4)(x - 4)
\]

Soit :
\begin{align*}
y - \frac{1}{4} &= -\frac{1}{16}(x - 4) \\
y &= -\frac{1}{16}x + \frac{4}{16} + \frac{1}{4} \\
y &= -\frac{1}{16}x + \frac{1}{4} + \frac{1}{4} \\
y &= -\frac{1}{16}x + \frac{1}{2}
\end{align*}

\textbf{Réponse :} L'équation de la tangente au point $A(4;\frac{1}{4})$ est :
\[
y = -\frac{1}{16}x + \frac{1}{2}
\]

\textit{Pour la représentation graphique, tracer la courbe $f(x) = \frac{1}{x}$ (hyperbole) et la droite $y = -\frac{1}{16}x + \frac{1}{2}$ qui touche la courbe au point $A(4;\frac{1}{4})$.}
}

