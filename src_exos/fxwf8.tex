\titre{}
\theme{dérivées}
\auteur{Nathan Scheinmann}
\niveau{3M}
\source{fundamentum}
\type{serie}
\piments{2}
\pts{}
\annee{2425}

\contenu{
\tcblower
Des mesures répétées d'une grandeur inconnue $x$ ont donné les résultats suivants: $x_1, x_2, x_3, \dots, x_n$. Montrer que la somme des carrés des écarts $(x-x_1)^2 + (x-x_2)^2 + \dots + (x-x_n)^2$ sera minimale si l'on estime $x$ par la moyenne des mesures $\dfrac{x_1+x_2+x_3+\dots+x_n}{n}$.
}
\correction{
\tcblower
%% GENERATED BY AI %%
{\scriptsize \textit{Correction générée par IA}}

Notons $S(x) = \sum_{i=1}^{n} (x - x_i)^2$ la somme des carrés des écarts.

Développons cette expression :
\begin{align*}
S(x) &= \sum_{i=1}^{n} (x - x_i)^2\\
&= \sum_{i=1}^{n} (x^2 - 2xx_i + x_i^2)\\
&= nx^2 - 2x\sum_{i=1}^{n} x_i + \sum_{i=1}^{n} x_i^2
\end{align*}

Pour minimiser $S(x)$, nous dérivons par rapport à $x$ :
\[S'(x) = 2nx - 2\sum_{i=1}^{n} x_i\]

L'annulation de la dérivée donne :
\[2nx - 2\sum_{i=1}^{n} x_i = 0\]
\[nx = \sum_{i=1}^{n} x_i\]
\[x = \frac{1}{n}\sum_{i=1}^{n} x_i = \frac{x_1 + x_2 + \cdots + x_n}{n}\]

Vérifions qu'il s'agit bien d'un minimum en calculant la dérivée seconde :
\[S''(x) = 2n > 0\]

La dérivée seconde est positive, donc nous avons bien un minimum.

\textbf{Conclusion :} La somme des carrés des écarts $(x-x_1)^2 + (x-x_2)^2 + \cdots + (x-x_n)^2$ est minimale lorsque $x$ est égal à la moyenne arithmétique des mesures :
\[\boxed{x = \frac{x_1 + x_2 + x_3 + \cdots + x_n}{n}}\]

Ce résultat justifie l'utilisation de la moyenne comme meilleur estimateur d'une grandeur à partir de mesures répétées, au sens des moindres carrés.
}

