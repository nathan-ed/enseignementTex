\titre{}
\theme{dérivées}
\auteur{Nathan Scheinmann}
\niveau{3M}
\source{fundamentum}
\type{serie}
\piments{2}
\pts{}
\annee{2425}

\contenu{
\tcblower
Déterminer les dimensions d'une boîte cylindrique sans couvercle, de volume donné, pour que sa construction demande le moins de matériau possible (on négligera l'épaisseur des parois et les déchets de construction). Même question pour une boîte avec couvercle.
}
\correction{
\tcblower
\textit{Generated by AI}
{\scriptsize \textit{Correction générée par IA}}

Notons $V$ le volume donné et $r$ le rayon de la base du cylindre, $h$ sa hauteur.

\textbf{Cas 1 : Sans couvercle}

Volume : $V = \pi r^2 h$, donc $h = \frac{V}{\pi r^2}$.

Aire de matériau : $A = \pi r^2 + 2\pi r h$ (fond + paroi latérale).

Substituons $h$ :
\[A(r) = \pi r^2 + 2\pi r \cdot \frac{V}{\pi r^2} = \pi r^2 + \frac{2V}{r}\]

Dérivons :
\[A'(r) = 2\pi r - \frac{2V}{r^2}\]

Annulation : $2\pi r = \frac{2V}{r^2} \implies \pi r^3 = V \implies r = \sqrt[3]{\frac{V}{\pi}}$

Donc $h = \frac{V}{\pi r^2} = \frac{V}{\pi \cdot \frac{V^{2/3}}{\pi^{2/3}}} = \frac{V^{1/3}}{\pi^{1/3}} = r$.

La boîte optimale sans couvercle a $\boxed{h = r = \sqrt[3]{\frac{V}{\pi}}}$ (hauteur égale au rayon).

\textbf{Cas 2 : Avec couvercle}

Aire de matériau : $A = 2\pi r^2 + 2\pi r h$ (fond + couvercle + paroi latérale).

Substituons $h = \frac{V}{\pi r^2}$ :
\[A(r) = 2\pi r^2 + 2\pi r \cdot \frac{V}{\pi r^2} = 2\pi r^2 + \frac{2V}{r}\]

Dérivons :
\[A'(r) = 4\pi r - \frac{2V}{r^2}\]

Annulation : $4\pi r = \frac{2V}{r^2} \implies 2\pi r^3 = V \implies r = \sqrt[3]{\frac{V}{2\pi}}$

Donc $h = \frac{V}{\pi r^2} = \frac{V}{\pi \cdot \frac{V^{2/3}}{(2\pi)^{2/3}}} = \frac{V^{1/3} \cdot 2^{2/3}\pi^{2/3}}{\pi} = 2^{2/3} \cdot \frac{V^{1/3}}{\pi^{1/3}} = 2^{2/3} r = \sqrt[3]{4} \cdot r \approx 1{,}587r$.

La boîte optimale avec couvercle a $\boxed{r = \sqrt[3]{\frac{V}{2\pi}}, \; h = 2r}$ (hauteur égale au diamètre).
}

