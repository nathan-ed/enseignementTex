\titre{}
\theme{limites}
\auteur{Nathan Scheinmann}
\niveau{3M}
\source{sesamath3e}
\type{serie}
\piments{2}
\pts{}
\annee{2425}

\contenu{
\tcblower
\noindent Montrer que la fonction définie par  
\[
f(x)=\frac{1}{x}
\]
est continue en \(a=1\) et qu'elle n'est pas continue en \(a=0\).
}
\correction{
\tcblower
\begin{itemize}
  \item  \( D_f=\mathbb{R} \setminus \{0\} \)
  \item En \( a = 1 \) : \( \lim_{x \to 1} \dfrac{1}{x} = \dfrac{1}{\lim_{x \to 1} x} = \dfrac{1}{1} = 1 \) (L2, P4) et \( f(1) = 1 \), donc \( \lim_{x \to 1} f(x) = f(1) \), donc \( f \) est continue en $1$.
  \item En \( a = 0 \), \( f \) n’est pas définie, donc pas continue.
\end{itemize}
}

