\titre{}
\theme{equations}
\auteur{Nathan Scheinmann}
\niveau{1M}
\source{musy-2017}
\type{serie}
\piments{1}
\pts{}
\annee{2425}

\contenu{
	\tcblower
	Pour chaque système d'équations, donner l'équation obtenue après avoir élimé une des variables. 

		\begin{center}
			{\bfseries Exemple:} \systeme*[][:]{3x-2y=5 : -3x-4y=8}
			\vspace{5pt}

			On élimine $x$ en additionnant la première équation à la deuxième équation. On obtient 
\[
	\begin{array}{cccc}
    &3x-2y&=&5\\
		+&&&\\
    &-3x-4y&=&8\\
    \hline
    &0x+2y&=&13
	\end{array}
\]
		\end{center}
\begin{tasks}(3)
	\task \systeme*[][:]{2 x+3 y=7 : 4 x-3 y=5}
	\task \systeme*[][:]{7 x-4 y=9 : 2 x+4 y=3}
	\task \systeme*[][:]{ 2 x+8 y  =-3 : -2 x+5 y  =10}
\task \systeme*[][:]{3 x+4 y=8 : -3 x+5 y=-1}
\task \systeme*[][:]{ 4 x-5 y  =9 : -3 x+5 y  =-7}
\task \systeme*[][:]{-4 x-3 y=-1 : 4 x+8 y=5}
\end{tasks}
}

\correction{
	\tcblower
On remarque que dans tous ces systèmes, une des variables de la première équation apparaît avec un coefficient opposé dans l'autre équation. On additionne donc à chaque fois la première équation à la deuxième pour éliminer une des variables. 
	\begin{tasks}(3)
\task $6x=12$
\task $9x=12$
\task $13y=7$
\task $9y=7$
\task $x=2$
\task $5y=4$
	\end{tasks}

}

