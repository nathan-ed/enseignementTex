\titre{9}
\theme{trigo}
\auteur{Nathan Scheinmann}
\niveau{1M}
\source{sesamath-1M-trigo}
\type{serie}
\piments{2}
\pts{}
\annee{2425}

\contenu{
	\tcblower
	Charlotte navigue le long d'une falaise. Pour des questions de sécurité, elle ne doit pas aller au delà du point $C$. Elle a jeté l'ancre au point $B$. On a $\overline{SH}=100~\text{m}, \widehat{HCS}=75^\circ$ et $\widehat{HBS}=65^\circ$. 
	\begin{center}
	\includegraphics[scale=1]{../medias/1M/trigo/1M-exo-9}
	\end{center}
	À quelle distance du point $C$ le bateau de Charlotte se trouve-t-il~? Doner la valeur approchée au mètre près.
}
\correction{
\tcblower
\textit{Generated by AI}

Pour trouver la distance entre le bateau de Charlotte (point $B$) et le point limite $C$, nous devons déterminer la longueur $\overline{BC}$.

\textbf{Données :}
\begin{itemize}
\item $\overline{SH} = 100$ m (hauteur de la falaise)
\item $\widehat{HCS} = 75°$
\item $\widehat{HBS} = 65°$
\end{itemize}

\textbf{Étape 1 : Calcul de $\overline{HC}$}

Dans le triangle rectangle $\triangle SHC$ (rectangle en $H$), nous avons :
\[\tan(\widehat{HCS}) = \frac{\overline{SH}}{\overline{HC}}\]

Donc :
\[\overline{HC} = \frac{\overline{SH}}{\tan(75°)} = \frac{100}{\tan(75°)}\]

Calculons :
\[\overline{HC} = \frac{100}{\tan(75°)} \approx \frac{100}{3{,}732} \approx 26{,}79~\text{m}\]

\textbf{Étape 2 : Calcul de $\overline{HB}$}

Dans le triangle rectangle $\triangle SHB$ (rectangle en $H$), nous avons :
\[\tan(\widehat{HBS}) = \frac{\overline{SH}}{\overline{HB}}\]

Donc :
\[\overline{HB} = \frac{\overline{SH}}{\tan(65°)} = \frac{100}{\tan(65°)}\]

Calculons :
\[\overline{HB} = \frac{100}{\tan(65°)} \approx \frac{100}{2{,}145} \approx 46{,}63~\text{m}\]

\textbf{Étape 3 : Calcul de $\overline{BC}$}

Puisque les points $C$, $H$ et $B$ sont alignés sur une ligne horizontale au pied de la falaise, nous avons :
\[\overline{BC} = \overline{HB} - \overline{HC} \approx 46{,}63 - 26{,}79 \approx 19{,}84~\text{m}\]

\textbf{Réponse :} Le bateau de Charlotte se trouve à environ $\boxed{20~\text{m}}$ du point $C$ (valeur arrondie au mètre près).
}

