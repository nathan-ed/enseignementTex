\titre{22}
\theme{trigo}
\auteur{Nathan Scheinmann}
\niveau{1M}
\source{sesamath-1M-trigo}
\type{serie}
\piments{2}
\pts{}
\annee{2425}

\contenu{
\tcblower
Soit $\triangle RST$ un triangle rectangle en $R$. On a $\overline{ST}=23{,}43~\text{cm}$ et $\overline{RT}=12{,}30~\text{cm}$. Calculer $\widehat{RST}$ et $\overline{RS}$.
}
\correction{
\tcblower

\textit{Generated by AI}

Dans le triangle rectangle $\triangle RST$, rectangle en $R$, nous connaissons :
\begin{itemize}
\item L'hypoténuse : $\overline{ST} = 23{,}43~\text{cm}$
\item Un côté de l'angle droit : $\overline{RT} = 12{,}30~\text{cm}$
\end{itemize}

\textbf{Calcul de $\widehat{RST}$ :}

Dans le triangle rectangle, nous pouvons utiliser le sinus :
\[\sin(\widehat{RST}) = \dfrac{\text{côté opposé}}{\text{hypoténuse}} = \dfrac{\overline{RT}}{\overline{ST}} = \dfrac{12{,}30}{23{,}43} \approx 0{,}525\]

Donc :
\[\widehat{RST} = \arcsin(0{,}525) \approx 31{,}7^\circ\]

\textbf{Calcul de $\overline{RS}$ :}

Nous pouvons utiliser le théorème de Pythagore :
\[\overline{ST}^2 = \overline{RS}^2 + \overline{RT}^2\]

Donc :
\[\overline{RS}^2 = \overline{ST}^2 - \overline{RT}^2 = (23{,}43)^2 - (12{,}30)^2\]
\[\overline{RS}^2 = 549{,}16 - 151{,}29 = 397{,}87\]
\[\overline{RS} = \sqrt{397{,}87} \approx 19{,}95~\text{cm}\]

\textbf{Vérification avec le cosinus :}
\[\cos(\widehat{RST}) = \dfrac{\overline{RS}}{\overline{ST}} = \dfrac{19{,}95}{23{,}43} \approx 0{,}851\]

Ce qui correspond bien à $\cos(31{,}7^\circ) \approx 0{,}851$.

\textbf{Réponses :}
\begin{itemize}
\item $\widehat{RST} \approx 31{,}7^\circ$
\item $\overline{RS} \approx 19{,}95~\text{cm}$
\end{itemize}

}

