\titre{}
\theme{derivation}
\auteur{Nathan Scheinmann}
\niveau{3M}
\source{sesamath3e}
\type{serie}
\piments{2}
\pts{}
\annee{2425}

\contenu{
\tcblower
Étudier les fonctions suivantes, y compris les asymptotes obliques :

\begin{tasks}(2)
\task $f(x) = \dfrac{x^2 + 3}{x + 1}$
\task $f(x) = \dfrac{x^3 - 3x^2}{2x(1 - x)}$
\task $f(x) = \dfrac{x^2 - 6x + x}{4 - x^2}$
\end{tasks}
}
\correction{
\tcblower
\textit{Generated by AI}

\begin{tasks}(1)
\task $f(x) = \dfrac{x^2 + 3}{x + 1}$

\textbf{Domaine :} $\mathbb{R} \setminus \{-1\}$

\textbf{Division polynomiale :}
\[f(x) = \frac{x^2 + 3}{x + 1} = x - 1 + \frac{4}{x+1}\]

\textbf{Asymptote verticale :} $x = -1$ (car $\lim_{x \to -1} f(x) = \pm\infty$)

\textbf{Asymptote oblique :} $y = x - 1$ (car $\lim_{x \to \pm\infty} [f(x) - (x-1)] = 0$)

\textbf{Dérivée :}
\[f'(x) = \frac{(2x)(x+1) - (x^2+3)(1)}{(x+1)^2} = \frac{2x^2 + 2x - x^2 - 3}{(x+1)^2} = \frac{x^2 + 2x - 3}{(x+1)^2} = \frac{(x+3)(x-1)}{(x+1)^2}\]

$f'(x) = 0$ pour $x = -3$ ou $x = 1$

\task $f(x) = \dfrac{x^3 - 3x^2}{2x(1 - x)} = \dfrac{x^2(x - 3)}{-2x(x - 1)} = \dfrac{x(x - 3)}{-2(x - 1)}$ (simplification par $x$)

\textbf{Domaine :} $\mathbb{R} \setminus \{0, 1\}$

\textbf{Division :}
\[f(x) = \frac{-x^2 + 3x}{2(x-1)} = \frac{-x^2 + 3x}{2x - 2}\]

En développant : $f(x) \approx -\frac{x}{2}$ pour les grandes valeurs

\textbf{Asymptotes verticales :} $x = 0$ et $x = 1$

\textbf{Asymptote oblique :} $y = -\dfrac{x}{2} - \dfrac{1}{4}$

\task $f(x) = \dfrac{x^2 - 6x + x}{4 - x^2} = \dfrac{x^2 - 5x}{4 - x^2}$

\textbf{Domaine :} $\mathbb{R} \setminus \{-2, 2\}$

\textbf{Asymptotes verticales :} $x = -2$ et $x = 2$

Pour $x \to \infty$ : $f(x) \approx \dfrac{x^2}{-x^2} = -1$

\textbf{Asymptote horizontale :} $y = -1$
\end{tasks}
}

