\titre{}
\theme{derivation}
\auteur{Nathan Scheinmann}
\niveau{3M}
\source{sesamath3e}
\type{serie}
\piments{2}
\pts{}
\annee{2425}

\contenu{
\tcblower
Comment faut-il partager 200m de clôture pour faire un enclos circulaire et un enclos carré de manière que l'aire totale soit :

\begin{tasks}(1)
\task minimale ? Quelle est alors la valeur de ce minimum ?
\task maximale ? Quelle est alors la valeur de ce maximum ?
\end{tasks}
}
\correction{
\tcblower
\textit{Generated by AI}

Soit $x$ la longueur de clôture pour l'enclos circulaire, où $0 \leq x \leq 200$. La longueur restante $200 - x$ est pour l'enclos carré.

\textbf{Enclos circulaire :}
Périmètre : $2\pi r = x \Rightarrow r = \dfrac{x}{2\pi}$

Aire : $A_{\text{cercle}} = \pi r^2 = \pi \left(\dfrac{x}{2\pi}\right)^2 = \dfrac{x^2}{4\pi}$

\textbf{Enclos carré :}
Périmètre : $4c = 200 - x \Rightarrow c = \dfrac{200 - x}{4}$

Aire : $A_{\text{carré}} = c^2 = \left(\dfrac{200 - x}{4}\right)^2 = \dfrac{(200 - x)^2}{16}$

\textbf{Aire totale :}
$$A(x) = \dfrac{x^2}{4\pi} + \dfrac{(200 - x)^2}{16}$$

Calculons la dérivée :
$$A'(x) = \dfrac{2x}{4\pi} + \dfrac{2(200 - x)(-1)}{16} = \dfrac{x}{2\pi} - \dfrac{200 - x}{8}$$

Résolvons $A'(x) = 0$ :
$$\dfrac{x}{2\pi} = \dfrac{200 - x}{8}$$
$$\dfrac{4x}{\pi} = 200 - x$$
$$4x = \pi(200 - x)$$
$$4x = 200\pi - \pi x$$
$$x(4 + \pi) = 200\pi$$
$$x = \dfrac{200\pi}{4 + \pi} \approx 87{,}96 \text{ m}$$

Vérifions : $A''(x) = \dfrac{1}{2\pi} + \dfrac{1}{8} > 0$, donc $x$ correspond à un minimum.

\begin{tasks}(1)
\task \textbf{Aire minimale :} Utiliser $x = \dfrac{200\pi}{4 + \pi} \approx 87{,}96$ m pour le cercle et $200 - x \approx 112{,}04$ m pour le carré.

$$A_{\min} = \dfrac{x^2}{4\pi} + \dfrac{(200 - x)^2}{16} \approx 1962{,}5 \text{ m}^2$$

\task \textbf{Aire maximale :} L'aire est maximale aux bornes, soit $x = 0$ (tout pour le carré) ou $x = 200$ (tout pour le cercle).

$A(0) = \dfrac{200^2}{16} = 2500$ m² et $A(200) = \dfrac{200^2}{4\pi} \approx 3183{,}1$ m²

L'aire est maximale quand $\boxed{x = 200}$ (toute la clôture pour le cercle) : $\boxed{A_{\max} \approx 3183{,}1 \text{ m}^2}$
\end{tasks}
}

