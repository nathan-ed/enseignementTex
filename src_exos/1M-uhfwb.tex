\titre{}
\theme{derivation}
\auteur{Nathan Scheinmann}
\niveau{3M}
\source{sesamath3e}
\type{serie}
\piments{2}
\pts{}
\annee{2425}

\contenu{
\tcblower
Un mobile se déplace sur un axe selon la loi \(p(t)=4t-\dfrac{t^2}{2}\), où \(p(t)\) représente la position du mobile au temps \(t\).
\begin{tasks}(1)
  \task Calculer la vitesse moyenne du mobile entre les instants :
  \begin{multicols}{2}
  \begin{enumerate}
    \item \(t_1=2\) et \(t_2=3\).
    \item \(t_1=a\) et \(t_2=t\).
    \item \(t_1=2\) et \(t_2=t\).
    \item \(t_1=t\) et \(t_2=t+h\).
    \item \(t_1=2\) et \(t_2=2+h\).
  \end{enumerate}
\end{multicols}
  \task Calculer la vitesse instantanée du mobile :
  \begin{enumerate}
    \item à l’instant \(t=2\).
    \item à l’instant \(t\).
  \end{enumerate}
\end{tasks}
}
\correction{

}

