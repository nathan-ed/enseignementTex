\titre{}
\theme{limites}
\auteur{Nathan Scheinmann}
\niveau{3M}
\source{sesamath3e}
\type{serie}
\piments{4}
\pts{}
\annee{2425}

\contenu{
\tcblower
\noindent Déterminer si la fonction \(f\) définie par  
\[
	f(x)=\begin{cases}[c]
\dfrac{|x|}{x}, & x\neq 0,\\
1, & x=0
\end{cases}
\]
est continue en \(a=0\). Justifier et esquisser une représentation graphique de \(f\).
}
\correction{
	\tcblower
\begin{itemize}
  \item Pour \( x \ne 0 \), \( f(x) = \dfrac{|x|}{x} = \begin{cases}
  1 & \text{si } x > 0 \\
  -1 & \text{si } x < 0
  \end{cases} \)
  \item Donc \( \lim_{x \to 0^-} f(x) = -1 \), \( \lim_{x \to 0^+} f(x) = 1 \)
  \item Les limites à gauche et à droite sont différentes, donc \( \lim_{x \to 0} f(x) \) n’existe pas
  \item Ainsi, \( f \) n’est pas continue en \( a = 0 \)
\end{itemize}
}

