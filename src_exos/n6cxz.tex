\titre{27}
\theme{trigo}
\auteur{Nathan Scheinmann}
\niveau{1M}
\source{sesamath-1M-trigo}
\type{serie}
\piments{2}
\pts{}
\annee{2425}

\contenu{
	\tcblower
	\begin{minipage}[t]{0.4\textwidth}{
	\vspace{0pt}
	Un ballon vole à une altitude de $700~\text{m}$ en survolant un lac. Si les angles de dénivellation des rives du lac sont $\alpha=48^\circ$ et $\beta=39^\circ$, trouver la largeur $L$ du lac.
	}
	\end{minipage}
	\hfill
	\begin{minipage}[t]{0.55\textwidth}{
	\vspace{0pt}
	\includegraphics[scale=1]{../medias/1M/trigo/1M-exo-27}
	}
	\end{minipage}
}
\correction{
\tcblower
\textit{Generated by AI}

Le ballon se trouve à une altitude de $h = 700~\text{m}$ au-dessus du lac.

Notons $A$ et $B$ les deux rives du lac, et $P$ le point directement sous le ballon.

L'angle de dénivellation est l'angle formé par l'horizontale et la ligne de visée vers le bas.

\textbf{Configuration :}

Depuis le ballon, on observe :
\begin{itemize}
\item La rive gauche avec un angle de dénivellation $\alpha = 48°$
\item La rive droite avec un angle de dénivellation $\beta = 39°$
\end{itemize}

\textbf{Calcul des distances horizontales :}

Pour la distance horizontale $d_1$ entre le ballon et la rive gauche :
\[
\tan(\alpha) = \frac{h}{d_1} \quad \Rightarrow \quad d_1 = \frac{h}{\tan(\alpha)} = \frac{700}{\tan(48°)}
\]

Pour la distance horizontale $d_2$ entre le ballon et la rive droite :
\[
\tan(\beta) = \frac{h}{d_2} \quad \Rightarrow \quad d_2 = \frac{h}{\tan(\beta)} = \frac{700}{\tan(39°)}
\]

\textbf{Largeur du lac :}

La largeur $L$ du lac est :
\[
L = d_1 + d_2 = \frac{700}{\tan(48°)} + \frac{700}{\tan(39°)}
\]

\textbf{Calcul numérique :}

\begin{align*}
d_1 &= \frac{700}{\tan(48°)} \approx \frac{700}{1{,}1106} \approx 630{,}4~\text{m} \\
d_2 &= \frac{700}{\tan(39°)} \approx \frac{700}{0{,}8098} \approx 864{,}4~\text{m}
\end{align*}

Donc :
\[
L \approx 630{,}4 + 864{,}4 = 1494{,}8~\text{m} \approx 1{,}5~\text{km}
\]

\textbf{Réponse :} La largeur du lac est d'environ $1{,}5~\text{km}$ ou $1495~\text{m}$.
}

