\titre{}
\theme{derivation}
\auteur{Nathan Scheinmann}
\niveau{3M}
\source{sesamath3e}
\type{serie}
\piments{2}
\pts{}
\annee{2425}

\contenu{
\tcblower
Soit $f$ la fonction définie par
$f(x) = \dfrac{\sin(x) + \cos(x)}{1 + \cos(x)}$. Montrer que, pour
tout $a \in \mathbb{R}$, $f'$ s'annule au moins une fois sur
l'intervalle $]a,a + 2\pi[$.
}
\correction{
\tcblower
{\scriptsize \textit{Correction générée par IA}}

On a $f(x) = \dfrac{\sin(x) + \cos(x)}{1 + \cos(x)}$.

\textbf{Périodicité~:} Puisque $\sin$ et $\cos$ sont périodiques de période $2\pi$, il suffit de montrer que $f'$ s'annule au moins une fois sur $]a, a+2\pi[$ pour tout $a \in \mathbb{R}$.

\textbf{Valeurs particulières~:} Calculons $f$ en deux points espacés de $2\pi$~:
\begin{itemize}
\item $f(0) = \dfrac{\sin(0) + \cos(0)}{1 + \cos(0)} = \dfrac{0 + 1}{1 + 1} = \dfrac{1}{2}$
\item $f(2\pi) = \dfrac{\sin(2\pi) + \cos(2\pi)}{1 + \cos(2\pi)} = \dfrac{0 + 1}{1 + 1} = \dfrac{1}{2}$
\end{itemize}

On a donc $f(0) = f(2\pi)$.

\textbf{Application du théorème de Rolle~:} La fonction $f$ est continue et dérivable sur $[0, 2\pi]$ (le dénominateur ne s'annule pas sur cet intervalle car $\cos(x) \neq -1$ pour $x \in ]0, 2\pi[$), et $f(0) = f(2\pi)$.

D'après le théorème de Rolle, il existe au moins un point $c \in ]0, 2\pi[$ tel que $f'(c) = 0$.

\textbf{Généralisation~:} Pour tout $a \in \mathbb{R}$, en considérant l'intervalle $[a, a+2\pi]$, on a de même $f(a) = f(a+2\pi)$ par périodicité. Le théorème de Rolle s'applique donc sur $[a, a+2\pi]$, ce qui montre que $f'$ s'annule au moins une fois sur $]a, a+2\pi[$.
}

