\titre{}
\theme{derivation}
\auteur{Nathan Scheinmann}
\niveau{3M}
\source{sesamath3e}
\type{serie}
\piments{2}
\pts{}
\annee{2425}

\contenu{
	{\bfseries Entraînement individuel}
\tcblower
Calculer, à partir de la définition de la fonction dérivée, la dérivée des fonctions réelles ci-dessous ; représenter \(f\) et \(f'\) dans un même repère et interpréter graphiquement le résultat (sauf pour les points d) et f))~:
\begin{tasks}(2)
  \task \(f(x)=7\).
  \task \(f(x)=x\).
  \task \(f(x)=3x\).
  \task \(f(x)=x^2\).
  \task \(f(x)=ax^2+bx+c, \,\, a,b,c\in\mathbb{R}\).
  \task \(f(x)=x^3\).
  \task \(f(x)=\dfrac{1}{x}\).
%  \task \(f(x)=\dfrac{1}{x^2+1}\).
\end{tasks}
}
\correction{
	\tcblower

\begin{tasks}
\task $f(x)=7$. 
$
f'(x)=\displaystyle\lim_{h\to 0}\dfrac{7-7}{h}=0.
$

\task $f(x)=x$. 
$
f'(x)=\displaystyle\lim_{h\to 0}\dfrac{(x+h)-x}{h}=1.
$

\task $f(x)=3x$. 
$
f'(x)=\displaystyle\lim_{h\to 0}\dfrac{3(x+h)-3x}{h}=3.
$


\task $f(x)=x^{2}\quad f'(x)=\displaystyle\lim_{h\to 0}\dfrac{(x+h)^{2}-x^{2}}{h}
=\displaystyle\lim_{h\to 0}(2x+h)=2x.$

\task $f(x)=ax^{2}+bx+c,\ a,b,c\in\mathbb{R}$.

$
\begin{aligned}
f'(x)
&=\displaystyle\lim_{h\to 0}\frac{a(x+h)^{2}+b(x+h)+c-\bigl(ax^{2}+bx+c\bigr)}{h} \\[6pt]
&=\displaystyle\lim_{h\to 0}\frac{a(x^{2}+2xh+h^{2})+bx+bh+c-ax^{2}-bx-c}{h} \\[6pt]
&=\displaystyle\lim_{h\to 0}\frac{2axh+ah^{2}+bh}{h} \\[6pt]
&=\displaystyle\lim_{h\to 0}\bigl(2ax+ah+b\bigr) \\[6pt]
&=2ax+b.
\end{aligned}
$

\task $f(x)=x^{3}$. 
$
f'(x)=\displaystyle\lim_{h\to 0}\dfrac{(x+h)^{3}-x^{3}}{h}
=\displaystyle\lim_{h\to 0}(3x^{2}+3xh+h^{2})=3x^{2}.
$
\task $f(x)=\dfrac{1}{x}$, $x\neq 0$. 
\[
f'(x)=\displaystyle\lim_{h\to 0}\dfrac{\dfrac{1}{x+h}-\dfrac{1}{x}}{h}
=\displaystyle\lim_{h\to 0}\dfrac{-1}{x(x+h)}=-\dfrac{1}{x^{2}}.
\]

%\task $f(x)=\dfrac{1}{x^{2}+1}$. 
%\[
%f'(x)=\displaystyle\lim_{h\to 0}\dfrac{\dfrac{1}{(x+h)^{2}+1}-\dfrac{1}{x^{2}+1}}{h}
%=-\dfrac{2x}{(x^{2}+1)^{2}}.
%\]
\end{tasks}

}

