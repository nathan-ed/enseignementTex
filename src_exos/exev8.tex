\titre{}
\theme{dérivées}
\auteur{Nathan Scheinmann}
\niveau{3M}
\source{fundamentum}
\type{serie}
\piments{2}
\pts{}
\annee{2425}

\contenu{
\tcblower
On désire fabriquer une tente en forme de pyramide régulière de base carrée. On dispose de S m$^2$ de toile pour fabriquer les quatre faces. On désigne par $V$ le volume de la tente, par $x$ le côté du carré de la base et par $h$ la hauteur de la tente. Montrer que $V$ est maximum lorsque $\dfrac{x}{h} = \sqrt{2}$.
}
\correction{

\tcblower
{\scriptsize \textit{Correction générée par IA}}

une pyramide régulière de base carrée avec :
\begin{itemize}
\item $x$ : côté du carré de base
\item $h$ : hauteur de la tente
\item $S$ : surface totale des quatre faces latérales
\end{itemize}

\textbf{Calcul de l'aire d'une face latérale :}

Chaque face latérale est un triangle isocèle. Notons $a$ l'apothème (hauteur d'une face latérale).


Dérivons :
\[f'(u) = S^2 - 3u^2\]

L'annulation de la dérivée donne :
\[S^2 - 3u^2 = 0 \quad \Rightarrow \quad u^2 = \dfrac{S^2}{3} \quad \Rightarrow \quad u = \dfrac{S}{\sqrt{3}}\]

Donc $x^2 = \dfrac{S}{\sqrt{3}}$, et :
\[h^2 = \dfrac{S^2 - x^4}{4x^2} = \dfrac{S^2 - \dfrac{S^2}{3}}{4 \cdot \dfrac{S}{\sqrt{3}}} = \dfrac{\dfrac{2S^2}{3}}{\dfrac{4S}{\sqrt{3}}} = \dfrac{2S^2}{3} \cdot \dfrac{\sqrt{3}}{4S} = \dfrac{S\sqrt{3}}{6}\]

Calculons le rapport :
\[\dfrac{x^2}{h^2} = \dfrac{\dfrac{S}{\sqrt{3}}}{\dfrac{S\sqrt{3}}{6}} = \dfrac{S}{\sqrt{3}} \cdot \dfrac{6}{S\sqrt{3}} = \dfrac{6}{3} = 2\]

Donc $\dfrac{x}{h} = \sqrt{2}$, ce qui est bien le résultat demandé. $\boxed{\dfrac{x}{h} = \sqrt{2}}$

}

