\titre{}
\theme{dérivées}
\auteur{Nathan Scheinmann}
\niveau{3M}
\source{fundamentum}
\type{serie}
\piments{2}
\pts{}
\annee{2425}

\contenu{
\tcblower
On colle les côtés $[AB]$ et $[DC]$ d'un rectangle $ABCD$ de périmètre donné pour former un cylindre ouvert. Quelle doit être la mesure de l'angle entre le côté $[AB]$ et la diagonale $[AC]$ pour que le volume du cylindre soit maximal ?
}
\correction{

\tcblower
\textit{Generated by AI}
{\scriptsize \textit{Correction générée par IA}}

Notons $P$ le périmètre du rectangle $ABCD$, et $\alpha$ l'angle entre le côté $[AB]$ et la diagonale $[AC]$.

Si $AB = x$ et $BC = y$, alors le périmètre est $P = 2(x + y)$, donc $x + y = \dfrac{P}{2}$ (constante).

De plus, $\tan\alpha = \dfrac{BC}{AB} = \dfrac{y}{x}$, donc $y = x\tan\alpha$.

En collant les côtés $[AB]$ et $[DC]$, on forme un cylindre :
\begin{itemize}
\item La circonférence de la base est $2\pi r = x$, donc $r = \dfrac{x}{2\pi}$
\item La hauteur du cylindre est $h = y$
\end{itemize}

Le volume du cylindre est :
\[V = \pi r^2 h = \pi \left(\dfrac{x}{2\pi}\right)^2 y = \dfrac{x^2y}{4\pi}\]

Avec la contrainte $y = x\tan\alpha$ et $x + y = \dfrac{P}{2}$, nous avons :
\[x + x\tan\alpha = \dfrac{P}{2} \quad \Rightarrow \quad x = \dfrac{P}{2(1 + \tan\alpha)}\]
\[y = \dfrac{P}{2} - x = \dfrac{P\tan\alpha}{2(1 + \tan\alpha)}\]

Le volume devient :
\[V(\alpha) = \dfrac{x^2y}{4\pi} = \dfrac{1}{4\pi} \cdot \dfrac{P^2}{4(1 + \tan\alpha)^2} \cdot \dfrac{P\tan\alpha}{2(1 + \tan\alpha)} = \dfrac{P^3\tan\alpha}{32\pi(1 + \tan\alpha)^3}\]

Pour maximiser $V$, posons $t = \tan\alpha$ et étudions :
\[f(t) = \dfrac{t}{(1 + t)^3}\]

Dérivons :
\[f'(t) = \dfrac{(1 + t)^3 - t \cdot 3(1 + t)^2}{(1 + t)^6} = \dfrac{(1 + t)^2[(1 + t) - 3t]}{(1 + t)^6} = \dfrac{1 - 2t}{(1 + t)^4}\]

L'annulation de la dérivée donne :
\[1 - 2t = 0 \quad \Rightarrow \quad t = \dfrac{1}{2}\]

Donc $\tan\alpha = \dfrac{1}{2}$, ce qui donne :
\[\alpha = \arctan\left(\dfrac{1}{2}\right) \approx 26{,}57^\circ\]

L'angle optimal est $\boxed{\alpha = \arctan\left(\dfrac{1}{2}\right) \approx 26{,}57^\circ}$.

}

