\titre{}
\theme{dérivées}
\auteur{Nathan Scheinmann}
\niveau{3M}
\source{fundamentum}
\type{serie}
\piments{2}
\pts{}
\annee{2425}

\contenu{
\tcblower
Parmi toutes les boîtes cylindriques d'aire totale donnée, caractériser celle dont le volume est maximal.
}
\correction{

\tcblower
\textit{Generated by AI}
{\scriptsize \textit{Correction générée par IA}}

Soit $r$ le rayon et $h$ la hauteur du cylindre. L'aire totale donnée est :
\[A = 2\pi r^2 + 2\pi r h = A_0\]

D'où :
\[h = \dfrac{A_0 - 2\pi r^2}{2\pi r} = \dfrac{A_0}{2\pi r} - r\]

Le volume du cylindre est :
\[V = \pi r^2 h = \pi r^2 \left(\dfrac{A_0}{2\pi r} - r\right) = \pi r^2 \cdot \dfrac{A_0}{2\pi r} - \pi r^3 = \dfrac{A_0 r}{2} - \pi r^3\]

Pour maximiser le volume, dérivons :
\[V'(r) = \dfrac{A_0}{2} - 3\pi r^2\]

Annulation :
\[\dfrac{A_0}{2} - 3\pi r^2 = 0 \implies r^2 = \dfrac{A_0}{6\pi} \implies r = \sqrt{\dfrac{A_0}{6\pi}}\]

Calculons la hauteur correspondante :
\[h = \dfrac{A_0}{2\pi r} - r = \dfrac{A_0}{2\pi \sqrt{\dfrac{A_0}{6\pi}}} - \sqrt{\dfrac{A_0}{6\pi}}\]

Simplifions le premier terme :
\[\dfrac{A_0}{2\pi \sqrt{\dfrac{A_0}{6\pi}}} = \dfrac{A_0}{2\pi} \cdot \sqrt{\dfrac{6\pi}{A_0}} = \dfrac{A_0}{2\pi} \cdot \dfrac{\sqrt{6\pi}}{\sqrt{A_0}} = \dfrac{\sqrt{A_0} \cdot \sqrt{6\pi}}{2\pi} = \dfrac{\sqrt{6\pi A_0}}{2\pi}\]

Posons $r = \sqrt{\dfrac{A_0}{6\pi}}$. Vérifions le rapport $\dfrac{h}{r}$ :

\[h = \dfrac{A_0}{2\pi r} - r\]

Multiplions par $\dfrac{1}{r}$ :
\[\dfrac{h}{r} = \dfrac{A_0}{2\pi r^2} - 1 = \dfrac{A_0}{2\pi \cdot \dfrac{A_0}{6\pi}} - 1 = \dfrac{A_0 \cdot 6\pi}{2\pi A_0} - 1 = 3 - 1 = 2\]

Donc $h = 2r$.

La boîte cylindrique de volume maximal pour une aire totale donnée a $\boxed{h = 2r}$ (la hauteur est égale au diamètre).

}

