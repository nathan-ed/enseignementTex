\titre{}
\theme{calcLit}
\auteur{Nathan Scheinmann}
\niveau{1M}
\source{musy}
\type{serie}
\piments{2}
\pts{}
\annee{2425}

\contenu{
\tcblower
Développer directement à l'aide des identités remarquables sans écrire l'étape intermédiaire.

Exemple~: $(x-3)(x+2)=x^2-x-6$.
	\begin{tasks}(4)
\task $(x-1)(x-2)$
\task $(x+3)(x+1)$
\task $(x-4)(x+4)$
\task $(y+6)(y-8)$
\task $(a+1)(a-12)$
\task $(y+9)(y-4)$
\task $(a+7)(a+3)$
\task $(x-3)(x-10)$
	\end{tasks}
}
\correction{
\tcblower
	\begin{tasks}(4)
\task $ x^2 - 3x + 2$
\task $ x^2 + 4x + 3$
\task $ x^2 - 16$
\task $ y^2 - 2y - 48$
\task $ a^2 - 11a - 12$
\task $ y^2 + 5y - 36$
\task $ a^2 + 10a + 21$
\task $ x^2 - 13x + 30$
\end{tasks}
}

