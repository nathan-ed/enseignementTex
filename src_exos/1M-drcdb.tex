\titre{}
\theme{ensemble}
\auteur{Nathan Scheinmann}
\niveau{1M}
\source{musy}
\type{serie}
\piments{2}
\pts{}
\annee{2425}

\contenu{
\tcblower
	Quel est le nombre réel situé à égale distance des bornes de l'intervalle $\interval{\sqrt{27}}{\sqrt{75}}$ ~?

Réponse sous forme simplifiée; s'il s'agit d'une racine carrée: de quel entier~?
}
\correction{
\tcblower
On a $\sqrt{27}=3\sqrt{3}$ et $\sqrt{75}=5\sqrt{3}$.

$\begin{aligned}
	\sqrt{27}+\dfrac{\sqrt{75}-\sqrt{27}}{2}&=3\sqrt{3}+\dfrac{2\sqrt{3}}{2}\\
						&=3\sqrt{3}+\sqrt{3}\\
						&=4\sqrt{3}
\end{aligned}$

On aurait pu le déduire directement depuis l'écriture simplifiée de $\sqrt{27}$ et $\sqrt{75}$. 
}

