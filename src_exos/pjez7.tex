\titre{}
\theme{dérivées}
\auteur{Nathan Scheinmann}
\niveau{3M}
\source{analysis}
\type{serie}
\piments{2}
\pts{}
\annee{2425}

\contenu{
\tcblower
Montrer que l’équation \(x^3 + 9x^2 + 33x - 8 = 0\) a exactement une racine réelle.
}
\correction{
\tcblower
\textit{Generated by AI}

Considérons la fonction $f(x) = x^3 + 9x^2 + 33x - 8$.

\textbf{Méthode : Étude de la monotonie}

Calculons la dérivée :
\[
f'(x) = 3x^2 + 18x + 33
\]

Étudions le signe de $f'(x)$ en calculant le discriminant :
\[
\Delta = 18^2 - 4 \cdot 3 \cdot 33 = 324 - 396 = -72 < 0
\]

Comme $\Delta < 0$ et que le coefficient dominant est positif ($3 > 0$), on a $f'(x) > 0$ pour tout $x \in \mathbb{R}$.

Donc $f$ est strictement croissante sur $\mathbb{R}$.

\textbf{Conséquence :}

Une fonction strictement monotone sur $\mathbb{R}$ coupe l'axe des abscisses au plus une fois. Donc $f$ admet au plus une racine réelle.

\textbf{Existence d'une racine :}

Calculons les limites :
\begin{align*}
\lim_{x \to -\infty} f(x) &= -\infty \\
\lim_{x \to +\infty} f(x) &= +\infty
\end{align*}

De plus, $f$ est continue sur $\mathbb{R}$ (polynôme).

Par le théorème des valeurs intermédiaires, comme $f$ est continue et strictement croissante avec $f(x)$ passant de $-\infty$ à $+\infty$, il existe un unique $c \in \mathbb{R}$ tel que $f(c) = 0$.

\textbf{Vérification numérique :}

\begin{align*}
f(0) &= -8 < 0 \\
f(1) &= 1 + 9 + 33 - 8 = 35 > 0
\end{align*}

Donc la racine se trouve dans l'intervalle $]0 ; 1[$.

\textbf{Conclusion :} L'équation $x^3 + 9x^2 + 33x - 8 = 0$ a exactement une racine réelle (située entre 0 et 1).
}

