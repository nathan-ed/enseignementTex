\titre{}
\theme{derivation}
\auteur{Nathan Scheinmann}
\niveau{3M}
\source{sesamath3e}
\type{serie}
\piments{2}
\pts{}
\annee{2425}

\contenu{
\tcblower
Vrai ou faux ? Justifier.

\begin{tasks}
\task Il n'existe pas de fonction à la fois croissante et décroissante sur un intervalle $I$.
\task Si $f$ est nulle sur un intervalle ouvert $I$, alors $f'(x) > 0$ sur $I$.
\task Si $f$ est strictement croissante sur un intervalle ouvert $I$, alors $f'(x) > 0$ sur $I$.
\task Si $f$ est strictement décroissante et dérivable sur un intervalle ouvert $I$, alors $f'(x) < 0$ sur $I$.
\end{tasks}
}
\correction{

}

