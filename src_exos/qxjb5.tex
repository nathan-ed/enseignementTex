\titre{}
\theme{fonctions}
\auteur{Nathan Scheinmann}
\niveau{1M}
\source{ns}
\type{serie}
\piments{2}
\pts{}
\annee{2425}

\contenu{
	\tcblower
Déterminer la pente et l'ordonnée à l'origine des fonctions affines suivantes. Puis la représenter graphiquement.
		\begin{alignat*}{3}
			&f_1: x \mapsto-2 x+3& \quad\quad\quad\quad  & f_4: x \mapsto-2 &\quad\quad\quad\quad & f_7: x \mapsto 3 x \\
			&f_2: x \mapsto 5-x&  & f_5: x \mapsto x & &f_8: x \mapsto 0 \\
			&f_3: x \mapsto \frac{5 x}{4} & & f_6: x \mapsto \frac{2 x-3}{4}  & & f_9: x \mapsto \frac{2-x}{3}
\end{alignat*}
}
\correction{
	\tcblower
\begin{tasks}(3)
		\task[] $f_1:$ o.o. $3$ et pente $-2$
		\task[] $f_2:$ o.o. $5$ et pente $-1$
		\task[] $f_3:$ o.o. $0$ et pente $\dfrac{5}{4}$
		\task[] $f_4:$ o.o. $-2$ et pente $0$
		\task[] $f_5:$ o.o. $0$ et pente $1$
		\task[] $f_6:$ o.o. $-\dfrac{3}{4}$ et pente $\dfrac{1}{2}$
		\task[] $f_7:$ o.o. $0$ et pente $3$
		\task[] $f_8:$ o.o. $0$ et pente $0$
		\task[] $f_9:$ o.o. $\dfrac{2}{3}$ et pente $-\dfrac{1}{3}$
\end{tasks}
}

