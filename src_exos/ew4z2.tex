\titre{}
\theme{ensemble}
\auteur{Nathan Scheinmann}
\niveau{1M}
\source{musy}
\type{serie}
\piments{2}
\pts{}
\annee{2425}

\contenu{
\tcblower
Dans l'ensemble $T$ des triangles, on considère $I$, le sous-ensemble des triangles isocèles\,; $E$, le sous-ensemble des triangles équilatéraux\,; $R$, le sous-ensemble des triangles rectangles
\begin{tasks}
\task Représenter ces quatre ensembles à l'aide d'un diagramme.
\task Décrire par des mots les ensembles $I \cap E, R \cap E$ et $I \cap R$.
\end{tasks}
}
\correction{
\tcblower

\def\trianglecircle{(0,0) ellipse (3cm and 2cm)}
\def\rectcircle{(-1,0) ellipse (1.5cm and 1.25cm)}
\def\isocircle{(0:1.2cm) circle (1cm)}
\def\equicircle{(0:1.5cm) circle (0.5cm)}

\begin{tasks}(2)
	\task La taille des diagrammes n'est pas représentative de la \enquote{taille} des ensembles.	

\begin{center}
\begin{tikzpicture}
        % Draw the shapes
        \draw \trianglecircle node[above, yshift=2cm] {$T$};
        \draw \rectcircle node[above, yshift=1.3cm,xshift=0.3cm] {$R$};
        \draw \isocircle node[above, yshift=1cm] {$I$};
        \draw \equicircle node[] {$E$};

        % Find and mark the intersection between isocircle and rectcircle
        %\path [name path=isocircle] \isocircle;
        %\path [name path=rectcircle] \rectcircle;
        %\path [name intersections={of=isocircle and rectcircle, by=intersection}];
        %\fill[red] (intersection) circle (2pt);
    \end{tikzpicture}
\end{center}
\task 
\begin{itemize}
	\item $I\cap E=E$, car l'ensemble des triangles \\ équilatéraux est contenu dans l'ensemble de triangles isocèles. 
	\item $R\cap E=\emptyset$, car il n'existe aucun triangle qui est équilatéral et rectangle (par le théorème de Pythagore, si $a\in \R^*_+$ est la longueur du côté du triangle, alors $a^2+a^2\neq a^2$).
	\item $I\cap R$ est l'ensemble des triangles dont les deux cathètes mesure $a\in \R^*_+$ et l'hypoténuse mesure $a\sqrt{2}$ (par Pythagore).
\end{itemize}
\end{tasks}

}

