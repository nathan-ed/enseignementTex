\titre{}
\theme{derivation}
\auteur{Nathan Scheinmann}
\niveau{3M}
\source{crm}
\type{serie}
\piments{1}
\pts{}
\annee{2526}

\contenu{
\tcblower
Soit la fonction $f(x)=x^2$.
\begin{tasks}
\task Déterminer l'équation de la tangente au graphe de $f$ au point $(2;4)$.
\task La droite d'équation $y=4x-3$ est-elle tangente au graphe de $f$ au point d'abscisse $2$ ? Justifier.
\end{tasks}
}
\correction{
\tcblower
\begin{tasks}
\task On a $f(x)=x^2$, donc $f'(x)=2x$.

Pour $x=2$, on a $f'(2)=2\cdot 2=4$.

L'équation de la tangente au point $(2,4)$ est :
\[
y=f'(2)(x-2)+f(2)=4(x-2)+4=4x-8+4=4x-4
\]

\task La tangente au point d'abscisse $2$ a pour équation $y=4x-4$ (d'après la question précédente).

La droite d'équation $y=4x-3$ n'est donc \textbf{pas} tangente au graphe de $f$ au point d'abscisse $2$, car les équations sont différentes.

On peut aussi vérifier que le point $(2,3)$ n'appartient pas au graphe de $f$, car $f(2)=4\neq 3$.
\end{tasks}
}

