\titre{}
\theme{dérivées}
\auteur{Nathan Scheinmann}
\niveau{3M}
\source{fundamentum}
\type{serie}
\piments{2}
\pts{}
\annee{2425}

\contenu{
\tcblower
On fait tourner un rectangle de périmètre $2p$ autour de l'un de ses axes de symétrie. Déterminer les dimensions du rectangle pour que le corps ainsi obtenu ait
\begin{tasks}
    \task le plus grand volume;
    \task la plus grande aire latérale;
    \task la plus grande aire totale.
\end{tasks}
}
\correction{

\tcblower
\textit{Generated by AI}
{\scriptsize \textit{Correction générée par IA}}

Notons $a$ et $b$ les dimensions du rectangle. Le périmètre vaut $2a + 2b = 2p$, d'où $a + b = p$.

\begin{tasks}
\task \textbf{Volume maximal :}

En faisant tourner autour de l'axe de longueur $a$, nous obtenons un cylindre de rayon $r = \dfrac{b}{2}$ et de hauteur $h = a$.

Volume : $V = \pi r^2 h = \pi \left(\dfrac{b}{2}\right)^2 a = \dfrac{\pi}{4}ab^2$

Avec $a = p - b$, nous avons :
\[V(b) = \dfrac{\pi}{4}(p-b)b^2 = \dfrac{\pi}{4}(pb^2 - b^3)\]

Dérivons : $V'(b) = \dfrac{\pi}{4}(2pb - 3b^2) = \dfrac{\pi b}{4}(2p - 3b)$

Annulation : $2p - 3b = 0 \implies b = \dfrac{2p}{3}$, donc $a = \dfrac{p}{3}$.

Le rectangle de dimensions $\boxed{a = \dfrac{p}{3}, \; b = \dfrac{2p}{3}}$ donne le volume maximal.

\task \textbf{Aire latérale maximale :}

Aire latérale du cylindre : $A_{\text{lat}} = 2\pi r h = 2\pi \cdot \dfrac{b}{2} \cdot a = \pi ab$

Avec $a = p - b$ : $A(b) = \pi b(p-b) = \pi(pb - b^2)$

Dérivons : $A'(b) = \pi(p - 2b)$

Annulation : $p - 2b = 0 \implies b = \dfrac{p}{2}$, donc $a = \dfrac{p}{2}$.

Le rectangle de dimensions $\boxed{a = b = \dfrac{p}{2}}$ (carré) donne l'aire latérale maximale.

\task \textbf{Aire totale maximale :}

Aire totale : $A_{\text{tot}} = 2\pi r^2 + 2\pi r h = 2\pi\left(\dfrac{b}{2}\right)^2 + \pi ab = \dfrac{\pi b^2}{2} + \pi ab$

Avec $a = p - b$ : $A(b) = \dfrac{\pi b^2}{2} + \pi b(p-b) = \dfrac{\pi b^2}{2} + \pi pb - \pi b^2 = \pi pb - \dfrac{\pi b^2}{2}$

Dérivons : $A'(b) = \pi p - \pi b$

Annulation : $p - b = 0 \implies b = p$, mais cela n'est pas possible (car $a > 0$).

Le maximum est donc à la frontière, soit $\boxed{a \to 0, \; b \to p}$ (rectangle très fin).
\end{tasks}

}

