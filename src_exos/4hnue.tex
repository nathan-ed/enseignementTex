\titre{1}
\theme{trigo}
\auteur{Nathan Scheinmann}
\niveau{1M}
\source{sesamath-1M-trigo}
\type{serie}
\piments{2}
\pts{}
\annee{2425}

\contenu{
\tcblower
Dans chaque cas, calculer la valeur arrondie au millième des longueurs manquantes.
\begin{tasks}(3)
	\task 
\includegraphics[scale=1]{../medias/1M/trigo/1M-exo-1-1}	
	\task 
\includegraphics[scale=1]{../medias/1M/trigo/1M-exo-1-2}	
	\task 
\includegraphics[scale=1]{../medias/1M/trigo/1M-exo-1-3}	
\end{tasks}
}
\correction{
\tcblower
\textit{Generated by AI}

\begin{tasks}(1)
\task \textbf{Triangle LOS :}

Dans ce triangle rectangle en $O$, nous connaissons l'hypoténuse $LS = 5{,}5$ cm et l'angle $\widehat{LSO} = 27°$. Nous cherchons le côté opposé $LO$.

Utilisons la relation trigonométrique :
\[\sin(27°) = \frac{LO}{LS} = \frac{LO}{5{,}5}\]

D'où :
\[LO = 5{,}5 \times \sin(27°) = 5{,}5 \times 0{,}453990... \approx 2{,}497\text{ cm}\]

\textbf{Réponse :} $\boxed{LO \approx 2{,}497\text{ cm}}$

\task \textbf{Triangle OSS :}

Dans ce triangle rectangle en $S$, nous connaissons l'angle $\widehat{O} = 56°$ et le côté adjacent $SS = 7$ cm. Nous cherchons l'hypoténuse $OS$.

Utilisons la relation trigonométrique :
\[\cos(56°) = \frac{SS}{OS} = \frac{7}{OS}\]

D'où :
\[OS = \frac{7}{\cos(56°)} = \frac{7}{0{,}559192...} \approx 12{,}521\text{ cm}\]

\textbf{Réponse :} $\boxed{OS \approx 12{,}521\text{ cm}}$

\task \textbf{Triangle LOO :}

Dans ce triangle rectangle en $L$, nous connaissons l'angle $\widehat{O} = 83°$ et le côté adjacent $LS = 5$ cm. Nous cherchons l'hypoténuse $LO$.

Utilisons la relation trigonométrique :
\[\cos(83°) = \frac{LS}{LO} = \frac{5}{LO}\]

D'où :
\[LO = \frac{5}{\cos(83°)} = \frac{5}{0{,}121869...} \approx 41{,}025\text{ cm}\]

\textbf{Réponse :} $\boxed{LO \approx 41{,}025\text{ cm}}$
\end{tasks}
}
