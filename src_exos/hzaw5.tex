\titre{}
\theme{derivation}
\auteur{Nathan Scheinmann}
\niveau{3M}
\source{sesamath3e}
\type{serie}
\piments{2}
\pts{}
\annee{2425}

\contenu{
\tcblower
Soit $P = (3;2)$, $A = (x;0)$ avec $x > 0$ et $B = (0;y)$ avec $y > 0$. Déterminer l'équation d'une droite passant par $P$ et telle que le triangle $\triangle OAB$ ait une aire :

\begin{tasks}(2)
\task minimale
\task maximale
\end{tasks}
}
\correction{
\tcblower
\textit{Generated by AI}

La droite passant par $P = (3;2)$ coupe l'axe des abscisses en $A = (x;0)$ avec $x > 0$ et l'axe des ordonnées en $B = (0;y)$ avec $y > 0$.

L'équation de la droite passant par $A$ et $B$ est :
\[
\frac{X}{x} + \frac{Y}{y} = 1
\]

Comme cette droite passe par $P = (3;2)$, on a :
\[
\frac{3}{x} + \frac{2}{y} = 1 \quad \Rightarrow \quad y = \frac{2x}{x-3}
\]

Cette relation est valide pour $x > 3$ (car $y > 0$).

L'aire du triangle $\triangle OAB$ est :
\[
\mathcal{A}(x) = \frac{1}{2} \cdot x \cdot y = \frac{1}{2} \cdot x \cdot \frac{2x}{x-3} = \frac{x^2}{x-3}
\]

\textbf{Question 1 : Aire minimale}

Calculons la dérivée :
\[
\mathcal{A}'(x) = \frac{2x(x-3) - x^2}{(x-3)^2} = \frac{2x^2 - 6x - x^2}{(x-3)^2} = \frac{x^2 - 6x}{(x-3)^2} = \frac{x(x-6)}{(x-3)^2}
\]

Étudions le signe de $\mathcal{A}'(x)$ pour $x > 3$ :
\begin{itemize}
\item Si $3 < x < 6$ : $\mathcal{A}'(x) < 0$ donc $\mathcal{A}$ est décroissante
\item Si $x > 6$ : $\mathcal{A}'(x) > 0$ donc $\mathcal{A}$ est croissante
\end{itemize}

L'aire est minimale en $x = 6$, ce qui donne $y = \frac{2 \cdot 6}{6-3} = \frac{12}{3} = 4$.

L'aire minimale est : $\mathcal{A}(6) = \frac{36}{3} = 12$

L'équation de la droite est : $\dfrac{X}{6} + \dfrac{Y}{4} = 1$ soit $2X + 3Y = 12$.

\textbf{Question 2 : Aire maximale}

Quand $x \to 3^+$ : $y \to +\infty$ donc $\mathcal{A} \to +\infty$

Quand $x \to +\infty$ : $y \to 2$ donc $\mathcal{A} \to +\infty$

Il n'existe pas d'aire maximale.

\textbf{Réponses :}
\begin{itemize}
\item L'équation de la droite donnant l'aire minimale est $2x + 3y = 12$ avec une aire de 12
\item Il n'existe pas d'aire maximale
\end{itemize}
}

