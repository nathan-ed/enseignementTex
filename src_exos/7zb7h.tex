\titre{}
\theme{derivation}
\auteur{Nathan Scheinmann}
\niveau{3M}
\source{sesamath3e}
\type{serie}
\piments{2}
\pts{}
\annee{2425}

\contenu{
\tcblower
Déterminer la distance minimale du point $A = (4 ; 2)$ à la parabole $y^2 = 8x$.
}
\correction{
\tcblower
Soit $A(4,2)$ et la parabole $y^2=8x$.
 Pour $M\!\left(\dfrac{y^2}{8},y\right)$ un point de la parabole sa distance au carré à $A$ est
$d^2=\!\left(\dfrac{y^2}{8}-4\right)^2+(y-2)^2$.
On dérive en fonction de $y$, puis on cherche les zéros.
Le minimum est atteint pour $y=4, x=2$, donc la distance minimale est $\boxed{d_{\min}=2\sqrt{2}}$.
}

