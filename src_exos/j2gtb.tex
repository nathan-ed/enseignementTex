\titre{29}
\theme{trigo}
\auteur{Nathan Scheinmann}
\niveau{1M}
\source{sesamath-1M-trigo}
\type{serie}
\piments{2}
\pts{}
\annee{2425}

\contenu{
\tcblower
\begin{minipage}[t]{0.4\textwidth}{
\vspace{0pt}
On considère le triangle $\widehat{MST}$ tel que $\overline{MS}=23~\text{cm}$ et $\overline{TM}=15~\text{cm}$. Les droites $AH$ et $MS$ sont parallèles.
}
\end{minipage}
\hfill
\begin{minipage}[t]{0.55\textwidth}{
\vspace{0pt}
\includegraphics[scale=1]{../medias/1M/trigo/1M-exo-29}
}
\end{minipage}
\begin{tasks}(1)
	\task Écrire les rapports de longueurs qui sont égaux en justifiant.
	\task Écrire la relation donnant le sinus de l'angle $\widehat{AHT}$. 
	\task Déduire des questions précédentes la mesure arrondie au degré près de l'angle $\widehat{AHT}$.
\end{tasks}
}
\correction{
\tcblower
\textit{Generated by AI}

\textbf{Question 1 : Rapports de longueurs égaux}

Les droites $AH$ et $MS$ étant parallèles, le théorème de Thalès s'applique dans le triangle $\triangle MST$.

On a donc :
\[
\frac{\overline{TA}}{\overline{TM}} = \frac{\overline{TH}}{\overline{TS}} = \frac{\overline{AH}}{\overline{MS}}
\]

\textbf{Question 2 : Relation donnant le sinus de l'angle $\widehat{AHT}$}

Dans le triangle rectangle $\triangle AHT$ (rectangle en $H$), on a :
\[
\sin(\widehat{AHT}) = \frac{\overline{AT}}{\overline{AH}}
\]

Ou bien, si l'angle considéré est $\widehat{TAH}$ :
\[
\sin(\widehat{TAH}) = \frac{\overline{TH}}{\overline{AT}}
\]

\textbf{Question 3 : Mesure de l'angle $\widehat{AHT}$}

Par le théorème de Thalès et les droites parallèles, les angles correspondants sont égaux.

Dans le triangle $\triangle MST$, on peut calculer :
\[
\sin(\widehat{MST}) = \frac{\overline{TM}}{\overline{MS}} = \frac{15}{23}
\]

Comme $AH \parallel MS$, on a $\widehat{AHT} = \widehat{MST}$ (angles correspondants).

Donc :
\[
\sin(\widehat{AHT}) = \frac{15}{23}
\]

Ce qui donne :
\[
\widehat{AHT} = \arcsin\left(\frac{15}{23}\right) \approx 41°
\]

\textbf{Réponse :} La mesure de l'angle $\widehat{AHT}$ est d'environ $41°$.
}

