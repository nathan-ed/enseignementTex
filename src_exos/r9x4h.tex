\titre{}
\theme{derivation2}
\auteur{Nathan Scheinmann}
\niveau{3M}
\source{sesamath3e}
\type{serie}
\piments{2}
\pts{}
\annee{2425}

\contenu{
\tcblower
Vrai ou faux ? Justifier.
\begin{tasks}(2)
  \task Il n’existe pas de fonction à la fois croissante et décroissante sur un intervalle \(I\).
  \task Si \(f\) est nulle sur un intervalle ouvert \(I\), alors \(f'(x)>0\) sur \(I\).
  \task Si \(f\) est strictement croissante sur un intervalle ouvert \(I\), alors \(f'(x)>0\) sur \(I\).
  \task Si \(f\) est strictement décroissante et dérivable sur un intervalle ouvert \(I\), alors \(f'(x)<0\) sur \(I\).
\end{tasks}
}
\correction{
\tcblower
\textit{Generated by AI}

\textbf{Question a) : Il n'existe pas de fonction à la fois croissante et décroissante sur un intervalle $I$.}

\textbf{FAUX.} La fonction constante est à la fois croissante (au sens large) et décroissante (au sens large) sur tout intervalle.

Si on parle de fonctions \textit{strictement} croissantes et \textit{strictement} décroissantes, alors c'est vrai : une fonction ne peut pas être à la fois strictement croissante et strictement décroissante.

\textbf{Question b) : Si $f$ est nulle sur un intervalle ouvert $I$, alors $f'(x) > 0$ sur $I$.}

\textbf{FAUX.} Si $f$ est nulle sur $I$, c'est-à-dire $f(x) = 0$ pour tout $x \in I$, alors $f$ est constante et donc $f'(x) = 0$ sur $I$ (et non $f'(x) > 0$).

\textbf{Question c) : Si $f$ est strictement croissante sur un intervalle ouvert $I$, alors $f'(x) > 0$ sur $I$.}

\textbf{FAUX.} Une fonction peut être strictement croissante sans que sa dérivée soit strictement positive partout.

Contre-exemple : $f(x) = x^3$ sur $\mathbb{R}$. La fonction est strictement croissante, mais $f'(0) = 0$.

Ce qui est vrai : si $f'(x) > 0$ pour tout $x \in I$, alors $f$ est strictement croissante. Mais la réciproque est fausse.

Le bon énoncé serait : si $f$ est strictement croissante sur $I$, alors $f'(x) \geq 0$ sur $I$ (avec égalité possible en certains points isolés).

\textbf{Question d) : Si $f$ est strictement décroissante et dérivable sur un intervalle ouvert $I$, alors $f'(x) < 0$ sur $I$.}

\textbf{FAUX.} Par le même raisonnement que la question c), une fonction strictement décroissante peut avoir une dérivée nulle en certains points.

Contre-exemple : $f(x) = -x^3$ sur $\mathbb{R}$. La fonction est strictement décroissante, mais $f'(0) = 0$.

Le bon énoncé : si $f$ est strictement décroissante et dérivable sur $I$, alors $f'(x) \leq 0$ sur $I$ (avec égalité possible en certains points isolés).
}

