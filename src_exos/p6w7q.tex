\titre{}
\theme{ensembles}
\auteur{Nathan Scheinmann}
\niveau{1M}
\source{crm}
\type{serie}
\piments{2}
\pts{}
\annee{2425}

\contenu{
\tcblower
 Décrire les ensembles suivants à l'aide d'intervalles.
\begin{tasks}(2)
\task $A = \{x \in \mathbb{R} \mid -3 \leq x \leq 5\}$
\task $B = \{x \in \mathbb{R} \mid 4 < x < 5\}$
\task $C = \{x \in \mathbb{R} \mid x < 1\}$
\task $D = \{x \in \mathbb{R} \mid x \geq 10\}$
\task $E = \{x \in \mathbb{R} \mid x \geq -2 \text{ et } x \leq 2\}$
\task $F = \mathbb{R}$
\task $G = \{2\}$
\end{tasks}
}
\correction{
\tcblower
\begin{tasks}(3)
\task $A = [-3; 5]$
\task $B = ]4; 5[$
\task $C = ]-\infty; 1[$
\task $D = [10; +\infty[$
\task $E = [-2; 2]$
\task $F = ]-\infty; +\infty[$
\task* Un intervalle contient une infinité de nombre, donc pas possible.
\end{tasks}
}

