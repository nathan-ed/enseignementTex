\titre{}
\theme{limits}
\auteur{Nathan Scheinmann}
\niveau{3M}
\source{nathan}
\type{serie}
\piments{1}
\pts{}
\annee{2526}

\contenu{
\tcblower
Application du théorème des deux gendarmes.
\begin{tasks}
	\task Montrer que si $\lim_{x\to a}|f(x)|=0$ alors $\lim_{x\to a}f(x)=0$.
	\task Montrer que $\lim_{x\to 0} x\sin\left(\dfrac{1}{x}\right)=0$. 
	\task En déduire que la fonction $f(x)=
\begin{cases}
x\sin\left(\tfrac{1}{x}\right),&x\neq 0,\\
0,&x=0
\end{cases}$ est continue en $x=0$. 
\end{tasks}
}
\correction{
\tcblower
\begin{tasks}
	\task On a que $-|f(x)|\leq f(x)\leq |f(x)|, \forall x\in \R$, donc par le théorème des deux gendarmes, $\lim_{x\to a}f(x)=0$. 
	\task On a $-1\leq \sin\left(\dfrac{1}{x}\right)\leq 1$ $,\forall x \in \R^*$ et donc $0\leq |\sin\left(\dfrac{1}{x}\right)|\leq 1$. On multiplie l'inégalité par $|x|$ et on obtient 

	$\begin{aligned}
		0\leq |x|\cdot |\sin\left(\dfrac{1}{x}\right)|\leq |x|\\
		0\leq |x\cdot \sin\left(\dfrac{1}{x}\right)|\leq |x|\\
	\end{aligned}
		$

		Par le théorème des deux gendarme, $\lim_{x\to 0}|\sin\left(\dfrac{1}{x}\right)|=0$ et par le point précédent $\lim_{x\to 0}\sin\left(\dfrac{1}{x}\right)=0$
	\task On a bien $\lim_{x\to 0}f(x)=f(0)$, donc la fonction $f$ est continue en $0$.
\end{tasks}
}

