\titre{}
\theme{dérivées}
\auteur{Nathan Scheinmann}
\niveau{3M}
\source{analysis}
\type{serie}
\piments{2}
\pts{}
\annee{2425}

\contenu{
\tcblower
Montrer que l’équation \(6x^5 + 13x + 1 = 0\) a exactement une racine réelle. (Utiliser le théorème de Rolle et le théorème de la valeur intermédiaire.)
}
\correction{
\tcblower
\textit{Generated by AI}

Soit $f(x) = 6x^5 + 13x + 1$.

\textbf{Existence d'au moins une racine (Théorème de la valeur intermédiaire) :}

$f$ est un polynôme, donc continue sur $\mathbb{R}$.

Calculons quelques valeurs :
\begin{itemize}
\item $f(0) = 1 > 0$
\item $f(-1) = -6 - 13 + 1 = -18 < 0$
\end{itemize}

Puisque $f$ est continue sur $[-1, 0]$ et que $f(-1) < 0 < f(0)$, le théorème de la valeur intermédiaire garantit l'existence d'au moins un $c \in ]-1, 0[$ tel que $f(c) = 0$.

\textbf{Unicité de la racine (Théorème de Rolle) :}

Supposons par l'absurde qu'il existe deux racines distinctes $a$ et $b$ avec $a < b$.

Alors $f(a) = f(b) = 0$.

Le théorème de Rolle affirme que si $f$ est continue sur $[a,b]$, dérivable sur $]a,b[$, et si $f(a) = f(b)$, alors il existe $c \in ]a,b[$ tel que $f'(c) = 0$.

Calculons $f'(x)$ :
\[f'(x) = 30x^4 + 13\]

Or, pour tout $x \in \mathbb{R}$ :
\[f'(x) = 30x^4 + 13 \geq 13 > 0\]

La dérivée est toujours strictement positive, donc $f'(c) \neq 0$ pour tout $c$. Ceci contredit le théorème de Rolle.

Donc il ne peut pas exister deux racines distinctes.

\textbf{Conclusion :} L'équation $6x^5 + 13x + 1 = 0$ possède exactement une racine réelle (située dans $]-1, 0[$).
}

