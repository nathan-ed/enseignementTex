\titre{}
\theme{calcLit}
\auteur{Nathan Scheinmann}
\niveau{1M}
\source{musy}
\type{serie}
\piments{2}
\pts{}
\annee{2425}

\contenu{
\tcblower
Soit le polynôme $x^6-1$.
	\begin{tasks}
\task Le factoriser de deux manières différentes (indications: $x^6=\left(x^3\right)^2=\left(x^2\right)^3$ et utiliser l'activité 1).
\task En déduire une factorisation pour le polynôme $x^4+x^2+1$.
	\end{tasks}
}
\correction{
\tcblower
\textit{Generated by AI}
{\scriptsize \textit{Correction générée par IA}}

\begin{tasks}
\task Nous avons $x^6 - 1$.

\textbf{Première factorisation :} Utilisons $x^6 = (x^3)^2$, ce qui donne une différence de deux carrés :
\[x^6 - 1 = (x^3)^2 - 1^2 = (x^3 - 1)(x^3 + 1)\]

Nous pouvons ensuite factoriser chaque terme en utilisant les identités remarquables :
\begin{align*}
x^3 - 1 &= (x-1)(x^2 + x + 1)\\
x^3 + 1 &= (x+1)(x^2 - x + 1)
\end{align*}

D'où :
\[\boxed{x^6 - 1 = (x-1)(x+1)(x^2 + x + 1)(x^2 - x + 1)}\]

\textbf{Deuxième factorisation :} Utilisons $x^6 = (x^2)^3$, ce qui donne une différence de deux cubes :
\[x^6 - 1 = (x^2)^3 - 1^3 = (x^2 - 1)((x^2)^2 + x^2 + 1) = (x^2 - 1)(x^4 + x^2 + 1)\]

Nous pouvons factoriser $x^2 - 1$ :
\[x^2 - 1 = (x-1)(x+1)\]

D'où :
\[\boxed{x^6 - 1 = (x-1)(x+1)(x^4 + x^2 + 1)}\]

\task En comparant les deux factorisations, nous avons :
\[(x-1)(x+1)(x^2 + x + 1)(x^2 - x + 1) = (x-1)(x+1)(x^4 + x^2 + 1)\]

En simplifiant par $(x-1)(x+1)$, nous obtenons :
\[(x^2 + x + 1)(x^2 - x + 1) = x^4 + x^2 + 1\]

Développons le membre de gauche pour vérifier :
\begin{align*}
(x^2 + x + 1)(x^2 - x + 1) &= x^4 - x^3 + x^2 + x^3 - x^2 + x + x^2 - x + 1\\
&= x^4 + x^2 + 1
\end{align*}

Donc :
\[\boxed{x^4 + x^2 + 1 = (x^2 + x + 1)(x^2 - x + 1)}\]
\end{tasks}
}

