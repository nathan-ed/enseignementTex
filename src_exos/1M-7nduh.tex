\titre{}
\theme{fonctions}
\auteur{Nathan Scheinmann}
\niveau{1M}
\source{}
\type{serie-form}
\piments{1}
\pts{}
\annee{2425}

\contenu{
	\tcblower
Déterminer le domaine de définition de la fonction 
\[f:x\mapsto \dfrac{\sqrt{x^2+2x-2}}{x^2-4x+1}.\]
}
\correction{
	\tcblower
Soit la fonction :
\[
f(x) = \frac{\sqrt{x^2+2x-2}}{x^2-4x+1}
\]
Les conditions de définition sont :
\[
x^2 + 2x - 2 \geq 0 \quad \text{et} \quad x^2 - 4x + 1 \neq 0.
\]
Les racines du numérateur sont :
\[
x = -1 \pm \sqrt{3}
\]
et \( f(x) \) est défini pour :
\[
x \in ]-\infty, -1-\sqrt{3}] \cup [ -1+\sqrt{3}, +\infty[.
\]
Les racines du dénominateur sont :
\[
x = 2 \pm \sqrt{3}.
\]
Seule \( x = 2 - \sqrt{3} \) est dans le domaine, donc on l'exclut.
Le domaine final est :
\[
D_f = ]-\infty, -1-\sqrt{3}] \cup [ -1+\sqrt{3}, 2-\sqrt{3}[ \cup ]2-\sqrt{3}, +\infty[.
\]
}
