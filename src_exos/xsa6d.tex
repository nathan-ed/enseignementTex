\titre{11}
\theme{trigo}
\auteur{Nathan Scheinmann}
\niveau{1M}
\source{sesamath-1M-trigo}
\type{serie}
\piments{2}
\pts{}
\annee{2425}

\contenu{
\tcblower
\begin{minipage}[t]{0.5\textwidth}{
\vspace{0pt}
$SABCD$ est une pyramide régulière dont la base est le carré $ABCD$ de côté $230~\text{m}$ et de centre $I$. La hauteur $[SI]$ de la pyramide a pour longueur $\overline{SI}=147~\text{m}$. $M$ est le milieu de $[BC]$. 
}
\end{minipage}
\hfill
\begin{minipage}[t]{0.45\textwidth}{
\vspace{0pt}
\includegraphics[scale=1.2]{../medias/1M/trigo/1M-exo-11}
}
\end{minipage}
\begin{tasks}
	\task Calculer le volume de la pyramide. 
	\task Calculer les mesures des angles $\widehat{IAS}$ et $\widehat{SMI}$ arrondies au degré près. 
\end{tasks}
}
\correction{
\tcblower
\textit{Generated by AI}

\begin{tasks}
\task \textbf{Volume de la pyramide :}

Le volume d'une pyramide est donné par :
$$V = \dfrac{1}{3} \times \text{Aire de la base} \times \text{Hauteur}$$

L'aire de la base carrée $ABCD$ est :
$$A_{\text{base}} = 230^2 = 52\,900 \text{ m}^2$$

Le volume est donc :
$$V = \dfrac{1}{3} \times 52\,900 \times 147 = \dfrac{7\,776\,300}{3} = 2\,592\,100 \text{ m}^3$$

$$\boxed{V = 2\,592\,100 \text{ m}^3}$$

\task \textbf{Calcul des angles :}

\textbf{Angle $\widehat{IAS}$ :}

$I$ est le centre du carré $ABCD$, donc $IA$ est la moitié de la diagonale du carré.

La diagonale du carré vaut $230\sqrt{2}$, donc :
$$IA = \dfrac{230\sqrt{2}}{2} = 115\sqrt{2} \approx 162{,}63 \text{ m}$$

Dans le triangle rectangle $SIA$ (rectangle en $I$) :
$$\tan(\widehat{IAS}) = \dfrac{SI}{IA} = \dfrac{147}{115\sqrt{2}} \approx \dfrac{147}{162{,}63} \approx 0{,}904$$

$$\widehat{IAS} = \arctan(0{,}904) \approx 42°$$

$$\boxed{\widehat{IAS} \approx 42°}$$

\textbf{Angle $\widehat{SMI}$ :}

$M$ est le milieu de $[BC]$, donc $IM$ est perpendiculaire à $BC$ et $IM = \dfrac{230}{2} = 115$ m.

Dans le triangle rectangle $SMI$ (rectangle en $I$) :
$$\tan(\widehat{SMI}) = \dfrac{SI}{IM} = \dfrac{147}{115} \approx 1{,}278$$

$$\widehat{SMI} = \arctan(1{,}278) \approx 52°$$

$$\boxed{\widehat{SMI} \approx 52°}$$
\end{tasks}
}

