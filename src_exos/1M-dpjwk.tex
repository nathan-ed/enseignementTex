\titre{}
\theme{limites}
\auteur{Nathan Scheinmann}
\niveau{3M}
\source{sesamath3e}
\type{serie}
\piments{4}
\pts{}
\annee{2425}

\contenu{
\tcblower
\noindent Soit \(f\) la fonction définie par  
\[
f(x)=\begin{cases}
2+k, & x<2,\\
x^2-2, & x\ge2.
\end{cases}
\]
Pour quelle(s) valeur(s) de \(k\) la fonction \(f\) est-elle continue ?
}
\correction{
	\tcblower
\begin{itemize}
  \item Chaque expression définissant \( f \) est polynomiale, donc continue sur son domaine.
  \item Le seul point critique est \( x = 2 \), où les deux morceaux se rejoignent.
  \item \( \lim_{x \to 2^-} f(x) = 2 + k \), \quad \( \lim_{x \to 2^+} f(x) = 2^2 - 2 = 2 \)
  \item Pour que $f$ soit continue en $2$, on doit avoir $\lim_{x\to 2}f(x)=f(2)$. On impose \( 2 + k = 2 \), donc \( k = 0 \).
\end{itemize}
}

