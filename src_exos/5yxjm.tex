\titre{}
\theme{dérivées}
\auteur{Nathan Scheinmann}
\niveau{3M}
\source{fundamentum}
\type{serie}
\piments{2}
\pts{}
\annee{2425}

\contenu{
\tcblower
Une fenêtre a la forme d'un rectangle surmonté d'un demi-cercle. Si le périmètre de la fenêtre est de 6 m, quelles seront les dimensions de la fenêtre laissant passer un maximum de lumière ?
}
\correction{
\tcblower
%% GENERATED BY AI %%
{\scriptsize \textit{Correction générée par IA}}

Soit $x$ le rayon de la base du demi-cercle (largeur de la fenêtre) et $h$ la hauteur du rectangle.

Le périmètre de la fenêtre comprend :
\begin{itemize}
\item La base du rectangle : $2x$
\item Les deux côtés verticaux : $2h$
\item Le demi-cercle : $\pi x$
\end{itemize}

Donc : $2x + 2h + \pi x = 6$, d'où :
\[h = \frac{6 - 2x - \pi x}{2} = 3 - x - \frac{\pi x}{2} = 3 - x\left(1 + \frac{\pi}{2}\right)\]

L'aire de la fenêtre est :
\[A = A_{\text{rectangle}} + A_{\text{demi-cercle}} = 2xh + \frac{\pi x^2}{2}\]

Substituons l'expression de $h$ :
\[A(x) = 2x\left[3 - x\left(1 + \frac{\pi}{2}\right)\right] + \frac{\pi x^2}{2}\]
\[= 6x - 2x^2\left(1 + \frac{\pi}{2}\right) + \frac{\pi x^2}{2}\]
\[= 6x - 2x^2 - \pi x^2 + \frac{\pi x^2}{2}\]
\[= 6x - 2x^2 - \frac{\pi x^2}{2}\]
\[= 6x - x^2\left(2 + \frac{\pi}{2}\right)\]

Pour maximiser l'aire, dérivons :
\[A'(x) = 6 - 2x\left(2 + \frac{\pi}{2}\right) = 6 - x(4 + \pi)\]

Annulation :
\[6 - x(4 + \pi) = 0 \implies x = \frac{6}{4 + \pi}\]

Donc :
\[h = 3 - \frac{6}{4 + \pi}\left(1 + \frac{\pi}{2}\right) = 3 - \frac{6(2 + \pi)}{2(4 + \pi)} = 3 - \frac{3(2 + \pi)}{4 + \pi}\]
\[= \frac{3(4 + \pi) - 3(2 + \pi)}{4 + \pi} = \frac{12 + 3\pi - 6 - 3\pi}{4 + \pi} = \frac{6}{4 + \pi}\]

Les dimensions optimales sont :
\[\boxed{\text{Largeur} = 2x = \frac{12}{4 + \pi} \approx 1{,}68 \text{ m}, \quad \text{Hauteur} = h = \frac{6}{4 + \pi} \approx 0{,}84 \text{ m}}\]
}

