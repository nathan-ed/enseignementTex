\titre{}
\theme{numerique}
\auteur{Nathan Scheinmann}
\niveau{1M}
\source{musy}
\type{serie}
\piments{1}
\pts{}
\annee{2425}

\contenu{
\tcblower
Leonhard EULER énonça en 1772:
\enquote{Le nombre $n^2+n+41$ est premier pour $n \leq 39.$} $(n \in \mathbb{N})$
\begin{tasks}(1)
	\task Vérifier son affirmation pour $0 \leq n \leq 6$.
\task (*) Montrer que $n^2+n+41$ n'est premier ni pour 41 ni pour 40, sans calculer la valeur du nombre pour 40 ni 41 et sans la liste, mais uniquement par factorisation.
\end{tasks}
%\end{enumerate}
}
\correction{
\tcblower
	\begin{tasks}(3)
		\task[] Pour $n=0$ on obtient $41$.	
		\task[] Pour $n=1$ on obtient $43$.	
		\task[] Pour $n=2$ on obtient $47$.	
		\task[] Pour $n=3$ on obtient $53$.	
		\task[] Pour $n=4$ on obtient $61$.	
		\task[] Pour $n=5$ on obtient $71$.	
		\task[] Pour $n=6$ on obtient $83$.	
	\end{tasks}
}

