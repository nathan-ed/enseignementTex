\titre{}
\theme{derivation}
\auteur{Nathan Scheinmann}
\niveau{3M}
\source{sesamath3e}
\type{serie}
\piments{2}
\pts{}
\annee{2425}

\contenu{
\tcblower
Trouver $x$ et $y$ des nombres positifs dont la somme soit égale à 60 et tels que $xy^3$ soit :

\begin{tasks}(2)
\task maximal
\task minimal.
\end{tasks}
}
\correction{
\tcblower

\textit{Generated by AI}

Nous cherchons $x, y > 0$ tels que $x + y = 60$ et nous voulons optimiser $f(x,y) = xy^3$.

En utilisant la contrainte, nous pouvons écrire $y = 60 - x$, donc :
\[f(x) = x(60-x)^3\]

avec $x \in [0; 60]$.

\begin{tasks}(1)
\task \textbf{Cas maximal :}

Pour trouver le maximum, calculons $f'(x)$ :
\begin{align*}
f'(x) &= (60-x)^3 + x \cdot 3(60-x)^2 \cdot (-1)\\
&= (60-x)^3 - 3x(60-x)^2\\
&= (60-x)^2 \left[(60-x) - 3x\right]\\
&= (60-x)^2(60 - 4x)
\end{align*}

Pour $f'(x) = 0$ :
\[(60-x)^2(60 - 4x) = 0\]

Cela donne $x = 60$ (minimum car $f(60) = 0$) ou $60 - 4x = 0$, soit $x = 15$.

Vérifions qu'il s'agit d'un maximum :
\begin{itemize}
\item Pour $x < 15$ : $f'(x) > 0$ (fonction croissante)
\item Pour $15 < x < 60$ : $f'(x) < 0$ (fonction décroissante)
\end{itemize}

Donc $x = 15$ donne un maximum.

Avec $x = 15$, nous avons $y = 60 - 15 = 45$.

\[\boxed{x = 15 \text{ et } y = 45}\]

La valeur maximale de $xy^3$ est :
\[f(15) = 15 \times 45^3 = 15 \times 91125 = 1\,366\,875\]

\task \textbf{Cas minimal :}

La fonction $f(x) = x(60-x)^3$ sur $[0;60]$ :
\begin{itemize}
\item Aux bords : $f(0) = 0$ et $f(60) = 0$
\item Le maximum est atteint en $x = 15$ (trouvé ci-dessus)
\end{itemize}

Pour des nombres positifs avec $x + y = 60$, le minimum de $xy^3$ est atteint aux extrémités du domaine :

\[\boxed{x = 0, y = 60 \quad \text{ou} \quad x = 60, y = 0}\]

Dans les deux cas, $xy^3 = 0$.

\textbf{Remarque :} Si on exige $x, y > 0$ (strictement positifs), alors le minimum n'est pas atteint, mais l'infimum est $0$ (on peut se rapprocher arbitrairement de $0$ en prenant $x$ ou $y$ très proche de $0$).
\end{tasks}

}

