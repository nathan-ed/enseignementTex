\titre{13}
\theme{trigo}
\auteur{Nathan Scheinmann}
\niveau{1M}
\source{sesamath-1M-trigo}
\type{serie}
\piments{2}
\pts{}
\annee{2425}

\contenu{
\tcblower
Donner des approximations au millième des angles et longueurs manquants du triangle suivant. 
\begin{center}
\includegraphics[scale=1]{../medias/1M/trigo/1M-exo-13}	
\end{center}
}
\correction{
\tcblower
\textit{Generated by AI}

Nous avons un triangle avec deux côtés et un angle connus :
\begin{itemize}
\item Le côté opposé à l'angle $\alpha$ mesure 6 cm
\item Le côté opposé à l'angle de $40°$ mesure 7 cm
\item L'angle en bas à droite mesure $40°$
\end{itemize}

\textbf{Calcul de l'angle $\alpha$ :}

D'abord, nous trouvons l'angle $\gamma$ en utilisant la somme des angles d'un triangle :
$$\alpha + \gamma + 40° = 180°$$

Pour trouver $\alpha$, nous utilisons la loi des sinus :
$$\frac{6}{\sin(\alpha)} = \frac{7}{\sin(40°)}$$

D'où :
$$\sin(\alpha) = \frac{6 \times \sin(40°)}{7} = \frac{6 \times 0{,}643}{7} \approx 0{,}551$$

Donc :
$$\alpha = \arcsin(0{,}551) \approx 33{,}436° \approx \boxed{33{,}436°}$$

\textbf{Calcul de l'angle $\gamma$ :}

$$\gamma = 180° - 40° - 33{,}436° = \boxed{106{,}564°}$$

\textbf{Calcul du côté $c$ :}

En utilisant à nouveau la loi des sinus :
$$\frac{c}{\sin(\gamma)} = \frac{7}{\sin(40°)}$$

D'où :
$$c = \frac{7 \times \sin(106{,}564°)}{\sin(40°)} = \frac{7 \times 0{,}960}{0{,}643} \approx \boxed{10{,}459 \text{ cm}}$$
}

