\titre{limites}
\theme{limites}
\auteur{Nathan Scheinmann}
\niveau{3M}
\source{nathan}
\type{serie}
\piments{1}
\pts{}
\annee{2526}

\contenu{
\tcblower
Dans chacun des cas suivants, déterminer l'expression d'une fonction rationnelle répondant aux conditions données. 
\begin{tasks}
	\task Une fonction rationnelle passant par le point $(0;9)$ et ayant une asymptote horizontale en $y=4$ et deux asymptotes verticales en $x=-1$ et $x=2$. 
	\task Une fonction rationnelle passant par les points $(-3;1)$ et $(-1;0)$ et ayant une asymptote horizontale en $y=-3$ et trois asymptotes verticales en $x=0$, $x=-4$ et $x=5$.
\end{tasks}
}
\correction{
\tcblower
\begin{tasks}(2)
	\task Par exemple, $f(x)=\dfrac{4x^2-18}{(x+1)(x-2)}$
	\task Par exemple, $f(x)=\dfrac{-3(x+1)(x-1)(x+2)}{x(x+4)(x-5)}$ 
\end{tasks}
}

