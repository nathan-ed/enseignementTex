\titre{}
\theme{numerique}
\auteur{Nathan Scheinmann}
\niveau{1M}
\source{musy}
\type{serie}
\piments{1}
\pts{}
\annee{2425}

\contenu{
\tcblower
Sur les multiples de $3$~:
\begin{enumerate}
\item Trouver le plus grand multiple de $3$, formé de cinq chiffres et terminant par $24$.
\item Trouver le plus petit multiple de $3$, formé de quatre chiffres et terminant par $24$.
\item Trouver le plus petit multiple de $3$, formé de quatre chiffres pairs distincts.
\item Trouver le plus grand multiple de $3$, formé de quatre chiffres impairs distincts.
\end{enumerate}
}
\correction{
\tcblower
	On note un nombre à cinq chiffres  \[a+b\cdot 10+c\cdot 10^2+d\cdot 10^3+e\cdot 10^4 \quad\text{ où } a,b,c,d,e\in \N, e\neq 0\]
	Si le nombre a quatre chiffres, alors on prend $e=0$ et $d\neq 0$.

	\begin{tasks}
	\task On a $a=4$ et $b=2$. Par ailleurs la somme $a+b+c+d+e$ doit être divisible par $3$ pour que le nombre soit un multiple de $3$. On a $2+4=6$ qui est déjà un multiple de $3$. Le nombre recherché est donc $99924$. 
	\task Le nombre recherché est $1224$.
	\task Le nombre recherché est $2046$.
	\task Le nombre recherché est $9753$.
	\end{tasks}
}

