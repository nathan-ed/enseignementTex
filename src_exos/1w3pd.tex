\titre{}
\theme{limites}
\auteur{Nathan Scheinmann}
\niveau{3M}
\source{sesamath3e}
\type{serie}
\piments{2}
\pts{}
\annee{2425}

\contenu{
\tcblower
\noindent Tracer le graphe d'une fonction réelle \(f\) satisfaisant simultanément toutes les conditions suivantes :  
\begin{tasks}
\task \(f(2)=-4\)
\task l'ensemble des antécédents de 1 est \(\{5;8\}\)
\task l'ensemble des zéros de \(f\) est \(\{-4;7\}\)
\task \(\displaystyle\lim_{x\to-1}f(x)=0\) et \(\displaystyle\lim_{x\to1}f(x)=-\infty\)
\task \(f\) n'est pas définie en \(x=-3\) et \(\displaystyle\lim_{x\to-3}f(x)=-2\)
\task \(f(1)=-5\) et \(\displaystyle\lim_{x\to1}f(x)\) n'existe pas
\task \(\displaystyle\lim_{x\to+\infty}f(x)=3\)
\end{tasks}
}
\correction{
\tcblower
À vérifier avec l'enseignant. 
}

