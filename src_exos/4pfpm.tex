\titre{}
\theme{derivation}
\auteur{Nathan Scheinmann}
\niveau{3M}
\source{sesamath3e}
\type{serie}
\piments{2}
\pts{}
\annee{2425}

\contenu{
\tcblower
Utiliser les formules de dérivation pour  déterminer les dérivées suivantes en justifiant toutes les étapes, et en donnant des réponses sans exposant négatif ou fractionnaire :
\begin{tasks}(4)
\task $\displaystyle (x^{2}-3)'$
\task $\displaystyle \left(\dfrac{2}{x^{5}}\right)'$
\task $\displaystyle \left(\sqrt{2x^{3}-3}\right)'$
\task $\displaystyle \left(\sqrt[3]{x^{3}+1}\right)'$
\end{tasks}
}
\correction{
\tcblower

\textit{Generated by AI}

\begin{tasks}(1)
\task $\displaystyle (x^{2}-3)'$

Utilisons la formule $(u+v)' = u' + v'$ et $(x^n)' = nx^{n-1}$ :
\begin{align*}
(x^{2}-3)' &= (x^{2})' - (3)'\\
&= 2x^{2-1} - 0\\
&= \boxed{2x}
\end{align*}

\task $\displaystyle \left(\dfrac{2}{x^{5}}\right)'$

Réécrivons d'abord avec un exposant négatif : $\dfrac{2}{x^{5}} = 2x^{-5}$.

Utilisons les formules $(cu)' = cu'$ et $(x^n)' = nx^{n-1}$ :
\begin{align*}
\left(\dfrac{2}{x^{5}}\right)' &= (2x^{-5})'\\
&= 2 \cdot (-5)x^{-5-1}\\
&= -10x^{-6}
\end{align*}

Sans exposant négatif :
\[\boxed{-\dfrac{10}{x^{6}}}\]

\task $\displaystyle \left(\sqrt{2x^{3}-3}\right)'$

Réécrivons avec un exposant fractionnaire : $\sqrt{2x^{3}-3} = (2x^{3}-3)^{1/2}$.

Utilisons la formule de dérivation composée $(u^n)' = nu^{n-1} \cdot u'$ :
\begin{align*}
\left(\sqrt{2x^{3}-3}\right)' &= \left[(2x^{3}-3)^{1/2}\right]'\\
&= \frac{1}{2}(2x^{3}-3)^{1/2-1} \cdot (2x^{3}-3)'\\
&= \frac{1}{2}(2x^{3}-3)^{-1/2} \cdot 6x^{2}\\
&= \frac{6x^{2}}{2(2x^{3}-3)^{1/2}}\\
&= \frac{3x^{2}}{(2x^{3}-3)^{1/2}}
\end{align*}

Sans exposant fractionnaire :
\[\boxed{\dfrac{3x^{2}}{\sqrt{2x^{3}-3}}}\]

\task $\displaystyle \left(\sqrt[3]{x^{3}+1}\right)'$

Réécrivons avec un exposant fractionnaire : $\sqrt[3]{x^{3}+1} = (x^{3}+1)^{1/3}$.

Utilisons la formule de dérivation composée :
\begin{align*}
\left(\sqrt[3]{x^{3}+1}\right)' &= \left[(x^{3}+1)^{1/3}\right]'\\
&= \frac{1}{3}(x^{3}+1)^{1/3-1} \cdot (x^{3}+1)'\\
&= \frac{1}{3}(x^{3}+1)^{-2/3} \cdot 3x^{2}\\
&= \frac{3x^{2}}{3(x^{3}+1)^{2/3}}\\
&= \frac{x^{2}}{(x^{3}+1)^{2/3}}
\end{align*}

Sans exposant fractionnaire :
\[\boxed{\dfrac{x^{2}}{\sqrt[3]{(x^{3}+1)^{2}}}}\]

Ou encore, en utilisant $\sqrt[3]{a^2} = (\sqrt[3]{a})^2$ :
\[\boxed{\dfrac{x^{2}}{\left(\sqrt[3]{x^{3}+1}\right)^{2}}}\]
\end{tasks}

}

